[ಅ]
<ಅಕಾರಾದಿ ವರ್ಣಾನುಕ್ರಮಣಿಕೆ–ಋಗ್ವೇದ ಮಂತ್ರಗಳ>
1-717

<ಅಂಕಾ ಎಂಬ ಶಬ್ದದ ವಿವರಣೆ>
12-195

<ಅಕೂಪಾರಸ್ಯ ಎಂಬ ಶಬ್ದದ ನಿರ್ವಚನ>
19-311

<ಅಕ್ರಾನ್ತ್ಸಮುದ್ರಃ ಎಂಬ ಋಕ್ಕಿಗೆ ಅರ್ಥದ್ವಯ ವಿವರಣೆ>
26-833

<ಅರ್ಕ ಶಬ್ದದ ನಾನಾರ್ಥಗಳು ಮತ್ತು ವಿವರಣೆ>
11-218 
13-416 
13-35 
14-139 
19-190

<ಅರ್ಕ, ಅಗ್ನಿ, ಗಾಯತ್ರೀ ಇವುಗಳಿಗಿರುವ ಸಂಬಂಧ>
12-446

<ಅಗಸ್ತ್ಯ ಪುರೋಹಿತನಾದ ಖೇಲನೆಂಬ ರಾಜನ ಸಂಬಂಧಿಯಾದ ವಿಶ್ಪಲಾ ಎಂಬ ಸ್ತ್ರೀಯ ಕತ್ತರಿಸಿ ಹೋದ ಕಾಲನ್ನು ಅಶ್ವಿನೀ ದೇವತೆಗಳು ಸರಿಮಾಡಿದ ವಿಚಾರ>
9-262 
9-355

<ಅಗಸ್ತ್ಯ ಋಷಿಯ ಪಶ್ಚಾತ್ತಾಪ>
13-213

<ಅಗಸ್ತ್ಯ ಋಷಿಯ ಸೂಕ್ಷ್ಮ ಪರಿಚಯ>
13-1

<ಅಗಸ್ತ್ಯ ಋಷಿಗೂ ಅವನ ಪತ್ನಿಯಾದ ಲೋಪಾಮುದ್ರೆಗೂ ರತಿವಿಷಯವಾಗಿ ನಡೆದ ಸಂಭಾಷಣೆಯಲ್ಲಿ ಅಗಸ್ತ್ಯನು ಲೋಪಾಮುದ್ರೆಗೆ ಹೇಳುವ ಪ್ರತ್ಯುತ್ತರ>
13-479

<ಅಗಸ್ತ್ಯ ಋಷಿಗೂ ಅವನ ಪತ್ನಿಯಾದ ಲೋಪಾಮುದ್ರೆಗೂ ರತಿವಿಷಯವಾಗಿ ನಡೆದ ಸಂಭಾಷಣೆ>
13-470

<ಅಗಸ್ತ್ಯನು ಮಿತ್ರಾವರುಣರ ಪುತ್ರನೆಂಬ ವಿಷಯದಲ್ಲಿ ಬೃಹದ್ದೇವತಾಕಾರರ ವಿವರಣೆ>
13-492

<ಅಗಸ್ತ್ಯ ಋಷಿಯನ್ನು ಅವನ ಪತ್ನಿಯಾದ ಲೋಪಾಮುದ್ರೆಯು ರತಿವಿಷಯವಾಗಿ ಪ್ರಾರ್ಥಿಸುವುದು>
13-473

<ಅಗಸ್ತ್ಯ ಶಿಷ್ಯನು ಗುರು ಮತ್ತು ಗುರುಪತ್ನಿಯ ರಹಸ್ಯ ಸಂಭಾಷಣೆಯನ್ನು ಕೇಳಿ ಪಶ್ಚಾತ್ತಾಪವನ್ನು ಸೂಚಿಸುವುದು>
13-488

<ಅಗಸ್ತ್ಯ ಋಷಿಗೆ ಮಾನ್ಯನೆಂಬ ಹೆಸರು ಹೇಗೆ ಬಂದಿತೆಂಬ ವಿಚಾರ>
13-621

<ಅಗಸ್ತ್ಯ ವಸಿಷ್ಠ ಋಷಿಗಳ ಜನ್ಮಕಥನವು>
9-355

<ಅಗಸ್ತ್ಯ ಋಷಿ, ಇಂದ್ರ ಮತ್ತು ಮರುತ್ತುಗಳು (ಪ್ರಥಮ ಮಂಡಲದ  169–170 ಸೂಕ್ತಗಳು–ಬೃಹದ್ದೇವತಾ)>
28-817

<ಅಗಸ್ತ್ಯ ಋಷಿ ಮತ್ತು ಅವನ ಪತ್ನಿಯಾದ ಲೋಪಾಮುದ್ರಾ–ಬೃಹದ್ದೇವತಾ>
28-819

<ಅಗ್ನಾಯೀ ಎಂಬ ಅಗ್ನಿ ಪತ್ನಿಯ ವಿಷಯ>
3-100

<ಅಗ್ನಿ ಶಬ್ದದ ರೂಪನಿಷ್ಪತ್ತಿ, ಅವಯವಾರ್ಥ ಇತ್ಯಾದಿ>
1-612

<ಅಗ್ನಿಗೆ ಅಂಗಿರ ಎಂಬ ಹೆಸರು ಬರಲು ಕಾರಣ>
1-640 
17-752 
20-577 
25-247

<ಅಗ್ನಿಷ್ಟೋಮ ಯಾಗದ ಸಂಕ್ಷೇಪವಾದ ಸ್ವರೂಪ ಸಂಗ್ರಹವು>
1-667

<ಅಗ್ನಿಯು ನೀರಿನಲ್ಲಿ ಅಡಗಿಕೊಂಡಿದ್ದನೆಂಬ ಉಪಾಖ್ಯಾನ ವಿವರಣೆ–ಈ ವಿಷಯದಲ್ಲಿ ಋಗ್ವೇದ ಮತ್ತು ಯಜುರ್ವೇದಗಳಲ್ಲಿರುವ ಪೂರ್ವೇತಿಹಾಸ ಕಥನವು>
2-60

<ಅಗ್ನಿಯು ಮಾತರಿಶ್ವನಿಗೆ ಆವಿರ್ಭೂತನಾದ ವಿಚಾರ>
3-530 
11-298

<ಅಗ್ನಿ ಎಂಬ ದೇವತೆಯ ವಿಷಯ>
5-606

<ಅಗ್ನಿಯು ನೀರಿನಲ್ಲಿ ಗೂಢವಾಗಿ ಅಡಗಿಕೊಂಡಿದ್ದ ವಿಚಾರ>
6-159
15-506

<ಅಗ್ನಿಯು ವಿವಾಹ ಕಾಲದಲ್ಲಿ ಕನ್ಯೆಗೆ ಪತಿಯಾಗುವ ವಿಚಾರ>
6-196

<ಅಗ್ನಿಯ ಉತ್ಪತ್ತಿ, ವರ್ಣನೆ ಇತ್ಯಾದಿ>
8-11

<ಅಗ್ನಿಯ ಮೂರು ವಿಧವಾದ ಜನ್ಮಗಳು>
8-17 
11-165 
15-510
28-694

<ಅಗ್ನಿಯ ವಿದ್ಯುದಾದಿ ಸ್ವರೂಪಗಳು>
8-23

<ಅಗ್ನಿಯ ದೇವಬಂಧುತ್ವ>
8-500

<ಅಗ್ನಿ ಶಬ್ದದ ನಿರ್ವಚನ>
10-260
14-429 
14-381
16-45

<ಅಗ್ನಿಯು ಅಂಗಿರಾ ಋಷಿಗಳಲ್ಲಿ ಶ್ರೇಷ್ಠನೆಂಬ ವಿಚಾರ>
10-269

<ಅಗ್ನಿಜಿಹ್ವ ಶಬ್ದಾರ್ಥ ವಿವರಣೆ>
4-539

<ಅಗ್ನಿಜ್ವಾಲೆಗಳ ವೈಶಿಷ್ಟ್ಯವು>
11-113

<ಅಗ್ನಿಜ್ವಾಲೆಯ ಪ್ರಸರಣಶಕ್ತಿ ವರ್ಣನೆ>
11-122

<ಅಗ್ನಿಯ ವಿವಿಧ ರೂಪಗಳ ಸ್ವರೂಪ>
11-426

<ಅಗ್ನಿಮಿಂಧಃ ಎಂಬ ಋತ್ವಿಜನ ಕರ್ತವ್ಯ>
12-166

<ಅಗ್ನಿಯ ಸ್ಥಾನತ್ರಯಗಳ ವರ್ಣನೆ>
12-356

<ಅಗ್ನಿಗೂ ಆದಿತ್ಯನಿಗೂ ಇರುವ ಸಂಬಂಧ>
12-403

<ಅಗ್ನಿ, ಅರ್ಕ, ಗಾಯತ್ರೀ ಇವುಗಳಿಗಿರುವ ಸಂಬಂಧ ಅಗ್ನಿಯಲ್ಲಿ ಹೋಮ>
12-446

<ಅಗ್ನಿಯಲ್ಲಿ ಹೋಮ ಮಾಡಿದ ಆಹುತಿಯು ಆದಿತ್ಯನನ್ನು ಸೇರಿ ವೃಷ್ಟ್ಯಾದಿ ಫಲವನ್ನುಂಟು ಮಾಡುವುದು>
12-531

<ಅಗ್ನಿಯು ಯಜ್ಞ ಶಬ್ದವಾಚ್ಯನು>
12-652

<ಅಗ್ನಿಗೂ ಪಶುವಿಗೂ ಇರುವ ಸಾದೃಶ್ಯ>
12-654

<ಅಗ್ನಿಯ ದಾವಾನಲಸ್ವರೂಪ>
14-478

<ಅಗ್ನಿಪ್ರಣಯನವೆಂಬ ಕರ್ಮದ ವಿವರಣೆ (ಐತರೇಯ ಬ್ರಾಹ್ಮಣದಲ್ಲಿರುವಂತೆ)>
14-541

<ಅಗ್ನಿಯು ದೇವತೆಗಳಿಗೆ ನಾಯಕನೆಂಬ ವಿಷಯ>
15-586

<ಅಗ್ನಿಷ್ಟೋಮಕ್ಕೂ ಅಗ್ನಿಗೂ ಇರುವ ಸಂಬಂಧ>
15-603

<ಅಗ್ನಿಯ ಎಂಟು ಹೆಸರುಗಳು>
15-647

<ಅಗ್ನಿಯು ಓಷಧಿ ವನಸ್ಪತಿಗಳನ್ನು ವ್ಯಾಪಿಸಿರುವ ವಿಷಯ>
15-764

<ಅಗ್ನಿಯ ಸಪ್ತಜಿಹ್ವೆಗಳು>
15-777

<ಅಗ್ನಿರಸ್ಮಿ ಜನ್ಮನಾ ಎಂಬ ಋಕ್ಕಿನ ಎರಡು ವಿಧ ಅರ್ಥ ಮತ್ತು ವಿಷಯ ವಿಮರ್ಶೆ>
16-520

<ಅಗ್ನಿಯು ಸಹಸಸ್ಪುತ್ರ (ಶಕ್ತಿಯ ಪುತ್ರ)ನೆಂಬ ವಿಷಯ>
16-574

<ಅಗ್ನಿಯ ಕರ್ಮಗಳು>
18-683 
28-246

<ಅಗ್ನಿ ದೇವತೆಯ ಸ್ವರೂಪ ಮತ್ತು ವಿಶೇಷ ವರ್ಣನೆ>
18-683

<ಅಗ್ನಿಯು ನೀರಿನಲ್ಲಿ ಅವಿತುಕೊಂಡಿದ್ದಾಗ ದೇವತೆಗಳು ಅವನನ್ನು ಹುಡುಕಿಕೊಂಡು ಹೋಗಿ ಅವನೊಡನೆ ಸಂಭಾಷಣೆ ನಡೆಸಿದ ವಿವರ (ಋ.ಸಂ. ೧೦-೫೧ನೇ ಸೂಕ್ತದಂತೆ)>
18-687

<ಅಗ್ನಿಯು ನೀರಿನಲ್ಲಿ ಅವಿತುಕೊಂಡಿದ್ದ ವಿಷಯ (ತೈತ್ತಿರೀಯ ಸಂಹಿತೆಯಲ್ಲಿರುವಂತೆ)>
18-688
19-4 
22-296 
28-250

<ಅಗ್ನಿಯ ಉತ್ಪತ್ತಿ ಮತ್ತು ಮೂರು ಸ್ಥಾನಗಳು>
18-690

<ಅಗ್ನಿಗೆ ಇರುವ ದ್ವಿಜನ್ಮಾ, ಭೂರಿಜನ್ಮಾ ಇತ್ಯಾದಿ ವಿಶೇಷಣಗಳ ವಿವರಣೆ>
18-696

<ಅಗ್ನಿಯ ಸ್ವಾಭಾವಿಕ ಕರ್ಮಗಳನ್ನು ವರ್ಣಿಸುವ ವಿಶೇಷಣಗಳು>
18-698

<ಅಗ್ನಿಯ ದಿವ್ಯ ಕರ್ಮಗಳು>
18-701

<ಅಗ್ನಿಯು ತನ್ನ ಭಕ್ತರಲ್ಲಿಟ್ಟಿರುವ ಸಾಕ್ಷಾತ್ಸಂಬಂಧ>
18-704

<ಅಗ್ನಿಗೂ ಇಂದ್ರನಿಗೂ ಇರುವ ಸಾಹಚರ್ಯ>
18-708

<ಅಗ್ನಿಯು ಯುವಕ, ಯವಿಷ್ಠ ಎಂಬ ಹೆಸರಿನಿಂದ ಕರೆಯಲ್ಪಡುವ ವಿಚಾರ>
18-727

<ಅಗ್ನಿಗೆ ಕುಮಾರನೆಂಬ ಹೆಸರು ಹೇಗೆ ಬಂದಿತೆಂಬ ವಿಚಾರ>
18-739

<ಅಗ್ನಿಯು ಇಂದ್ರ ಶಬ್ದ ವಾಚ್ಯನೆಂಬ ವಿಷಯ ಸಮರ್ಥನೆ>
18-747

<ಅಗ್ನಿಯು ಇತರ ಎಲ್ಲಾ ದೇವತೆಗಳ ಸ್ವರೂಪನೆಂಬ ವಿಷಯ>
18-770

<ಅಗ್ನಿಯನ್ನು ಪಿತೃ ಪುತ್ರ ಭಾವದಿಂದ ಸ್ತುತಿಸಿರುವ ವಿಷಯ>
18-784

<ಅಗ್ನಿಗೆ ಹವ್ಯವಾಹನ, ಕವ್ಯವಾಹನ, ಸಹರಕ್ಷ ಎಂಬ ಮೂರು ವಿಧ ಹೆಸರುಗಳು ಬರಲು ಕಾರಣ>
18-849

<ಅಗ್ನಿಯ ಪ್ರಶಂಸೆ>
19-1

<ಅಗ್ನಿಯ ಹವ್ಯವಾಹಕತ್ವ>
19-2

<ಅಗ್ನಿಗೂ ದೇವತೆಗಳಿಗೂ ನಡೆದ ಸಂವಾದ>
19-2

<ಅಗ್ನಿಗೂ ಅಂಗಿರಾ ಋಷಿಗಳಿಗೂ ಇರುವ ಸಂಬಂಧ>
19-5

<ಅಗ್ನಿಯ ಪ್ರಸರಣಶಕ್ತಿ>
19-7

<ಅಗ್ನಿಯ ಸರ್ವಾಧಾರತ್ವ ಮತ್ತು ಸರ್ವಶಕ್ತಿತ್ವ>
19-9

<ಅಗ್ನಿಯನ್ನು ಅರಣಿಗಳಲ್ಲಿ ಉತ್ಪತ್ತಿ ಮಾಡುವ ವಿಚಾರ>
19-37

<ಅಗ್ನಿರೈತ ಪ್ರಥಮೋ ದೇವತಾನಾಂ ಎಂಬ ೧೦ ನೇ ಪರಿಶಿಷ್ಟ ಸೂಕ್ತ (೫ ಋಕ್ಕುಗಳು) ಅರ್ಥನಿವರಣೆಸಹಿತ>
20-265

<ಅಗ್ನೇ ವಿಶ್ವೇಭಿಃ ಎಂಬ ಋಕ್ಕಿಗೆ ಅಗ್ನಿಪ್ರಣಯನಕರ್ಮದಲ್ಲಿ ಐತರೇಯಬ್ರಾಹ್ಮಣದಲ್ಲಿ ವಿವರಿಸಿರುವ ಅರ್ಥವಿವರಣೆ>
20-557

<ಅಗ್ನಿಮಥನವೆಂಬ (ಯಜ್ಞಾಂಗಕರ್ಮದ) ವಿವರಣೆ–ಐತರೇಯಬ್ರಾಹ್ಮಣದಲ್ಲಿ ಹೇಳಿರುವಂತೆ>
20-628

<ಅಗ್ನಿದಗ್ಧಾಃ ಅನಗ್ನಿದಗ್ಧಾಃ ಎಂಬ ಪಿತೃಗಳು>
27-441

<ಅಗ್ನಿ, ಗಾಯತ್ರೀ ಶ್ಯೇನ ಇವುಗಳಿಗಿರುವ ತಾದಾತ್ಮ್ಯ ಮತ್ತು ಸವನಕಾಲಗಳಲ್ಲಿ ಇವುಗಳ ಸ್ಥಾನ>
17-578

<ಅಗ್ನಿಯು ಮಮತಾ ಎಂಬ ಸ್ತ್ರೀಯ ಪುತ್ರನಾದ ದೀರ್ಘತಮಾಃ ಎಂಬ ಋಷಿಯ ಅಂಧತ್ವವನ್ನು ಪರಿಹಾರಮಾಡಿದ ವಿಚಾರ>
17-782

<ಅಗ್ನಿಜಿಹ್ವಾಃ ಎಂಬ ಶಬ್ದದ ವಿವರಣೆ>
7-790

<ಅಗ್ನಿಯ ಮಹದ್ಗುಣಗಳ ವರ್ಣನೆ>
28-80

<ಅಗ್ನಿಯು ಮರಣರಹಿತನು ಎಂಬ ವಿಚಾರ>
28-86

<ಅಗ್ನಿಗೂ ಅಂಗಿರಾಋಷಿಗಳಿಗೂ ಇರುವ ಸಂಬಂಧ>
28-115

<ಅಗ್ನಿಗೂ ದೇವತೆಗಳಿಗೂ ನಡೆದ ಸಂಭಾಷಣೆ>
28-253

<ಅಗ್ನಿ, ಇಂದ್ರ ಅಥವಾ ವಾಯು, ಸೂರ್ಯ ಎಂಬ ದೇವತೆಗಳ ಪರವಾದ ಸೂಕ್ತಗಳ ವರ್ಣನೆ (ಬೃ)>
28-699

<ಅಗ್ನಿ, ಜಾತವೇದಾಃ ವೈಶ್ವಾನರ ಎಂಬ ಹೆಸರುಗಳು ಒಂದೇ ಅಗ್ನಿಯನ್ನು ಸೂಚಿಸುವುವಾದರೂ ಆ ಅಗ್ನಿಗಳ ಗುಣವರ್ಣನೆಯಲ್ಲಿ ಭೇದವು ಕಂಡುಬರುವುದು (ಬೃ)>
28-702

<ಅಗ್ನಿಯ ಐದು ಹೆಸರುಗಳು, ಅಗ್ನಿ ಶಬ್ದದ ರೂಪನಿಷ್ಪತ್ತಿ, ದ್ರವಿಣೋದಾಃ ಮತ್ತು ತನೂನಪಾತ್‍ ಎಂಬ ಶಬ್ದಗಳ ನಿಷ್ಪತ್ತಿ (ಬೃಹದ್ದೇವತಾಗ್ರಂಥದಲ್ಲಿ ಹೇಳಿರುವಂತೆ)>
28-717

<ಅಗ್ನಿಯು ದೇವತೆಗಳನ್ನು ಬಿಟ್ಟುಹೋಗಿ ನೀರಿನಲ್ಲಿ ಅಡಗಿಕೊಂಡಿದ್ದ ವಿಚಾರ (ಬೃ.ದೇ)>
28-964

<ಅಗ್ನಿಮಾಠರಶಾಖೆ>
1-63

<ಅಗ್ರುವಶಬ್ದಾರ್ಥವಿವರಣೆ>
11-132

<ಅಗ್ರು ಎಂಬುವಳ ಪುತ್ರನಾದ ಪರಾವೃಕ್ತನೆಂಬುವನನ್ನು ಇಂದ್ರನು ಹುತ್ತದಿಂದ ಕಾಪಾಡಿದ ವಿಚಾರ>
18-24

<ಅಗ್ರೇ ಬೃಹನ್‍ ಎಂಬ ಋಕ್ಕಿನ ವಿಶೇಷ ವಿವರಣೆ>
27-199

<ಅಂಗ ಎಂಬ ಶಬ್ದದ ವಿವರಣೆ>
7-99

<ಅಂಗಿರ ಎಂಬ ಹೆಸರು ಅಗ್ನಿಗೆ ಬರಲು ಕಾರಣ>
3-520 
6-442

<ಅಂಗಿರಸರು>
5-672 
10-522 
14-635

<ಅಂಗಿರಸ್ವತ್‍>
14-797 
6-514

<ಅಂಗಿರಾಋಷಿಯ ವೃತ್ತಾಂತ>
4-552 
6-375 
8-187

<ಅಂಗಿರಸರಿಗೂ ವರಾಹರೆಂದು ಹೆಸರು>
9-121 
28-569

<ಅಂಗಿರಸ್ಸುಗಳು (ದೇವತೆಗಳಿಂದ) ಗೋವುಗಳನ್ನು ಪಡೆದ ವೃತ್ತಾಂತ>
11-71

<ಅಂಗಿರಾಋಷಿಗಳ ಉತ್ಪತ್ತಿ>
27-365

<ಅಂಗಿರಸಃ ಎಂಬ ಪಿತೃಗಣಗಳು>
27-467

<ಅಂಗಿರಾಋಷಿಗಳಿಗೂ ಅಗ್ನಿಗೂ ಇರುವ ಸಂಬಂಧ>
28-115

<ಅಂಗಿರಸರು, ಆದಿತ್ಯ, ಭೃಗು ಇವರ ಉತ್ಪತ್ತಿ (ಬೃ)>
28-417 
28-874

<ಅಂಗಿರಾಶಬ್ದದ ರೂಪನಿಷ್ಪತ್ತಿ>
28-452

<ಅಘ್ನ್ಯಾ ಎಂಬ ಶಬ್ದದ ಅರ್ಥವಿವರಣೆ>
12-471

<ಅಚ್ಛ ಎಂಬ ಶಬ್ದದ ವಿವರಣೆ>
15-496

<ಅಜ>
5-683

<ಅಜ ಏಕಪಾತ್‍ ಎಂಬ ದೇವತೆಯ ವಿಷಯ>
5-588 
15-173

<ಅರ್ಜುನಶಬ್ದದ ಅರ್ಥವಿವರಣೆ>
17-7

<ಅಂಜಂತಿ ತ್ವಾ ಎಂಬ ಋಕ್ಕಿನ ವಿಷಯದಲ್ಲಿ ಐತರೇಯಬ್ರಾಹ್ಮಣದಲ್ಲಿರುವ ವಿವರಣೆ>
16-77

<ಅತಿಷ್ಠಂತೀನಾಂ ಎಂಬ ಋಕ್ಕಿನ ನಿರುಕ್ತ>
3-625

<ಅತ್ರಿ>
11-84 
22-488 
5-676

<ಅತ್ರಿಋಷಿಯನ್ನು ಅಶ್ವಿನೀದೇವತೆಗಳು ಕಾಪಾಡಿದ ವಿಚಾರ>
8-757

<ಅತ್ರಿಯೆಂಬ ಋಷಿಯನ್ನು ಅಸುರರು ತುಷಾಗ್ನಿಯಲ್ಲಿ ಹಾಕಿ ದಹಿಸುತ್ತಿರುವಾಗ ಅಶ್ವಿನೀದೇವತೆಗಳು ಬಂದು ಶೀತೋದಕದಿಂದ ಆ ಅಗ್ನಿಯನ್ನು ಆರಿಸಿ ಅವನನ್ನು ಕಾಪಾಡಿದ ವಿಚಾರ>
9-222
20-143

<ಅತಿಥಿಶಬ್ದದ ನಿರ್ವಚನ ಮತ್ತು ಅರ್ಥವಿವರಣೆ>
10-304 
18-725 
20-533
14-472

<ಅಂತರ್ಯಾಮಗ್ರಹದ ವಿವರಣೆ>
10-650

<ಅತ್ಯಶಬ್ದಾರ್ಥ>
11-454

<ಅತಿಥಿಗ್ವ ಎಂಬುವನ ವಿಷಯ>
5-173 
14-741
20-694

<ಅತ್ರಿಋಷಿಯ ಪರಿಚಯ>
18-669
18-674

<ಅಂತರಿಕ್ಷ, ದ್ಯುಲೋಕ ಪೃಥಿವೀಲೋಕಗಳ ಮೂರು ಅವಾಂತರಪ್ರಭೇದಗಳು>
20-187

<ಅಂತರಾಭರಃ ಎಂಬ ಶಬ್ದದ ಅರ್ಥವಿವರಣೆ>
24-238

<ಅಂತ್ಯಕ್ರಿಯೆ, ಅಂತ್ಯಸಂಸ್ಕಾರ>
27-485

<ಅತ್ಯತಿಷ್ಠತ್‍ ಎಂಬ ಶಬ್ದದ ವಿವರಣೆ>
29-720

<ಅಂತರಿಕ್ಷಸ್ಥನಾದ ವೈದ್ಯುತಾಗ್ನಿಯ ಪ್ರಶಂಸೆ>
28-59

<ಅತ್ರಿ, ವಿರೂಪಾಃ, ಅಂಗಿರಾಃ ಎಂಬ ಋಷಿಗಳ ವಿಷಯ>
4-552

<ಅತ್ರಿ, ಅಂಗಿರಸ, ಭೃಗು ಇವರ ಉತ್ಪತ್ತಿ (ಬೃ)>
28-874

<ಅತ್ರಿಋಷಿಗೆ ಅಶ್ವಿನೀದೇವತೆಗಳು ಸಹಾಯ ಮಾಡಿದ ವಿಚಾರ>
13-512 
13-643
27-960

<ಅಂತ್ಯಸಂಸ್ಕಾರಕ್ಕಾಗಿ ಉಪಯೋಗಿಸುವ ಕೆಲವು ಮಂತ್ರಗಳ ವಿವರಣೆ (ಬೃ)>
28-950

<ಅತಿಥಿಯನ್ನು ಸತ್ಕಾರಮಾಡದೆ ಅಲಕ್ಷ್ಯಮಾಡಿದರೆ ಉಂಟಾಗುವ ಅನರ್ಥ ಇತ್ಯಾದಿ>
30-848

<ಅಥರ್ವವೇದಸಂಹಿತಾ>
1-22

<ಅಥರ್ವವೇದ ಮತ್ತು ಅದರ ಶಾಖೆಗಳು>
1-157

<ಅಥರ್ವವೇದದ ಪದಪಾಠಕಾರರು>
1-291

<ಅರ್ಥವಾದಲಕ್ಷಣ>
1-394

<ಅಥರ್ವಾ>
5-671
6-617


<ಅರ್ಥಾನುಸಾರವಾಗಿ ಮಂತ್ರಗಳಲ್ಲಿರುವ ದೈವತಮಂತ್ರ, ಸ್ತುತಿಮಂತ್ರ, ಆಶೀರ್ಮಂತ್ರ ಇತ್ಯಾದಿ ಭೇದಗಳು (ಬೃ)>
14-682
28-686

<ಅಥರ್ವಋಷಿಯು ಅಗ್ನಿಯನ್ನು ಪುಷ್ಕರದಿಂದ ಮಥಿಸಿದ ವಿಷಯ>
20-580

<ಅಥರ್ವಾಃ ಎಂಬ ಪಿತೃಗಣಗಳು>
27-467

<ಅಥರ್ಯುಂ ಎಂಬ ಶಬ್ದದ ನಿರ್ವಚನ ಮತ್ತು ಅರ್ಥವಿವರಣೆ>
21-619

<ಅರ್ಥವಿವರಣೆ (ಶಬ್ದಗಳ) ಬೃ>
28-740

<ಅರ್ಥ, ಪದ, ಕ್ರಿಯಾಶಬ್ದಾರ್ಥ (ಬೃ)>
28-743

<ಅರ್ಥರಹಿತವಾದ ವೇದಾದ್ಯಯನವು ನಿಷ್ಪ್ರಯೋಜನ>
12-578

<ಅದಬ್ಧೇಭಿಃ ಎಂಬ ಶಬ್ದದ ವಿವರಣೆ>
8-51

<ಅದಿತಿ ಶಬ್ದಾರ್ಥ>
8-559 
14-67 
17-266 
28-491

<ಅದಿತೇರನೀಕಂ ಎಂಬ ಶಬ್ದಗಳ ಅರ್ಥಾನುವಾದ>
9-92

<ಅದಿತಿಯೆಂಬ ದೇವತೆಯ ವಿಷಯ–ಅದಿತಿಯ ಸ್ವರೂಪವನ್ನು ವರ್ಣಿಸುವ ವಿಶೇಷಣಗಳು>
18-558

<ಅದಿತಿಯ ಮೂಲರೂಪದ ವಿಷಯದಲ್ಲಿ Roth ಮತ್ತು Max-muller ಎಂಬ ಪಾಶ್ಚಾತ್ಯ ಪಂಡಿತರು ತಿಳಿಸಿರುವ ಅಭಿಪ್ರಾಯಗಳು>
18-389

<ಅದಿತಿಯು ಆದಿತ್ಯರ ಮಾತೆ ಎಂದು ವರ್ಣಿಸಿರುವ ಪ್ರಕರಣಗಳು>
18-590

<ಅದಿತಿಗೂ ಅಂತರಿಕ್ಷಕ್ಕೂ ತಾದಾತ್ಮ್ಯಭಾವವಿದೆಯೇ>
18-591

<ಅದಿತಿಯು ಪೃಥ್ವಿಗಿಂತಲೂ ಬೇರೆಯಾದ ದೇವತೆ ಎಂಬುದಕ್ಕೆ ಆಧಾರಗಳು>
18-592

<ಅದಿತಿ ಮತ್ತು ದಿತಿ>
5-645
18-594

<ಅದಿತಿಯೇ ವಿಶ್ವಪ್ರಕೃತಿಯನ್ನು ಪ್ರತಿಬಿಂಬಿಸುವ ದೇವತಾತತ್ತ್ವ>
18-595

<ಅದಿತಿಯು ಪಾಪವನ್ನು ಕ್ಷಮಿಸತಕ್ಕ ದೇವತೆ>
18-596

<ಅದಿತಿಯೂ ಸಹ (ವಿಶ್ವನಿಯಾಮಕಳಾದ) ಇತರ ಶಕ್ತಿಗಳಿಂದ ನಿಯಂತ್ರಿತಳಾದವಳು ಎಂದು ವರ್ಣಿಸಿರುವ ಕೆಲವು ಪ್ರಕರಣಗಳು>
18-598 
29-23

<ಅದಿತಿ, ದಕ್ಷ, ಆದಿತ್ಯರು ಮತ್ತು ಇತರ ದೇವತೆಗಳ ಉತ್ಪತ್ತಿಕ್ರಮ (ಋ. ಸಂ. ೧೦ ನೇ ಮಂಡಲದ ೭೨ ನೇ ಸೂಕ್ತದಲ್ಲಿ ವರ್ಣಿತವಾಗಿರುವಂತೆ ವಿಶ್ವಸೃಷ್ಟಿಯ ಕ್ರಮ)>
18-598

<ಅದಿತಿಯಲ್ಲಿ ದ್ವಿಪ್ರಕಾರವಾಗಿ (ಅವಳಿ) ಜನಿಸಿದ ದೇವತೆಗಳ ವಿಚಾರ>
25-240

<ಅದಿತಿಯಲ್ಲಿ ವಿವಸ್ವಂತನು ಹುಟ್ಟಿದ ಬಗೆ>
27-539

<ಅದಿತಿಯ ಸ್ವರೂಪ ಇತ್ಯಾದಿ>
29-25

<ಅದೋ ಯದ್ದಾರು ಪ್ಲವತೇ ಎಂಬ ಋಕ್ಕಿಗೆ ಅಧ್ಯಾತ್ಮಪರವಾದ ಅರ್ಥ>
30-1047

<ಅದೋ ಯದ್ದಾರು ಪ್ಲವತೇ ಎಂಬ ಋಕ್ಕಿಗೆ ದುರ್ಭಿಕ್ಷದೇವತೆಯಪರವಾದ ಅರ್ಥ>
30-1047

<ಅದ್ರಿ ಶಬ್ದದ ವಿವರಣೆ>
5-161 
6-408 
8-633 
9-580 
10-410 
11-446
12-692 
15-498 
26-717 
17-23

<ಅದ್ರಿ ಮತ್ತು ಗ್ರಾವ ಶಬ್ದಗಳ ವಿವರಣೆ>
10-614

<ಅಧಮದೇವತೆಗಳು>
5-645

<ಅಧಿಕಮಾಸೋತ್ಪತ್ತಿಕ್ರಮ>
3-313

<ಅಧ್ವರ್ಯುವಿನ ಕರ್ತವ್ಯ>
12-166

<ಅಧ್ವರ್ಯವೋ ಭರತ ಎಂಬ ಋಕ್ಕಿಗೆ ನಿರುಕ್ತವ್ಯಾಖ್ಯಾಕಾರರ ವ್ಯಾಖ್ಯಾನ ಮತ್ತು ಅರ್ಥವಿವರಣೆ>
14-724

<ಅಂಧಃ ಎಂಬ ಶಬ್ದದ ಅರ್ಥವಿವರಣೆ>
14-722 
14-839 
17-8

<ಅಧ್ವರ್ಯು ಶಬ್ದಾರ್ಥವಿವರಣೆ>
14-721

<ಅಧ್ರಿಗು ಶಬ್ದವಿವರಣೆ>
16-430

<ಅಧೂಷತ ಎಂಬ ಶಬ್ದದ ಅರ್ಥವಿವರಣೆ>
7-40

<ಅನುಕ್ರಮಣಿಕೆಗಳು>
1-42

<ಅನುವಾಕಸಂಖ್ಯೆಯ ವಿವರಣೆ–ಪ್ರಥಮಮಂಡಲದ>
2-2

<ಅನನುದ ಎಂಬ ಶಬ್ದದ ವಿವರಣೆ>
5-292

<ಅನು>
8-604

<ಅನರ್ವಾಣಂ ಎಂಬ ಶಬ್ದದ ವಿವರಣೆ ಮತ್ತು ಆಂಗ್ಲಪಂಡಿತರ ಅಭಿಪ್ರಾಯಗಳು ಮತ್ತು ವಿಮರ್ಶೆ>
10-694

<ಅನವಭ್ರರಾಧಸಃ ಎಂಬ ಶಬ್ದದ ಅರ್ಥವಿವರಣೆ>
13-35

<ಅನಾನುದಃ ಎಂಬ ಶಬ್ದದ ವಿವರಣೆ>
14-683

<ಅನಾಗಸಃ ಎಂಬ ಶಬ್ದದ ಅವರಣೆ ಮತ್ತು ನಿರ್ವಚನ>
14-927

<ಅನಿತಭಾ ಎಂಬ ನದಿ>
19-643

<ಅನುವಷಟ್ಕಾರವೆಂದರೇನು? ಅದರ ವಿವರಣೆ>
22-205

<ಅನರ್ಶರಾತಿಂ ಎಂಬ ಶಬ್ದದ ನಿರ್ವಚನ>
25-494

<ಅನ್ಯಮೂ ಷು ತ್ವಂ ಎಂಬ ಋಕ್ಕಿನ ನಿರುಕ್ತ>
27-382

<ಅನುಯಾಜಾಃ ಎಂಬ ಆಹುತಿಗಳ ವಿಷಯದಲ್ಲಿ ಯಾಸ್ಕರ ನಿರ್ವಚನ>
28-268

<ಅನ್ನವಿಚಾರವಾದ ವಿಷಯವಿಮರ್ಶೆ>
30-513

<ಅನ್ನದಾನಪ್ರಶಂಸೆ>
30-561

<ಅನ್ನದಾನಕ್ಕೆ ಅರ್ಹರಾರು?>
30-563

<ಅನ್ನಗಳು–ಮೂರುವಿಧವಾದ>
18-459

<ಅಪಾಂನಪಾತ್‍ ಎಂಬ ದೇವತೆ>
5-585
11-293 
16-120
22-280
28-725

<ಅಪಸಃ ಎಂಬ ಶಬ್ದದ ವಿವರಣೆ>
16-221

<ಅಪತ್ಯಶಬ್ದಾರ್ಥವಿಚಾರ ಮತ್ತು ನಿರ್ವಚನ>
21-688

<ಅಪ್ಸರಸ್ತ್ರೀಯರು>
5-662

<ಅಪ್ಸರಾಃ ಎಂಬ ಶಬ್ದದ ನಿರ್ವಚನ>
22-141 
30-15

<ಅಪಾಗೂಹನ್‍ ಎಂಬ ಋಕ್ಕಿನ ನಿರುಕ್ತ>
27-549

<ಅಪೋನಪ್ತ್ರೀಯವೆಂಬ ಯಜ್ಞಾಂಗಕರ್ಮ>
27-786

<ಅಪೋನಪ್ತ್ರೀಯವೆಂಬ ಯಜ್ಞಾಂಗಕರ್ಮ ಮತ್ತು ವಸತೀವರೀ ಉದಕಗಳ ವಿಷಯ>
26-841

<ಅಪೋನಪ್ತ್ರೀಯವೆಂಬ ಯಜ್ಞಾಂಗಕರ್ಮದ ವಿವರಣೆ–ಐತರೇಯಬ್ರಾಹ್ಮಣದಲ್ಲಿರುವಂತೆ>
27-787

<ಅಪ್ರತಿಷ್ಕುತಃ ಎಂಬ ಶಬ್ದದ ವಿವರಣೆ>
19-816

<ಅಪಾಂಗರ್ಭಃ ಎಂಬ ಶಬ್ದಗಳ ವಿವರಣೆ>
17-398

<ಅಪ್ನವಾನ ಎಂಬ ಋಷಿಯ ವೃತ್ತಾಂತ>
17-242

<ಅಪೀಚ್ಯಂ ಎಂಬ ಶಬ್ದದ ವಿವರಣೆ>
7-122

<ಅಪಾಲಾ ಎಂಬ ಸ್ತ್ರೀಯ ವೃತ್ತಾಂತ (ಬೃ. ದೇ.)>
28-927

<ಅಪಾಲಾ ಎಂಬ ಅತ್ರಿಪುತ್ರಿಯು ಇಂದ್ರನಿಗೆ ಸೋಮವನ್ನು ಅರ್ಪಿಸಿದ ವಿಚಾರ ಈ ವಿಷಯದಲ್ಲಿ ಶಾಟ್ಯಾಯನ ಬ್ರಾಹ್ಮಣದಲ್ಲಿ ಹೇಳಿರುವ ಪೂರ್ವೇತಿಹಾಸ>
25-314

<ಅಪತ್ಯಂ ಗೋಪಾಂ ಎಂಬ ಋಕ್ಕಿನ ಅಧ್ಯಾತ್ಮಪರವಾದ ಅರ್ಥ>
30-1207

<ಅಪಶ್ಯಂ ಗೋಪಾಂ ಎಂಬ ಋಕ್ಕಿನ ದೇವತಾಪರವಾದ ಅರ್ಥ>
30-1205

<ಅಪಶ್ಯಂ ಗೋಪಾಂ ಎಂಬ ಋಕ್ಕಿನ ವಿಷಯದಲ್ಲಿ ನಿರುಕ್ತ ಪರಿಶಿಷ್ಟದ ವಿವರಣೆ>
30-1209

<ಅರ್ಬುದ ಶಬ್ದಾರ್ಥ ವಿವರಣೆ>
14-633

<ಅರ್ಬುದ ಎಂಬ ಅಸುರನನ್ನು ಇಂದ್ರನು ಸಂಹಾರ ಮಾಡಿದ ವಿಷಯ>
14-731

<ಅಭಿಪ್ಲವಷಳಹ ಎಂಬ ಯಾಗದ ವಿವರಣೆ>
2-137 
14-49
17-485

<ಅಭಿಜ್ಞಾ>
6-355

<ಅಭಿಪಿತ್ವೇ ಎಂಬ ಶಬ್ದದ ವಿವರಣೆ ಮತ್ತು ಪ್ರಯೋಗ>
10-230

<ಅಭ್ವ ಶಬ್ದಾರ್ಥವಿವರಣೆ>
14-63

<ಅಭಿಪ್ರವಂತ ಎಂಬ ಋಕ್ಕಿನ ಯಾಸ್ಕರ ನಿರ್ವಚನ>
18-661

<ಅಭಿಧೇತನ ಎಂಬ ಶಬ್ದದ ನಿರ್ವಚನ ಮತ್ತು ಅರ್ಥವಿವರಣೆ>
24-781

<ಅಭ್ಯಸ್ತ (ಪುನರಾವೃತ್ತಿ ಉಕ್ತವಾದ) ಪದಗಳ ವಿಷಯ>
27-207

<ಅಭೀ ಇದಂ ಎಂಬ ಋಕ್ಕಿನ ನಿರುಕ್ತ>
28-153


<ಅಭ್ಯಾವತೀ ಮತ್ತು ಪ್ರಸ್ತೋಕಸಾಂರ್ಜಯ ಎಂಬುವರ ವೃತ್ತಾಂತ (ಬೃ. ದೇ.)>
28-882

<ಅಮತ್ರ ಶಬ್ದದ ವಿವರಣೆ>
14-771

<ಅಮತಿಃ>
6-389

<ಅಮರ್ತ್ಯ>
6-489

<ಅಮಿನತೀ ದೈವ್ಯಾನಿ ಎಂಬ ಋಕ್ಕಿನ ವಿಷಯದಲ್ಲಿ Bloom-field ಪಂಡಿತನ ವಿಮರ್ಶೆ ಮತ್ತು ಅಭಿಪ್ರಾಯ>
10-138

<ಅಮರ್ತ್ಯನಾದ ಆತ್ಮನಿಗೂ ಮರ್ತ್ಯವಾದ ದೇಹಾದಿಗಳಿಗೂ ಇರುವ ಸಂಬಂಧ ಇತ್ಯಾದಿ>
12-557

<ಅಮೃತ ಶಬ್ದದ ನಾನಾರ್ಥಗಳು>
13-33

<ಅಮಾಜೂಃ ಎಂಬ ಶಬ್ದದ ವಿವರಣೆ>
14-812

<ಅಮಿನಃ ಎಂಬ ಶಬ್ದದ ನಿರ್ವಚನ>
20-701

<ಅಈವಹಾ ಎಂಬ ಋಕ್ಕಿನ ನಿರುಕ್ತ>
22-326

<ಅಮೂರ ಎಂಬ ಶಬ್ದದ ನಿರ್ವಚನ>
27-254

<ಅಮೃತತ್ವಸ್ಯೇಶಾನಃ ಎಂಬ ವಾಕ್ಯದ ವಿವರಣೆ>
29-747

<ಅಈಷಾಂ ಚಿತ್ತಂ ಎಂಬ ಋಕ್ಕಿನ ನಿರುಕ್ತ>
30-210

<ಅಈವಾ ಎಂಬ ಶಬ್ದದ ವಿವರಣೆ>
30-1096

<ಅಯಂ ಮಾತಾಯಂ ಎಂಬ ಋಕ್ಕಿನ ವಿಷಯದಲ್ಲಿ ಶಾಟ್ಯಾಯನ ಬ್ರಾಹ್ಮಣ ವಿವರಣೆಯು>
28-393

<ಅಯಂ ವೇನಶ್ಚೋದಯತ್‍ ಎಂಬ ಋಕ್ಕಿನ ನಿರುಕ್ತ>
30-663

<ಅಯಾ ಎಂಬ ಶಬ್ದದ ವೈಶಿಷ್ಟ್ಯ>
17-785

<ಅಯಾಸ್ಯ ಶಬ್ದದ ಅರ್ಥಾನುವಾದ ವಿವರಣೆ>
24-927
6-37

<ಅರಾತಿ ಶಬ್ದದ ವಿವರಣೆ>
5-698
14-927

<ಅರುಣಪ್ಸವಃ ಎಂಬ ಶಬ್ದದ ವಿವರಣೆ>
5-93


<ಅರುಷೇಭಿಃ ಎಂಬ ಶಬ್ದದ ವಿವರಣೆ>
11-193
16-309

<ಅರ್ಯಮಾ ಎಂಬ ದೇವತೆಯ ವಿಷಯ>
11-200 
19-668
17-226

<ಅರಣಃ ಎಂಬ ಶಬ್ದದ ಅರ್ಥವಿವರಣೆ>
18-752

<ಅರ್ಯಮಾ, ಮಿತ್ರ, ವರುಣರ ವಿಷಯ>
14-119

<ಅರ್ಯಃ ಎಂಬ ಶಬ್ದದ ಅರ್ಥವಿವರಣೆ>
14-13

<ಅರಣಿಗಳಲ್ಲಿ ಅಗ್ನಿಯನ್ನು ಉತ್ಪತ್ತಿಮಾಡುವ ವಿಚಾರ>
19-37

<ಅರ್ವನ್‍>
8-712
11-454

<ಅರಿಷ್ಠನೇಮಿಶಬ್ದದ ವಿವರಣೆ>
30-954

<ಅರಣ್ಯಾನೀ ಎಂಬ ಶಬ್ದದ ನಿರ್ವಚನ>
30-973

<ಅರಣ್ಯಾನೀ ಎಂಬ ಋಕ್ಕಿನ ನಿರುಕ್ತ>
30-975

<ಅರಾಯಿ ಕಾಣೇ ಎಂಬ ಋಕ್ಕಿನ ನಿರುಕ್ತ>
30-1043

<ಅರಾಯಿ ಎಂಬ ಶಬ್ದದ ವಿವರಣೆ>
30-1044

<ಅಲಾತೃಣಾಸಃ ಎಂಬ ಶಬ್ದದ ವಿವರಣೆ>
13-33

<ಅಲಕ್ಷ್ಮಿ ಅಥವಾ ದರಿದ್ರದೇವತೆಯ ವಿಷಯ>
30-1040

<ಅವಮಶಬ್ದಾರ್ಥ>
8-459

<ಅವಯಾಃ ಎಂಬ ಶಬ್ದದ ವಿವರಣೆ>
13-305

<ಅವಿದ್ಯಾ ಮತ್ತು ವಿದ್ಯೆಗಳ ಸ್ವರೂಪ>
12-572

<ಅವ ಎಂಬ ಉಪಸರ್ಗದ ಅರ್ಥವಿವರಣೆ>
19-704

<ಅಂಶು ಶಬ್ದದ ನಾನಾರ್ಥ ಮತ್ತು ಪ್ರಯೋಗಗಳು>
10-201 
11-12
14-686

<ಅಶ್ಮಾಸ್ಯಂ ಎಂಬ ಶಬ್ದದ ವಿವರಣೆ>
15-22

<ಅಶ್ನ ಎಂಬ ಅಸುರ>
14-733

<ಅಶ್ವಮೇಧಯಜ್ಞದಲ್ಲಿ ಉಪಯೋಗಿಸುವ ಅಶ್ವಸಂಬಂಧವಾದ ಆಪ್ರೀಸೂಕ್ತ>
12-199

<ಅಶ್ವಶರೀರಚ್ಛೇದನ ಇತ್ಯಾದಿ>
12-210

<ಅಶ್ವಸ್ತುತಿ>
12-218

<ಅಶ್ವಬಂಧನಸ್ವರೂಪ>
12-242

<ಅಶ್ವಸ್ತುತಿವರ್ಣನೆ ಇತ್ಯಾದಿ>
12-273

<ಅಶ್ವಮೇಧಯಾಗದ ಸ್ವರೂಪವರ್ಣನೆ ಇತ್ಯಾದಿ>
12-282

<ಅಶ್ವಾಜನಿ ಅಥವಾ ಕಶಾವಿವರಣೆ>
21-583

<ಅಶ್ವಿನೀದೇವತೆಗಳ ಸೂಕ್ಷ್ಮಪರಿಚಯ>
2-11 
4-75 
5-563 
14-168 
20-78
30-931

<ಅಶ್ವಿನೀದೇವತೆಗಳ ರಥಸ್ವರೂಪ>
4-114
5-7 
8-732 
9-417 
13-503 
13-626 
20-104 
22-484 
27-934

<ಅಶ್ವ ದಧಿಕ್ರಾ>
5-680

<ಅಶ್ವಿನಾ ಎಂಬ ಶಬ್ದದ ಅರ್ಥವಿವರಣೆ ಮತ್ತು ನಿರ್ವಚನ>
8-639
16-416

<ಅಶ್ವಿನೀದೇವತೆಗಳ ರಥವು ದೇವತೆಗಳ ಸ್ಪರ್ಧೆಯಲ್ಲಿ ಗೆದ್ದ ವಿಚಾರ>
9-271

<ಅಶ್ವಿನೀದೇವತೆಗಳ ವಾಹನವಾದ ರಾಸಭವು ದೇವತೆಗಳ ಪಂದ್ಯದಲ್ಲಿ ಗೆದ್ದ ವಿಚಾರ>
9-187

<ಅಶ್ವಿನೀದೇವತೆಗಳು ತಮ್ಮ ಕುದುರೆಯ ಗೊರಸಿನಿಂದ ಮದ್ಯವನ್ನು ಸ್ರವಿಸುವಂತೆ ಮಾಡಿದ ವಿಚಾರ>
9-216

<ಅಶ್ವಿನೀದೇವತೆಗಳ ರಥದ ಮಹಿಮೆ>
9-455

<ಅಶ್ವಿನೀದೇವತೆಗಳ ರಥಕ್ಕೆ ಮೂರು ಚಕ್ರಗಳು ಹೇಗೆ ಬಂದವೆಂಬ ವಿಚಾರ>
11-681

<ಅಶ್ವಿನೀದೇವತೆಗಳು ದೇವವೈದ್ಯರೆಂಬ ವಿಚಾರ>
11-691

<ಅಶ್ವಿನೀದೇವತೆಗಳು ಸುಕನ್ಯೆಯ ಪತಿಯನ್ನು ಯುವಕನನ್ನಾಗಿ ಮಾಡಿದ ವಿಚಾರ>
11-992

<ಅಶ್ವಿನೌ ಎಂಬ ಶಬ್ದದ ರೂಪನಿಷ್ಪತ್ತಿ, ಅರ್ಥವಿವರಣೆ ಇತ್ಯಾದಿ>
12-472

<ಅಶ್ವಿನೀ ದೇವತೆಗಳು ಶಯುವೆಂಬ ರಾಜರ್ಷಿಯ ಧೇನುವನ್ನು ಹಾಲು ಕೊಡುವಂತೆ ಮಾಡಿದ ವಿಚಾರ>
13-509

<ಅಶ್ವಿನೀ ದೇವತೆಗಳು ದೇವತೆಗಳ ಯಜ್ಞವನ್ನು ಪೂರ್ಣಗೊಳಿಸಿದ ವಿಚಾರ>
13-553

<ಅಶ್ವಿನೀ ದೇವತೆಗಳು ಯಾರು, ಇವರ ಸ್ವರೂಪವೇನು ಎಂಬ ವಿಷಯ ವಿಮರ್ಶೆ>
13-566

<ಅಶ್ವಿನೀ ದೇವತೆಗಳ ಉತ್ಪತ್ತಿ ವಿಚಾರ>
13-568
22-432
27-931
27-534

<ಅಶ್ವಿನೀ ದೇವತೆಗಳ ರಥ ಸಂಚಾರ ಕ್ರಮ>
13-573

<ಅಶ್ವಿನೀ ದೇವತೆಗಳನ್ನು ಸೃಷ್ಟಿಕರ್ತರೆಂದು ವರ್ಣಿಸಿರುವ ವಿಚಾರ>
13-581

<ಅಶ್ವಿನೀದೇವತೆಗಳು ಪುರುಈಳ್ಹ, ಅತ್ರಿ, ಗೋತಮ ಎಂಬುವರನ್ನು ರಕ್ಷಿಸಿದ ವಿಚಾರ–ಪುರುಈಳ್ಹನ ವೃತ್ತಾಂತ>
13-643

<ಅಶ್ವಿನೀದೇವತೆಗಳ ನಾನಾಕಾರ್ಯಗಳು>
14-170

<ಅಶ್ವಿನೀದೇವತೆಗಳ ಸಂಬಂಧದಲ್ಲಿ ಉಪಯೋಗಿಸಿರುವ ತ್ರಿಚಕ್ರ, ತ್ರಿಬರ್ಹಿಃ, ತ್ರಿಷಧಸ್ಥ, ತ್ರಿವಂಧುರ, ತ್ರಿವಿಕ್ರಮ, ತ್ರಿಧಾತುನಾ ಇತ್ಯಾದಿ ಶಬ್ದಗಳ ವಿವರಣೆ>
20-107

<ಅಶ್ವಿನೀದೇವತೆಗಳು ತಮ್ಮನ್ನು ಸ್ತುತಿಸಿದವರಿಗೆ ಮಾಡಿದ ಸಹಾಯಗಳು>
22-486
27-936

<ಅಶ್ವಿನೀದೇವತೆಗಳು ಚ್ಯವಾನ, ಕಲಿ, ವಿಮದ, ಕೃಷ್ಣ, ಭುಜ್ಯು, ವಿಶ್ಪಲಾ, ಋಜ್ರಾಶ್ವ, ಪಾರಾವೃಜ, ರೇಭ, ವಂದನ, ಕಕ್ಷೀವಾನ್‍, ಅತ್ರಿ, ವಧ್ರಿಮತೀ, ಘೋಷಾ, ಶಯು, ಪೇದು ಮೊದಲಾದವರಿಗೆ ಮಾಡಿದ ಸಹಾಯಗಳ ವಿವರಣೆ>
22-486

<ಅಶ್ವಿನೀ ದೇವತೆಗಳಿಗೂ ವಸಿಷ್ಠ ಋಷಿಗೂ ಇರುವ ಬಂಧುತ್ವದ ವಿಮರ್ಶೆ>
22-539

<ಅಶ್ವಿನೀ ದೇವತೆಗಳ ಬಾಹುಗಳ ವೈಶಿಷ್ಟ್ಯ>
17-391

<ಅಶ್ವಿನೀ ದೇವತೆಗಳು ಸೂರ್ಯಾ ವಿವಾಹದ ಸಮಯದಲ್ಲಿ ಸ್ಪರ್ಧೆಯಲ್ಲಿಗೆದ್ದ ವಿಚಾರ>
29-343

<ಅಶ್ವಿನೀ ದೇವತೆಗಳು ಅತ್ರಿ ಋಷಿಯನ್ನು ಕಾಪಾಡಿದ ವಿಚಾರ>
30-947

<ಅಷ್ಟಾಚತ್ವಾರಿಂಶಸ್ತೋಮದ ವಿವರಣೆ>
2-148

<ಅಷ್ಟ ವಸುಗಳು>
17-386
19-577

<ಅಷ್ಟಿ ಎಂಬ ಛಂದಸ್ಸಿನ ಲಕ್ಷಣ>
17-669

<ಅಂಸ ಶಬ್ದಾರ್ಥ>
13-45

<ಅಸುರರು>
5-688

<ಅಸ್ಮೇ ಎಂಬ ಶಬ್ದದ ನಾನಾರ್ಥಗಳು ಮತ್ತು ಸಪ್ತ ವಿಭಕ್ತಿಗಳಲ್ಲಿಯೂ ಈ ಶಬ್ದದ ಪ್ರಯೋಗಗಳು>
6-340
6-538 
9-82 
10-63 
10-170 
14-20 
16-171 
19-677
26-753

<ಅಸುರ ಶಬ್ದಾರ್ಥ ವಿವರಣೆ ಮತ್ತು ಪ್ರಯೋಗಗಳು>
8-569 
10-8 
11-491
13-315
14-403 
17-277
17-376

<ಅಸು ಶಬ್ದಾರ್ಥ ವಿವರಣೆ>
9-77

<ಅಸ್ಯವಾಮಸ್ಯ ಎಂಬ ಋಕ್ಕಿಗೆ ಆದಿತ್ಯ ಪರವಾದ ಅರ್ಥ ವಿವರಣೆ>
12-294

<ಅಸ್ಯವಾಮಸ್ಯ ಎಂಬ ಋಕ್ಕಿಗೆ ಅಧ್ಯಾತ್ಮಪರ ವಿವರಣೆ>
12-296

<ಅಸಿನ್ವನ್‍ ಶಬ್ದದ ವಿವರಣೆ>
14-695

<ಅಸ್ಕೃಧೋಯುಃ ಎಂಬ ಶಬ್ದದ ನಿರ್ವಚನ ಮತ್ತು ಅರ್ಥ ವಿವರಣೆ>
20-760

<ಅಸತ್‍, ಸತ್‍ ಎಂಬ ತತ್ತ್ವಗಳ ವಿಷಯ>
27-279
30-755

<ಅಸತ್‍ನಿಂದ ಸತ್‍ ಉತ್ಪತ್ತಿಯಾದ ವಿಚಾರ>
29-20

<ಅಸಾಮಿಶವಸಃ ಎಂಬ ಶಬ್ದದ ಅರ್ಥ ವಿವರಣೆ>
19-612

<ಅಸುನೀತಿ (ಬೃ)>
28-725

<ಅಸ್ತ್ರಬುಧ್ನ ಎಂಬವನ ವಿಷಯ>
30-1162

<ಅಸುನೀತೇ ಎಂಬ ಋಕ್ಕಿನ ನಿರುಕ್ತ>
28-379

<ಅಹಿರ್ಬುಧ್ನ್ಯಃ>
5-588 
14-142 
15-173
19-385

<ಅಹಿಭಾನವಃ ಎಂಬ ಶಬ್ದದ ಅರ್ಥಾನುವಾದ>
13-230

<ಅಹಿ>
5-685

<ಅಹಿಮನ್ಯವಃ ಎಂಬ ಶಬ್ದದ ವಿವರಣೆ>
6-125

<ಅಹಿಂ>
6-605 
8-364 
14-653
12-698

<ಅಂಹುರ ಶಬ್ದಾರ್ಥ>
8-518

<ಅಂಹಃ ಎಂಬ ಶಬ್ದದ ಅರ್ಥ ವಿವರಣೆ>
14-921

<ಅಹಿರಿವ ಭೋಗೈಃ ಎಂಬ ಋಕ್ಕಿನ ನಿರುಕ್ತ>
21-586

<ಅಹಂ ಸುವೇ ಪಿತರಮಸ್ಯ ಎಂಬ ಋಕ್ಕಿನ ವಿಶೇಷ ವಿವರಣೆ>
30-700

<ಅಕ್ಷಣ್ವಂತಃ ಕರ್ಣವಂತಃ ಎಂಬ ಋಕ್ಕಿನ ನಿರುಕ್ತ>
28-665

<ಅಕ್ಷರ ಶಬ್ದಾರ್ಥ ವಿವರಣೆ>
12-573
17-276

<ಅಕ್ಷಾಃ ಎಂಬ ಶಬ್ದದ ನಿರ್ವಚನ ಮತ್ತು ಅರ್ಥ>
27-760

<ಅಕ್ಷರ, ಕ್ಷರ ಎಂಬ ತತ್ತ್ವಗಳ ವಿವರಣೆ>
29-759

[ಆ]
<ಆ ಎಂಬ ನಿಪಾತದ ನಾನಾರ್ಥಗಳು>
19-607

<ಆಗಧಿತಾ ಎಂಬ ಶಬ್ದದ ವಿವರಣೆ>
10-243

<ಆಗಃ ಎಂಬ ಶಬ್ದದ ವಿವರಣೆ>
14-90

<ಆಘೃಣೇ ಎಂಬ ಶಬ್ದದ ನಿರ್ವಚನ>
21-356 
11-33

<ಆ ಜಂಘಂತಿ ಎಂಬ ಋಕ್ಕಿನ ನಿರ್ವಚನ>
21-583

<ಆಜ್ಯಭಾಗಗಳು ಯಜ್ಞದ ಚಕ್ಷುಸ್ಸುಗಳೆಂಬ ವಿಚಾರ>
6-376

<ಆರ್ಜುನೇಯ ಕುತ್ಸ>
8-821

<ಆಜ್ಯದ ಮಹತ್ತ್ವ ಇತ್ಯಾದಿ>
29-889

<ಆಜುಹ್ವಾನ ಈಡ್ಯಃ ಎಂಬ ಋಕ್ಕಿನ ನಿರುಕ್ತ>
30-343

<ಆಗ್ನಿಮಾರುತ ಶಸ್ತ್ರದಲ್ಲಿ ವೈಶ್ವಾನರಾಗ್ನಿಯ ಸ್ಥಾನ>
15-615

<ಆತ್‍ ಶಬ್ದದ ವಿವರಣೆ>
14-804

<ಆತ್ಮ>
5-7OO
27-486

<ಆತ್ಮಸ್ವರೂಪ>
12-427

<ಆತ್ಮನ ನಾನಾವಿಧ ಅವಸ್ಥೆಗಳು>
12-434

<ಆತ್ಮನು ಅಶಬ್ದ, ಅಸ್ಪರ್ಶ, ಅರೂಪ ಇತ್ಯಾದಿ ಶಬ್ದ ವಾಚ್ಯನು>
12-566

<ಆತೇ ಗರ್ಭೋ ಯೋನಿಮೈತು ಎಂಬ ೯ನೇ ಪರಿಶಿಷ್ಟ ಸೂಕ್ತ (೫ ಋಕ್ಕಗಳ) ಅರ್ಥವಿವರಣೆ ಸಹಿತ>
20-264

<ಆತ್ಮಾ ಯಕ್ಷ್ಯಸ್ಯ ನಶ್ಯತಿ ಎಂಬ ವಾಕ್ಯದ ನಿರುಕ್ತ>
30-85

<ಆ ತೂ ಷಿಂಚ ಎಂಬ ಋಕ್ಕಿನ ನಿರುಕ್ತ>
30-167

<ಆದಿತ್ಯ ಶಬ್ದದ ರೂಪನಿಷ್ಪತ್ತಿ>
3-21

<ಆದಿತ್ಯರು>
5-557

<ಆದಿತ್ಯ ಶಬ್ದಾರ್ಥ ಮತ್ತು ನಿರ್ವಚನ>
8-536 
19-583 
22-303
27-520

<ಆದಿತ್ಯನ ವಿವರಣೆ>
11-535

<ಆದಿತ್ಯಮಂಡಲದ ಸ್ವರೂಪ ವರ್ಣನೆ>
12-329

<ಆದಿತ್ಯರಶ್ಮಿಗಳ ನಾನಾಕಾರ್ಯಗಳು>
12-332

<ಆದಿತ್ಯನಿಗೂ ಪೃಥಿವಿಗೂ ಇರುವ ಸಂಬಂಧ>
12-338

<ಆದಿತ್ಯನಿಗೂ ಕಾಲಕ್ಕೂ ಇರುವ ಸಂಬಂಧ>
12-349

<ಆದಿತ್ಯನ ಗತಿವರ್ಣನೆ>
12-361

<ಆದಿತ್ಯನ ಸಪ್ತರಶ್ಮಿಗಳು>
12-365

<ಆದಿತ್ಯರಶ್ಮಿಗೂ ಯಜ್ಞಾಹುತಿಗೂ ಇರುವ ಸಂಬಂಧ>
12-398

<ಆದಿತ್ಯನಿಗೂ ಅಗ್ನಿಗೂ ಇರುವ ಸಂಬಂಧ>
12-403

<ಆದಿತ್ಯನಿಗೂ ರಥಂತರಸಾಮಕ್ಕೂ ಇರುವ ಸಂಬಂಧ>
12-461

<ಆದಿತ್ಯಮಂಡಲಾಂತರ್ಗತನಾದ ಪುರುಷನೇ ಪರಮಾತ್ಮನು>
12-505

<ಆದಿತ್ಯನ ಸ್ಥಾನ ಮತ್ತು ವಿಶ್ವಕ್ಕೂ ಅವನಿಗೂ ಇರುವ ರಕ್ಷ್ಯ ರಕ್ಷಕಸಂಬಂಧ>
12-509

<ಆದಿತ್ಯವೇ ಸಕಲಕ್ಕೂ ಆಶ್ರಯನು, ಇವನೇ ಪರಬ್ರಹ್ಮನು ಇತ್ಯಾದಿ>
12-612

<ಆದಿತ್ಯನ ಎರಡುವಿಧವಾದ ಗತಿಗಳು>
12-633

<ಆದಿತ್ಯರು ಯಾರು ಎಂಬ ವಿಷಯವಿಮರ್ಶೆ>
16-104 
22-305
27-164

<ಆದಿತ್ಯರು ಎಷ್ಟು ಮಂದಿ?>
19-578

<ಆದಿತ್ಯನು ವೃಷ್ಟಿಗೆ (ಮಳೆಗೆ) ಕಾರಣನೆಂಬ ವಿಷಯ>
22-187

<ಆದಿತ್ಯನೇ ಬ್ರಹ್ಮನೆಂಬ ವಿಷಯ>
26-823

<ಆದಿತ್ಯನ ಉತ್ಪತ್ತಿ>
27-365 
28-174

<ಆದಿತ್ಯನ ಕಾಲಸ್ವರೂಪ, ಋತುಸ್ವರೂಪ, ಸಂವತ್ಸರಸ್ವರೂಪ, ಇತ್ಯಾದಿ>
17-330

<ಆದಿತ್ಯನ ಅಥವಾ ಸವಿತೃವಿನ ವಿಶ್ವಸಂಚಾರವಿಷಯ>
17-552

<ಆದಿತ್ಯಾಂತರ್ಗತವೂ ಸವಿತ್ರಾತ್ಮಕವೂ ಆದ ಹಿರಣ್ಮಯ ಪುರುಷ>
17-565

<ಆದಿತ್ಯರೂಪವಾದ ಬ್ರಹ್ಮನ ದಿವ್ಯಗಾನವನ್ನು ಪ್ರಶಂಸಿಸುವ ಗಾಯತ್ರೀ ಮತ್ತು ಪ್ರಣವವೂ ಗಾನಾತ್ಮಕವಾಗಿಯೇ ಇದೆ>
17-622

<ಆಧ್ಯಾತ್ಮಿಕವಾದ ಅಂತಸ್ತತ್ತ್ವದೊಡನಿರುವ ತಾದಾತ್ಮ್ಯ ವರ್ಣನೆ>
17-601

<ಆಧ್ಯಾತ್ಮಿಕ ಮಂತ್ರ>
2-205

<ಆನಂದತೀರ್ಥ>
1-240

<ಆನುಷಕ್‍ ಎಂಬ ಶಬ್ದದ ನಿರ್ವಚನ ಮತ್ತು ಅರ್ಥವಿವರಣೆ>
24-452

<ಆ ನೋ ಯಜ್ಞಂ ಭಾರತೀ ಎಂಬ ಋಕ್ಕಿನ ನಿರುಕ್ತ>
30-359

<ಆಪ್ರೀ ಸೂಕ್ತಗಳ ವಿವರಣೆ>
2-73 
15-665 
28-602
28-756 
30-322

<ಆಪ್ರಶಬ್ದದ ವಿವರಣೆ>
10-509

<ಆಪ್ರೀಸೂಕ್ತಗಳ ವಿಷಯವಾಗಿ ಕೆಲವು ವಿಶೇಷವಿಷಯಗಳು>
11-223

<ಆಪ್ರೀಸೂಕ್ತಗಳ ವೈಶಿಷ್ಟ್ಯ>
11-283

<ಆಪ್ರೀ ಮತ್ತು ಆಪ್ರ ಎಂಬ ಶಬ್ದಗಳಿಗೆ ಇರುವ ವ್ಯತ್ಯಾಸ>
15-672

<ಆಪ್ರೀಸೂಕ್ತಗಳ ವಿಷಯದಲ್ಲಿ ಪೀಠಿಕೆ>
21-648

<ಆಪಿತ್ವೇ ಪ್ರಪಿತ್ವೇ ಎಂಬ ಶಬ್ದಗಳ ನಿರ್ವಚನ, ಅರ್ಥವಿವರಣೆ>
23-415

<ಆಪೋ ಹಿ ಷ್ಠ ಎಂಬ ಸೂಕ್ತದ ಪೀಠಿಕೆ>
27-319

<ಆಪೋ ಹಿ ಷ್ಠ ಎಂಬ ಸೂಕ್ತದ ಜಪವಿಧಾನ ಮತ್ತು ಈ ಸೂಕ್ತದ ಜಪದಿಂದ ಇಂದ್ರನ ಬ್ರಹ್ಮಹತ್ಯಾದೋಷವು ಪರಿಹಾರವಾದ ಬಗೆ>
27-320

<ಆಪೋ ಹಿ ಷ್ಠ ಎಂಬ ಋಕ್ಕಿನ ನಿರುಕ್ತ>
27-323

<ಆಪಃ ಎಂಬ ಶಬ್ದದ ನಾನಾರ್ಥಗಳು>
3-217
17-367

<ಆಪಾಂತಮನ್ಯುಃ ಎಂಬ ಋಕ್ಕಿನ ನಿರುಕ್ತ>
29-665

<ಆಪ್ರೀಸೂಕ್ತಗಳು ಹನ್ನೊಂದು (ಬೃ)>
28-753

<ಆಪ್ರೀಸೂಕ್ತಗಳ ಮೂರು ಗುಂಪುಗಳು ಅಥವಾ ಪ್ರಭೇದಗಳು (ಬೃ)>
28-754

<ಆಮಾಸು ಪಕ್ವಂ ಎಂಬ ಋಕ್ಕಿನ ನಿರುಕ್ತ>
25-301

<ಆಯುಧಗಳು, ಲಾಂಛನಗಳು>
5-687

<ಆಯುರ್ನ ಪ್ರಾಣಃ>
6-182


<ಆಯುಃ ಶಬ್ದಾರ್ಥ>
8-411 
9-134 
10-288 
13-343

<ಆಯತೀನಾಂ ಎಂಬ ಶಬ್ದದ ಅರ್ಥವಿವರಣೆ>
9-42

<ಆಯು ಎಂಬುವನ ವಿಷಯ>
14-741

<ಆಯೂಥೇವ ಎಂಬ ಋಕ್ಕಿಗೆ Wilson ಎಂಬ ಆಂಗ್ಲಪಂಡಿತನ ಅರ್ಥವಿವರಣೆ>
17-728

<ಆರ್ಯ ಮತ್ತು ದಾಸ ಎಂಬ ಜನರು ಯಾರು>
29-912

<ಆರ್ಯ ಶಬ್ದ ವಿವರಣೆ>
5-183

<ಆರಣ್ಯಕಗಳು>
1-24

<ಆ ವೃಕ್ಷನ್‍ ಹಂತಿ ಎಂಬ ಋಕ್ಕಿನ ನಿರುಕ್ತ>
21-206

<ಆಶಿರದ್ರವ್ಯಗಳ ವಿವರಣೆ–ಸೋಮರಸಕ್ಕೆ ಮಿಶ್ರಮಾಡುವ>
2-172

<ಆಶ್ವಲಾಯನ ಶಾಖೆಗಳು>
1-67

<ಆಶಿರ ದ್ರವ ವಿಚಾರ>
11-5 
14-209 
19-298
26-72

<ಆಶುಶುಕ್ಷಣಿಃ ಎಂಬ ವಿವರಣೆ>
14-382

<ಅಶ್ವಿನ ಶಸ್ತ್ರಮಂತ್ರಗಳಿಗೂ ಅಗ್ನಿಗೂ ಇರುವ ಸಂಬಂಧ>
16-251

<ಆರ್ಷ್ಟಿಷೇಣೋ ಹೋತ್ರಂ ಎಂಬ ಋಕ್ಕಿನ ನಿರುಕ್ತ>
30-108

<ಆಸಂಗನೆಂಬ ರಾಜನಿಗೆ ಪುಂಸತ್ವವು ಪ್ರಾಪ್ತವಾದ ವಿಚಾರದಲ್ಲಿ ಪೂರ್ವೇತಿಹಾಸವು ಮತ್ತು ಅವನ ಪತ್ನಿಯ ಅಭಿನಂದನೆ>
23-332

<ಆ ಸುಷ್ವಯಂತೀ ಎಂಬ ಋಕ್ಕಿನ ನಿರುಕ್ತ>
30-353

<ಆಹವನೀಯವು ದೇವತೆಗಳ ಯೋನಿಯು>
12-667

<ಆಹನ ಶಬ್ದಾರ್ಥ>
14-686

[ಇ]
<ಇತ್‍ ಎಂಬ ನಿಪಾತದ ಅರ್ಥ ಮತ್ತು ಪ್ರಯೋಗ>
26-726

<ಇದಂ ಯಮಸ್ಯ ಎಂಬ ಋಕ್ಕಿಗೆ ಆದಿತ್ಯ ಪರವಾದ ಅರ್ಥ>
30-866

<ಇದಂ ಯಮಸ್ಯ ಎಂಬ ಋಕ್ಕಿಗೆ ಯಮನ ಪರವಾದ ಅರ್ಥ>
30-866

<ಇದಂ ಶ್ರೇಷ್ಠಂ ಎಂಬ ಋಕ್ಕಿಗೆ ಯಾಸ್ಕರ ನಿರ್ವಚನ>
20-153

<ಇಂದು ಶಬ್ದ ನಿರ್ವಚನ>
10-392

<ಇಂದ್ರಃ>
5-569

<ಇಂದ್ರ ಶಬ್ದ ನಿಷ್ಪತ್ತಿ ಮತ್ತು ಅರ್ಥ ವಿವರಣೆ>
2-33 
10-367
17-5

<ಇಂದ್ರ ಶಬ್ದ ನಿರ್ವಚನ–ಯಾಸ್ಕರ ನಿರ್ವಚನದಂತೆ>
20-648

<ಇಂದ್ರಾಣೀ ಎಂಬ ಇಂದ್ರನ ಪತ್ನಿಯ ವಿಷಯ>
3-100

<ಇಂದ್ರನ ವಿಷಯದ ಜ್ಞಾನವು ಮನುಷ್ಯರಿಗೆ ಹೆಚ್ಚಾಗಿ ತಿಳಿಯುವುದಿಲ್ಲವೆಂಬ ವಿಚಾರ>
6-611

<ಇಂದ್ರನಿಗೂ ಮರುತ್ತಗಳಿಗೂ ಇರುವ ಸಖ್ಯತ್ವವಿಚಾರ>
8-172

<ಇಂದ್ರನ ಎರಡುವಿಧವಾದ ಶಕ್ತಿಗಳು>
8-225

<ಇಂದ್ರನ ವಜ್ರಾಯುಧ>
8-232 
20-666

<ಇಂದ್ರನ ಸಾಮರ್ಥ್ಯವಿಚಾರ>
8-344

<ಇಂದ್ರನ ಸೋಮಪಾನ>
8-436 
13-397
14-606 
16-644

<ಇಂದ್ರಾಗ್ನಿಗಳ ಸಾಹಚರ್ಯ>
8-575

<ಇಂದ್ರನು ಅಧಿಕವಾಗಿ ಸೋಮಪಾನಮಾಡುವ ವಿಚಾರ>
8-581

<ಇಂದ್ರಾಗ್ನಿಗಳ ವೀರ್ಯಸಾಮರ್ಥ್ಯಾದಿಗಳು>
8-592

<ಇಂದ್ರಾದಿದೇವತೆಗಳ ವಿಧವಿಧವಾದ ವಾಹನಗಳು>
9-188
14-566

<ಇಂದ್ರನು ಹೆಣ್ಣುಕುದುರೆಯಲ್ಲಿ ಗೋವು ಉತ್ಪತ್ತಿಯಾಗುವಂತೆ ಮಾಡಿದ ವಿಚಾರ>
9-546

<ಇಂದ್ರನು ಕೃಷ್ಣನೆಂಬ ಅಸುರನನ್ನು ಸಂಹಾರಮಾಡಿದ ವಿಚಾರ>
10-453
25-443

<ಇಂದ್ರಾಪರ್ವತಾ ಎಂಬ ಶಬ್ದದ ಅರ್ಥಾನುವಾದ>
10-531
1-7173

<ಇಂದ್ರಾವಿಷ್ಣುಗಳಿಗೂ ಆದಿತ್ಯನಿಗೂ ಇರುವ ತಾದಾತ್ಮ್ಯ>
11-615


<ಇಂದ್ರಶಬ್ದದ ರೂಪನಿಷ್ಪತ್ತಿ>
11-655
12-711
13-75
13-598

<ಇಂದ್ರ ಮಿತ್ರ ವರುಣ ಅಗ್ನಿ ಮುಂತಾದ ಸಕಲ ಶಬ್ದಗಳೂ ಒಬ್ಬನೇ ಆದ ಪರಬ್ರಹ್ಮನ ನಾನಾ ನಾಮಧೇಯಗಳು>
12-627


<ಇಂದ್ರನಿಗೂ ಮರುತ್ತುಗಳಿಗೂ ನಡೆದ ಸಂಭಾಷಣೆ>
12-679

<ಇಂದ್ರನಿಗೂ ಮರುತ್ತುಗಳು ವೃತ್ರಹನನಕಾಲದಲ್ಲಿ ಸಹಾಯ ಮಾಡಿದ ವಿಚಾರ>
12-681

<ಇಂದ್ರಾಗಸ್ತ್ಯ ಸಂವಾದ–ಅಗಸ್ತ್ಯ ಋಷಿಯ ವಾಕ್ಯವು>
13-182

<ಇಂದ್ರನ ವಾಕ್ಯವು>
13-184

<ಇಂದ್ರಾಗಸ್ತ್ಯರಿಗೆ ಉಂಟಾದ ಅಸಮಾಧಾನಕ್ಕೆ ಕಾರಣ>
13-193

<ಇಂದ್ರನಿಗೂ ಮರುತ್ತಗಳಿಗೂ ಇರುವ ಸಾಹಚರ್ಯ, ಮರುತ್ತುಗಳು ಇಂದ್ರನಿಗೆ ಮಾಡಿದ ಸಹಾಯ ಇತ್ಯಾದಿ>
13-210 
19-725 
22-325
28-750

<ಇಂದ್ರನು ಮರುದ್ದೇವತೆಗಳೊಡನೆ ಸೋಮಪಾನಮಾಡಿದ ವಿಚಾರ>
14-617

<ಇಂದ್ರನು ಗೃತ್ಸಮದ ಋಷಿಯ ರೂಪದಲ್ಲಿದ್ದ ವಿಷಯ>
14-643

<ಇಂದ್ರನ ನಾನಾವಿಧ ಗುಣಗಳ ವರ್ಣನೆ>
14-646

<ಇಂದ್ರನು ಸೋಮಪಾನದಿಂದ ಉಂಟಾದ ಹರ್ಷದಿಂದ ಮಾಡಿದ ಅನೇಕ ಸಾಹಸಕೃತ್ಯಗಳ ವರ್ಣನೆ>
14-755

<ಇಂದ್ರನು ದಭೀತಿಯನ್ನು ರಕ್ಷಿಸಿದ ವಿಚಾರ>
14-741

<ಇಂದ್ರನು ಉಷೋದೇವತೆಯ ರಥವನ್ನು ಧ್ವಂಸಮಾಡಿದ ವಿಷಯ>
14-766

<ಇಂದ್ರನು ದೇವತೆಗಳಲ್ಲೆಲ್ಲಾ ಶ್ರೇಷ್ಠನೆಂಬ ವಿಚಾರ>
14-776

<ಇಂದ್ರನ ರಥದ ಅಪ್ರತಿಹತವಾದ ಸಂಚಾರಶಕ್ತಿ>
14-781

<ಇಂದ್ರನು ಪರ್ವತಗಳನ್ನು ಭೂಮಿಯಲ್ಲಿ ಸ್ಥಿರವಾಗಿರುವಂತೆ ಮಾಡಿದ ವಿಚಾರ>
14-807

<ಇಂದ್ರನು ಕ್ರಿವಿ ಎಂಬ ಅಸುರನನ್ನು ಸಂಹರಿಸಿದ ವಿಚಾರ>
14-810
14-897

<ಇಂದ್ರನು ಸೂರ್ಯನೊಡನೆ ಹೋರಾಡುತ್ತಿದ್ದ ಏತಶಋಷಿಗೆ ಸಹಾಯಮಾಡಿದ ವಿಚಾರ>
14-846

<ಇಂದ್ರನಿಗೂ ಅಂಗಿರಸರಿಗೂ ಇರುವ ಮಿತ್ರತ್ವ ವಿಷಯ>
14-864

<ಇಂದ್ರನಿಗೆ ದೇವತೆಗಳು ವೃತ್ರಹನನಾರ್ಥವಾಗಿ ಬಲವನ್ನು ಕೊಟ್ಟ ವಿಚಾರ>
14-871

<ಇಂದ್ರನಿಗೂ ವಿಷ್ಣುವಿಗೂ ಇರುವ ಸಾಹಚರ್ಯ>
14-893

<ಇಂದ್ರನ ಉತ್ಪತ್ತಿ ಮತ್ತು ಮಾತಾ ಪಿತೃಗಳು, ಪತ್ನಿ>
16-631
17-106

<ಇಂದ್ರನಭೌತಿಕವೂ ಮಾನಸಿಕವೂ ಆದ ಗುಣಗಳು>
16-638

<ಇಂದ್ರನ ರಥ ಮತ್ತು ಅಶ್ವಗಳು>
16-640

<ಇಂದ್ರನ ವಜ್ರಾಯುಧ ಮತ್ತು ಇತರ ಸಾಧನಗಳು>
16-641

<ಇಂದ್ರನ ಮಹತ್ತ್ವದ ವರ್ಣನೆ>
16-655

<ಇಂದ್ರನು ವೃತ್ರಾಸುರನನ್ನು ಪೌರ್ಣಮಾಸಿಯದಿವಸ ಸಂಹಾರಮಾಡಿದನೆಂಬ ವಿಚಾರ>
18-7

<ಇಂದ್ರನು ಅಗ್ರು ಎಂಬ ಕನ್ಯೆಯ ಪುತ್ರನಾದ ಪರಾವೃಕ್ತನೆಂಬುವನನ್ನು ಹುತ್ತದಿಂದ ಕಾಪಾಡಿದವಿಚಾರ>
18-24
18-234

<ಇಂದ್ರನ ಪತ್ನಿಯಾದ ಪ್ರಾಸಹಾದೇವಿಯ ಕಥೆ>
18-51

<ಇಂದ್ರನು ಪ್ರಜಾಪತಿಯ ಸ್ಥಾನಮಾನಗಳನ್ನಪೇಕ್ಷಿಸುವುದು>
18-48

<ಇಂದ್ರನಿಗೂ ಮರುತ್ತುಗಳಿಗೂ ಇರುವ ರಾಜಪ್ರಜಾ ಸಂಬಂಧ ಮತ್ತು ಸಾಹಚರ್ಯ>
17-12
18-73

<ಇಂದ್ರನು ಓರ್ವ ರಾಕ್ಷಸಸ್ತ್ರೀಯನ್ನು ಸಂಹಾರಮಾಡಿದ ವಿಚಾರ>
18-125

<ಇಂದ್ರನು ಸೂರ್ಯರಥ ಚಕ್ರವನ್ನು ಅಪಹರಿಸಿದ ವಿಚಾರ>
18-204

<ಇಂದ್ರನು ಏತಶನೆಂಬ ಋಷಿಯನ್ನು ರಕ್ಷಿಸಿದ ವಿಚಾರ>
18-222

<ಇಂದ್ರನು ಉಷೋದೇವತೆಯ ರಥವನ್ನು ಪುಡಿಪುಡಿ ಮಾಡಿದ ವಿಚಾರ>
18-227

<ಇಂದ್ರನ ಕುದುರೆಗಳ ಸ್ತುತಿ ವರ್ಣನೆ>
18-288

<ಇಂದ್ರನು ನಮುಚಿಯೆಂಬ ಅಸುರನನ್ನು ಸಂಹಾರ ಮಾಡಿದ ವಿಷಯ-ಋಗ್ವೇದ ಮತ್ತು ಯಜುರ್ವೇದದಲ್ಲಿ ಹೇಳಿರುವಂತೆ>
19-193 
28-173
29-647

<ಇಂದ್ರನು ಪಶುಗಳ ಮಾಂಸವನ್ನು ಭಕ್ಷಿಸಿದ ವಿಷಯ>
20-667

<ಇಂದ್ರನಿಗೆ ಮಹಾನ್‍ ಇಂದ್ರ ಎಂಬ ಹೆಸರು ಬರಲು ಕಾರಣ-ಈ ವಿಷಯದಲ್ಲಿ ಐತರೇಯ ಬ್ರಾಹ್ಮಣದಲ್ಲಿರುವ ಪೂರ್ವೇತಿಹಾಸ ಮತ್ತು ಪ್ರಜಾಪತಿಗೆ ಕಃ ಎಂಬ ಹೆಸರು ಬರಲು ಕಾರಣ>
20-702
29-639

<ಇಂದ್ರನು ಕುತ್ಸರಿಗೆ ಸಹಾಯ ಮಾಡುವುದಕ್ಕಾಗಿ ಶುಷ್ಣನೆಂಬ ಅಸುರನನ್ನು ಸಂಹಾರ ಮಾಡಿದ ವಿಷಯ>
20-724

<ಇಂದ್ರನು ಅನೇಕ ರೂಪಗಳನ್ನು ಧರಿಸಬಲ್ಲನೆಂಬ ವಿಚಾರದಲ್ಲಿ ವಿಷಯ ವಿಮರ್ಶೆ, ಪ್ರತ್ಯುದಾಹರಣೆ ಇತ್ಯಾದಿ>
17-111
21-209

<ಇಂದ್ರನ ಬ್ರಹ್ಮಹತ್ಯಾದೋಷ ಪರಿಹಾರಾರ್ಥವಾಗಿ ಏತೋ ನ್ವಿಂದ್ರಂ ಇತ್ಯಾದಿ ಮೂರು ಋಕ್ಕುಗಳ ವಿಷಯದಲ್ಲಿ ಭಾಷ್ಯಕಾರರು ಹೇಳಿರುವ ಪೂರ್ವೇತಿಹಾಸ>
25-418

<ಇಂದ್ರನು ಸಪ್ತ ಪರ್ವತಗಳನ್ನು ಸೀಳಿದ ವಿಚಾರ>
25-425

<ಇಂದ್ರನಿಗೆ ಮರುದ್ದೇವತೆಗಳು ಸಹಾಯ ಮಾಡಿದ ವಿಚಾರ ಮತ್ತು ಮರುತ್ತುಗಳಿಗೂ ಇಂದ್ರನಿಗೂ ಸಖ್ಯವು ಉಂಟಾದ ಬಗೆ ಈ ವಿಷಯದಲ್ಲಿ ಪೂರ್ವೇತಿಹಾಸ>
25-433

<ಇಂದ್ರನು ವೃತ್ರವಧಾರ್ಥವಾಗಿ ವಿಷ್ಣುವಿನ ಸಹಾಯವನ್ನು ಪ್ರಾರ್ಥಿಸಿರುವ ವಿಷಯ>
25-515

<ಇಂದ್ರನಿಗೆ ಋಜೀಷೀ ಎಂಬ ಹೆಸರು ಬರಲು ಕಾರಣ>
26-86 
28-175
29-649

<ಇಂದ್ರನು ವರಾಹಾಸುರನನ್ನು ಸಂಹಾರ ಮಾಡಿದ ವಿಷಯ>
26-763 
28-181
29-655

<ಇಂದ್ರನಿಗೆ ಪ್ರಾಪ್ತವಾದ ಬ್ರಹ್ಮಹತ್ಯಾದೋಷದ ಪರಿಹಾರದ ವಿವರಣೆ>
27-304

<ಇಂದ್ರ ತ್ವಾ ವೃಷಭಂ ಎಂಬ ಸೂಕ್ತದ ಪೀಠಿಕೆ>
17-1

<ಇಂದ್ರನು ತ್ವಷ್ಟೃವಿನ ಯಜ್ಞಗೃಹದಲ್ಲಿ ಬಲಾತ್ಕಾರದಿಂದ ಸೋಮಪಾನ ಮಾಡಿದ ವಿಚಾರ>
17-111

<ಇಂದ್ರನು ಶರ್ಯಾತಿ ಎಂಬುವನ ಯಜ್ಞದಲ್ಲಿ ಸೋಮಪಾನ ಮಾಡಿದ ವಿಚಾರ ಮತ್ತು ಶರ್ಯಾತಿ ಎಂಬ ರಾಜರ್ಷಿಯ ವಿಷಯ>
17-147

<ಇಂದ್ರನಿಗೂ ಋಭುಗಳಿಗೂ ಇರುವ ಸಂಬಂಧ>
17-506

<ಇಂದ್ರ, ಇಂದ್ರಾಣೀ ಮತ್ತು ಇಂದ್ರಪುತ್ರನಾದ ವೃಷಾಕಪಿ ಎಂಬುವರಿಗೆ ನಡೆದ ಸಂವಾದವು>
29-451

<ಇಂದ್ರಾಣೀಮಾಸು ನಾರಿಷು ಎಂಬ ಋಕ್ಕಿನ ನಿರುಕ್ತ>
22-471

<ಇಂದ್ರನ ವಿಷಯ ಸಾಮಾನ್ಯ ವರ್ಣನೆ>
29-577

<ಇಂದ್ರಾವರುಣರಿಗಿರುವ ಪರಸ್ಪರ ಸಂಬಂಧ>
29-628

<ಇಂದ್ರನಿಗೆ ಮರುತ್ವಾನ್‍ ಎಂಬ ಹೆಸರು ಬರಲು ಕಾರಣ>
28-170
29-645

<ಇಂದ್ರನ ಸಾಹಸಕೃತ್ಯಗಳು-ವೃತ್ರಾಸುರ ವಧ ವರ್ಣನೆ>
28-161

<ಇಂದ್ರನು ವಿಶ್ವರೂಪವನನ್ನು ಸಂಹಾರಮಾಡಿದ ವಿಚಾರ>
28-184

<ಇಂದ್ರನಿಗೆ ಬ್ರಹ್ಮಹತ್ಯಾದೋಷವು ಪ್ರಾಪ್ತವಾದ ವಿಷಯ ಮತ್ತು ಅದರ ಪರಿಹಾರಕ್ಕಾಗಿ ಇಂದ್ರನು ಮಾಡಿದ ಪ್ರಯತ್ನ>
28-186

<ಇಂದ್ರವೈಕುಂಠನ ವಿಷಯ>
28-189
28-961

<ಇಂದ್ರನ ಇಪ್ಪತ್ತಾರು ಹೆಸರುಗಳು-ವಾಯು ವರುಣ ರುದ್ರ ಮೊದಲಾದವರು (ಬೃ.ದೇ.)>
28-719

<ಇಂದ್ರದೇವತಾಕವಾದ ಸೂಕ್ತಗಳು (ಬೃ.ದೇ.)>
28-748

<ಇಂದ್ರ ಮರುತ್ತುಗಳು ಮತ್ತು ಅಗಸ್ತ್ಯಋಷಿ (ಬೃ.ದೇ. ೧೬೯-೧೭೦) ಸೂಕ್ತಗಳು>
28-817

<ಇಂದ್ರ ಮತ್ತು ದೈತ್ಯರು (ಬೃ.ದೇ.)>
28-823

<ಇಂದ್ರ ಮತ್ತು ಗೃತ್ಸಮದ (ಬೃ.ದೇ.)>
28-824

<ಇಂದ್ರ ಮತ್ತು ಋಷಿಗಳು, ತಪಸ್ಸಾಮರ್ಥ್ಯ (ಬೃ.ದೇ.)>
28-939

<ಇಂದ್ರ ಮತ್ತು ತ್ರಿಶಿರಾಃ (ಬೃ.ದೇ.)>
28-941

<ಇಂದ್ರನು ರುದ್ರಧನುಸ್ಸಿನ ಹೆದೆಯನ್ನು ಕತ್ತರಿಸಿದ ವಿಚಾರ>
30-1160

<ಇಂದ್ರನ ಸಂಬಂಧವಾದ ಅನೇಕ ವಿಷಯಗಳು>
30-369

<ಇಧ್ಮ ಶಬ್ದಾರ್ಥ ವಿವರಣೆ>
16-371

<ಇಪ್ಪತ್ತೊಂದು ವಿಧ ಯಜ್ಞ ಪ್ರಭೇದಗಳು>
3-32
6-359

<ಇಪ್ಪತ್ತೊಂದು ವಿಧವಾದ ಛಂದಸ್ಸುಗಳ ವಿವರಣೆ>
17-695

<ಇಪ್ಪತ್ತೇಳು ನಕ್ಷತ್ರಗಳು ಮತ್ತು ಅವುಗಳ ಅಧಿದೇವತೆಗಳು>
29-386

<ಇಭನ ವಿಷಯ>
20-730

<ಇಮಂ ತಂ ಪಶ್ಯ ಎಂಬ ಋಕ್ಕಿನ ನಿರುಕ್ತ>
30-188


<ಇಯತ್ತಕಃ ಎಂಬ ಶಬ್ದದ ಅರ್ಥವಿವರಣೆ>
14-369

<ಇಯಂ ಶುಷ್ಮೇಭಿಃ ಎಂಬ ಋಕ್ಕಿನ ನಿರುಕ್ತ>
21-411

<ಇರಧ್ಯೈ ಎಂಬ ಶಬ್ದದ ಅರ್ಥವಿವರಣೆ>
10-576

<ಇರಿಣಂ ಎಂಬ ಶಬ್ದದ ಅರ್ಥವಿವರಣೆ>
14-160
27-857

<ಇವ ಎಂಬ ಶಬ್ದದ ಅರ್ಥ ಮತ್ತು ಪ್ರಯೋಗ>
19-664

<ಇಷುಧಿ ಶಬ್ದಾರ್ಥ ವಿವರಣೆ>
21-568

<ಇಷಿರೇಣ ತೇ ಎಂಬ ಋಕ್ಕಿನ ನಿರುಕ್ತ ಮತ್ತು ಅರ್ಥವಿವರಣೆ>
24-555

<ಇಷಿಕಾ ಎಂಬ ಶಬ್ದದ ವಿವರಣೆ>
27-857

<ಇಲೀಬಿಲಸ್ಯ ಎಂಬ ಶಬ್ದದ ವಿವರಣೆ>
4-53

<ಇಳಃ>
28-617

<ಇಳಾ (ಬೃ.ದೇ.)>
28-756

<ಇಳಾ ಶಬ್ದ ವಿವರಣೆ>
10-357
16-49

<ಇಳಾ, ಸರಸ್ವತೀ, ಭಾರತೀ ಎಂಬ ದೇವತೆಗಳ ವಿಷಯ>
11-263

<ಇಳಸ್ಪದೇ ಎಂಬ ಶಬ್ದಗಳ ವಿವರಣೆ>
14-563
16-589

<ಇಳಾ ಶಬ್ದದ ವಿವಿಧಾರ್ಥಗಳು>
16-465
18-794

<ಇಳಾ ದೇವತೆಯ ವಿಷಯ>
16-49

<ಇಳಾ ಸ್ವರೂಪ ವಿಚಾರ>
17-452

<ಇಳೆಯ ಉತ್ಪತ್ತಿ ಮತ್ತು ಪಾರ್ಥಿವ ಸ್ವರೂಪ>
17-453

<ಇಳೆಯ ಯಾಜ್ಞಿಕ ಸ್ವರೂಪ>
17-455

<ಇಳೆಯ ಲೌಕಿಕವಾದ ಶಕ್ತಿಸ್ವರೂಪಗಳು>
17-455

<ಇಳೆಯ ಆಧ್ಯಾತ್ಮಿಕ ಸ್ವರೂಪ>
17-457

<ಇಳಾ ರೂಪಿಯಾದ ಧೇನುವಿನ ಕ್ಷೀರ>
17-457

[ಈ]
<ಈಡ್ಯಃ ಎಂಬ ಶಬ್ದದ ಅರ್ಥವಿವರಣೆ>
14-233
15-683

<ಈರ್ಮಾಂತಾಸಃ ಎಂಬ ಶಬ್ದದ ಅರ್ಥವಿವರಣೆ>
12-270

<ಈಶಾನಾಸಃ ಎಂಬ ಶಬ್ದದ ನಾನಾರ್ಥಗಳು>
11-171

<ಈಳಃ ಎಂಬ ದೇವತೆಯ ವಿಷಯ>
30-343

<ಈಳಿತ ಶಬ್ದಾರ್ಥ ವಿವರಣೆ>
11-242

<ಈಕ್ಷೇ ಎಂಬ ಶಬ್ದದ ನಿರ್ವಚನ, ಅರ್ಥವಿವರಣೆ ಇತ್ಯಾದಿ>
20-714

[ಉ]
<ಉ ಎಂಬ ಶಬ್ದದ ಅರ್ಥವಿವರಣೆ>
18-804 
19-742

<ಉಕ್ಥಮಂತ್ರಸ್ವರೂಪ ಮತ್ತು ವಿವರಣೆ>
8-252 
14-582
10-695 
11-153 
18-79

<ಉಖಾಶಬ್ದವಿವರಣೆ>
12-194

<ಉತ ತ್ವಃ ಪಶ್ಯನ್ನ ದದರ್ಶ ವಾಚಃ ಎಂಬ ಋಕ್ಕಿನ ಅರ್ಥ ವಿವರಣೆ>
1-527

<ಉತ ತ್ವಃ ಪಶ್ಯನ್‍ ಎಂಬ ಋಕ್ಕಿನ ನಿರುಕ್ತ>
28-658

<ಉತ ತ್ವಂ ಸಖ್ಯೇ ಸ್ಥಿರಪೀತಮಾಹುಃ ಎಂಬ ಋಕ್ಕಿನ ಅರ್ಥವಿವರಣೆ>
1-527

<ಉತ ತ್ವಂ ಸಖ್ಯೇ ಎಂಬ ಋಕ್ಕಿನ ನಿರುಕ್ತ>
28-661

<ಉತ್ತರವೇದಿಯನ್ನು (ಯಜ್ಞದಲ್ಲಿ) ರಚಿಸುವ ಕ್ರಮ ಮತ್ತು ಅದರ ಮಹತ್ವ ಇತ್ಯಾದಿ>
10-678

<ಉತ್ಸ ಶಬ್ದದ ಅರ್ಥವಿವರಣೆ>
11-591 
14-789

<ಉತ ಗ್ನಾ ವ್ಯಂತು ಎಂಬ ಋಕ್ಕಿಗೆ ಯಾಸ್ಕರ ನಿರ್ವಚನ>
19-523

<ಉತಾಪಿ ಮೈತ್ರಾವರುಣ ಎಂಬ ಋಕ್ಕಿನ ನಿರುಕ್ತ>
22-140

<ಉದರ್ಕ ಅಥವಾ ಅಭ್ಯಾಸಗಳ ಉಪಯೋಗ ಇತ್ಯಾದಿ>
13-225

<ಉದಕತತ್ತ್ವಕ್ಕೂ ವರುಣನಿಗೂ ಇರುವ ಸಂಬಂಧ>
20-14

<ಉದೀರತಾಂ ಎಂಬ ಸೂಕ್ತದ ಪೀಠಿಕೆ>
27-484

<ಉದೀರತಾಂ ಎಂಬ ಋಕ್ಕಿನ ನಿರುಕ್ತ>
27-490

<ಉನ್ನೀಯಮಾನ ಸೂಕ್ತಗಳ ವಿವರಣೆ>
15-696 
26-49

<ಉಪನಿಷತ್ತುಗಳು>
1-24

<ಉಪಕರಣಗಳು>
5-686

<ಉಪೋ ಅದರ್ಶಿ ಎಂಬ ಋಕ್ಕಿನ ನಿರುಕ್ತ ಮತ್ತು ಅರ್ಥವಿವರಣೆ>
10-148

<ಉಪಾಂಶುಗ್ರಹವಿವರಣೆ>
10-650

<ಉಪ ವ ಏಷೇ ಎಂಬ ಋಕ್ಕಿನ ನಿರುಕ್ತ ಮತ್ತು ಅರ್ಥ ವಿವರಣೆ>
14-137

<ಉಪಸರ್ಗ ಎಂದರೇನು? ಅವ ಎಂಬ ಉಪಸರ್ಗದ ವಿವರಣೆ>
19-704

<ಉಪಪ್ಲವತ ಮಂಡೂಕಿ ಎಂಬ ಪರಿಶಿಷ್ಟ ಮಂತ್ರ ಮತ್ತು ಅದಕ್ಕೆ ಯಾಸ್ಕರು ಹೇಳಿರುವ ನಿರ್ವಚನ>
23-240

<ಉಪಪ್ಲವತ ಎಂಬ ಪರಿಶಿಷ್ಟ ಮಂತ್ರ>
23-278

<ಉಪಜಿಹ್ವಿಕಾ ಎಂಬ ಶಬ್ದದ ನಿರ್ವಚನ ಮತ್ತು ಅರ್ಥವಿವರಣೆ>
25-555

<ಉಪಸರ್ಗಗಳು ಮತ್ತು ಲಿಂಗಗಳು (ಬೃ. ದೇ.)>
28-736

<ಉಪಾವಸೃಜ ತ್ಮನ್ಯಾ ಎಂಬ ಋಕ್ಕಿನ ನಿರುಕ್ತ>
30-365

<ಉಭಯ ಲಿಂಗದೇವತಾ ವಿಷಯ>
12-482

<ಉರಣ>
5-693

<ಉರಣ ಎಂಬ ಅಸುರನನ್ನು ಇಂದ್ರನು ಸಂಹಾರ ಮಾಡಿದ ವಿಷಯ>
14-731

<ಉರ್ವಶೀ ಶಬ್ದದ ನಿರ್ವಚನ ಮತ್ತು ಅರ್ಥವಿವರಣೆ>
22-140 
30-14

<ಉರ್ವೀ ಎಂಬ ಶಬ್ದದ ವಿವರಣೆ>
17-369

<ಉರ್ವಶೀಃ ಮತ್ತು ಪುರೂರವಾಃ (ಬೃ.ದೇ.)>
28-989 
30-1

<ಉರ್ವಶೀ ಮತ್ತು ಪುರೂರವರ ಸಂಭಾಷಣೆ>
30-10

<ಉಲೂಖಲ ಶಬ್ದ ವಿವರಣೆ>
3-413

<ಉಶಿಕ್‍ ಎಂಬ ಸ್ತ್ರೀಯ ಪುತ್ರನಾದ ಕಕ್ಷೀವಾನ್‍ ಎಂಬ ಋಷಿ>
10-24

<ಉಶನಾಃ ಕವಿಃ>
28-195

<ಉಶನಾಃ ಎಂಬ ಋಷಿಯ ವಿಷಯ>
5-190 
26-761

<ಉಷೋದೇವತೆಯ ಸ್ವರೂಪ>
5-32

<ಉಷಃ ಶಬ್ದಾರ್ಥ ನಿಷ್ಪತ್ತಿ>
5-32

<ಉಷಾಃ>
5-560

<ಉಷಃಕಾಲದ ವಿವರಣೆ>
5-401
18-537

<ಉಷಃಕಾಲದ ಉತ್ಪತ್ತಿವಿಚಾರ>
9-3

<ಉಷಃಕಾಲದ ಭೂತ ಭವಿಷ್ಯದ್ವರ್ತಮಾನಗಳ ಕಾಲಗಳ ವ್ಯಾಪ್ತಿತ್ವವು>
9-54

<ಉಷಸ್ಸಿನ ವರ್ತಮಾನ ಭೂತಕಾಲಗಳ ಪರಸ್ಪರ ಸಂಬಂಧ>
9-73

<ಉಷೋ ದೇವತೆಯ ಪರ್ಯಾಯನಾಮಗಳ ಅರ್ಥವಿವರಣೆ>
10-71

<ಉಷೋದೇವತೆಯ ವಿಷಯ>
18-525
20-151

<ಉಷಸ್ತಿಚಾಕ್ರಾಯಣ ಎಂಬುವನ ವೃತ್ತಾಂತ>
11-551

<ಉಷಾಸಾನಕ್ತಾ ಎಂಬ ಶಬ್ದದ ವಿವರಣೆ>
14-136

<ಉಷಸೌ (ನಕ್ತೋಷಾಸಾ) ಶಬ್ದದ ವಿವರಣೆ>
14-245
28-757
30-351


<ಉಷೋದೇವತೆಯ ರಥವನ್ನು ಇಂದ್ರನು ಧ್ವಂಸಮಾಡಿದ ವಿಚಾರ>
14-766
18-227

<ಉಷಸ್ಸಿಗೂ ಅಗ್ನಿಗೂ ಇರುವ ಸಂಬಂಧ>
16-300

<ಉಷೋದೇವತೆಯ ಸ್ವರೂಪವನ್ನು ವಿವರಿಸುವ ೧೬ ಶಬ್ದಗಳ ಅರ್ಥ ವಿವರಣೆ ಇತ್ಯಾದಿ>
18-526
20-152

<ಉಷಸ್ಸು ಅಗ್ನಿಯ ಜನನಿಯೆಂಬ ವಿಚಾರ>
18-744

<ಉಷೋದೇವತೆಗೂ ಪೃಥಿವಿಗೂ ಇರುವ ಮಾತೃದುಹಿತೃ ಸಂಬಂಧ>
19-528

<ಉಷಸ್ಸಿಗೂ ಯಜ್ಞಾಗ್ನಿಗೂ ಇರುವ ಸಂಬಂಧ>
17-414

<ಉಷಸ್ಸಿನ ಅನಾದಿತ್ವ ಮತ್ತು ಪುನಃ ಆವಿರ್ಭಾವ>
17-575

<ಉಸ್ರಃ>
6-258

<ಉಕ್ಷಾ ಶಬ್ದದ ವಿವರಣೆ>
11-387

[ಊ]
<ಊರ್ಜೋನಪಾತ್‍ ಎಂಬ ಶಬ್ದದ ಅರ್ಥ ವಿವರಣೆ>
25-249

<ಊರ್ಣೋತ್‍ ಎಂಬ ಶಬ್ದದ ಅರ್ಥ ವಿವರಣೆ>
6-224

<ಊರ್ದರ ಎಂಬ ಶಬ್ದದ ನಿರ್ವಚನ>
14-750

<ಉರ್ಮ್ಯಾಶಬ್ದದ ಅರ್ಥವಿವರಣೆ>
14-19


[ಋ]
<ಋಕ್‍ ಶಾಖೆಗಳು ೨೧>
1-49

<ಋಕ್‍ ಶಬ್ದಾರ್ಥ ಇತ್ಯಾದಿ>
12-570

<ಋಕ್ಕುಗಳಿಗೂ ಸಾಮಕ್ಕೂ ಇರುವ ನಿಕಟ ಸಂಬಂಧ>
18-55

<ಋಕ್‍ ಮತ್ತು ಸಾಮ ಮಂತ್ರಗಳ ಐದುವಿಧ ಪ್ರಭೇದಗಳು>
18-57

<ಋಗ್ವೇದಸಂಹಿತಾ>
1-11

<ಋಗ್ವೇದದ ಶಾಖೆಗಳು>
1-47

<ಋಗ್ವೇದದ ನಾಲ್ಕು ವೇದಗಳ ಸೂಕ್ಷವಿವರಣೆ>
1-4

<ಋಗ್ವೇದದ ಬ್ರಾಹ್ಮಣಗಳು ಮತ್ತು ಆರಣ್ಯಕಗಳು>
1-168

<ಋಗ್ವೇದದ ಋಷಿಗಳು>
1-176

<ಋಗ್ವೇದದ ಋಷಿಗಳ ಹೆಸರುಗಳ ಅಕಾರಾದಿವರ್ಣಾನುಕ್ರಮಸೂಚೀ>
1-191

<ಋಗ್ವೇದದ ದೇವತೆಗಳು>
1-205

<ಋಗ್ವೇದದ ಭಾಷ್ಯಕಾರರು>
1-232

<ಋಗ್ವೇದದ ಪದ ಪಾಠಕಾರರು>
1-286

<ಋಗ್ವೇದ–ಅದರ ಮತ>
1-324

<ಋಗ್ವೇದಭಾಷ್ಯಭೂಮಿಕಾ–ಕರ್ನಾಟಕಾನುವಾದ ಮತ್ತು ಆಂಗ್ಲ ಭಾಷಾಂತರಸಹಿತಾ>
1-345

<ಋಗ್ವೇದ ಮಂತ್ರಗಳ ಅಕಾರಾದಿ ವರ್ಣಾನುಕ್ರಮಣಿಕೆ>
1-717

<ಋಗಾದಿ ವೇದಗಳ ಉತ್ಪತ್ತಿ>
29-898

<ಋಚಾಂತ್ವಃ ಪೋಷಮಾಸ್ತೇ ಎಂಬ ಋಕ್ಕಿನ ನಿರುಕ್ತ>
28-672

<ಋಜಿತ್ವನ ವಿಷಯ>
5-169
20-728

<ಋಜ್ರಾಶ್ವ>
8-247
22-488

<ಋಜ್ರಾಶ್ವನೆಂಬುವನು ಅಶ್ವಿನೀದೇವತೆಗಳ ವಾಹನವಾದ ರಾಸಭವು ತೋಳರೂಪದಿಂದಿರುವಾಗ ಅದಕ್ಕೆ ನೂರು ಆಡುಗಳನ್ನು ಕತ್ತರಿಸಿ ಆಹಾರವಾಗಿ ಕೊಟ್ಟಿದ್ದರಿಂದ ಕೋಪಗೊಂಡ ಅವನ ತಂದೆಯು ಋಜ್ರಾಶ್ವನನ್ನೂ ಕುರುಡನನ್ನಾಗಿ ಮಾಡಲು ಅಶ್ವಿನೀ ದೇವತೆಗಳು ಅವನಿಗೆ ಪೂರ್ವದಂತೆ ಕಣ್ಣುಗಳನ್ನು ಅನುಗ್ರಹಿಸಿದ ವಿಚಾರ>
9-266 
9-382

<ಋಜ್ರಾಶ್ವನು ಅಂಧನಾಗಿದ್ದಾಗ ದೃಷ್ಟಿಪ್ರದಾನಕ್ಕಾಗಿ ಅಶ್ವಿನೀದೇವತೆಗಳನ್ನು ಸ್ತುತಿಸಿದ ವಿಚಾರ>
9-517

<ಋಜೀಷಿಣಃ ಎಂಬ ಶಬ್ದದ ವಿವರಣೆ>
15-227

<ಋಜೀಷೀ ಎಂಬ ಹೆಸರು ಇಂದ್ರನಿಗೆ ಬರಲು ಕಾರಣ>
19-322

<ಋಜೀಷ ಶಬ್ದವಿವರಣೆ–ಶ್ಯೇನ ಪಕ್ಷಿಯು ಸ್ವರ್ಗದಿಂದ ಸೋಮವನ್ನು ತಂದ ವಿಚಾರ>
17-930 
19-322 
24-227

<ಋಜೀಷಿನ್‍ ಶಬ್ದಾರ್ಥ ವಿವರಣೆ>
20-721

<ಋಣತ್ರಯದ ವಿಷಯ>
24-242 
28-349

<ಋಣಂಚಯ ರಾಜನು ಬಭ್ರುವಿಗೆ ಮಾಡಿದ ದಾನದ ಪ್ರಶಂಸೆ (ಬೃ.ದೇ.)>
28-857

<ಋತಜ್ಞಾಃ>
6-368

<ಋತಪ್ರವೀತಂ>
6-273

<ಋತಶಬ್ದದ ನಾನಾರ್ಥಗಳು ಮತ್ತು ಪ್ರಯೋಗಗಳು>
10-143 
11-45

<ಋತ್ವಿಯಃ ಎಂಬ ಶಬ್ದದ ವಿವರಣೆ>
10-621

<ಋತ ಶಬ್ದದ ವಿಶೇಷಾರ್ಥ>
10-673

<ಋತಸಾಪ ಶಬ್ದಾರ್ಥ>
13-474

<ಋತುಯಾಜ ಹೋಮಗಳ ವಿವರಣೆ–ಐತರೇಯ ಬ್ರಾಹ್ಮಣ ಮತ್ತು ತೈತ್ತಿರೀಯ ಸಂಹಿತೆಗಳಲ್ಲಿರುವಂತೆ>
15-307

<ಋತಧೀತಯಃ ಎಂಬ ಶಬ್ದದ ಅರ್ಥವಿವರಣೆ>
19-564

<ಋತು ಶಬ್ದ ವಿವರಣೆ>
17-97

<ಋತು ಸ್ವರೂಪ>
17-330

<ಋತುಗಳು ಆದಿತ್ಯನ ಅವಯವಗಳು>
17-344

<ಋತುವಿನಲ್ಲಿ (ಯಾವಯಾವ) ಯಾವ ಯಾವ ವರ್ಣದವರು ಯಜ್ಞಮಾಡಬೇಕೆಂಬ ವಿಚಾರ>
17-347

<ಋತಸ್ಯ ಗರ್ಭ ಇತ್ಯಾದಿ ವಿವರಣೆ>
17-398

<ಋತೇನ ಎಂಬ ಶಬ್ದದ ವಿವರಣೆ>
17-417

<ಋತುದೇವಕಾಕವಾದ ಸೂಕ್ತ–೧ನೇ ಮಂಡಲದ ೧೫ನೇ ಸೂಕ್ತ (ಬೃ.ದೇ.)>
28-765

<ಋದೂದರಃ ಎಂಬ ಶಬ್ದದ ವಿವರಣೆ>
15-205

<ಋದೂದರೇಣ ಎಂಬ ಶಬ್ದದ ನಿರ್ವಚನ>
24-560

<ಋಭುಗಳ ಪೂರ್ವೇತಿಹಾಸ>
3-4

<ಋಭುಶಬ್ದದ ರೂಪನಿಷ್ಪತ್ತಿ>
3-16
5-156

<ಋಭುಗಳು>
5-658 
17-165
18-294 
17-477

<ಋಭುಗಳು ಆದಿತ್ಯನ ಆತಿಥ್ಯವನ್ನು ಪಡೆದು ಯಜ್ಞದಲ್ಲಿ ಸೋಮಪಾನಾರ್ಹರಾದ ವಿಚಾರ>
8-666

<ಋಭುಗಳು ಚಮಸಾದಿ ಯಜ್ಞಪಾತ್ರೆಗಳನ್ನು ನಿರ್ಮಿಸಿದ ವಿಚಾರ>
8-671

<ಋಭು ಶಬ್ದಾರ್ಥ ವಿವರಣೆ>
8-685

<ಋಭುಗಳು ತಾಯಿಯಿಲ್ಲದ ಕರುವನ್ನು ಕಾಪಾಡಿದ ವಿಚಾರ>
8-693

<ಋಭುಗಳೊಡನೆ ಇಂದ್ರನ ಸಾಹಚರ್ಯ>
8-699

<ಋಭುಗಳ ಕಲಾಕೌಶಲ್ಯ ವಿವರಣೆ>
8-704

<ಋಭುಗಳು ಸೋಮಪಾನಾರ್ಹತೆಯನ್ನು ಪಡೆದ ವಿಚಾರ>
12-89 
26-92

<ಋಭುಗಳು ಒಂದು ಚಮಸಪಾತ್ರೆಯನ್ನು ನಾಲ್ಕಾಗಿ ಮಾಡಿದ ವಿಷಯ>
12-93 
17-493


<ಋಭುಗಳ ರಥನಿರ್ಮಾಣ ಕೌಶಲ್ಯ>
12-108

<ಋಭುಗಳು (ಆಪೋ ಭೂಯಿಷ್ಠಾಃ) ನೀರೇ ಎಲ್ಲಕ್ಕೂ ಶ್ರೇಷ್ಠವು ಅಥವಾ ಮೂಲವು ಎಂಬ ತತ್ತ್ವವನ್ನು ಪ್ರತಿಪಾದನೆ ಮಾಡಿದ ವಿಷಯ>
12-120

<ಋಭುಗಳ ಕಲಾಕೌಶಲ್ಯದ ವಿವರಣೆ>
17-492
25-393

<ಋಭುಗಳು ದೇವತ್ವವನ್ನು ಪಡೆದ ವಿಚಾರ>
17-477

<ಋಭುಗಳು ಇಂದ್ರನ ಸಖ್ಯವನ್ನು ಪಡೆದ ವಿಚಾರ>
17-497

<ಋಭುಗಳಿಗೂ ಇಂದ್ರನಿಗೂ ಇರುವ ಸಂಬಂಧ>
17-506

<ಋಭುಗಳ ಮತ್ತು ತ್ವಷ್ಟೃವಿನ ವೃತ್ತಾಂತ (ಬೃ.ದೇ.)>
28-779

<ಋಷಿಗಳ ಸ್ವರೂಪ ಇತ್ಯಾದಿ>
13-494

<ಋಷ್ಟಿ ಶಬ್ದವಿವರಣೆ>
13-128

<ಋಷಿ ಶಬ್ದವಿವರಣೆ>
17-57
26-663

<ಋಷಿಕಾ ಅಥವಾ ಮಂತ್ರದ್ರಷ್ಟೃಗಳಾದ ಸ್ತ್ರೀಋಷಿಗಳು–ಇವರಲ್ಲಿ ಮೂರು ಪಂಗಡಗಳು (ಬೃ.ದೇ.)>
28-753

[ಎ]
<ಎಂಟನೆಯ ಮಂಡಲದ ಪೀಠಿಕೆ>
23-279

<ಎಂಟನೆಯ ಮಂಡಲದ ಋಷಿಗಳು ಮತ್ತು ಅವರಿಂದ ದೃಷ್ಟವಾದ ಸೂಕ್ತಗಳ ವಿವರಣೆ ಇತ್ಯಾದಿ>
23-284

<ಎರಡನೆಯ ಮಂಡಲದ ೩೨ ನೇ ಸೂಕ್ತದ ಕೊನೆಯಲ್ಲಿ ಪಠಿಸಬೇಕಾದ ಪರಿಶಿಷ್ಟ ಮಂತ್ರಗಳು–ಅರ್ಥಸಹಿತ>
15-490

<ಎರಡನೆಯ ಮಂಡಲದ ಕೊನೆಯಲ್ಲಿ ಪಠಿಸಬೇಕಾದ ಪರಿಶಿಷ್ಟ ಮಂತ್ರಗಳು>
15-492

[ಏ]
<ಏಕವಿಂಶಸ್ತೋಮದ ವಿವರಣೆ>
2-146

<ಏಕತ ಎಂಬುವನ ವಿಷಯ>
5-230 
8-443

<ಏಕಃ ಸುಪರ್ಣಃ ಎಂಬ ಋಕ್ಕಿನ ನಿರುಕ್ತ>
30-497

<ಏಕಯಾ ಪ್ರತಿಭಾ ಎಂಬ ಋಕ್ಕಿನ ನಿರುಕ್ತ>
25-162

<ಏಕ ಎಂಬ ಶಬ್ದದ ರೂಪನಿಷ್ಪತ್ತಿ>
28-154

<ಏಕಾದಶರುದ್ರರ ವಿಷಯ>
16-106

<ಏತಶ>
5-682 
9-593

<ಏತಶಋಷಿಯು ಸೂರ್ಯನೊಡನೆ ಹೋರಾಡುತ್ತಿದ್ದಾಗ ಇಂದ್ರನು ಅವನಿಗೆ ಸಹಾಯಮಾಡಿದ ವಿಚಾರ>
14-846

<ಏತಶಃ ಎಂಬ ಋಷಿಯ ವಿಚಾರ>
14-845

<ಏತಶನನ್ನು ಇಂದ್ರನು ರಕ್ಷಿಸಿದ ವಿಚಾರ>
18-222

<ಏತೋನ್ವಿಂದ್ರಂ (ಇಂದ್ರನ ಶುದ್ಧಿಗಾಗಿ) ಇತ್ಯಾದಿ ಮೂರು ಋಕ್ಕುಗಳ ವಿಷಯದಲ್ಲಿ ಭಾಷ್ಯಕಾರರು ಹೇಳಿರುವ ಪೂರ್ವೇತಿಹಾಸವು>
25-418

<ಏತಶನೆಂಬ ಋಷಿಯು ಸ್ವಶ್ವನೆಂಬ ರಾಜನ ಪುತ್ರನೊಡನೆ ಯುದ್ಧಮಾಡಿದಾಗ ಇಂದ್ರನು ಏತಶನಿಗೆ ಸಹಾಯ ಮಾಡಿದ ವಿಚಾರ>
17-975

<ಏತಾವಾನಸ್ಯ ಮಹಿಮಾ ಎಂಬ ಋಕ್ಕಿನ ಅರ್ಥವಿವರಣೆ>
29-757

<ಏನಾ ಶಬ್ದದ ಅರ್ಥ ಮತ್ತು ಪ್ರಯೋಗ>
21-779

<ಏವಾಸಃ>
13-228

<ಏವೈಃ>
13-228

<ಏವಾನ ಇಂದ್ರಃ ಎಂಬ ಋಕ್ಕಿಗೆ ಐತರೇಯ ಬ್ರಾಹ್ಮಣದಲ್ಲಿರುವ ವಿವರಣೆಯು>
17-983

<ಏಳುಮಂದಿ ಸೋಮಪಾಲಕರ ಹೆಸರು>
28-496

<ಏಳನೆಯ ಮಂಡಲದ ಕೊನೆಯಲ್ಲಿ ಪಠಿಸಬೇಕಾದ ಪರಿಶಿಷ್ಟಮಂತ್ರಗಳು>
23-276

[ಐ]
<ಐತರೇಯಶಾಖಾ>
1-90

<ಐತರೇಯಬ್ರಾಹ್ಮಣ>
1-171

<ಐತರೇಯಾರಣ್ಯಕ>
1-173

<ಐದನೆಯ ಮಂಡಲದ ಋಷಿಗಳು ಮತ್ತು ಅವರಿಂದ ದೃಷ್ಟವಾದ ಋಕ್ಕುಗಳ ವಿವರಣೆ ಇತ್ಯಾದಿ>
18-670

<ಐದನೆಯ ಮಂಡಲದ ಕೊನೆಯಲ್ಲಿ ಪಠಿಸಬೇಕಾದ ಪರಿಶಿಷ್ಟಮಂತ್ರಗಳು ಮತ್ತು ಪರಿಶಿಷ್ಟ ಮಂತ್ರಗಳ ಸ್ವರೂಪ>
20-263

<ಐಂದ್ರವಾಯವಗ್ರಹದ ವಿವರಣೆ>
10-650

[ಒ]
<ಒಂದೇ ಮನೆಯಲ್ಲಿರುವ ಬಂಧುಬಾಂಧವರ ಗುಣಗಳು ಒಂದೇ ವಿಧವಾಗಿರುವುದಿಲ್ಲ>
30-568

<ಒಂಭತ್ತುವಿಧ ಸ್ತೋಮಗಳು, ಅವುಗಳ ಸ್ವರೂಪ>
14-587

<ಒಂಭತ್ತನೆಯ ಮಂಡಲದ ಪೀಠಿಕೆ>
26-1

<ಒಂಭತ್ತನೆಯ ಮಂಡಲದ ಋಷಿಗಳು>
26-98

<ಒಂಭತ್ತನೆಯ ಮಂಡಲದ ಪರಿಶಿಷ್ಟಮಂತ್ರಗಳು>
27-174

[ಓ]
<ಓಂ ಎಂಬ ವ್ಯಾಹೃತಿ ಮತ್ತು ಅದರ ದೇವತೆಗಳು (ಬೃ.ದೇ.)>
28-745

<ಓಂಕಾರದ ಉತ್ಪತ್ತಿ ವಿಚಾರ>
12-583

<ಓಂಕಾರವೇ ಪರಬ್ರಹ್ಮ, ಇದೇ ಸಕಲವೇದಗಳ ಸಾರವು ಎಂಬ ವಿಚಾರ>
12-584

<ಓಚಿತ್ಸಖಾಯಂ ಎಂಬ ಸೂಕ್ತದ ಪೀಠಿಕೆ>
27-334

<ಓಷಧಿ ವನಸ್ಪತಿಗಳು>
5-686

<ಓಷಧಿಶಬ್ದನಿರ್ವಚನ>
7-395 
14-212 
30-69

<ಓ ಷೂ ಣೋ ಅಗ್ನೇ ಎಂಬ ಋಕ್ಕಿನ ವಿಶೇಷವಿನಿಯೋಗ>
11-70

[ಔ]
<ಔದ್ದಾಲಕಿಶಾಖಾ>
1-87

[ಕ]
<ಕಂ ಈಂ ಇತ್‍ ಉ ಎಂಬ ನಿಪಾತಗಳ ವಿಷಯ>
14-154 
19-189

<ಕಂಕತಃ ಎಂಬ ಶಬ್ದದ ಅರ್ಥವಿವರಣೆ>
14-335

<ಕಂಕತೋ ನ ಎಂಬ ಸೂಕ್ತದ ವಿಷಯದಲ್ಲಿ Wilson ಪಂಡಿತನ ಅಭಿಪ್ರಾಯ>
14-372

<ಕಃ ಕುಮಾರಂ ಎಂಬ ಋಕ್ಕಿಗೆ ಆದಿತ್ಯಪರವಾದ ಅರ್ಥ>
30-862

<ಕಃ ಕುಮಾರಂ ಎಂಬ ಋಕ್ಕಿಗೆ ಯಮನ ಪರವಾದ ಅರ್ಥ>
30-862

<ಕಚ್ಛಪ ಶಬ್ದಕ್ಕೆ ಯಾಸ್ಕರ ಅವಯವಾರ್ಥ ವಿವರಣೆ>
19-312

<ಕಠೋಪನಿಷತ್ತಿನ ಇತಿಹಾಸ>
30-845

<ಕಠೋಪನಿಷತ್ತಿನ ವಿವರಣೆ–ಯಮನಿಗೂ ನಚಿಕೇತಋಷಿಗೂ ನಡೆದ ಸಂಭಾಷಣೆ>
27-582

<ಕಣ್ವ ಎಂಬ ಋಷಿಯ ವಿಷಯ>
5-677 
11-84
23-280

<ಕಣ್ವ ಮತ್ತು ಪ್ರಗಾಥ ಎಂಬುವರ ವೃತ್ತಾಂತ (ಬೃ.ದೇ.)>
28-908

<ಕಣ್ವ ಎಂಬುವನನ್ನು ಅಶ್ವಿನೀದೇವತೆಗಳು ರಕ್ಷಿಸಿದ ವಿಚಾರ>
8-746

<ಕಣ್ವನೆಂಬ (ನೃಷದಪುತ್ರನ) ಋಷಿಗೆ ಅಶ್ವಿನೀದೇವತೆಗಳು ಕಣ್ಣನ್ನು ಕೊಟ್ಟು ರಕ್ಷಿಸಿದ ವಿಚಾರ>
9-341
9-436

<ಕಣ್ವಹೋತಾ ತ್ರಿತಃ>
19-358

<ಕದಾವಸೋ ಎಂಬ ಋಕ್ಕಿನ ನಿರುಕ್ತ>
30-228

<ಕದ್ರೂಸುಪರ್ಣಿಯರ ಕಥೆ–ಸುಪರ್ಣಿಯ ಪುತ್ರರೇ ಛಂದಸ್ಸುಗಳು>
17-572

<ಕನ್ಯಾಶಬ್ದದ ನಿರ್ವಚನ>
10-114

<ಕನೀನಾಂ ಎಂಬ ಶಬ್ದದ ವಿವರಣೆ>
11-530

<ಕನೀನಿಕೆ ಎಂಬ ಶಬ್ದದ ನಿರ್ವಚನ>
18-289

<ಕಪಿಂಜಲವೆಂಬ ಪಕ್ಷಿಯ ವಿಷಯ>
15-478

<ಕಪಿಂಜಲ (ಅಥವಾ ಶಕುನ) ಪಕ್ಷಿಯು ಶುಭಶಕುನವನ್ನು ನುಡಿಯುವ ವಿಷಯ>
15-488

<ಕಪಿಂಜಲರೂಪನಾದ ಇಂದ್ರ (ಬೃ.ದೇ.)>
28-830

<ಕಪೋತನೈಋತನೆಂಬ ಋಷಿ (ಬೃ.ದೇ.)>
28-1010

<ಕಪೋತನೈಋತನೆಂಬ ಋಷಿಯ ವಿಷಯ>
30-1114

<ಕರಂಜಶಬ್ದ ವಿವರಣೆ>
5-292

<ಕರಂಭಶಬ್ದದ ಅರ್ಥವಿವರಣೆ>
14-212

<ಕರಂಭಿಣಂ ಎಂಬ ಶಬ್ದ>
17-157

<ಕರಸ್ನಾ ಎಂಬ ಶಬ್ದದ ಅರ್ಥ ಮತ್ತು ನಿರ್ವಚನ>
16-380

<ಕರ್ಕಂಧು ಎಂಬ ಋಷಿಯನ್ನು ಅಶ್ವಿನೀದೇವತೆಗಳು ರಕ್ಷಿಸಿದ ವಿಚಾರ>
20-156

<ಕರ್ಮಾತ್ಮಕವಾದ ಯಜ್ಞವೇ ಬ್ರಹ್ಮಸ್ವರೂಪವು>
17-590

<ಕಲಿ>
8-792
22-487 
27-956

<ಕಲ್ಪವೆಂಬ ವೇದಾಂಗದ ವಿಷಯ ನಿರೂಪಣೆ>
1-553

<ಕವಚ ಮತ್ತು ಜ್ಯಾ (ಹೆದೆ) ಗಳ ವಿವರಣೆ>
21-562

<ಕವಷ ಐಲೂಷನ ವಿಷಯ (ಐತರೇಯ ಬ್ರಾಹ್ಮಣದಲ್ಲಿರುವಂತೆ)>
27-784

<ಕವಷ ಐಲೂಷನ ವಿಷಯ (ಕೌಷೀತಕೀ ಬ್ರಾಹ್ಮಣದಲ್ಲಿರುವಂತೆ)>
27-785

<ಕವಿ ಶಬ್ದದ ನಾನಾರ್ಥಗಳು>
13-377 
14-58
14-913

<ಕವಿಶಬ್ದದ ಅರ್ಥವಿವರಣೆ ಉದಾಹರಣೆಸಹಿತ>
17-358
19-749

<ಕವಿ ಮತ್ತು ಕ್ರತುಶಬ್ದದ ನಾನಾರ್ಥಗಳು>
16-288

<ಕವಿಃ ಉಶನಾಃ ಎಂಬುವರ ವಿಷಯ>
28-195

<ಕವ್ಯವಾಹನ ಎಂಬ ಪಿತೃಗಳಿಗೆ ಕವ್ಯವನ್ನು ವಹಿಸುವ ಅಗ್ನಿಯು>
18-849

<ಕಶಾಶಬ್ದದ ಅರ್ಥಾನುವಾದ ವಿಮರ್ಶೆ>
11-685
21-583

<ಕಶ್ಯಪನ ಪತ್ನಿಯರು (ಬೃ.ದೇ.)>
28-888

<ಕಕ್ಷೀವಾನ್‍ ಎಂಬ ಋಷಿಯ ವಿಷಯ>
5-202
22-483

<ಕಕ್ಷೀವಾನ್‍ ಎಂಬ ಋಷಿಯ ಜನ್ಮಕಥನ>
9-180
18-167

<ಕಕ್ಷೀವಾನ್‍ ಎಂಬ ಶಬ್ದದ ರೂಪನಿಷ್ಪತ್ತಿ>
9-215
10-191

<ಕಕ್ಷೀವಾನ್‍ ಎಂಬ ಋಷಿಯು ಶ್ರೇಷ್ಠರಾದ ಮೂವತ್ತಮೂರು ಆಂಗಿರಸ ವಂಶಸ್ಥರಲ್ಲಿ ಪ್ರಮುಖನೆಂಬ ವಿಚಾರ>
9-439

<ಕಕ್ಷೀವಾನ್‍ ಎಂಬ ಋಷಿಯು ಉಶಿಕ್‍ ಎಂಬ ಸ್ತ್ರೀಯ ಪುತ್ರನೆಂಬ ವಿಷಯದಲ್ಲಿ ಪೂರ್ವೇತಿಹಾಸವು>
10-24

<ಕಕ್ಷೀವಾನ್‍ ಮತ್ತು ಸ್ವನಯ ಎಂಬುವರ ವೃತ್ತಾಂತ (ಬೃ.ದೇ.)>
28-798

<ಕಕ್ಷೀವಂತನ ಪುತ್ರಿಯಾದ ಘೋಷಾ ಎಂಬ ಸ್ತ್ರೀಯ ವಿಚಾರ>
10-29

<ಕಾಣೇ ಎಂಬ ಶಬ್ದದ ವಿವರಣೆ>
30-1044

<ಕಾಣುಕಾ ಎಂಬ ಶಬ್ದದ ಅರ್ಥವಿವರಣೆ>
25-162

<ಕಾಣ್ವಶಾಖೆಯ ಪ್ರವರ್ತಕರು>
1-114

<ಕಾಣ್ವಸಂಹಿತೆಯ ಪದಪಾಠಕಾರರು>
1-288

<ಕಾರು, ಕಾರವಃ ಎಂಬ ಶಬ್ದಗಳ ಅರ್ಥವಿವರಣೆ>
8-726 
11-432 
12-722 
13-445
55-775

<ಕಾಲಸ್ವರೂಪ ವಿವರಣೆ>
3-313

<ಕಾಲಚಕ್ರದ ಸ್ವರೂಪ ವಿವರಣೆ>
12-369
12-638

<ಕಾಲಚಕ್ರಕ್ಕೂ ಆದಿತ್ಯಮಂಡಲಕ್ಕೂ ಇರುವ ತಾದಾತ್ಮ್ಯ>
12-377

<ಕಾಲಕ್ಕೂ ಆದಿತ್ಯನಿಗೂ ಇರುವ ಸಂಬಂಧ>
12-349

<ಕಾಲ, ಸಂವತ್ಸರ ಇವುಗಳ ವಿವರಣೆ>
11-633

<ಕಾಲಕ್ಕೂ ಮಹಾತತ್ತ್ವಕ್ಕೂ ಇರುವ ಸಂಬಂಧ>
8-792 
22-487
27-956

<ಕಾಲಕಂಜರೆಂಬ ಅಸುರರ ವಿಷಯ>
7-116

<ಕಾಲವಿಭಾಗ>
29-13

<ಕಾಲವಿಭಾಗ, ಮೃತ್ಯು ಇತ್ಯಾದಿ>
30-762

<ಕಾವ್ಯ ಉಶನಾಃ>
5-679

<ಕಾಷ್ಠಾಸು ಎಂಬ ಶಬ್ದದ ವಿವರಣೆ>
11-401

<ಕಿಮಿಚ್ಛಂತೀ ಎಂಬ ಋಕ್ಕಿನ ನಿರುಕ್ತ>
30-296

<ಕಿಮಿದೀ>
5-698

<ಕಿಯೇದಾಃ ಎಂಬ ಶಬ್ದದ ಅರ್ಥ ವಿವರಣೆ>
5-497

<ಕೀಕಟ ಎಂಬ ಜನಾಂಗದವರು>
17-208

<ಕುತ್ಸನ ವಿಷಯ>
4-62 
5-173 
5-678 
8-768 
8-821 
13-335
20-694

<ಕುತ್ಸ ಮತ್ತು ಶುಷ್ಣಾಸುರನ ವಿಷಯ>
13-380

<ಕುತ್ಸ, ಆಯು, ಅತಿಥಿಗ್ವ ಎಂಬುವರ ವಿಷಯ>
14-741

<ಕುತ್ಸನಿಗೆ ಸಹಾಯ ಮಾಡುವುದಕ್ಕಾಗಿ ಇಂದ್ರನು ಶುಷ್ಣನೆಂಬ ಅಸುರನನ್ನು ಸಂಹರಿಸಿದ ವಿಷಯ>
20-724

<ಕುತ್ಸನಿಗೂ ಇಂದ್ರನಿಗೂ ಸ್ನೇಹವಾದ ವಿಷಯ>
17-942

<ಕುಭಾ ಎಂಬ ನದಿ>
19-643

<ಕುಮಾರ ಎಂಬ ಋಷಿ>
18-730


<ಕುಮಾರಂ ಮಾತಾ ಯುವತಿಃ ಎಂಬ ಋಕ್ಕಿನಲ್ಲಿ ಹೇಳಿರುವ ಕಥಾಸಂಧರ್ಭದಲ್ಲಿ ಸಾಮವೇದದ ಶಾಟ್ಯಾಯನ ಬ್ರಾಹ್ಮಣದಲ್ಲಿ ಹೇಳಿರುವ ಪೂರ್ತೀತಿಹಾಸದ ವಿವರಣೆ>
18-736

<ಕುಮಾರಂ ಮಾತಾ ಯುವತಿಃ ಎಂಬ ಋಕ್ಕಿಗೆ Wilson ಪಂಡಿತನ ಅಭಿಪ್ರಾಯ>
18-741

<ಕುಮಾರನೆಂಬ ಹೆಸರು ಅಗ್ನಿಗೆ ಹೇಗೆ ಬಂದಿತೆಂಬ ವಿಚಾರ>
18-739

<ಕುಯವ ಎಂಬ ಅಸುರನ ವಿಚಾರ>
8-392

<ಕುಶಿಕನೆಂಬ ರಾಜನ ವಿಷಯ>
17-199

<ಕುಶಿಕ ಶಬ್ದಾರ್ಥ ವಿಚಾರ>
17-199

<ಕುಶಿಕಾಸಃ ಎಂಬ ಶಬ್ದದ ವಿವರಣೆ>
17-47

<ಕುಷುಂಭಕಃ ಎಂಬ ಶಬ್ದದ ವಿವರಣೆ>
14-369

<ಕುಹಸ್ವಿದ್ದೋಷಾ ಎಂಬ ಋಕ್ಕಿನ ನಿರುಕ್ತವು>
27-973

<ಕೂರ್ಮ>
5-684

<ಕೃಣುಷ್ವ ಪಾಜಃ ಎಂಬ ಋಕ್ಕಿನ ನಿರ್ವಚನ>
17-764

<ಕೃದರ ಎಂಬ ಶಬ್ದದ ನಿರ್ವಚನ>
14-750

<ಕೃಶಾನುವಿನ ವಿಷಯ>
8-816
11-605

<ಕೃಷ್ಟಿಶಬ್ದಾರ್ಥವಿವರಣೆ>
13-429

<ಕೃಷ್ಣನೆಂಬ ಅಸುರನನ್ನು ಇಂದ್ರನು ಸಂಹಾರಮಾಡಿದ ವಿಷಯ>
22-487

<ಕೃಷ್ಣನೆಂಬ ಅಸುರನನ್ನು ಇಂದ್ರನು ಸಂಹಾರಮಾಡಿದ ವಿಷಯದಲ್ಲಿ ಪೂರ್ತೀತಿಹಾಸವು>
25-443

<ಕೃಷ್ಣ ಯಜುರ್ವೇದದ ಶಾಖೆಗಳು>
1-122

<ಕೃಷ್ಣ ಯಜುರ್ವೇದದ ಶಾಖಾಭೇದಗಳು>
1-100

<ಕೇತು>
6-290

<ಕೇತು ಶಬ್ದದ ವಿವರಣೆ>
10-293

<ಕೇಶಿ ಶಬ್ದಾರ್ಥ ವಿಚಾರ>
12-610
30-867

<ಕೇಶ್ಯ ೧ ಗ್ನಿಂ ಎಂಬ ಋಕ್ಕಿನ ನಿರುಕ್ತ>
30-872

<ಕೋನು ಮರ್ಯಾ ಎಂಬ ಋಕ್ಕಿಗೆ ಯಾಸ್ಕರ ನಿರ್ವಚನ>
24-483

<ಕೋಶ ಶಬ್ದಾರ್ಥ ವಿವರಣೆ–ಯಾಸ್ಕರ ನಿರ್ವಚನ>
19-638

<ಕೌಶೀತಕೀಶಾಖಾ>
1-76

<ಕೌಷೀತಕೀ ಆರಣ್ಯಕ>
1-175

<ಕೌಷೀತಕೀ ಬ್ರಾಹ್ಮಣ>
1-172

<ಕ್ರತು ಶಬ್ದದ ವಿಸ್ತಾರವಾದ ವಿವರಣೆ>
13-385 
16-288 
17-77

<ಕ್ರಿವಿ ಎಂಬ ಅಸುರನನ್ನು ಇಂದ್ರನು ಸಂಹರಿಸಿದ ವಿಷಯ>
14-810

[ಖ]
<ಖಲ ಶಬ್ದಾರ್ಥ ವಿವರಣೆ>
28-154


<ಖೇಲನೆಂಬ ರಾಜನ ಸಂಬಂಧಿಯಾದ ವಿಶ್ಪಲಾ ಎಂಬ ಸ್ತ್ರೀಯ ಕತ್ತರಿಸಿಹೋದ ಕಾಲನ್ನು ಅಶ್ವಿನೀದೇವತೆಗಳು ಸರಿಮಾಡಿದ ವಿಚಾರ>
9-262

[ಗ]
<ಗಂಗೆ ಮೊದಲಾದ ಮುಖ್ಯ ನದಿಗಳ ವರ್ಣನೆ>
29-107

<ಗಣಪತಿ ಶಬ್ದಕ್ಕೆ ವೇದ ಮತ್ತು ಪುರಾಣಾದಿಗಳಲ್ಲಿರುವ ಅರ್ಥವ್ಯತ್ಯಾಸ>
14-912

<ಗಂಧರ್ವರು>
5-665

<ಗಭಸ್ತ್ಯೋಃ ಎಂಬ ಶಬ್ದದ ವಿವರಣೆ>
6-132

<ಗರ್ತಶಬ್ದದ ನಾನಾರ್ಥಗಳು>
10-161
19-845

<ಗರ್ಭಸ್ರಾವ (abration) ಉಂಟಾಗದಿರುವುದಕ್ಕಾಗಿ ಪರಿಹಾರ ಇತ್ಯಾದಿ>
30-1092

<ಗರ್ಭಸ್ರಾವಿಣ್ಯುಪನಿಷತ್ತೆಂದು ಪ್ರಸಿದ್ಧವಾದ ಋಕ್ಕುಗಳು, ಶಿಶುವು ಸಕಾಲದಲ್ಲಿ ಪ್ರಸವವಾಗದಿದ್ದರೆ ಪಠಿಸಬೇಕಾದ ಮಂತ್ರಗಳು ಇತ್ಯಾದಿ>
20-149

<ಗಯಸ್ಫಾನ ಶಬ್ದದ ವಿವರಣೆ>
7-364

<ಗವಯಃ ಗೌರಃ–ಎಂಬ ಶಬ್ದಗಳ ಅರ್ಥವಿವರಣೆ>
18-87 
20-140

<ಗವ್ಯವಃ ಎಂಬ ಶಬ್ದದ ವಿವರಣೆ>
11-478

<ಗವ್ಯೂತಿ ಶಬ್ದಾರ್ಥ ವಿವರಣೆ>
17-651

<ಗ್ರಹಪಾತ್ರೆಗಳು (ಸೋಮಪಾನಾರ್ಥವಾದ)>
30-499

<ಗ್ರಹಪಾತ್ರೆಗಳ ಸಂಖ್ಯೆ>
30-501

<ಗಾತು ಶಬ್ದದ ವಿವರಣೆ>
8-508 
11-503

<ಗಾಯತ್ರೀ, ಅಗ್ನಿ, ಅರ್ಕ ಇವುಗಳಿಗಿರುವ ಸಂಬಂಧ>
12-446

<ಗಾಯತ್ರಿಯ ಸ್ಥಾನ ಮತ್ತು ಮಹತ್ತ್ವ>
12-440

<ಗಾಯತ್ರಿಯು ಶ್ಯೇನಪಕ್ಷಿಯ ರೂಪದಿಂದ ಸ್ವರ್ಗದಲ್ಲಿದ್ದ ಸೋಮನನ್ನು ಭೂಮಿಗೆ ತಂದ ವಿಚಾರ (ತೈತ್ತಿರೀಯಸಂಹಿತೆಯಲ್ಲಿರುವಂತೆ)>
18-180

<ಗಾಯತ್ರೀ ಛಂದಸ್ಸಿನ ಪ್ರಾಶಸ್ತ್ಯ>
18-197

<ಗಾಯತ್ರಿಯ ವಿಜಯ>
17-575 
26-21

<ಗಾಯತ್ರಿಗೆ ಸವನಕಾಲದಲ್ಲಿರುವ ಪ್ರಾಶಸ್ತ್ಯ>
17-570 
26-22

<ಗಾಯತ್ರೀ ಮಂತ್ರದ ವಿವರಣೆ>
17-553

<ಗಾಯತ್ರೀ ಮಹಾಮಂತ್ರದ ಪೀಠಿಕೆ>
17-553

<ಗಾಯತ್ರೀ ಮಂತ್ರದ ವಿಶೇಷ ವಿನಿಯೋಗ>
17-556

<ಗಾಯತ್ರೀಗೆ ಸಾವಿತ್ರೀ ಎಂಬ ಹೆಸರು ಬರಲು ಕಾರಣ>
17-557

<ಗಾಯತ್ರಿಯ ಪ್ರಾಧಾನ್ಯವನ್ನು ತೋರಿಸುವ ಮೂರು ಅಂಶಗಳು>
17-569

<ಗಾಯತ್ರಿಯು ಬ್ರಹ್ಮಜ್ಞಾನಕ್ಕೆ ಕೇವಲ ಸಾಧನವೇ ಅಥವಾ ಜ್ಞಾನರೂಪವಾದ ಸಾಧ್ಯವೇ ಎಂಬ ವಿಷಯವಿಮರ್ಶೆ>
17-593

<ಗಾಯತ್ರಿಯನ್ನು ಉಪಾಸನೆ ಮಾಡತಕ್ಕ ಎರಡು ಕ್ರಮಗಳು ಮತ್ತು ಬಾಹ್ಯಪ್ರಪಂಚದೊಡನೆ ತಾದಾತ್ಮ್ಯ>
17-600

<ಗಾಯತ್ರೀ, ಅಗ್ನಿ, ಶ್ಯೇನ ಇವುಗಳ ತಾದಾತ್ಮ್ಯ ಮತ್ತು ಸವನತ್ರಯಗಳಲ್ಲಿ ಇವುಗಳ ಸ್ಥಾನ>
17-578

<ಗಾಯತ್ರಿಯ ಪಾದತ್ರಯ ಮತ್ತು ತುರೀಯಪಾದಗಳ ಸ್ವರೂಪ, ಪಾದೋಽಸ್ಯ ವಿಶ್ವಾ ಭೂತಾನಿ ಎಂಬ ಮಂತ್ರದ ತಾತ್ಪರ್ಯ>
17-602

<ಗಾಯತ್ರಿಯ ಪ್ರಶಂಸೆ>
17-623

<ಗಾಯತ್ರಿಯ ಮಹತ್ತ್ವ>
17-624

<ಗಾಯತ್ರೀ ಮಹಾಮಂತ್ರ ಪ್ರತಿಪದಾರ್ಥ ವರ್ಣನೆ>
17-627

<ಗಾಯತ್ರೀ ಮಂತ್ರಗಳಲ್ಲಿರುವ ಶಬ್ದಗಳ ವ್ಯಾಕರಣಪ್ರಕ್ರಿಯಾ>
17-636

<ಗಾಲವಶಾಖಾ>
1-56

<ಗ್ನಾ ಎಂಬ ಶಬ್ದದ ಅರ್ಥವಿವರಣೆ>
15-168

<ಗ್ರಾಮಶಬ್ದ>
14-664

<ಗ್ರಾವಾ ಎಂಬ ಶಬ್ದದ ವಿವರಣೆ>
3-413 
10-614
17-401

<ಗ್ರಾವಗ್ರಾಭಃ ಎಂಬ ಋತ್ವಿಜನ ಕರ್ತವ್ಯ>
12-166

<ಗ್ರಾವಗಳ ವಿಷಯ>
29-115 
29-1003
30-1181

<ಗೀಃ ಶಬ್ದಾರ್ಥ ವಿಚಾರ>
16-201

<ಗುರು ಶುಶ್ರೂಷೆ ಮಾಡದೆ ವೇದಾಧ್ಯಯನವು ಫಲಿಸುವುದಿಲ್ಲ>
1-538

<ಗುಹಾ ಹಿತಂ ಎಂಬ ಶಬ್ದಗಳ ಅರ್ಥವಿವರಣೆ>
3-192

<ಗೂರ್ತ ಶಬ್ದಾರ್ಥ>
13-255

<ಗೃತ್ಸಮದ ಋಷಿ (ಬೃ.ದೇ.)>
28-823

<ಗೃತ್ಸಮದ ಮತ್ತು ಇಂದ್ರ (ಬೃ.ದೇ.)>
28-824

<ಗೃಧ್ನು ಶಬ್ದಾರ್ಥ>
6-280

<ಗೃಹ್ಯಸೂತ್ರಗಳು>
1-35

<ಗೋ ಅಗ್ರಶಬ್ದದ ವಿವರಣೆ>
5-281

<ಗೋತಮಾಃ>
6-509

<ಗೋತಮನೆಂಬ ಋಷಿಯು ಬಾಯಾರಿಕೆಯಿಂದ ಬಳಲಿ ಜಲಪಾನಕ್ಕಾಗಿ ಅಶ್ವಿನೀದೇವತೆಗಳನ್ನು ಪ್ರಾರ್ಥಿಸಲು ಅವರು ದೂರದೇಶದಲ್ಲಿದ್ದ ನೀರಿನ ಒಂದು ಬಾವಿಯನ್ನೇ ಅವನ ಸಈಪಕ್ಕೆ ತಂದು ಕೊಟ್ಟು ಅವನ ದಾಹೋಪಶಮನಮಾಡಿದ ವಿಚಾರ>
9-230
19-727

<ಗೋತಮನನ್ನು ಅಶ್ವಿನೀದೇವತೆಗಳು ರಕ್ಷಿಸಿದ ವಿಚಾರ>
13-643

<ಗೋತಮನ (ರಹೂಗಣಪುತ್ರನ) ವೃತ್ತಾಂತ>
13-647

<ಗೋತಮನು ಇಂದ್ರನನ್ನು ಸ್ತುತಿಸಿ ಕುರು ಮತ್ತು ಸೃಂಜಯ ಎಂಬ ಜನಾಂಗದ ರಾಜರುಗಳಿಗೆ ಸಹಾಯ ಮಾಡಿದ ವಿಚಾರ>
7-11

<ಗೋತ್ರಶಬ್ದವಿವರಣೆ>
14-798

<ಗೋತ್ರಭಿದಂ ಎಂಬ ಶಬ್ದದ ಅರ್ಥವಿವರಣೆ>
14-919

<ಗೋವುಗಳ ವಿಷಯ>
5-683
30-1145

<ಗೋಶಬ್ದಾರ್ಥವಿವರಣೆ>
9-544
12-13
16-11
27-745

[ಘ]
<ಘೃತಶಬ್ದದ ನಿರ್ವಚನ>
10-668

<ಘೃತಪೃಷ್ಠಃ ಎಂಬ ಶಬ್ದದ ಅರ್ಥವಿವರಣೆ>
19-293

<ಘೃತಾಚೀಶಬ್ದದ ಅರ್ಥವಿವರಣೆ>
16-535

<ಘರ್ಮ ಮತ್ತು ಪ್ರವರ್ಗ್ಯಗಳ ಸ್ವರೂಪ>
13-516

<ಘೋಷಾ ಎಂಬ ಸ್ತ್ರೀಯ ವಿಚಾರ>
28-959
27-930
28-959

<ಘೋಷಾ ಎಂಬ ಸ್ತ್ರೀಗೆ ಅಶ್ವಿನೀದೇವತೆಗಳು ಪತಿಯನ್ನು ಒದಗಿಸಿಕೊಟ್ಟ ವಿಚಾರ>
9-337

[ಚ]
<ಚಕ್ರ, ರಥ, ಇತ್ಯಾದಿ ಶಬ್ದಗಳ ರೂಪನಿಷ್ಪತ್ತಿ>
12-301

<ಚಟಕ (ಗುಬ್ಬಚ್ಚಿ) ವೆಂಬ ಹಕ್ಕಿಯನ್ನು ಅಶ್ವಿನೀದೇವತೆಗಳು ತೋಳನ ಬಾಯಿಂದ ಬಿಡಿಸಿ ರಕ್ಷಿಸಿದ ವಿಚಾರ>
9-255

<ಚತುರ್ಥ>
28-154

<ಚತುರ್ಹೋತೃ>
29-856

<ಚತುಶ್ಚತ್ವಾರಿಂಶಸ್ತೋಮದ ವಿವರಣೆ>
2-148

<ಚತ್ವಾರಿ ವಾಕ್‍–ನಾಲ್ಕು ವಿಧಗಳಾದ ವಾಕ್ಯಗಳು ಎಂಬ ಶಬ್ದಗಳ ನಾನಾರ್ಥಗಳು>
12-617

<ಚತ್ವಾರಿ ವಾಕ್‍–ಓಂಕಾರ ಮತ್ತು ಭೂಃ, ಭುವಃ, ಸುವಃ ಎಂಬ ಮೂರು ವ್ಯಾಹೃತಿಗಳು>
12-67

<ಚತ್ವಾರಿ ವಾಕ್‍–ನಾಮ, ಆಖ್ಯಾತ, ಉಪಸರ್ಗ, ನಿಪಾತಗಳೆಂಬ ನಾಲ್ಕುವಿಧ ಶಬ್ದ ಸಮೂಹಗಳು>
12-620

<ಚತ್ವಾರಿ ವಾಕ್‍–ಮಂತ್ರ, ಕಲ್ಪ, ಬ್ರಾಹ್ಮಣ, ವ್ಯಾವಹಾರಿಕ ಭಾಷೆ>
12-623

<ಚತ್ವಾರಿ ವಾಕ್‍–ಋಗ್ವೇದ, ಯಜುರ್ವೇದ, ಸಾಮವೇದ ಮತ್ತು ವ್ಯಾವಹಾರಿಕ ಭಾಷೆ>
12-623

<ಚತ್ವಾರಿ ವಾಕ್‍–ಸರ್ಪಗಳು, ಪಕ್ಷಿಗಳು, ಕೃಮಿಗಳು, ಮನುಷ್ಯರು ಇವರುಗಳ ನಾಲ್ಕು ವಿಧವಾದ ಭಾಷೆ>
12-623

<ಚತ್ವಾರಿ ಶೃಂಗಾ ಎಂಬ ಋಕ್ಕಿಗೆ ಯಾಸ್ಕರ ನಿರ್ವಚನ>
18-651

<ಚತ್ವಾರಿ ಶೃಂಗಾ ಎಂಬ ಋಕ್ಕಿಗೆ ಅಗ್ನಿಪರವಾದ ಅರ್ಥವಿವರಣೆ>
18-651

<ಚತ್ವಾರಿ ಶೃಂಗಾ ಎಂಬ ಋಕ್ಕಿಗೆ ಸೂರ್ಯಪರವಾದ ಅರ್ಥವಿವರಣೆ>
18-653

<ಚಂದ್ರಮಾ ಮನಸೋ ಜಾತಃ ಎಂಬ ಋಕ್ಕಿನ ವಿವರಣೆ>
29-929

<ಚಂದ್ರರಥಾಃ ಎಂಬ ಶಬ್ದದ ವಿವರಣೆ>
17-513

<ಚಂದ್ರ ಶಬ್ದಾರ್ಥ ವಿವರಣೆ>
11-473
12-716

<ಚನಃ ಎಂಬ ಶಬ್ದದ ನಿರ್ವಚನ>
30-541

<ಚಮೂಃ ಎಂಬ ಶಬ್ದದ ವಿವರಣೆ>
19-567

<ಚರ್ಮ ಶಬ್ದದ ನಿರ್ವಚನ>
15-760

<ಚರ್ಷಣಿ ಶಬ್ದದ ವಿವರಣೆ>
8-647 
10-271 
13-402 
13-429
14-36

<ಚರ್ಷಣೀಧೃತ ಶಬ್ದದ ಅರ್ಥವಿವರಣೆ>
17-469

<ಚ್ಯವನನೆಂಬ ಋಷಿಯು ವಾರ್ಧಕ್ಯದಿಂದ ತೊಂದರೆಪಡುತ್ತಿದ್ದುದನ್ನು ಅಶ್ವಿನೀದೇವತೆಗಳು ನೋಡಿ ಅವನಿಗೆ ಯೌವನವನ್ನುಂಟುಮಾಡಿದ ವಿಚಾರ>
9-233 
9-365 
9-433 
20-117 
22-486 
27-946

<ಚಾರೂಣಿ ಎಂಬ ಶಬ್ದ>
6-341

<ಚಿಕೇತತ್‍ ಎಂಬ ಶಬ್ದ>
8-258

<ಚಿತ್‍ ಶಬ್ದದ ಅನೇಕಾರ್ಥಗಳು>
10-276 
13-99 
15-503 
18-4 
19-658

<ಚಿತ್ತ ಶಬ್ದಾರ್ಥ>
15-642

<ಚಿತ್ತಿ ಶಬ್ದದ ವಿವರಣೆ>
6-220

<ಚಿತ್ರ ಮತ್ತು ಸೋಭರಿ (ಬೃ.ದೇ)>
28-914

<ಚುಮುರಿ>
5-695 
10-550 
19-202 
20-687

<ಚೋಷ್ಕೂಯತೇ ಎಂಬ ಶಬ್ದಕ್ಕೆ ಯಾಸ್ಕರ ನಿರ್ವಚನ ಇತ್ಯಾದಿ>
2160-2

<ಚ್ಯೌತ್ನ್ಯ ಶಬ್ದಾರ್ಥ>
13-264

[ಛ]
<ಛಂದಃಶಾಸ್ತ್ರ>
1-29

<ಛಂದಸ್ಸುಗಳು>
1-221

<ಛಂದಸ್ಸುಗಳ ಉದಾಹರಣೆಗಳು>
1-222

<ಛಂದಶ್ಶಾಸ್ತ್ರದ ಪ್ರಯೋಜನ ಇತ್ಯಾದಿ>
1-581

<ಛಂದಸ್ಸುಗಳಲ್ಲಿ ಗಾಯತ್ರಿಯು ಸ್ವರ್ಗದಿಂದ ಸೋಮವನ್ನು ತಂದ ವಿಚಾರ>
12-464

<ಛಂದಸ್ಸುಗಳ ಅಕ್ಷರ ವಿಭಾಗ ಪ್ರಕರಣ>
17-577
26-23

<ಛಂದಸ್ಸುಗಳಲ್ಲಿ ಗಾಯತ್ರಿಯು ಬ್ರಹ್ಮಜ್ಞಾನಕ್ಕೆ ಸಾಧನವಾಗುವುದು ಹೇಗೆ (ಆಕ್ಷೇಪ ಸಮಾಧಾನಗಳು)>
17-589

<ಛಂದಸ್ಸಿನ ಆಧ್ಯಾತ್ಮಿಕ ಲಕ್ಷಣ ಮತ್ತು ಜ್ಞಾನಾನ್ವೇಷಣದಲ್ಲಿ ಗಾಯತ್ರಿಯ ಪ್ರಾಧಾನ್ಯತೆ>
17-596

<ಛಂದಸ್ಸುಗಳು–(ನಾನಾವಿಧ) ಅವುಗಳ ಹೆಸರುಗಳು, ಅಕ್ಷರ ಸಂಖ್ಯಾ ಇತ್ಯಾದಿ>
29-901

<ಛಂದಸ್ಸುಗಳ ದೇವತೆಗಳು (ಬೃ.ದೇ.)>
23-1022

<ಛಾಯಾ ಶಬ್ದದ ನಾನಾರ್ಥಗಳು, ವಿವರಣೆ ಇತ್ಯಾದಿ>
19-469

[ಜ]
<ಜಗತೀ ಛಂದಸ್ಸಿನ ಮಹತ್ವ>
12-459

<ಜಗತ್ತಿನ ಮೂಲತತ್ತ್ವ ಯಾವುದು?>
29-823

<ಜಗದುತ್ಪತ್ತಿಯು ಹೇಗಾಯಿತು?>
30-767

<ಜನಯಃ>
6-286

<ಜನಂಸಹಃ ಎಂಬ ಶಬ್ದದ ವಿವರಣೆ>
14-881

<ಜಮದಗ್ನಿ ಋಷಿಯ ವೃತ್ತಾಂತ>
17-218
26-854

<ಜಮದಗ್ನಿ ಋಷಿಯ ಪರಿಚಯ>
17-655

<ಜಯತೀರ್ಥರ ಟೀಕೆ>
1-242

<ಜರಯಂತೀ ಶಬ್ದದ ಅರ್ಥವಿವರಣೆ>
5-46

<ಜಲ್ಹವಃ ಎಂಬ ಶಬ್ದದ ಅರ್ಥವಿವರಣೆ ಮತ್ತು ನಿರ್ವಚನ>
24-700

<ಜಹ್ನು ಮಹರ್ಷಿಯ ಪುತ್ರಿಯಾದ ಜಾಹ್ನವೀ ಎಂಬುವಳಿಗೆ ಅಶ್ವಿನೀ ದೇವತೆಗಳು ಅನುಗ್ರಹಮಾಡಿದ ವಿಚಾರ>
9-280

<ಜ್ಮಯಾ ಅತ್ರ ಎಂಬ ಋಕ್ಕಿನ ನಿರುಕ್ತ>
22-221

<ಜಾತವೇದಾಃ ಎಂಬ (ಅಗ್ನಿಯ) ಸ್ವರೂಪ>
6-505 
8-160 
10-263 
16-315
28-719

<ಜಾತವೇದಶ್ಶಬ್ದದ ನಿರ್ವಚನ>
4-496 
7-512 
14-429 
15-443 
16-158
19-12

<ಜಾತವೇದಸೇ ಸುನವಾಮ ಎಂಬ ಋಕ್ಕಿನ ನಿರುಕ್ತ>
30-1260

<ಜಾತವೇದಾಃ ವೈಶ್ವಾನರ, ದ್ರವಿಣೋದಾಃ ಇತ್ಯಾದಿ ನಾನಾವಿಧ ಅಗ್ನಿಗಳು>
28-64

<ಜಾತೂಕರ್ಣ್ಯಶಾಖೆ>
1-64

<ಜಾತೂಭರ್ಮಾ>
8-370

<ಜಾತೂಸ್ಥಿರನ ವಿಷಯ>
14-712

<ಜಾನಃ ಎಂಬ ಋಷಿ>
18-730

<ಜಾಮಭಿಃ>
6-316

<ಜಾಮಿ ಶಬ್ದದ ರೂಪನಿಷ್ಪತ್ತಿ>
6-461 
8-221 
10-92 
10-156 
12-52
13-64

<ಜಾಹುಷನೆಂಬ ರಾಜನ ವಿಚಾರ>
.9-284

<ಜ್ಞಾಸ ಮತ್ತು ಸಜಾತರಿಗೆ ಇರುವ ವ್ಯತ್ಯಾಸ>
8-624

<ಜ್ಯಾ (ಹೆದೆ) ಶಬ್ದದ ವಿವರಣೆ>
21-562

<ಜಿನ್ವನ್‍>
6-286

<ಜೀರಾಶ್ವ ಶಬ್ದದ ಅರ್ಥ ವಿವರಣೆ ಇತ್ಯಾದಿ>
11-213

<ಜೀರದಾನು ಶಬ್ದದ ರೂಪನಿಷ್ಪತ್ತಿ ಮತ್ತು ವಿವರಣೆ>
13-309

<ಜೀವಧನ್ಯಾಃ>
6-577

<ಜೀವನಿಗೆ ಶರೀರವು ಪ್ರಾಕೃತವಾದದ್ದೆಂದೂ ಜೀವನು ನಿತ್ಯನಾಗಿರುನೆಂದೂ ಪ್ರತಿಪಾದನೆ ಮಾಡಿರುವ ವಿಷಯ>
12-492

<ಜೀವಶಂಸೇ>
8-423

<ಜುಗುರ್ಯಾತ್‍ ಎಂಬ ಶಬ್ದದ ಅರ್ಥವಿವರಣೆ>
13-255

<ಜುಹೂ ದೇವಿಯ ವೃತ್ತಾಂತ>
30-310

<ಜೂತಿ ಶಬ್ದವಿವರಣೆ>
15-657

<ಜೇನುನೊಣಗಳಿಗೆ ಅಶ್ವಿನೀ ದೇವತೆಗಳು ಮಧುವು ಸಿಕ್ಕುವಂತೆ ಮಾಡಿದ ವಿಚಾರ>
9-488

<ಜೇನ್ಯಃ ಎಂಬ ಶಬ್ದದ ವಿವರಣೆ>
6-300

<ಜೋಷವಾಕಂ ಎಂಬ ಶಬ್ದದ ನಿರ್ವಚನ, ವಿವರಣೆ ಇತ್ಯಾದಿ>
21-383

<ಜೋಹೂತ್ರಃ ಎಂಬ ಶಬ್ದದ ವಿವರಣೆ>
14-563

<ಜ್ಯೋಕ್ಚ ಸೂರ್ಯಂ ದೃಶೇ ಎಂಬ ವಾಕ್ಯದ ವಿವರಣೆ>
3-224

<ಜ್ಯೋತಿಶ್ಶಾಸ್ತ್ರ>
1-31

<ಜ್ಯೋತಿಶ್ಶಾಸ್ತ್ರದ ಪ್ರಯೋಜನ ಇತ್ಯಾದಿ>
1-583

[ತ]
<ತತ್‍ ಎಂಬ (ಗಾಯತ್ರಿಯಲ್ಲಿರುವ) ಶಬ್ದದ ವಿವರಣೆ>
17-627

<ತತ್ಸವಿತುಃ ಎಂಬ ಋಕ್ಕಿಗೆ ಸಾಯಣರ ತಾತ್ಪರ್ಯ ವಿವರಣೆ>
17-557

<ತನಯ ಶಬ್ದ ವಿವರಣೆ>
14-551

<ತನೂಕೃಧೇ ಎಂಬ ಶಬ್ದದ ವಿವರಣೆ>
25-267

<ತನೂತ್ಯಜೇವ ಎಂಬ ಋಕ್ಕಿನ ನಿರುಕ್ತ>
27-758

<ತನೂನಪಾತ್‍ ಎಂಬ ಶಬ್ದದ ರೂಪನಿಷ್ಪತ್ತಿ (ಬೃ.ದೇ.)>
28-717

<ತನೂನಪಾತ್‍ ಎಂಬ ದೇವತೆಯ ವಿಷಯ>
15-680
16-610 
30-338
28-756
28-764

<ತನೂನಪಾತ್ಪಥ ಎಂಬ ಋಕ್ಕಿನ ನಿರುಕ್ತ>
30-339

<ತಮಿದ್ವರ್ಧಂತು ನೋ ಗಿರಃ ಇತ್ಯಾದಿ ವಾಕ್ಯ ವಿವರಣೆ>
23-690

<ತಮೂಷು ಸಮನಾ ಎಂಬ ಋಕ್ಕಿನ ನಿರುಕ್ತ>
24-377

<ತರಣಿಃ ಎಂಬ ಶಬ್ದದ ವಿವರಣೆ>
5-118

<ತರಂತರಾಜನ ಮಹಿಷಿಯಾದ ಶಶೀಯಸಿಯ ದಾನಸ್ತುತಿ>
19-803

<ತಸ್ಮಾದಶ್ವಾ ಅಜಾಯಂತ ಎಂಬ ಋಕ್ಕಿನ ವಿವರಣೆ>
29-904

<ತಸ್ಮಾದ್ಯಜ್ಞಾತ್ಸರ್ವಹುತಃ ಎಂಬ ಋಕ್ಕಿನ ವಿವರಣೆ>
29-897

<ತಸ್ಮಾದ್ವಿರಾಳಜಾಯತ ಎಂಬ ಋಕ್ಕಿನ ವಿವರಣೆ>
29-809

<ತಕ್ಷ್ಯ ಶಬ್ದಾರ್ಥ ವಿವರಣೆ>
13-272

<ತಂತುಂ ತನ್ವನ್‍ ಎಂಬ ಋಕ್ಕಿನ ವಿಶೇಷ ವಿನಿಯೋಗ>
28-315

<ತಂ ಪ್ರತ್ನಥಾ ಪೂರ್ವಥಾ ಎಂಬ ಋಕ್ಕಿಗೆ ಯಾಸ್ಕರ ನಿರ್ವಚನ>
19-458

<ತಂ ಯಜ್ಞಂ ಬರ್ಹಿಷಿ ಎಂಬ ಋಕ್ಕಿನ ವಿವರಣೆ>
29-859

<ತಾಂಡ್ಯ ಬ್ರಾಹ್ಮಣ>
1-155

<ತಾರ್ಕ್ಷ್ಯ (ಬೃ.ದೇ)>
5-618
28-725

<ತಾರ್ಕ್ಷ್ಯ ಶಬ್ದ ವಿವರಣೆ>
7-286

<ತಿಲ್ವಲೇ ಎಂಬ ಶಬ್ದದ ಅರ್ಥವಿವರಣೆ>
19-848

<ತಿಸ್ರೋ ದೇವೀಃ ಎಂಬ ಶಬ್ದಗಳ ವಿವರಣೆ>
2-563

<ತಿಸ್ರೋ ದೇವೀಃ ಇಳಾ, ಸರಸ್ವತೀ, ಭಾರತೀ ಎಂಬ ದೇವತೆಗಳ ವಿಷಯ>
11-263 
14-254
15-727 
28-633
28-758
30-359

<ತಿಸ್ರೋ ವಾಚಃ ಎಂಬ ಋಕ್ಕಿಗೆ ಅಧಿದೈವತ ಮತ್ತು ಅಧ್ಯಾತ್ಮ ಪಕ್ಷಗಳಲ್ಲಿ ಅರ್ಥವಿವರಣೆ>
26-812

<ತೀವ್ರ ಎಂಬ ಶಬ್ದದ ಅರ್ಥವಿವರಣೆ>
25-221

<ತುಗ್ರನೆಂಬ ರಾಜನ ವಿಷಯ>
9-193

<ತುಗ್ರಪುತ್ರನ (ಭುಜ್ಯು) ವಿಚಾರ>
12-18

<ತುಗ್ರನೆಂಬ ಅಸುರನನ್ನು ಇಂದ್ರನು ಸೋಲಿಸಿದ ವಿಷಯ>
20-730

<ತುರ ಶಬ್ದಾರ್ಥ ವಿವರಣೆ>
13-196

<ತುವಿಜಾತ ಶಬ್ದದ ಅರ್ಥಾನುವಾದ ವಿವರಣೆ>
11-17 
22-163

<ತುವಿಕ್ಷಂ ತೇ ಎಂಬ ಋಕ್ಕಿನ ನಿರುಕ್ತ>
25-177

<ತುರ್ವಶ ಎಂಬುವನ ವಿಷಯ>
4-240 
5-322 
8-604
18-241

<ತುರ್ವಶ, ಯದು ಎಂಬುವರ ವಿಷಯ>
20-735

<ತುರ್ವಣಿ ಶಬ್ದಾರ್ಥ>
13-115

<ತುರ್ವೀತಿಯ ವಿಷಯ>
4-243 
5-322 
8-82 
14-814
18-19

<ತೂರ್ಣಾ ಶಬ್ದದ ನಿರ್ವಚನ ಇತ್ಯಾದಿ>
24-230

<ತೂತುಜ>
20-730

<ತೂರ್ವಯಾಣ ಎಂಬ ರಾಜನ ವಿಷಯ>
5-299

<ತೃಣಸ್ಕಂದಸ್ಯ ಎಂಬ ಶಬ್ದದ ಅರ್ಥವಿಚಾರ>
13-234

<ತೃತೀಯ ಸವನ>
26-48

<ತೃತೀಯ ಸವನದ ವಿವರಣೆ>
26-62

<ತೃತೀಯ ಸವನದ ಮಂತ್ರಗಳು>
15-695

<ತೃತೀಯ ಸವನಕಾಲದಲ್ಲಿ ಪಠಿಸಬೇಕಾದ ಉನ್ನೀಯಮಾನ ಸೂಕ್ತವು>
15-701
26-54

<ತೃತೀಯ ಸವನದ ಮಂತ್ರಗಳು ವಿವರಣೆ ಸಹಿತವಾಗಿ>
15-709
27-170

<ತೃತೀಯ ಸವನದಲ್ಲಿ ಸೋಮಾಭಿಷವಣದ ವಿಷಯವಾಗಿ ಕೆಲವು ವಿಶೇಷ ಸಂಗತಿಗಳು>
15-711

<ತೃತೀಯೋ ಅಗ್ನಿಷ್ಟೇ ಪತಿಃ ಎಂಬ ವಾಕ್ಯದ ನಿರ್ವಚನ>
29-429

<ತೇ ಆಚರಂತೀ ಎಂಬ ಋಕ್ಕಿನ ನಿರ್ವಚನ>
21-565

<ತೈತ್ತಿರೀಯ ಶಾಖಾ>
1-136

<ತೈತ್ತಿರೀಯ ಸಂಹಿತೆಯ ಪದಪಾಠಕಾರರು>
1-289

<ತೋಕ ಶಬ್ದದ ವಿವರಣೆ>
14-551

<ತೋಕ, ತುಕ್‍ ಎಂಬ ಶಬ್ದಗಳ ಅರ್ಥವ್ಯತ್ಯಾಸ>
17-674

<ತೋಕ ಮತ್ತು ತನಯ ಶಬ್ದಗಳಿಗೆ ಇರುವ ವ್ಯತ್ಯಾಸ>
15-149

<ತೋದ ಶಬ್ದದ ಅರ್ಥಾನುವಾದ>
11-466

<ತ್ಮನಾ ಎಂಬ ಶಬ್ದದ ವೈಶಿಷ್ಟ್ಯ>
17-785

<ತ್ಯಮೂಷು ವಾಜಿನಂ ಎಂಬ ಋಕ್ಕಿನ ನಿರುಕ್ತ>
30-1214

<ತ್ಯಂ ಸ್ಯಃ ಸ್ಮ ಎಂಬ ಶಬ್ದಗಳ ಅರ್ಥವಿವರಣೆ>
19-715

<ತ್ರಯಃ ಎಂಬ ಶಬ್ದದ ರೂಪನಿಷ್ಪತ್ತಿ>
28-154

<ತ್ರಯಃ ಕೇಶಿನಃ ಎಂಬ ಶಬ್ದಗಳ ವಿವರಣೆ>
12-609

<ತ್ರಯಸ್ತ್ರಿಂಶಸ್ತೋಮ ವಿವರಣೆ>
2-147

<ತ್ರಸದಸ್ಯುವಿನ ವಿಷಯ>
8-789

<ತ್ರಸದಸ್ಯು ಎಂಬ ರಾಜನ ಮತ್ತು ಪುರುಕುತ್ಸನ ವಿಷಯ>
18-368

<ತ್ರಸದಸ್ಯುವಿನ ಜನ್ಮವೃತ್ತಾಂತ>
18-430

<ತ್ರಸದಸ್ಯು ರಾಜನ ಆತ್ಮಪ್ರಶಂಸೆ ಇತ್ಯಾದಿ>
18-421

<ತ್ರಸದಸ್ಯು ರಾಜನ ದಾನಪ್ರಶಂಸೆ (ಬೃ.ದೇ.)>
23-912

<ತ್ರ್ಯರುಣ ಮತ್ತು ವೃಶಜಾನರ ವೃತ್ತಾಂತ (ಬೃ.ದೇ.)>
28-852

<ತ್ರ್ಯಂಬಕಂ ಯಜಾಮಹೇ ಎಂಬ ಋಕ್ಕಿನ ವಿಶೇಷಾರ್ಥ, ವಿನಿಯೋಗ ಜಪಕ್ರಮ ಇತ್ಯಾದಿ>
22-395

<ತ್ವಯಾ ಮನ್ಯೋ ಎಂಬ ಋಕ್ಕಿನ ನಿರ್ವಚನ>
29-331

<ತ್ವಷ್ಟಾ ಎಂಬ ದೇವತೆಯ ಸೂಕ್ಷ್ಮ ಪರಿಚಯ>
2-571 
5-639 
15-735
30-361

<ತ್ವಷ್ಟಾ ದುಹಿತ್ರೇ ಎಂಬ ಸೂಕ್ತದ ಪೀಠಿಕೆ>
27-531

<ತ್ವಷ್ಟಾ ದುಹಿತ್ರೇ ಎಂಬ ಋಕ್ಕಿನ ನಿರುಕ್ತ>
27-547

<ತ್ವಷ್ಟೃ ಶಬ್ದದ ವಿವರಣೆ>
14-261

<ತ್ವಷ್ಟೃವಿನ ಸ್ವರೂಪ>
11-269 
14-149 
28-637
28-760

<ತ್ವಷ್ಟೃವು ಇಂದ್ರನಿಗೆ ವಜ್ರಾಯುಧವನ್ನು ಮಾಡಿಕೊಟ್ಟ ವಿಚಾರ>
9-589

<ತ್ವಷ್ಟೃ ಪುತ್ರನಾದ ವಿಶ್ವರೂಪನನ್ನು ಇಂದ್ರನು ಸಂಹಾರ ಮಾಡಿದ ವಿಚಾರ>
14-626

<ತ್ರಿಕದ್ರುಕೇಷು ಎಂಬ ಶಬ್ದದ ಅರ್ಥವಿವರಣೆ ಇತ್ಯಾದಿ>
3-597
14-622

<ತ್ರಿಕಶಃ ಎಂಬ ಶಬ್ದದ ವಿವರಣೆ>
14-820

<ತ್ರಿಚಕ್ರದ ವಿವರಣೆ>
20-104

<ತ್ರಿಚಕ್ರ ತ್ರಿಧಾತು ಇತ್ಯಾದಿ ಶಬ್ದಗಳ ವಿವರಣೆ>
13-628

<ತ್ರಿಣವಸ್ತೋಮದ ವಿವರಣೆ>
2-146

<ತ್ರಿತ ಮತ್ತು ತ್ರೈತನ ವೃತ್ತಾಂತ>
5-230

<ತ್ರಿತನ ವಿಷಯ>
8-443
14-625

<ತ್ರಿತನ ವಿಷಯದಲ್ಲಿ Wilson ಎಂಬ ಪಾಶ್ಚಾತ್ಯ ಪಂಡಿತನ ಅಭಿಪ್ರಾಯ>
8-445

<ತ್ರಿತನ ಸ್ತುತಿ ಮಹಿಮೆ>
8-473

<ತ್ರಿತ ಆಪ್ತ್ಯಃ>
5-582 
27-188 
28-122
28-795

<ತ್ರಿಧಾತು ಶಬ್ದದ ಅರ್ಥ ವಿವರಣೆ>
4-99
11-583
2-107

<ತ್ರಿಧಾತು, ತ್ರಿಚಕ್ರ>
13-628

<ತ್ರಿನಾಭಿ ಶಬ್ದದ ವಿವರಣೆ>
12-305

<ತ್ರಿಪಾಚ್ಛಬ್ದದ ವಿವರಣೆ>
29-779

<ತ್ರಿಪಾದೂರ್ಧ್ವ ಉದೈತ್ಪುರುಷಃ ಎಂಬ ಋಕ್ಕಿನ ವಿವರಣೆ>
29-778

<ತ್ರಿಪುರ ಸಂಹಾರ (ರುದ್ರನು ಮಾಡಿದ) ವಿಷಯ–ಐತರೇಯ ಬ್ರಾಹ್ಮಣ ಮತ್ತು ತೈತ್ತಿರೀಯ ಸಂಹಿತೆಯಲ್ಲಿರುವಂತೆ>
20-625

<ತ್ರಿಮಾತಾ ಎಂಬ ಶಬ್ದದ ವಿವರಣೆ>
17-360

<ತ್ರಿಲೋಕಗಳ ಒಳಪ್ರಭೇದಗಳು>
17-374

<ತ್ರಿವೃತ್‍ಸ್ತೋಮದ ವಿವರಣೆ>
2-140

<ತ್ರಿವೃತಂ ಎಂಬ ಶಬ್ದ ವಿವರಣೆ>
30-678

<ತ್ರಿಶಿರಾಃ (ವಿಶ್ವರೂಪ) ಎಂಬುವನನ್ನು ಇಂದ್ರನು ಸಂಹಾರ ಮಾಡಿದ ವಿಷಯ (ಬೃ.ದೇ.)>
28-941 
27-300

<ತ್ರಿಶಿರಾಃ (ವಿಶ್ವರೂಪ) ಎಂಬುವನನ್ನು ಇಂದ್ರನು ಸಂಹಾರ ಮಾಡಿದ ವಿಷಯ (ತೈತ್ತಿರೀಯ ಸಂಹಿತೆಯಲ್ಲಿರುವಂತೆ)>
27-303

<ತ್ರಿಶೋಕ>
8-781

<ತ್ರಿಷಧಸ್ಥ ಶಬ್ದದ ಅರ್ಥವಿವರಣೆ>
17-359 
18-799

<ತ್ರಿಷ್ಟುಪ್‍ಛಂದಸ್ಸಿನ ಮಹತ್ತ್ವ>
12-452

<ತ್ರಿಸ್ಥಾನ ದೇವತೆಗಳು (ಬೃ.ದೇ.)>
28-678

<ತ್ರಿಸ್ಥಾನ ದೇವತೆಗಳು–ಆತ್ಮ ಮತ್ತು ವಾಕ್ಕಿನ ಮೂರು ರೂಪಗಳು (ಬೃ.ದೇ.)>
28-695

[ದ]
<ದತ್ತು ಪುತ್ರ (ಅಥವಾ) ಪುತ್ರಿಯ ವಿಷಯ (ಬೃ.ದೇ.)>
28-837

<ದಧ್ಯಂಚ>
5-671

<ದಧಿಕ್ರಾ ಎಂಬ ದೇವತೆಯ ವಿಷಯ>
5-680 
16-402 
16-415 
18-364
22-256
28-725

<ದಧೀಚಿಋಷಿಯ ವೃತ್ತಾಂತ>
6-615 
7-115 
9-245 
11-81 
13-601
20-110
20-597
28-760

<ದಧೀಚಿಋಷಿಯು ಅಶ್ವಿನೀದೇವತೆಗಳಿಗೆ ಮಧುವಿದ್ಯೆಯನ್ನುಪದೇಶಿಸಿದ ವಿಚಾರ>
9-401 
9-245

<ದಧೀಚಿಋಷಿಯ ವಿಷಯದಲ್ಲಿ ಸ್ಕಂದಸ್ವಾಮಿಯು ಹೇಳಿರುವ ಇತಿಹಾಸ>
9-247

<ದಧೀಚಿಋಷಿಗೆ ಅಶ್ವಶಿರಸ್ಸು ಹೇಗೆ ಬಂತೆಂಬ ವಿಚಾರ>
7-118 
9-247
28-761

<ದಧೀಚಋಷಿಯ ವಿಷಯದಲ್ಲಿ Wilson ಪಂಡಿತನ ಅಭಿಪ್ರಾಯ>
9-248

<ದಧ್ಯಙ್‍ ಅಥವಾ ದಧೀಚ ಋಷಿಯ ವೃತ್ತಾಂತ>
20-597

<ದಧ್ಯಙ್‍ ಹಮೇ ಎಂಬ ಋಕ್ಕಿನ ವಿಶೇಷವಿನಿಯೋಗ>
11-81

<ದಭೀತಿಯ ವಿಷಯ>
8-821 
14-707 
18-247 
20-738

<ದಮೂನಾಃ ಎಂಬ ಶಬ್ದದ ನಿರ್ವಚನ ಇತ್ಯಾದಿ>
15-538 
18-7

<ದರ್ಭೆಗಳು (ನಾನಾವಿಧ)>
14-340

<ದಶಗ್ವಾಃ>
5-675

<ದಶಗಶಬ್ದದ ವಿವರಣೆ>
6-24 
15-251 
18-534

<ದಶದ್ಯು ಮತ್ತು ಶ್ಚೈತ್ರೇಯ ಎಂಬ ಋಷಿಗಳ ವಿಷಯ>
4-64

<ದಶಯಜ್ಞಾಯುಧಗಳು>
3-414

<ದಶಾರಿತ್ರಃ ಎಂಬ ಶಬ್ದದ ವಿವರಣೆ>
14-840

<ದಶಾಂಗುಲಂ, ಅತ್ಯತಿಷ್ಠತ್‍ ಎಂಬ ಶಬ್ದಗಳ ವಿವರಣೆ>
29-720

<ದಶೋಣಿ>
20-730

<ದಂಸಯಃ ಎಂಬ ಶಬ್ದದ ನಿರ್ವಚನ>
30-891

<ದಸ್ಮಶಬ್ದಾರ್ಥ ವಿವರಣೆ>
11-32
13-264
28-199


<ದಸ್ಯು>
5-689
6-517 
13-374

<ದಸ್ಯುಗಳೆಂಬ ಕಪ್ಪುಜನರ ವಿಚಾರ>
14-656


<ದಸ್ರಶಬ್ದದ ವಿವರಣೆ>
7-462

<ದಹರಾಕಾಶ, ಹೃದಯಾಕಾಶ>
29-797

<ದಕ್ಷನ ವಿಷಯ>
29-38

<ದಕ್ಷಶಬ್ದವಿವರಣೆ>
17-77

<ದಕ್ಷಸ್ಯವಾದಿತೇ ಜನ್ಮನಿ ವ್ರತೇ ಎಂಬ ಋಕ್ಕಿನ ನಿರುಕ್ತ>
28-491

<ದಕ್ಷಿಣಾ ಎಂಬ ದೇವತೆಯ ವಿಷಯ, ದಕ್ಷಿಣಾಶಬ್ದದ ರೂಪನಿಷ್ಪತ್ತಿ>
2-768 
10-75 
10-213 
13-160
14-639

<ದಕ್ಷಿಣಾವಾಟ್‍ ಎಂಬ ಶಬ್ದದ ವಿವರಣೆ>
15-775


<ದ್ರಪ್ಸಃ ಎಂಬ ಶಬ್ದದ ನಾನಾರ್ಥಗಳು>
7-543 
27-565

<ದ್ರಪ್ಸಶ್ಚಸ್ಕಂದ ಎಂಬ ಋಕ್ಕಿಗೆ ಆದಿತ್ಯಪರವದ ಅರ್ಥ>
27-565

<ದ್ರವಿಣೋದಾಃ ಎಂಬ ದೇವತೆಯ ವಿಷಯ>
2-652
14-407
15-315 
27-221
28-717 
28-772

<ದ್ರವಿಣೋದಾಃ ಎಂಬ ಶಬ್ದದ ವಿವರಣೆ>
8-64


<ದಾನವಶಬ್ದ ವಿವರಣೆ>
14-603

<ದಾನಗಳಲ್ಲಿ ಮೂರು ವಿಧ–ಸತ್ತ್ವ, ರಜಃ, ತಮಃ ಎಂದು>
30-548

<ದಾವನೇ ಎಂಬ ಶಬ್ದದ ಅರ್ಥವಿವರಣೆ, ನಿರ್ವಚನ>
19-311

<ದಾವಾಗ್ನಿಯು ವನವನ್ನು ದಹಿಸುವ ವರ್ಣನೆ>
14-483
30-929

<ದಾವಾನಲರೂಪದ ಅಗ್ನಿಯ ಸ್ವರೂಪ>
14-478

<ದಾಸ>
5-689

<ದಾಸ, ದಸ್ಯು ಎಂಬ ಶಬ್ದಗಳ ವಿವರಣೆ, ದಸ್ಯು ಜನರು ಯಾರು ಎಂಬ ವಿಚಾರ>
14-580
19-199

<ದಾಸಪತ್ನೀಃ ಎಂಬ ಶಬ್ದದ ವಿವರಣೆ>
16-218
19-186

<ದಾಸ ಮತ್ತು ಆರ್ಯ ಎಂಬ ಜನರು ಯಾರು?>
29-912

<ದಾಸಪತ್ನೀಃ ಎಂಬ ಋಕ್ಕಿನ ನಿರುಕ್ತ>
3-629

<ದ್ಯಾವಾಪೃಥಿವಿಗಳ ವಿವರಣೆ>
3-104 
12-37
18-606

<ದ್ಯಾವಾಪೃಥಿವಿಗಳ ಉತ್ಪತ್ತಿ ವಿಚಾರ>
6-474

<ದ್ಯಾವಾಪೃಥಿವಿಯ ಉತ್ಪತ್ತಿಕ್ರಮವನ್ನು ವಿವರಿಸುವ ಪ್ರಕರಣಗಳು>
18-609

<ದ್ಯಾವಾಪೃಥಿವಿಗಳ ಉತ್ಪತ್ತಿ ವಿಷಯದಲ್ಲಿ ಕೆಲವು ಕಲ್ಪನೆಗಳು>
18-611

<ದ್ಯಾವಾಪೃಥಿವಿಗಳು ವಿಶ್ವಕ್ಕೆಲ್ಲಾ ಮಾತಾಪಿತೃಗಳು ಎಂದು ವರ್ಣಿಸಿರುವ ಶ್ರುತಿವಾಕ್ಯಗಳು>
18-607

<ದ್ಯಾವಾಪೃಥಿವಿಗಳು ಅಗ್ನಿಯ ಮಾತಾಪಿತೃಗಳೆಂಬ ವಿಚಾರ>
16-6

<ದ್ಯಾವಾಪೃಥಿವಿಗಳು ಅಗ್ನಿಯನ್ನು ಕಂಡು ಹೆದರಿದ ವಿಚಾರ>
10-559

<ದ್ಯಾವಾಪೃಥಿವಿಗಳು ಇಂದ್ರನ ಭಯದಿಂದ ನಡುಗಿದವು>
5-247

<ದ್ವಾರೋ ದೇವೀಃ ಎಂಬ ದೇವತೆಗಳ ವಿಷಯ>
11-249
30-348

<ದಿತಿಯ ವಿಷಯ>
5-648

<ದಿವಿಷ್ಟಿಷು ಎಂಬ ಶಬ್ದದ ಅರ್ಥವಿವರಣೆ>
5-61

<ದಿವೋದಾಸನ ವೃತ್ತಾಂತ>
9-276
20-570

<ದಿವೋದಾಸನನ್ನು ಅಶ್ವಿನೀದೇವತೆಗಳು ರಕ್ಷಿಸಿದ ವಿಚಾರ>
9-467

<ದ್ವಿಜನಾದವನು ತೀರಿಸಬೇಕಾದ ಋಣತ್ರಯಗಳ ವಿಷಯ>
24-242

<ದ್ವಿಜನ್ಮ ಶಬ್ದಾರ್ಥ ವಿವರಣೆ>
11-105
18-696

<ದ್ವಿತ>
8-443

<ದ್ವಿತನ ವೃತ್ತಾಂತ>
5-230

<ದ್ವಿಬರ್ಹಾ>
6-312 
9-143
20-701

<ದ್ವಿಬರ್ಹಸಃ ಎಂಬ ಶಬ್ದದ ಅರ್ಥವಿವರಣೆ>
13-416

<ದ್ವಿಮಾತಾಶಬ್ದದ ಅರ್ಥವಿವರಣೆ>
8-711
17-289

<ದೀರ್ಘಜಿಹ್ವೀ ಎಂಬ ರಾಕ್ಷಸಿಯ ವೃತ್ತಾಂತ>
8-113

<ದೀರ್ಘತಮಾಃ ಎಂಬ ಋಷಿಯ ಸೂಕ್ಷ್ಮ ಪರಿಚಯ>
12-4

<ದೀರ್ಘತಮಾಃ ಎಂಬ ಋಷಿಯ ಜನ್ಮವೃತ್ತಾಂತ>
11-43
28-807
28-809

<ದೀರ್ಘತಮಾಃ (ಮಮತಾ ಎಂಬ ಸ್ತ್ರೀಯಪುತ್ರನಾದ) ಎಂಬ ಋಷಿಯ ಅಂಧತ್ವವನ್ನು ಅಗ್ನಿಯು ಪರಿಹಾರ ಮಾಡಿದ ವಿಚಾರ>
17-782

<ದುಂದುಭಿ ಶಬ್ದಾರ್ಥ ವಿವರಣೆ>
21-223

<ದುರಃ (ದ್ವಾರೋ ದೇವೀಃ) ಎಂಬ ದೇವತೆ>
15-689

<ದುರೋಕಶೋಚಿಃ ಎಂಬ ಶಬ್ದದ ವಿವರಣೆ>
6-191

<ದುರ್ಣಾಮಾ ಎಂಬ ಶಬ್ದದ ವಿವರಣೆ>
30-1096

<ದುವಸ್ಯತಿ ಶಬ್ದ>
6-512 
12-512
15-623

<ದ್ಯುಭಕ್ತಾಃ>
6-408

<ದ್ಯುಮ್ನಂ>
6-505

<ದ್ಯುಮ್ನಂ ಸಜೋಷಸಃ ಇತ್ಯಾದಿ ಶಬ್ದಗಳ ವಿವರಣೆ>
10-471

<ದ್ಯುಲೋಕ, ಅಂತರಿಕ್ಷ, ಪೃಥಿವೀಲೋಕಗಳ ಮೂರು ಅವಾಂತರ ಪ್ರಭೇದಗಳು>
20-187

<ದ್ಯುಲೋಕಸ್ಥನಾದ ಅಗ್ನಿಯ ಉತ್ಪತ್ತಿ>
28-57

<ದ್ಯುಶಬ್ದದ ನಾನಾರ್ಥಗಳು>
18-69

<ದ್ಯುಸ್ಥಾನ ದೇವತೆಗಳಲ್ಲಿ (ಅಶ್ವಿನೀದೇವತೆಗಳದೇ) ಪ್ರಥಮ ಸ್ಥಾನವು>
9-315

<ದ್ರುಘಣದ (ಮುದ್ಗಲ ಋಷಿಯ) ವಿಷಯ>
30-171

<ದ್ರುಹ್ಯು>
8-604

<ದೃಳ್ಹಾ ಎಂಬ ಶಬ್ದ>
6-290

<ದೇವತಾಗಣಗಳು>
5-657

<ದೇವತಾದ್ವಂದ್ವಗಳು>
5-651

<ದೇವತೆಗಳ ನಾನಾವಿಧ ವಾಹನಗಳು>
14-156
19-180

<ದೇವತೆಗಳು ಪ್ರಜಾಪತಿಯ ಪುತ್ರಿಯಾದ ಸೂರ್ಯಾದೇವಿಯನ್ನು ಪಡೆಯಲು ನಡೆಸಿದ ಸ್ಪರ್ಧೆ>
11-333

<ದೇವತೆಗಳು ಪರೋಕ್ಷಪ್ರಿಯರೆಂಬ ವಿಷಯ>
15-779
17-357

<ದೇವತೆಗಳಿಗೆ ಮನುಷ್ಯರಿಗಿರುವಂತೆ ಹಸ್ತಾದ್ಯವಯವಗಳು-ಆಕಾರ ಮೊದಲಾದವು ಇರುವವೇ ಎಂಬ ವಿಚಾರದಲ್ಲಿ ವಿಷಯವಿಮರ್ಶೆ ಮತ್ತು ಯಾಸ್ಕರ ನಿರ್ವಚನ, ಅಭಿಪ್ರಾಯಗಳು>
21-191
30-535

<ದೇವತೆಗಳ ಋತ್ವಿಕ್ಕುಗಳು>
27-219

<ದೇವತೆಗಳು ಅಗ್ನಿಯನ್ನು ತಮ್ಮ ದೂತನನ್ನಾಗಿ ಮಾಡಿಕೊಂಡ ವಿಚಾರ>
18-849

<ದೇವತೆಗಳ ವಿಷಯದಲ್ಲಿ ಸರಿಯಾದ ತಿಳಿವಳಿಕೆಯು ಅವಶ್ಯಕವು (ಬೃ.ದೇ.)>
28-1029

<ದೇವತೆಗಳ ಸಂಖ್ಯೆ ಎಷ್ಟು ಎಂಬ ವಿಷಯ ವಿಮರ್ಶೆ>
16-146 
28-299

<ದೇವತೆಗಳು ಸೋಮಪಾನ ಮಾಡುವಾಗ ವಾಯುವಿಗೆ ಪ್ರಥಮಸ್ಥಾನವು ಹೇಗೆ ಲಭಿಸಿತೆಂಬ ವಿಷಯದಲ್ಲಿ ಐತರೇಯ ಬ್ರಾಹ್ಮಣದಲ್ಲಿ ಹೇಳಿರುವ ಪೂರ್ವೇತಿಹಾಸವು>
19-568

<ದೇವ ಪತ್ನಿಯರ ವಿಷಯ–ಇಂದ್ರಾಣೀ, ವರುಣಾನೀ, ಅಗ್ನಾಯೀ>
3-100

<ದೇವಪ್ಸರಸ್ತಮಂ ಎಂಬ ಶಬ್ದದ ವಿವರಣೆ>
6-456

<ದೇವಯಾನ, ಪಿತೃಯಾಣವೆಂಬ ಪರಲೋಕದ ಮಾರ್ಗಗಳು>
8-512 
12-563 
13-652
14-45 
27-232 
17-427 
29-561

<ದೇವರ ಶಬ್ದ ನಿರ್ವಚನ>
27-974 
29-489

<ದೇವಲೋಕಕ್ಕೂ ಮನುಷ್ಯ ಲೋಕಕ್ಕೂ ಇರುವ ಸಂಬಂಧ>
27-665

<ದೇವಶುನಿ ಸರಮಾ (ಬೃ.ದೇ.)>
28-999

<ದೇವಸುವ ಎಂಬ ಎಂಟು ಹವಿಸ್ಸುಗಳ ವಿಷಯ (ಶತಪಥ ಬ್ರಾಹ್ಮಣದಲ್ಲಿರುವಂತೆ)>
17-446

<ದೇವಸುವ ಎಂಬ ಎಂಟು ಹವಿಸ್ಸುಗಳ ವಿಷಯ (ತೈತ್ರಿರೀಯ ಸಂಹಿತೆಯಲ್ಲಿರುವಂತೆ)>
17-449 
28-219

<ದೇವಸ್ತ್ವಷ್ಟಾ ಎಂಬ ಋಕ್ಕಿಗೆ ನಿರುಕ್ತ ಮತ್ತು ಅರ್ಥವಿವರಣೆ>
17-314

<ದೇವಸ್ಯ (ಗಾಯತ್ರಿಯಲ್ಲಿರುವ) ಶಬ್ದದ ವಿವರಣೆ>
17-632

<ದೇವಾನಾಂ ಪತ್ನೀಃ ಎಂಬ ಋಕ್ಕಿನ ನಿರುಕ್ತ>
19-522

<ದೇವಾಪಿ, ಶಂತನು ಎಂಬ ರಾಜರ ವಿಷಯ>
14-301
28-991 
28-993 
30-97

<ದೇವಾನಾಂ ಮಾನೇ ಎಂಬ ಋಕ್ಕಿನ ನಿರುಕ್ತ>
27-751

<ದೇವೀಂ ವಾಚಮಜನಯಂತ ಎಂಬ ಋಕ್ಕಿನ ನಿರುಕ್ತ>
25-513

<ದೇವೀರ್ದ್ವಾರಃ (ದುರಃ) ಎಂಬ ಶಬ್ದಗಳ ವಿವರಣೆ>
14-239 
28-622
28-757

<ದೇವ್ಯೌ ಹೋತಾರೌ ಎಂಬ ದೇವತೆಗಳ ವಿಷಯ>
28-629
28-759 
29-568

<ದೇಹವು ನಶ್ಯ, ಆತ್ಮನು ಅವಿನಶ್ಯ ಇತ್ಯಾದಿ ವಿಚಾರ>
12-565

<ದೈವ್ಯಾ ಹೋತಾರಾ ಎಂಬ ಶಬ್ದಗಳ ಮತ್ತು ದೇವತೆಗಳ ವಿವರಣೆ>
2-561 
11-259 
14-250
15-723
30-357

<ದೈವ್ಯಾ ಹೋತಾರಾ ಪ್ರಥಮಾ ಎಂಬ ಋಕ್ಕಿನ ನಿರುಕ್ತ>
30-357

<ದೋಧತಃ ಎಂಬ ಶಬ್ದದ ವಿವರಣೆ>
6-579

<ದೌಃ>
5-531

<ದ್ವೌ ಎಂಬ ಶಬ್ದದ ರೂಪನಿಷ್ಪತ್ತಿ>
28-154

[ಧ]
<ಧನುಃ ಶಬ್ದ ವಿವರಣೆ ಮತ್ತು ನಿರ್ವಚನ>
21-560

<ಧರುಣ ಶಬ್ದಾರ್ಥ>
15-618

<ಧರ್ಣಸಿ ಎಂಬ ಶಬ್ದದ ವಿವರಣೆ>
11-209

<ಧರ್ಮಸೂತ್ರಗಳು>
1-39

<ಧ್ವಸಂತಿ>
8-821

<ಧಾತೃ (ಬೃ.ದೇ.)>
28-725

<ಧಾನಾವಂತಂ ಎಂಬ ಶಬ್ದದ ವಿವರಣೆ>
17-157

<ಧಾಮಚ್ಛತ್‍, ರಿಕ್ತ, ವಜ್ರ ಎಂಬ ವಷಟ್ಕಾರದ ಮೂರು ಪ್ರಭೇದಗಳು>
23-206

<ಧಾಸೇಃ ಎಂಬ ಶಬ್ದದ ವಿವರಣೆ>
17-381

<ಧಿಷಣಾ ಮತ್ತು ಧಿಷ್ಣಶಬ್ದಗಳ ವಿವರಣೆ>
15-323

<ಧಿಷಣಾ ಎಂಬ ಶಬ್ದದ ವಿವರಣೆ–ಉದಾಹರಣೆ ಸಹಿತ>
15-427

<ಧಿಷಣೇ ಎಂಬ ಶಬ್ದದ ಅರ್ಥ ವಿವರಣೆ>
17-364

<ಧಿಷ್ಣ್ಯಾ ಶಬ್ದದ ಅರ್ಥ ಮತ್ತು ಪ್ರಯೋಗ>
13-559
16-443

<ಧಿಯಃ (ಗಾಯತ್ರಿಯಲ್ಲಿರುವ) ಶಬ್ದದ ವಿವರಣೆ>
17-331

<ಧಿಯಾವಸು ಶಬ್ದಾರ್ಥ>
16-565

<ಧೀತಯಃ ಎಂಬ ಶಬ್ದದ ವಿವರಣೆ>
16-221

<ಧೀಮಹಿ (ಗಾಯತ್ರಿಯಲ್ಲಿರುವ) ಎಂಬ ಶಬ್ದದ ವಿವರಣೆ>
17-633

<ಧೀರ ಶಬ್ದದ ನಾನಾರ್ಥಗಳು>
11-397

<ಧೀಶಬ್ದಾರ್ಥ ವಿಚಾರ>
12-71

<ಧುನಿ>
5-695
10-550 
19-220
20-687

<ಧೂರ್ತಿಃ ಶಬ್ದಾರ್ಥ ವಿವರಣೆ>
10-357

<ಧೇನಾ ಶಬ್ದಾರ್ಥ ವಿವರಣೆ>
11-159

<ಧೇನು ಶಬ್ದ ವಿವರಣೆ>
11-538

<ಧೇನುವಿನ ಸ್ವರೂಪ>
12-480

<ಧೇನು ಶಬ್ದದ ರೂಪನಿಷ್ಪತ್ತಿ, ಅರ್ಥ ವಿವರಣೆ>
13-245
17-396

<ಧೇನುವಿನ ನಾಲ್ಕು ಸ್ತನಗಳು–ಸ್ವಾಹಾಕಾರ, ವಷಟ್ಕಾರ, ಹಂತಕಾರ, ಸ್ವಧಾಕಾರ ಅಥವಾ ಋಗ್ವೇದ, ಯಜುರ್ವೇದ, ಸಾಮವೇದ, ಅಥರ್ವವೇದ>
12-645

[ನ]
<ನ ಎಂಬ ಶಬ್ದದ ಅರ್ಥಗಳು ಮತ್ತು ಪ್ರಯೋಗ>
18-38 
19-770 
29-541

<ನಃ (ಗಾಯತ್ರಿಯಲ್ಲಿರುವ) ಎಂಬ ಶಬ್ದದ ವಿವರಣೆ>
17-634

<ನಕುಲ ಋಷಿಯ ಖಿಲ ಮಂತ್ರಗಳು (ಬೃ.ದೇ.)>
28-996

<ನಕ್ತೋಷಾಸಾ ಎಂಬ ದೇವತೆ>
28-622

<ನಕ್ತೋಷಾಸಾ ಎಂಬ ಶಬ್ದದ ವಿವರಣೆ>
2-557
11-255
15-717

<ನಚಿಕೇತನಿಗೂ ಯಮನಿಗೂ ನಡೆದ ಸಂಭಾಷಣೆಯ ವಿಷಯದಲ್ಲಿ ತೈತ್ತಿರೀಯ ಬ್ರಾಹ್ಮಣದಲ್ಲಿರುವ ವಿವರಣೆ>
30-847

<ನ ತಾಮಿನಂತಿ ಎಂಬ ಸೂಕ್ತದ ಪೀಠಿಕೆ>
17-324

<ನದಿಗಳು>
5-603

<ನದಿಗಳ ಸಂಖ್ಯೆ (ನವತಿ) ವಿಚಾರ>
6-590

<ನದಿಗಳ ವಿಷಾಪಹರಣ ಶಕ್ತಿ>
14-367

<ನಂದನ ಎಂಬುವನನ್ನು ಅಶ್ವಿನೀ ದೇವತೆಗಳು ರಕ್ಷಿಸಿದ ವಿಚಾರ>
9-433

<ನಂದನ ಋಷಿಯ ಜೀರ್ಣವಾದ ಶರೀರವನ್ನು ಹೋಗಲಾಡಿಸಿ ಅವನಿಗೆ ಯೌವನವುಂಟಾಗುವಂತೆ ಮಾಡಿದ ವಿಚಾರ>
9-479

<ನ ನೂನಂ ಎಂಬ ಋಕ್ಕಿನ ನಿರುಕ್ತಿ>
13-179

<ನಪಾತ್‍ ಶಬ್ದವಿವರಣೆ>
14-144

<ನಭನ್ಯಂ ಎಂಬ ಶಬ್ದದ ರೂಪನಿಷ್ಪತ್ತಿ>
13-240

<ನಮುಚಿ ಎಂಬ ಅಸುರನನ್ನು ಇಂದ್ರನು ಸಂಹಾರಮಾಡಿದ ವಿಷಯ>
5-287 
5-695 
10-550
14-733 
14-735 
19-193 
29-66

<ನ ಯಾತವ ಇಂದ್ರ ಎಂಬ ಋಕ್ಕಿನ ನಿರುಕ್ತ>
22-19

<ನರಕ>
5-704
27-512

<ನರಾಶಂಸ ಎಂಬ ದೇವತೆಯ ವಿಷಯ>
28-437
28-614
28-756
30-340

<ನರಾಶಂಸ ಶಬ್ದಾರ್ಥ>
8-546
11-237

<ನರಾಶಂಸಸ್ಯ ಮಹಿಮಾನಂ ಎಂಬ ಋಕ್ಕಿನ ನಿರುಕ್ತ>
21-656

<ನರಾಶಂಸ, ಪವಮಾನ, ಜಾತವೇದಾಃ ಎಂಬ ದೇವತೆಗಳು (ಬೃ.ದೇ.)>
28-719

<ನರಾಶಂಸ ಮತ್ತು ತನೂನಪಾತ್‍ ಎಂಬ ಎರಡು ದೇವತೆಗಳನ್ನು ಸ್ತುತಿಸಿರುವ ಆಪ್ರೀಸೂಕ್ತಗಳು (ಬೃ.ದೇ.)>
28-764

<ನವಗ್ವ ದಶಗ್ವ ಎಂಬ ಶಬ್ದಗಳ ಅರ್ಥ ವಿವರಣೆ>
4-29

<ನವಗ್ವಾಃ>
5-675 
6-24 
18-534 
19-501

<ನವೋನವೋ ಭವತಿ ಎಂಬ ಋಕ್ಕಿನ ನಿರುಕ್ತ>
29-398

<ನಹಿ ಗ್ರಭಾಯ ಎಂಬ ಋಕ್ಕಿನ ನಿರುಕ್ತ>
21-692

<ನಹುಷನೆಂಬ ರಾಜನ ವಿಷಯ>
3-561

<ನಕ್ಷತ್ರ ಶಬ್ದವಿವರಣೆ>
5-111 
29-357

<ನಕ್ಷತ್‍ ಶಬ್ದಾರ್ಥವಿವರಣೆ>
13-260

<ನಕ್ಷದ್ವಾಭಂ ಎಂಬ ಶಬ್ದದ ನಿರ್ವಚನ>
20-758

<ನಕ್ಷತ್ರಗಳು (ಇಪ್ಪತ್ತೇಳು) ಮತ್ತು ಅವುಗಳ ಅಧಿದೇವತೆಗಳು>
29-386

<ನ್ಯಕ್ರಂದಯನ್‍ ಎಂಬ ಋಕ್ಕಿನ ನಿರುಕ್ತ>
30-179

<ನಾಕಶಬ್ದದ ರೂಪನಿಷ್ಪತ್ತಿ>
10-209 
12-650 
15-596

<ನಾಕಶಬ್ದದ ನಾನಾವಿಧ ಪ್ರಯೋಗಗಳು>
11-56

<ನಾಕಶಬ್ದಾರ್ಥವಿಚಾರ–ಯಾಸ್ಕರ ವಿವರಣೆ ಮತ್ತು ನಾನಾರ್ಥಗಳು–ಉದಾಹರಣೆಸಹಿತವಾಗಿ>
19-614 
29-936

<ನಾನಾವಿಧ ದರ್ಭೆಗಳ ವಿಷಯ>
14-340

<ನಾನಾದೇವತೆಗಳ ವಿವಿಧ ವಾಹನಗಳು>
23-846 
15-414

<ನಾನಾದೇವತೆಗಳ ಹೆಸರುಗಳು (ಬೃ.ದೇ.)>
28-698

<ನಾನಾವಿಧ ಸೃಷ್ಟಿಕ್ರಮ ಇತ್ಯಾದಿ>
29-811

<ನಾನಾವಿಧ ಛಂದಸ್ಸುಗಳು, ಅವುಗಳ ಹೆಸರು ಅಕ್ಷರಸಂಖ್ಯೆ ಇತ್ಯಾದಿ>
29-901

<ನಾನಾವಿಧ ಸೂಕ್ತಗಳು (ಬೃ.ದೇ.)>
24-680

<ನಾಭಾನೇದಿಷ್ಠನ ವಿಷಯ–ಮನುವು ತನ್ನ ಪುತ್ರರಿಗೆ ದಾಯಭಾಗವನ್ನು ಕೊಟ್ಟ ವಿಚಾರ>
28-399

<ನಾಭ್ಯಾ ಆಸೀದಂತರಿಕ್ಷಂ ಎಂಬ ಋಕ್ಕಿನ ವಿವರಣೆ>
29-931

<ನಾಮಪದಗಳು, ಸರ್ವನಾಮಗಳು, ಅರ್ಥ, ವಾಕ್ಯ–ರಚನೆ ಇತ್ಯಾದಿ (ಬೃ.ದೇ.)>
28-738

<ನಾರಾಯಣಋಷಿಯ (ಪುರುಷಸೂಕ್ತದ್ರಷ್ಟೃವಾದ) ವಿಷಯ>
29-687

<ನಾರಾಯಣಶಬ್ದದ ರೂಪನಿಷ್ಪತ್ತಿ ಇತ್ಯಾದಿ>
29-689

<ನಾರಾಶಂಸೀ ಮಂತ್ರಗಳು-ರಾಜರ ದಾನಪ್ರಶಂಸೆ (ಬೃ.ದೇ.)>
28-800

<ನಾಸತ್ಯಾ ಎಂಬ ಶಬ್ದದ ಅರ್ಥವಿವರಣೆ>
4-105
14-13
15-423

<ನಾಸತ್ಯಾ ಎಂಬ ಶಬ್ದದ ರೂಪನಿಷ್ಪತ್ತಿ>
13-265

<ನಾಹಮಿಂದ್ರಾಣಿ ರಾರಣ ಎಂಬ ಋಕ್ಕಿನ ನಿರುಕ್ತ>
27-473

<ನಾಹಂ ತಂತು ಎಂಬ ಋಕ್ಕಿಗೆ ಅಧ್ಯಾತ್ಮ ಪರವಾದ ವಿವರಣೆ>
20-481

<ನಾಹಂ ತಂತು ಎಂಬ ಋಕ್ಕಿಗೆ ಯಜ್ಞಪರವಾದ ಅರ್ಥ ವಿವರಣೆ>
20-480

<ನಾಹುಷ>
8-247

<ನಾಹುಷ ಮತ್ತು ಸರಸ್ವತಿ (ಬೃ.ದೇ.)>
28-903

<ನಿಋತಿಯ ವಿಷಯ>
19-393

<ನಿಋತಿ ಶಬ್ದಾರ್ಥ ವಿವರಣೆ>
3-272

<ನಿಚುಂಪುಣಃ ಎಂಬ ಶಬ್ದಕ್ಕೆ ಯಾಸ್ಕರ ವಿವರಣೆ>
24-380

<ನಿಚೃದ್ಗಾಯತ್ರೀ ಇತ್ಯಾದಿ ಛಂದಸ್ಸುಗಳ ವಿಚಾರ>
6-360

<ನಿಣ್ಯಶಬ್ದಾರ್ಥ ವಿವರಣೆ>
12-551

<ನಿಪಾತಗಳು (ಬೃ.ದೇ.)>
28-734

<ನಿಪಾತಿನೀ ಮತ್ತು ಸೂಕ್ತಭಾಗಿನೀ ಮಂತ್ರಗಳು (ಬೃ.ದೇ.)>
28-770

<ನಿಯುತಃ>
3-405

<ನಿರಾವಿಧ್ಯತ್‍ ಎಂಬ ಋಕ್ಕಿನ ನಿರುಕ್ತ>
28-165

<ನಿರುಕ್ತ>
1-29

<ನಿರುಕ್ತದ ಲಕ್ಷಣ, ಪ್ರಯೋಜನ ಇತ್ಯಾದಿ>
1-875

<ನಿವಿತ್‍ ಎಂಬ ಮಂತ್ರ ವಿಶೇಷದ ಸ್ವರೂಪ>
8-74

<ನಿವಿತ್‍ ಮತ್ತು ಉಕ್ಥ ಮಂತ್ರಗಳ ಸ್ವರೂಪ ಇತ್ಯಾದಿ>
13-421

<ನಿವಿನ್ಮಂತ್ರಗಳ ವೈಶಿಷ್ಟ್ಯ ಮತ್ತು ಅವುಗಳ ಸ್ವರೂಪ>
13-390
15-301
15-562

<ನಿವಿತ್ಸೂಕ್ತಗಳು, ನಿಗದಗಳು ಮತ್ತು ಛಂದಸ್ಸುಗಳ ದೇವತೆಗಳು (ಬೃ.ದೇ.)>
28-1022

<ನಿಷ್ಕೇವಲ್ಯ ಶಸ್ತ್ರಮಂತ್ರಗಳ ವಿವರಣೆ (ಐತರೇಯ ಬ್ರಾಹ್ಮಣದಲ್ಲಿರುವಂತೆ)>
18-48

<ನಿಷ್ಕೇವಲ್ಯ ಶಸ್ತ್ರ ಮಂತ್ರಗಳಲ್ಲಿ ಯಾಜ್ಯಾ ಹೋಮದ ವಿಷಯವಾಗಿ ಪೂರ್ವೇತಿಹಾಸ>
18-50

<ನಿಷ್ಕೇವಲ್ಯ ಶಸ್ತ್ರಮಂತ್ರದಲ್ಲಿರುವ ಧಾಯ್ಯಾಮಂತ್ರದ ಪೂರ್ವೇತಿಹಾಸ>
18-52

<ನಿಷ್ಕೇವಲ್ಯ ಶಸ್ತ್ರಮಂತ್ರಗಳಲ್ಲಿರುವ ಋಕ್‍ ಮತ್ತು ಸಾಮ ಮಂತ್ರಗಳು>
18-54

<ನಿಷ್ಕೇವಲ್ಯ ಶಸ್ತ್ರದ ಐದು ವಿಭಾಗಗಳು>
18-59

<ನಿಷ್ಕೇವಲ್ಯ ಶಸ್ತ್ರದ ಭಾಗಗಳು ಮತ್ತು ಸ್ವರ ವಿಶೇಷಗಳು>
18-62

<ನಿಷ್ಕೇವಲ್ಯ ಶಸ್ತ್ರದ ಸ್ತೋತ್ರಿಯ, ಅನುರೂಪ ಧಾಯ್ಯಾ ಸಾಮ ಪ್ರಗಾಥ ಮತ್ತು ನಿವಿದ್ಧಾನೀಯಸೂಕ್ತಗಳು>
18-62

<ನು ಎಂಬ ಶಬ್ದದ ನಾನಾರ್ಥಗಳು ಮತ್ತು ಪ್ರಯೋಗ>
10-131 
14-159 
14-528 
18-27

<ನೂ ಚಿತ್‍ ಎಂಬ ಶಬ್ದದ ನಿಪಾತಗಳು>
8-100 
21-15

<ನೂನಂ ಎಂಬ ಶಬ್ದದ ನಾನಾರ್ಥಗಳು>
13-178 
15-341

<ನೃಮರ ಎಂಬ ಅಸುರನ ವಿಷಯ>
14-704

<ನ\ ನ್‍ ಎಂಬ ಶಬ್ದದ ಅರ್ಥಾನುವಾದ>
13-222 
19-745

<ನೇಜಮೇಷ ಎಂಬ ಖಿಲಸೂಕ್ತ (ಬೃ.ದೇ.)>
28-1016

<ನೇಮಃ ಎಂಬ ಶಬ್ದದ ವಿವರಣೆ ಮತ್ತು ನಿರ್ವಚನ>
19-808

<ನೈಚಾಶಾಖಂ ಎಂಬ ಶಬ್ದದ ವಿವರಣೆ>
17-209

<ನೆಪಾತಿಕ ಸೂಕ್ತ (ಬೃ.ದೇ.)>
28-681

<ನೋಧಾ ಗೌತಮ ಎಂಬ ಋಷಿಯ ವಿಷಯ>
6-59 
10-149

<ನೋಧಾ ಶಬ್ದದ ಅರ್ಥ ವಿವರಣೆ>
5-522

[ಪ]
<ಪಗಡೆ ದಾಳಗಳ ಪ್ರಶಂಸೆ (ಬೃ.ದೇ.)>
28-937

<ಪಚತ ಎಂಬ ಶಬ್ದದ ನಿರ್ವಚನ>
30-541

<ಪಂಚದಶಸ್ತೋಮದ ವಿವರಣೆ>
2-142

<ಪಂಚಜನ, ಪಂಚಕ್ಷಿತಿ, ಪಂಚಕೃಷ್ಟಿ ಎಂಬ ಶಬ್ದಗಳ ವಿವರಣೆ>
2-296 
7-300 
9-321
12-378 
13-407 
14-710 
14-449 
17-215 
17-472
19-240 
26-665
28-308

<ಪಂಚರಶ್ಮಿಗಳ ವಿವರಣೆ>
15-401

<ಪಂಚಹೋತೃಗಳ ವಿವರಣೆ>
15-255 
19-404
29-836

<ಪಂಚಮಹಾಯಜ್ಞಗಳ ವಿವರಣೆ>
30-1034

<ಪಂಚಯಾಮಂ ಎಂಬ ಶಬ್ದದ ವಿವರಣೆ>
30-677

<ಪಂಚಾಧ್ವರ್ಯುಗಳು>
16-30

<ಪಜ್ರರಿಗೂ ಅಂಗಿರಸ್ಸುಗಳಿಗೂ ಇರುವ ಸಂಬಂಧ>
8-775

<ಪಜ್ರಾ ಎಂದು ಅಂಗಿರಸರಿಗೆ ಹೆಸರಿರುವ ವಿಚಾರ>
9-215
10-209

<ಪಣಿ ಎಂಬ ಅಸುರರ ವೃತ್ತಾಂತ>
2-228
30-630

<ಪಣಿಗಳ ವಿಷಯ>
5-689 
6-15 
10-175 
13-536
15-26
20-680

<ಪಣಿ ಶಬ್ದವು ಪಠಿತವಾಗಿರತಕ್ಕ ಋಕ್ಕುಗಳು>
6-18

<ಪಣಿಗಳಿಗೂ ದೇವಶುನಿಯಾದ ಸರಮೆಗೂ ನಡೆದ ಸಂಭಾಷಣೆ>
28-999 
30-289
6-15

<ಪಣಿಗಳ ಮತ್ತು ವಲನ ವೃತ್ತಾಂತ>
14-634

<ಪಣಿಗಳು ದೇವತೆಗಳ ಗೋವುಗಳನ್ನು ಅಪಹರಿಸಿದ ವಿಚಾರ>
30-287

<ಪಣಿಗಳು ದೇವತೆಗಳ ಗೋವುಗಳನ್ನು ಅಪಹರಿಸಿದಾಗ ಅವುಗಳನ್ನು ಹಿಂದಕ್ಕೆ ತರುವುದಕ್ಕಾಗಿ ದೇವತೆಗಳ ಪ್ರಯತ್ನ>
26-829

<ಪತಂಗಮಕ್ತಂ ಎಂಬ ಋಕ್ಕಿಗೆ ದೇವತಾಪರವಾದ ಅರ್ಥವಿವರಣೆ>
30-1125

<ಪತಂಗಮಕ್ತಂ ಎಂಬ ಋಕ್ಕಿಗೆ ಅಧ್ಯಾತ್ಮಪರವಾದ ಅರ್ಥವಿವರಣೆ>
30-1198

<ಪತಂಗೋ ವಾಚಂ ಎಂಬ ಋಕ್ಕಿಗೆ ದೇವತಾಪರವಾದ ಅರ್ಥ>
30-1201

<ಪತಂಗೋ ವಾಚಂ ಎಂಬ ಋಕ್ಕಿಗೆ ಅಧ್ಯಾತ್ಮಪರವಾದ ಅರ್ಥ>
30-1203

<ಪದಪಾಠಕಾರರು>
1-285

<ಪದವ್ಯಃ>
6-341

<ಪದವೀ ಶಬ್ದದ ಅರ್ಥವಿವರಣೆ>
22-189

<ಪದ, ಅರ್ಥ ಮತ್ತು ಕ್ರಿಯಾಶಬ್ದ (ಬೃ.ದೇ.)>
28-743

<ಪದವಿಚ್ಛೇದ ಮತ್ತು ಆರುವಿಧ ಸಮಾಸಗಳು (ಬೃ.ದೇ.)>
28-738

<ಪಪೃಚಾಸಿ ಎಂಬ ಶಬ್ದದ ವಿವರಣೆ>
11-209

<ಪಯಃ ಶಬ್ದಾರ್ಥ ವಿವರಣೆ>
9-558

<ಪರಬ್ರಹ್ಮನ ವ್ಯಕ್ತಾವ್ಯಕ್ತ ಸ್ವರೂಪವು>
12-431

<ಪರಮಾತ್ಮನಲ್ಲಿ ಸರ್ವದೇವತೆಗಳೂ ಅಧಿಷ್ಠಿತರಾಗಿದಾರೆ ಎಂಬ ವಿಷಯ>
12-577

<ಪರಮಜ್ಯಾಃ ಎಂಬ ಶಬ್ದದ ವಿವರಣೆ>
25-304

<ಪರಮ ಪುರುಷನೇ ಕಾಲಸ್ವರೂಪನು>
26-826

<ಪರಮ ಪುರುಷನ ಪರಂಜ್ಯೋತಿಯೇ ಸವಿತೃವಿನ ಭರ್ಗ–ಈ ವಿಷಯದಲ್ಲಿ ಆಕ್ಷೇಪ ಸಮಾಧಾನಗಳು>
17-567

<ಪರಮ ಪುರುಷನ ವಿಶ್ವವ್ಯಾಪಕತ್ವ ಮತ್ತು ತದತಿರಿಕ್ತತ್ವ>
17-604

<ಪರತತ್ತ್ವದ ಜ್ಯೋತೀರೂಪ>
17-608

<ಪರಮಾತ್ಮನಿಗೆ ಪುರುಷನೆಂಬ ಹೆಸರು ಬರಲು ಕಾರಣ>
29-249

<ಪರಮಾತ್ಮನು ಸೃಷ್ಟಿಯನ್ನು ಹೇಗೆ ಮಾಡಿದನು?>
30-771

<ಪರಬ್ರಹ್ಮ ವಸ್ತುಸ್ವರೂಪ>
30-602

<ಪರಂ ಮೃತ್ಯೋ ಅನು ಎಂಬ ಸೂಕ್ತದ ಪೀಠಿಕೆ>
27-577

<ಪರಂ ಮೃತ್ಯೋ ಅನು ಎಂಬ ಸೂಕ್ತದ ವಿಷಯದಲ್ಲಿ ಬೃಹದ್ದೇವತಾಗ್ರಂಥದಲ್ಲಿ ಹೇಳಿರುವ ವಿವರಣೆ>
27-577

<ಪರಂ ಮೃತ್ಯೋ ಅನು ಎಂಬ ಸೂಕ್ತದ ವಿಷಯದಲ್ಲಿ ಋಗ್ವಿಧಾನವೆಂಬ ಗ್ರಂಥದಲ್ಲಿ ಈ ಸೂಕ್ತದ ಜಪ ವಿಷಯವಾಗಿ ಹೇಳಿರುವ ಅಂಶಗಳು>
27-578

<ಪರಂ ಮೃತ್ಯೋ ಅನು ಎಂಬ ಸೂಕ್ತದ ವಿಷಯದಲ್ಲಿ ಶತಪಥ ಬ್ರಾಹ್ಮಣದಲ್ಲಿ ಹೇಳಿರುವ ಇತಿಹಾಸವು>
27-580

<ಪರಂ ಮೃತ್ಯೋ ಅನು ಎಂಬ ಋಕ್ಕಿನ ನಿರುಕ್ತ>
27-597

<ಪರಂಜ್ಯೋತೀರೂಪವಾದ ಭರ್ಗವನ್ನರಿಯಲು ಗಾಯತ್ರಿಯ ಪಠನರೂಪವಾದ ಧ್ಯಾನವೇ ಮುಖ್ಯಸಾಧನ>
17-568

<ಪರಂಜ್ಯೋತೀರೂಪನಾದ ಬ್ರಹ್ಮನ ಮೂರ್ತರೂಪಗಳು (ಭೂರಾದಿವ್ಯಾಹೃತಿವಿಭಾಗಗಳು)>
17-610

<ಪರಾಶರಶಾಖೆ>
1-63

<ಪರಾವೃಜಶಬ್ದ ವಿವರಣೆ>
8-762

<ಪರಾವೃಜ ಎಂಬುವನ ವಿಷಯ>
8-763

<ಪರಾಯತೀನಾಂ ಆಯತೀನಾಂ ಮೊದಲಾದ ಶಬ್ದಗಳ ಅರ್ಥಾನುವಾದ ವಿವರಣೆ>
9-42

<ಪರಾವತಃ ಎಂಬ ಶಬ್ದದ ಅರ್ಥ ವಿವರಣೆ>
10-420

<ಪರಾವೃಕ್‍ ಎಂಬ ಕುಂಟನೂ ಕುರುಡನೂ ಆದ ಋಷಿಗೆ ಇಂದ್ರನು ಕಾಲುಗಳನ್ನೂ ಕಣ್ಣುಗಳನ್ನೂ ಕೊಟ್ಟು ರಕ್ಷಿಸಿದ ವಿಚಾರ>
14-714
14-769

<ಪರಾವೃಕ್ತನೆಂಬುವನನ್ನು (ಅಗ್ರುಪುತ್ರ) ಇಂದ್ರನು ಹುತ್ತದಿಂದ ಕಾಪಾಡಿದ ವಿಚಾರ>
18-24

<ಪರಾಶರಃ ಎಂಬ ಶಬ್ದಕ್ಕೆ ಯಾಸ್ಕರ ನಿರ್ವಚನ>
23-271

<ಪರಿಗಧಿತಾ ಎಂಬ ಶಬ್ದದ ವಿವರಣೆ>
10-243

<ಪರಿಶಿಷ್ಟಗಳು>
1-41

<ಪರಿಸಂಖ್ಯಾಲಕ್ಷಣ>
1-389

<ಪರಿದೇವನಾ ವಾಕ್ಯದಸ್ವರೂಪ>
12-547

<ಪರಿರಪಃ ಎಂಬ ಶಬ್ದದ ವಿವರಣೆ>
14-912

<ಪರಿಶಿಷ್ಟ ಮಂತ್ರಗಳ ಸ್ವರೂಪ>
20-263

<ಪರಿಷಧ್ಯಂ ಎಂಬ ಋಕ್ಕಿನ ನಿರುಕ್ತ>
21-688

<ಪರುಚ್ಛೇಪ ಋಷಿಯ ಶೈಲಿ ವಿವರಣೆ ಇತ್ಯಾದಿ>
10-249

<ಪರೇಯಿವಾಂಸಂ ಎಂಬ ಸೂಕ್ತದ ಪೀಠಿಕೆ>
27-428

<ಪರೋಕ್ಷ ಪ್ರಿಯರು ದೇವತೆಗಳು>
15-779 
17-357

<ಪರೋಕ್ಷ ಕೃತ, ಪ್ರತ್ಯಕ್ಷಕೃತ, ಆಧ್ಯಾತ್ಮಿಕ ಎಂದು ಮಂತ್ರಗಳಲ್ಲಿ ಮೂರು ಪ್ರಭೇದಗಳು>
2-205
19-747
23-262 
26-785 
30-1282

<ಪರ್ಜನ್ಯದೇವತೆಯ ವಿಷಯ>
5-606 
20-199
28-721

<ಪರ್ಜನ್ಯನು ಭೂಮಿಗೂ ಅಗ್ನಿಯು ಸ್ವರ್ಗಕ್ಕೂ ಚೈತನ್ಯದಾಯಕರು>
12-665

<ಪರ್ಜನ್ಯಾಗ್ನಿಗಳ ಸಂಬಂಧವು ಬೀಜಾಂಕುರಗಳ ಸಂಬಂಧದಂತಿದೆ>
12-663

<ಪರ್ಯಗ್ನಿಕರಣವೆಂಬ ಕರ್ಮದ ವಿವರಣೆಯು (ಐತರೇಯ ಬ್ರಾಹ್ಮಣದಲ್ಲಿರುವಂತೆ)>
17-913

<ಪರ್ವತ ಶಬ್ದದ ರೂಪನಿಷ್ಪತ್ತಿ>
10-18 
17-409
19-696

<ಪವಮಾನ (ಬೃ.ದೇ.)>
28-719

<ಪವಿಃ ಎಂಬ ಶಬ್ದದ ಅರ್ಥವಿವರಣೆ>
13-139

<ಪವಿತ್ರ ಶಬ್ದದ ನಿರ್ವಚನ, ಅರ್ಥವಿವರಣೆ ಇತ್ಯಾದಿ>
22-282 
26-28

<ಪಶುಗಳ ಉತ್ಪತ್ತಿ–ಪಶುಗಳಿಗೂ ವಾಯುವಿಗೂ ಇರುವ ಸಂಬಂಧ>
29-892

<ಪಕ್ಷಿಗಳ ವಿಷಯ>
30-1114

<ಪಕ್ಷಿದ್ವಯಸಾದೃಶ್ಯದಿಂದ ಜೀವಾತ್ಮ ಪರಮಾತ್ಮನ ವರ್ಣನೆ>
12-413

<ಪಕ್ಷ್ಯಾಕಾರವಾದ ಗಾಯತ್ರಿಯೇ ಶ್ಯೇನವು>
17-572

<ಪ್ರಉಗ ಶಸ್ತ್ರದ ಮಹಿಮೆ>
10-609

<ಪ್ರಕೃತಿಗೂ ಈಶ್ವರನಿಗೂ ಇರುವ ಸಂಬಂಧ>
12-314

<ಪ್ರಕೃತಿ ಮತ್ತು ವಿಕೃತಿಯಾಗಗಳ ವಿವರಣೆ>
2-151

<ಪ್ರಕೇತುನಾ ಎಂಬ ಸೂಕ್ತದ ಪೀಠಿಕೆ>
27-299

<ಪ್ರಗಾಥ ಮತ್ತು ಕಣ್ವ (ಬೃ.ದೇ.)>
28-908

<ಪ್ರಚೇತಾಃ ಎಂಬ ಶಬ್ದದ ಅರ್ಥವಿವರಣೆ>
16-482

<ಪ್ರಚೋದಯಾತ್‍ (ಗಾಯತ್ರಿಯಲ್ಲಿರುವ) ಎಂಬ ಶಬ್ದದ ವಿವರಣೆ>
17-635

<ಪ್ರಜಾಪತಿ>
5-642

<ಪ್ರಜಾಪತಿಯು ತನ್ನ ಪುತ್ರಿಯನ್ನೇ ಮೋಹಿಸಿದ ಪೂರ್ವೇತಿಹಾಸ>
11-290 
17-754 
27-342
28-414

<ಪ್ರಜಾಪತಿಯು ಪುಂಸ್ತ್ರೀರೂಪಗಳಿಂದ ಜಗತ್ತನ್ನು ವ್ಯಾಪಿಸಿರುವ ವಿಚಾರ>
12-392

<ಪ್ರಜಾಪತಿಯು ವಾಕ್ಕಿಗೆ ಆಧಾರನು ಅಥವಾ ರಕ್ಷಕನು–ಯಾಜ್ಞವಲ್ಕ್ಯರ ಸಂವಾದ>
12-532

<ಪ್ರಜಾಪತಿ ಮತ್ತು ಸಂವತ್ಸರದ ವಿಷಯ, ಅಗ್ನಿಯ ಎಂಟು ಹೆಸರುಗಳು>
15-647

<ಪ್ರಜಾಪತಿಗೆ ಕಃ ಎಂದು ಹೆಸರು ಬರಲು ಕಾರಣವು–ಯಜ್ಞದಲ್ಲಿ ಇಂದ್ರನ ಹವಿರ್ಭಾಗವು>
18-48 
29-639 
30-625

<ಪ್ರಜಾಪತಿಯ ವೀರ್ಯದಿಂದ ಆದಿತ್ಯ ಭೃಗುಋಷಿ, ಅಂಗಿರಾ ಋಷಿಗಳು ಇವರ ಉತ್ಪತ್ತಿಕ್ರಮ ಇತ್ಯಾದಿ>
27-365

<ಪ್ರಜಾಪತಿ ದುಹಿತೃಧ್ಯಾನೋಪಾಖ್ಯಾನವೆಂಬ ಇತಿಹಾಸ (ಐತರೇಯಬ್ರಾಹ್ಮಣದಲ್ಲಿರುವಂತೆ)>
17-754

<ಪ್ರಜಾಪತಿಯು ದಶಹೋತ್ರನೆಂಬ ವಿಚಾರ>
29-832

<ಪ್ರಜಾಪತಿಯು ದಶಹೋತೃವಾಗಿ ಸೃಷ್ಟಿಮಾಡಿದ ಕ್ರಮ>
29-836

<ಪ್ರಜಾಪತಿಯೇ ಸಂವತ್ಸರ (ಕಾಲ) ಸ್ವರೂಪನು>
29-847

<ಪ್ರಜಾಪತಿ ಶಬ್ದನಿರ್ವಚನ>
30-645

<ಪ್ರಣವದ ಸ್ವರೂಪ>
17-615

<ಪ್ರಣವ ಮತ್ತು ಗಾಯತ್ರೀ–ಈ ಎರಡೂ ಬ್ರಹ್ಮಸ್ವರೂಪಗಳೇ; ಭೂಃ ಭುವಃ ಸುವಃ ಎಂಬ ವ್ಯಾಹೃತಿರೂಪವಾದ ಬ್ರಹ್ಮದ ಉಪಾಸನೆಯ ಫಲ>
17-619

<ಪ್ರಣವದ (ಓಂಕಾರದ) ಉತ್ಪತ್ತಿ ವಿಚಾರ>
12-583

<ಪ್ರತದ್ವಸೂ ಎಂಬ ಶಬ್ದದ ನಿರ್ವಚನ ಮತ್ತು ಅರ್ಥವಿವರಣೆ>
23-699

<ಪ್ರತ್ಯಕ್ಷಕೃತ ಪರೋಕ್ಷಕೃತ ಆಧ್ಯಾತ್ಮಕ ಎಂಬ ಮೂರುವಿಧ ಮಂತ್ರಗಳು>
17-169

<ಪ್ರತ್ಯವರೋಹಕ್ರಮದಿಂದ ತ್ರಿಸ್ಥಾನದೇವತೆಗಳ ವರ್ಣನೆ (ಬೃ.ದೇ.)>
28-703

<ಪ್ರಥಮ ಮಂಡಲದ ಆಪ್ರೀಸೂಕ್ತಗಳು (ಬೃ.ದೇ.)>
28-751

<ಪ್ರ ದೇವತ್ರಾ ಬ್ರಹ್ಮಣೇ ಎಂಬ ಸೂಕ್ತದ ಪೀಠಿಕೆ>
27-783

<ಪ್ರ ದೇವತ್ರಾ ಬ್ರಹ್ಮಣೇ ಎಂಬ ಸೂಕ್ತದಲ್ಲಿರುವ ಋಕ್ಕುಗಳ ವಿಷಯದಲ್ಲಿ ಪೂರ್ವೇತಿಹಾಸ>
27-289

<ಪ್ರ ನೂನಂ ಎಂಬ ಋಕ್ಕಿನ ನಿರುಕ್ತ>
30-1261

<ಪ್ರ ಪರ್ವತಾನಾಂ ಎಂಬ ಸೂಕ್ತದ (ಋ. ಸಂ. ೩-೩೭) ಸಾರಾಂಶ>
17-195

<ಪ್ರಪಿತ್ವೇ ಎಂಬ ಶಬ್ದದ ನಿರ್ವಚನ ಇತ್ಯಾದಿ>
23-415

<ಪ್ರಭರ್ಮಣಿ ಎಂಬ ಶಬ್ದದ ವಿವರಣೆ>
25-219

<ಪ್ರಭುಂಜತೀ ಶಬ್ದದ ಅರ್ಥ ವಿವರಣೆ>
5-46

<ಪ್ರಯಾಜಾಃ ಎಂಬ ಆಹುತಿಗಳ ವಿಷಯದಲ್ಲಿ ಯಾಸ್ಕರ ವಿವರಣೆ>
28-268

<ಪ್ರಯೋಗನೆಂಬ ಋಷಿಯು ಅಗ್ನಿಯಲ್ಲಿ ಕಾಷ್ಠಾದಿಗಳನ್ನು ಅರ್ಪಿಸಿದ ವಿಷಯದಲ್ಲಿ ಯಜುರ್ವೇದದಲ್ಲಿರುವ ವಿವರಣೆ>
25-554

<ಪ್ರವಯಾಃ ಎಂಬ ಶಬ್ದದ ವಿವರಣೆ>
14-804

<ಪ್ರವರ್ಗ್ಯ ಯಾಗದ ಸ್ವರೂಪ ಮತ್ತು ಮಹತ್ತ್ವ>
12-485

<ಪ್ರವರ್ಗ್ಯ ಮತ್ತು ಘರ್ಮಗಳ ಸ್ವರೂಪ>
13-516

<ಪ್ರವತ, ನಿವತ, ಉದ್ವತ ಎಂಬ ವಿವರಣೆ>
15-591

<ಪ್ರವಾವೃಜೇ ಎಂಬ ಋಕ್ಕಿನ ನಿರುಕ್ತ>
22-219

<ಪ್ರಶಿಷಃ ಎಂಬ ಶಬ್ದದ ಅರ್ಥಾನುವಾದ>
11-361

<ಪ್ರಸ್ಕಣ್ವಋಷಿಯ ವಿಷಯ>
4-514

<ಪ್ರಸ್ತುತಿ ಎಂಬ ಪಾರಿಭಾಷಿಕಶಬ್ದದ ವಿವರಣೆ>
11-549

<ಪ್ರಸ್ತೋತಿ ಸಾಂರ್ಜಯ ಮತ್ತು ಅಭ್ಯಾವರ್ತೀ ಎಂಬುವರ ವೃತ್ತಾಂತ (ಬೃ.ದೇ.)>
28-882

<ಪಾಕಶಬ್ದಾರ್ಥ ವಿಚಾರ>
12-319
15-91

<ಪಾಜಃ ಶಬ್ದ ವಿವರಣೆ>
17-341

<ಪಾಥಃ ಶಬ್ದದ ಅರ್ಥವಿವರಣೆ>
11-589 
22-447

<ಪಾದಶಬ್ದದ ವಿಶೇಷಾರ್ಥ>
29-805

<ಪಾದುಃ ಎಂಬ ಶಬ್ದದ ನಿರ್ವಚನ>
27-758

<ಪಾಪಿಷ್ಠರು, ಪುಣ್ಯವಂತರು ಮತ್ತು ಅವರುಗಳಿಗೆ ಲಭಿಸುವಗತಿ>
27-437

<ಪಾರಾವೃಜ>
22-488

<ಪಾವಕಶೋಚಿಷಂ ಮತ್ತು ಶೇರಂ ಎಂಬ ಶಬ್ದಗಳ ನಿರ್ವಚನ>
24-422

<ಪಾವಮಾನೀ ಸೂಕ್ತಗಳನ್ನು ಅಧ್ಯಯನ ಮಾಡಿದವರಿಗೆ ಪರಿಹಾರವಾಗುವ ಪಾಪಗಳ ವಿವರಣೆ>
27-274

<ಪಾರ್ಥಿವಾಗ್ನಿಯ ಮಹಿಮೆ>
28-60

<ಪಾರ್ಥಿವಗ್ನಿಯನ್ನು ಪ್ರತಿಬಿಂಬಿಸುವ ದೇವತೆಗಳು (ಬೃ.ದೇ.)>
28-705

<ಪ್ರಾಕೃತ ಮತ್ತು ವೈಕೃತ ಸೃಷ್ಟಿಗಳು>
29-7

<ಪ್ರಾಚೀನಂ ಬರ್ಹಿಃ ಎಂಬ ಋಕ್ಕಿನ ನಿರುಕ್ತ>
30-346

<ಪ್ರಾಣದ ಮಹತ್ವ ಇತ್ಯಾದಿ>
29-255

<ಪ್ರಾತರನುವಾಕಮಂತ್ರಗಳ ಸ್ವರೂಪ ಮತ್ತು ಅವುಗಳನ್ನು ಪಠಿಸುವ ವಿಧಾನ>
8-827

<ಪ್ರಾತರನುವಾಕ ಮಂತ್ರಗಳನ್ನು ಪಠಿಸುವ ಕ್ರಮ (ಐತರೇಯ ಬ್ರಾಹ್ಮಣದಲ್ಲಿರುವಂತೆ)>
16-254

<ಪ್ರಾತರನುವಾಕ ಮಂತ್ರಗಳ ವಿವರಣೆ>
11-671 
13-643
16-253

<ಪ್ರಾತರನುವಾಕ ಮಂತ್ರಗಳಲ್ಲಿ ಆಗ್ನೇಯಕ್ರತು, ಉಷಸ್ಯಕ್ರತು, ಆಶ್ವಿನಕ್ರತು ಎಂಬ ಮೂರು ವಿಧ>
]6-253

<ಪ್ರಾತರನುವಾಕ ಮಂತ್ರಗಳಲ್ಲಿ ಪಠಿಸಬೇಕಾದ ಏಳುವಿಧ ಛಂದಸ್ಸುಗಳು ಮತ್ತು ಮಂತ್ರಗಳು–ಉದಾಹರಣೆ ಸಹಿತವಾಗಿ>
16-253

<ಪ್ರಾತರ್ಜಿತಂ ಎಂಬ ಋಕ್ಕಿನ ನಿರುಕ್ತ>
22-237

<ಪ್ರಾತಸ್ಸವನ, ಮಾಧ್ಯಂದಿನ ಸವನ, ತೃತೀಯ ಸವನ ಎಂಬ ಸವನತ್ರಯಗಳಲ್ಲಿ ಉಪಯೋಗಿಸುವ ಸ್ತೋತ್ರ ಮತ್ತು ಶಸ್ತ್ರಮಂತ್ರಗಳು>
13-290

<ಪ್ರಾತಸ್ಸವನ ಎಂಬ ಕರ್ಮದ ವಿವರಣೆ (ಐತರೇಯ ಬ್ರಾಹ್ಮಣಾನುಸಾರವಾಗಿ ಪಠಿಸಬೇಕಾದ ಮಂತ್ರಗಳು)>
15-693
26-56

<ಪ್ರಾತಸ್ಸವನಕಾಲದಲ್ಲಿ ಪಠಿಸಬೇಕಾದ ಉನ್ನೀಯಮಾನ ಸೂಕ್ತವು>
15-697
26-50

<ಪ್ರಾತಸ್ಸವನಕಾಲದಲ್ಲಿ ಸಪ್ತಹೋತೃಗಳು ಪಠಿಸಬೇಕಾದ ಮಂತ್ರಗಳು>
15-703
26-56

<ಪ್ರಾತಸ್ಸವನಕಾಲದಲ್ಲಿ ಋತ್ವಿಕ್ಕುಗಳು ಪಠಿಸಬೇಕಾದ ಮಂತ್ರಗಳು>
27-168

<ಪ್ರಾತಾ ರಥಃ ಎಂಬ ಋಕ್ಕಿನ ಎರಡು ವಿಧವಾದ ಅರ್ಥವಿವರಣೆ>
14-819

<ಪ್ರಾಥಂತಮಃ ಎಂಬ ಶಬ್ದದ ಅರ್ಥವಿವರಣೆ>
11-473

<ಪ್ರಾರ್ಥನೆ ಅಥವ ಸ್ತುತಿಯ ಸ್ವರೂಪ (ಬೃ.ದೇ.)>
28-679

<ಪ್ರಾವೇಪಾಮಾ ಎಂಬ ಋಕ್ಕಿನ ನಿರುಕ್ತ>
27-856

<ಪ್ರಾಸಕ್ರಮ–ಕೆಲವು ಋಕ್ಕುಗಳಲ್ಲಿ ಕಂಡುಬರುವ>
2-188

<ಪ್ರಾಸಹಾದೇವಿಯ (ಇಂದ್ರನ ಪತ್ನಿಯಾದ) ಕಥೆ>
18-51

<ಪಿಪ್ರು ಎಂಬ ಅಸುರನ ವೃತ್ತಾಂತ>
5-168 
5-694 
8-275 
10-550 
19-201
20-687

<ಪಿತರಾ ಎಂಬ ಶಬ್ದದ ವಿವಿದಾರ್ಥಗಳು>
15-664

<ಪಿತು ಶಬ್ದದ ಅರ್ಥವಿವರಣೆ>
14-187

<ಪಿತೃಗಳು>
5-705
17-280


<ಪಿತೃಗಳ ಸ್ವರೂಪವರ್ಣನೆ>
8-540

<ಪಿತೃ ಶಬ್ದಾರ್ಥ ವಿವರಣೆ>
14-491

<ಪಿತೃಗಳು, ಪಿತೃಲೋಕಗಳು ಇತ್ಯಾದಿ>
27-429

<ಪಿತೃಗಳ ವಿಷಯದಲ್ಲಿ ಕೆಲವು ವಿವರಣೆಗಳು>
27-451

<ಪಿತೃಗಳಲ್ಲಿರುವ ಪ್ರಭೇದಗಳು–ಆಂಗಿರಸರು, ಅಥರ್ವಾಣರು, ಭೃಗುಗಳು>
27-467

<ಪಿತೃಗಳಲ್ಲಿ ಉತ್ತಮ, ಮಧ್ಯಮ, ಅಧಮ ಎಂದು ಮೂರು ಪ್ರಭೇದಗಳು>
27-490

<ಪಿತೃಗಳ ವಿವಿಧಸ್ವರೂಪ>
27-493

<ಪಿತೃಯಾಣ, ಮತ್ತು ದೇವಯಾನಗಳ ವಿಮರ್ಶೆ>
27-232

<ಪಿತೃಶ್ರಾದ್ಧ–ಪಿಂಡಪಿತೃಯಜ್ಞ, ಮಹಾಪಿತೃಯಜ್ಞ ಇತ್ಯಾದಿ>
27-451

<ಪಿತ್ರ್ಯಾಣಿ ಎಂಬ ಶಬ್ದದ ವಿವರಣೆ>
6-330

<ಪಿಶಾಚ>
5-698

<ಪಿಶಾಚಿ, ರಕ್ಷಃ ಇತ್ಯಾದಿ ವಿಷಯಗಳು>
10-554
10-549

<ಪ್ರಿಯಮೇಧನ ವಿಷಯ>
4-551
11-84

<ಪೀಯು ಎಂಬ ಅಸುರನ ವಿಷಯ>
14-851

<ಪ್ರೀಣೀತಾಶ್ವಾನ್‍ ಹಿತಂ ಎಂಬ ಋಕ್ಕಿನ ನಿರುಕ್ತ>
30-160

<ಪುತ್ರಶಬ್ದಾರ್ಥ ವಿವರಣೆ>
6-183
6-238


<ಪುತ್ರ ಪುತ್ರಿ ಇವರಿಗೆ ಸಲ್ಲಬೇಕಾದ ದಾಯಭಾಗದ ವಿಷಯದಲ್ಲಿ ಧರ್ಮಶಾಸ್ತ್ರದ ಅಭಿಪ್ರಾಯಗಳು>
16-720

<ಪುತ್ರನಿಲ್ಲದವನಿಗೆ ಪುತ್ರಿಯೇ ಪುತ್ರನೆಂಬ ವಿಷಯ>
16-718

<ಪುನರುತ್ಪತ್ತಿ ಮತ್ತು ಜನ್ಮಾಂತರ>
12-440

<ಪುನರೇಹಿ ವೃಷಾಕಪೇ ಎಂಬ ಋಕ್ಕಿನ ನಿರುಕ್ತ>
29-485

<ಪುನಃ ಪುನರ್ಜಾಯಮಾನಾ ಎಂಬ ಉಷೋದೇವತೆಯ ವರ್ಣನ>
7-441

<ಪುರಂದರ ಶಬ್ದಾರ್ಥ ವಿವರಣೆ>
8-339

<ಪುರಂಧಿ ಶಬ್ದಾರ್ಥ ವಿವರಣೆ>
9-216 
10-96 
10-582 
12-13 
13-528 
17-510
26-816

<ಪುರಸ್ಸದಃ ಎಂಬ ಶಬ್ದದ ವಿವರಣೆ>
6-394

<ಪುರಾಣಾನಾ ಎಂಬ ಋಕ್ಕಿನ ಆದಿತ್ಯಪರವಾದ ಅರ್ಥ>
30-855

<ಪುರಾಣಾನಾ ಎಂಬ ಋಕ್ಕಿಗೆ ಯಮನಪರವಾದ ಅರ್ಥ>
30-854

<ಪುರೀಷ ಶಬ್ದ ವಿವರಣೆ>
12-362

<ಪುರುಕುತ್ಸನೆಂಬ ರಾಜನ ವಿಚಾರ>
8-757 
13-323
18-368

<ಪುರುಈಳ್ಹರಾಜನ ವೃತ್ತಾಂತ>
11-482
13-643

<ಪುರುಈಳ್ಹರಾಜನ ದಾನಪ್ರಶಂಸೆ>
19-811

<ಪುರುಈಳ್ಹನೆಂಬುವನನ್ನು ಅಶ್ವಿನೀದೇವತೆಗಳು ರಕ್ಷಿಸಿದ ವಿಚಾರ>
13-643

<ಪುರುಶ್ಚಂದ ಶಬ್ದದ ವಿವರಣೆ>
5-281

<ಪುರುಷಂತಿ>
8-821

<ಪುರುಷ ಶಬ್ದಾರ್ಥ>
29-249

<ಪುರುಷ ಸೂಕ್ತಕ್ಕೆ ಪುರುಷ ಸೂಕ್ತವೆಂಬ ಹೆಸರು ಬರಲು ಕಾರಣ, ಪುರುಷ ಶಬ್ದಾರ್ಥ ವಿವರಣೆ ಇತ್ಯಾದಿ>
29-683

<ಪುರುಷ ಸೂಕ್ತದ ಋಷಿಯಾದ ನಾರಾಯಣ ಋಷಿಯ ವಿಷಯ>
29-688

<ಪುರುಷ ಶಬ್ದದ ನಾನಾವಿಧ ಅರ್ಥಗಳು>
29-692

<ಪುರುಷ ಸೂಕ್ತದ ವಿಷಯದಲ್ಲಿ ವಿನಿಯೋಗ, ಫಲಶ್ರುತಿ ಮಹಾತ್ಮೆ ಇತ್ಯಾದಿ>
29-699

<ಪುರುಷ ಏವೇದಂ ಸರ್ವಂ ಎಂಬ ಋಕ್ಕಿನ ವಿವರಣೆ>
29-731

<ಪುರ್ವಣೀಕ ಎಂಬ ಶಬ್ದದ ವಿವರಣೆ>
6-542

<ಪುರೂರವಸ್ಸಿನ ವಿಷಯ>
3-537

<ಪುರೂರವಾಃ ಮತ್ತು ಉರ್ವಶಿಯ ವೃತ್ತಾಂತ (ಬೃ.ದೇ.)>
28-989

<ಪುರೂರವಾಃ ಮತ್ತು ಉರ್ವಶೀ>
30-1

<ಪುರೂರವಾಃ ಮತ್ತು ಉರ್ವಶಿ ಇವರ ಸಂವಾದ>
30-10

<ಪುರೂರವಾಃ, ಮೃತ್ಯು, ಸೂರ್ಯನ ಹೆಸರುಗಳು (ಬೃ.ದೇ.)>
28-727

<ಪುರೋಹಿತ ಶಬ್ದದ ನಿರ್ವಚನ, ರೂಪನಿಷ್ಪತ್ತಿ, ಅರ್ಥವಿವರಣೆ ಇತ್ಯಾದಿ>
4-526
15-629
16-170

<ಪುರೋಹಿತ ಪ್ರಶಂಸಾ–ಪುರೋಹಿತನ ಸ್ಥಾನಮಾನಗಳು (ಐತರೇಯ ಬ್ರಾಹ್ಮಣದಲ್ಲಿರುವಂತೆ)>
18-510

<ಪುರೋಹಿತನ ಸಹಾಯದಿಂದ ಯಜ್ಞಾದಿಗಳನ್ನು ಮಾಡುವ ರಾಜನನ್ನು ಅಗ್ನಿಯು ರಕ್ಷಿಸುವನು>
18-514

<ಪುರೋಹಿತನ ಯೋಗ್ಯತೆ ಇತ್ಯಾದಿ–ಮೂರು ವಿಧ ಪುರೋಹಿತರು, ಪುರೋಹಿತನ ಸ್ಥಾನಕ್ಕೆ ಇರುವ ಅರ್ಹತೆ, ಪುರೋಹಿತನನ್ನು ನೇಮಿಸಿಕೊಳ್ಳುವ ಕ್ರಮ ಇತ್ಯಾದಿ>
18-515

<ಪೂತದಕ್ಷ–ಎಂಬ ಶಬ್ದದ ನಾನಾರ್ಥಗಳು>
15-504

<ಪೂರು>
8-604

<ಪೂರ್ವೀಃ–ಎಂಬ ಶಬ್ದ>
6-262

<ಪೂಷನ್‍ ಎಂಬ ದೇವತೆ>
4-437 
5-547 
15-394 
17-167
17-547

<ಪೂಷನ್‍ ಶಬ್ದದ ಅರ್ಥ ವಿವರಣೆ>
3-188 
11-17 
12-163 
14-24 
14-164
17-545

<ಪೂಷಾ, ವಿಷ್ಣು, ಕೇಶಿನ್‍, ವಿಶ್ವಾನರ, ವೃಷಾಕಪಿ (ಬೃ.ದೇ.)>
28-728

<ಪೂಷಾ ಶ್ವೇತಶ್ಚ್ಯಾವಯತು ಎಂಬ ಋಕ್ಕಿನ ನಿರುಕ್ತ>
27-552

<ಪೃಥಿ>
8-757

<ಪೃಥಿವೀಶಬ್ದದ ನಾನಾರ್ಥಗಳು>
13-167

<ಪೃಥಿವೀಶಬ್ದದ ರೂಪನಿಷ್ಪತ್ತಿ, ಅರ್ಥ ಇತ್ಯಾದಿ>
3-113

<ಪೃಥಿವಿಗೂ ಆದಿತ್ಯನಿಗೂ ಇರುವ ಸಂಬಂಧ>
12-338

<ಪೃಥಿವಿಯ ಎರಡು ವಿಧರೂಪಗಳು>
20-219

<ಪೃಥಿವೀ, ಅಂತರಿಕ್ಷ, ದ್ಯುಲೋಕಗಳ ಮೂರು ಅವಾಂತರ ಪ್ರಭೇದಗಳು>
20-187

<ಪೃಥಿವೀ ಮತ್ತು ಮಧ್ಯಮಸ್ಥಾನದ ರೂಪಗಳು (ಬೃ.ದೇ.)>
28-730

<ಪೃಥಿವ್ಯಂತರಿಕ್ಷ ಸ್ವರ್ಗಗಳಲ್ಲಿರುವ ಸರ್ಪಗಳ ವಿಷಯ>
14-347

<ಪೃಥಿವ್ಯಾದಿಜಗದ್ರೂಪಗಳ ವ್ಯಾಹೃತಿಗಳೇ ತನ್ಮೂಲವಾದ ಬ್ರಹ್ಮದ ಉಪಾಸನೆಗೆ ಮುಖ್ಯ ಸಾಧನ>
17-612

<ಪೃಥ್ವೀ>
5-605

<ಪೃಥ್ವಿಯವರ್ಣನೆ>
17-337

<ಪೃಥು ಮತ್ತು ಅಸ್ತ್ರಬುಧ್ನ ಎಂಬುವರ ವಿಷಯ>
30-1162

<ಪೃಥುಶ್ರವಸ್ಸೆಂಬ ರಾಜನನ್ನೂ ವಶನೆಂಬುವನನ್ನೂ ಅಶ್ವಿನೀ ದೇವತೆಗಳು ರಕ್ಷಿಸಿದ ವಿಚಾರ>
9-289

<ಪೃಶ್ನಿಗು>
8-757

<ಪೃಶ್ನಿಶಬ್ದವಿವರಣೆ>
7-153 
13-142

<ಪೃಶ್ನಿಮಾತರಃ ಎಂಬ ಶಬ್ದದ ರೂಪನಿಷ್ಪತ್ತಿ ಅರ್ಥವಿವರಣೆ, ಈ ಹೆಸರು ಮರುದ್ದೇವತೆಗಳಿಗೆ ಬಂದ ವಿಚಾರ ಇತ್ಯಾದಿ>
3-179 
19-626

<ಪೃಷತೀಶಬ್ದದ ವಿವರಣೆ>
4-258 
6-125

<ಪೃಷದ್ಯೋನಿಶಬ್ದಾರ್ಥ ವಿವರಣೆ>
19-402

<ಪೃಷ್ಠ್ಯಷಡಹವೆಂಬ ಯಾಗದ ವಿವರಣೆ>
2-137 
17-485

<ಪೃಕ್ಷಃ ಎಂಬ ಶಬ್ದದ ವಿವರಣೆ>
6-316

<ಪೇದುವೆಂಬ ರಾಜನಿಗೆ ಅಶ್ವಿನೀದೇವತೆಗಳು ಒಂದು ಬಿಳೀಕುದುರೆಯನ್ನು ಕೊಟ್ಟ ವಿಚಾರ>
9-209 
9-346
9-443 
9-495
22-489 
27-962

<ಪ್ರೇತಸಂಸ್ಕಾರ–ಅಗ್ನಿದಗ್ಧಾಃ, ಅನಗ್ನಿದಗ್ಧಾಃ ಎಂಬ ಶಬ್ದಗಳ ವಿವರಣೆ>
27-441


<ಪೈಂಗ್ಯಶಾಖೆ>
1-86

<ಪೈದ್ವ ಎಂಬ ಶಬ್ದದ ವಿವರಣೆ>
5-681

<ಪ್ರೈತೇ ವದಂತು ಎಂಬ ಋಕ್ಕಿನ ನಿರುಕ್ತ>
29-1007

<ಪೌಂಸ್ಯಶಬ್ದದ ವಿವರಣೆ>
11-613

[ಫ]
<ಫಲಿಗಳ ಶಬ್ದಾರ್ಥ ವಿವರಣೆ>
9-581

[ಬ]
<ಬತಃ ಎಂಬ ಶಬ್ದದ ನಿರ್ವಚನ>
27-380

<ಬತೋ ಬತಾಸಿ ಎಂಬ ಋಕ್ಕಿನ ನಿರುಕ್ತ>
27-380

<ಬದ್ಬದೇ ಎಂಬ ಶಬ್ದ>
6-605

<ಬರ್ಹಿ ಎಂಬ ದೇವತೆಯ ವಿಷಯ>
15-686 
28-620
28-756
30-346

<ಬರ್ಹಿಃ ಶಬ್ದಾರ್ಥ ವಿವರಣೆ>
9-1R2 
11-245 
14-237 
15-435 
17-26
19-845

<ಬಹ್ವೀನಾಂ ಪಿತಾ ಎಂಬ ಋಕ್ಕಿನ ನಿರುಕ್ತ>
21-568

<ಬ್ರಹ್ಮಣಸ್ಪತಿಯ ವಿಷಯ>
4-379

<ಬ್ರಹ್ಮಣಸ್ಪತಿ ಮತ್ತು ಬೃಹಸ್ಪತಿ ಎಂಬ ದೇವತೆಗಳ ವಿಷಯದಲ್ಲಿ ವಿಸ್ತಾರವಾದ ವಿವರಣೆಯು>
2-279
14-904

<ಬ್ರಹ್ಮಣಸ್ಪತಿಯ ಮಂತ್ರಶಕ್ತಿ ಮಹಿಮೆ>
15-24

<ಬ್ರಹ್ಮಣಸ್ಪತಿ (ಬೃ.ದೇ.)>
28-721

<ಬ್ರಹ್ಮಣಸ್ಕವೇ ಎಂಬ ಶಬ್ದದ ಅರ್ಥಾನುವಾದ>
20-615

<ಬ್ರಹ್ಮನ್‍ ಶಬ್ದದ ವಿವರಣೆ>
16-244

<ಬ್ರಹ್ಮನೆಂಬ ಋತ್ವಿಜನ ಕರ್ತವ್ಯ>
26-709

<ಬ್ರಹ್ಮವಾಹ>
8-305

<ಬ್ರಹ್ಮಜ್ಞಾನದ ರಸಾನುಭವವರ್ಣನೆ>
12-553

<ಬ್ರಹ್ಮ ಶಬ್ದದ ನಾನಾರ್ಥಗಳು>
10-186
11-538

<ಬ್ರಹ್ಮ ವಸ್ತುವಿನ ಸ್ವರೂಪ>
30-602

<ಬ್ರಹ್ಮವಲ್ಲದೆ ಬೇರೆ ವಸ್ತುವಿಲ್ಲ>
30-765

<ಬ್ರಹ್ಮೋದ್ಯದ ವಿಷಯ>
28-225

<ಬ್ರಹ್ಮೋದ್ಯ ರೂಪವಾದ ಪ್ರಶ್ನೆಗಳಿಗೆ ಉತ್ತರ>
12-519

<ಬ್ರಹ್ಮೋಪಾಸನೆ ಇಲ್ಲದೆ ಬ್ರಹ್ಮಸಾಕ್ಷಾತ್ಕಾರವಿಲ್ಲ ಎಂಬ ತತ್ತ್ವ ಪ್ರತಿಪಾದನೆ>
12-585

<ಬಾಣಸ್ತುತಿ, ಸುಪರ್ಣಂ ವಸ್ತೇ ಎಂಬ ಋಕ್ಕಿನ ನಿರುಕ್ತ>
21-580

<ಬಾಲಾಕಿ, ಸ್ವೇತಕೇತು ಎಂಬುವರ ವಿಷಯ>
12-327

<ಬಾಷ್ಕಲ ಸಂಹಿತಾ>
1-64

<ಬಾಷ್ಕಲ ಸಂಹಿತಾ ಕ್ರಮ>
1-46

<ಬಾಷ್ಕಲ ಶಾಖೆಗಳು>
1-61

<ಬಾಹ್ಯಯಜ್ಞಕ್ಕಿಂತಲೂ ಜ್ಞಾನಯಜ್ಞವು ಉತ್ತಮವಾದದ್ದು>
12-650

<ಬ್ರಾಹ್ಮಣಗಳು>
1-23

<ಬ್ರಾಹ್ಮಣ ಮಂತ್ರಭಾಗಗಳ ಪ್ರಾಮಾಣ್ಯ ವಿಚಾರ ವಿಮರ್ಶೆ>
1-412

<ಬ್ರಾಹ್ಮಣಾದಿವರ್ಣಗಳ ಉತ್ಪತ್ತಿ>
29-911

<ಬ್ರಾಹ್ಮಣೋಽಸ್ಯ ಮುಖಮಾಸೀತ್‍ ಎಂಬ ಋಕ್ಕಿನ ವಿವರಣೆ>
29-909

<ಬೀರಿಟ ಎಂಬ ಶಬ್ದದ ಅರ್ಥ ವಿವರಣೆ>
22-218

<ಬುಂದ ಎಂಬ ಶಬ್ದದ ವಿವರಣೆ>
25-165
25-177

<ಬುಧ್ನ ಶಬ್ದದ ವಿವರಣೆ>
8-96
22-158

<ಬೃಬದುಕ್ಥಂ ಮತ್ತು ಸೃಪಃ ಎಂಬ ಶಬ್ದಗಳ ನಿರ್ವಚನ ಮತ್ತು ಅರ್ಥವಿವರಣೆ>
24-236

<ಬೃಹದ್ದೇವತಾಗ್ರಂಥದಲ್ಲಿ ಉಕ್ತವಾಗಿರುವ ಉಪಾಖ್ಯಾನ ಮತ್ತು ಇತಿಹಾಸಗಳು (ಬೃ.ದೇ.)>
28-1031

<ಬೃಹದ್ದೇವತಾಗ್ರಂಥದಲ್ಲಿ ಉಕ್ತವಾಗಿರುವ ಋಗ್ವೇದಮಂತ್ರಗಳು ಆಕಾರಾದಿವರ್ಣಾನುಕ್ರಮಣಿಕೆ (ಬೃ.ದೇ.)>
28-1035

<ಬೃಹದ್ರಥನ ವಿಷಯ>
4-243

<ಬೃಹಸ್ಪತಿ ಎಂಬ ದೇವತೆಯ ವಿಷಯ>
5-622 
14-299 
18-495 
28-553
28-721

<ಬೃಹಸ್ಪತಿ ಶಬ್ದದ ನಿರ್ವಚನ>
14-299

<ಬೃಹಸ್ಪತಿ ದೇವತೆಗೆ ಸಂಬಂಧಪಟ್ಟ ವಿಷಯಗಳು>
14-300

<ಬೃಹಸ್ಪತಿಯ ನಾನಾಕಾರ್ಯಗಳ ಸೂಕ್ಷ್ಮಪರಿಚಯ>
14-302

<ಬೃಹಸ್ಪತಿಯ ಆಯುಧವಿಷಯ>
14-318

<ಬೃಹಸ್ಪತಿಯೇ ಎಲ್ಲಾ ಮಂತ್ರಗಳಿಗೂ ಕರ್ತೃವು>
14-917

<ಬೃಹಸ್ಪತಿಗೂ ವಿಶ್ವೇದೇವತೆಗಳಿಗೂ ಇರುವ ನಿಕಟಸಂಬಂಧ>
17-538

<ಬೃಹಸ್ಪತೇ ಯದರ್ಯಃ ಎಂಬ ಋಕ್ಕಿನ ಅರ್ಥವಿವರಣೆ, ವಿಮರ್ಶೆ ಮತ್ತು ಐತರೇಯ ಬ್ರಾಹ್ಮಣದಲ್ಲಿ ಹೇಳಿರುವಂತೆ ಈ ಋಕ್ಕಿನ ವಿನಿಯೋಗ, ಅರ್ಥವಿಮರ್ಶೆ ಇತ್ಯಾದಿ>
14-942

<ಬೃಹಸ್ಪತಿಯ ವಾಹನವಾದ ವಿಶ್ವರೂಪವು>
17-543

<ಬೃಹಸ್ಪತಿಯು ಅಗ್ನಿಯನ್ನು ಕಂಡ ವಿಚಾರ>
28-115

<ಬೃಹಸ್ಪತಿಯು ಮನು ಅಪಹರಿಸಿದ್ದ ಗೋವುಗಳನ್ನು ಹಿಂದಕ್ಕೆ ತಂದ ವಿಚಾರ>
28-561

<ಬೇಕನಾಟಾನ್‍ ಎಂಬ ಶಬ್ದದ ನಿರ್ವಚನ ಮತ್ತು ಅರ್ಥವಿವರಣೆ>
24-771

<ಬೌಧ್ಯಶಾಖಾ>
1-62

[ಭ]
<ಭಗನೆಂಬ ದೇವತೆಯ ವಿಷಯ>
19-516 
22-235

<ಭಗಶಬ್ದಾರ್ಥ>
10-92

<ಭಟ್ಟಭಾಸ್ಕರ>
1-175
1-259

<ಭದ್ರ ಶಬ್ದಾರ್ಥ ವಿವರಣೆ>
8-35
10-123

<ಭರತನೆಂಬ ರಾಜನ ವಿಷಯ–ಇವನ ಪಟ್ಟಾಭಿಷೇಕ ವರ್ಣನೆ (ಐತರೇಯ ಬ್ರಾಹ್ಮಣದಲ್ಲಿರುವಂತೆ)>
20-568

<ಭರತರಾಜನಿಗೂ ವಿಶ್ವಾಮಿತ್ರನಿಗೂ ಇರುವ ಸಂಬಂಧ>
17-204

<ಭರದ್ವಾಜಋಷಿಯ ಪರಿಚಯ>
20-340

<ಭರದ್ವಾಜಋಷಿಯ ಜನ್ಮವೃತ್ತಾಂತ (ಬೃ.ದೇ.)>
28-875

<ಭರದ್ವಾಜ ಎಂಬ ಋಷಿಯನ್ನು ಅಶ್ವಿನೀದೇವತೆಗಳು ರಕ್ಷಿಸಿದ ವಿಚಾರ>
8-784

<ಭರ್ಗ (ಗಾಯತ್ರಿಯಲ್ಲಿರುವ) ಎಂಬ ಶಬ್ದದ ವಿವರಣೆ>
17-630

<ಭಾನು ಶಬ್ದದ ಅರ್ಥವಿವರಣೆ>
5-61

<ಭಾವನಾ ದೇವತೆಗಳ ವಿಷಯ–ಶ್ರದ್ಧಾ, ಮನ್ಯು, ಮೇಧಾ, ಪ್ರಜ್ಞಾ, ಸತ್ಯ, ತಪಸ್ಸು, ದಮ, ಶಮ, ದಾನ, ಧರ್ಮ, ಯಜ್ಞ, ಮನಸ್ಸು, ಸಂವತ್ಸರ ಅಥವಾ ಕಾಲ, ದಕ್ಷಿಣಾ, ಆನಂದ, ಶಾಂತಿ ಇತ್ಯಾದಿ>
29-311

<ಭಾವಯವ್ಯನಿಗೂ ಅವನ ಪತ್ನಿ ರೋಮಶೆಗೂ ನಡೆದ ಸಂವಾದ>
10-242

<ಭಾರತ ಶಬ್ದಾರ್ಥ ವಿವರಣೆ>
14-519

<ಭಾರತೀ ಇಳಾ ಸರಸ್ಪತೀ ಎಂಬ ದೇವತೆಗಳ ವಿಷಯ>
11-263 
14-254

<ಭಿಷಜಾ ಎಂಬ ಶಬ್ದದ ವಿವರಣೆ>
25-267

<ಭುಜ್ಯುವಿನ ವಿಷಯ>
22-487

<ಭುಜ್ಯು (ತುಗ್ರಪುತ್ರನಾದ) ವನ್ನು ಅಶ್ವಿನೀದೇವತೆಗಳು ಸಮುದ್ರ ಮಧ್ಯದಲ್ಲಿ ರಕ್ಷಿಸಿದ ವಿಚಾರ>
8-812
8-753
9-194 
9-369 
9-433 
9-467 
9-484
13-523
13-616
27-948

<ಭುರಣ್ಯುಃ>
6-224

<ಭುರಣ್ಯಂತಂ ಎಂಬ ಶಬ್ದದ ವಿವರಣೆ>
5-124

<ಭುರಣ್ಯತಿ ಶಬ್ದಾರ್ಥ ವಿವರಣೆ>
11-617

<ಭುವನಸ್ಯ ನಾಭಿಂ ಎಂಬ ಶಬ್ದಗಳ ವಿವರಣೆ>
14-79

<ಭೂತ, ಪಿಶಾಚ ಇತ್ಯಾದಿ>
10-554

<ಭೂತಾಂಶ ಕಾಶ್ಯಪ (ಬೃ.ದೇ.)>
28-997

<ಭೂಮಿಃ ಎಂಬ ಖಿಲಸೂಕ್ತ (ಬೃ.ದೇ.)>
28-1007

<ಭೂಮಿಯ ಮೇಲೆ ವೃಷ್ಟಿಯಾಗುವುದನ್ನು ದೇವತೆಗಳು ಹೇಗೆ ತಿಳಿಯುವರು ಎಂಬ ವಿಷಯ>
19-777

<ಭೂರಾದಿಲೋಕಗಳ ಸೃಷ್ಟಿಕ್ರಮ>
29-221

<ಭೂರಿಜನ್ಮಾ ಎಂಬ ಶಬ್ದ>
18-696

<ಭೃಗವಃ ಎಂಬ ಶಬ್ದದ ವಿಶೇಷ ವಿವರಣೆ>
11-309 
20-535

<ಭೃಗವಃ ಎಂಬ ಪಿತೃಗಣಗಳು>
27-467

<ಭೃಗುಋಷಿಯ ಉತ್ಪತ್ತಿ>
27-365 
28-417

<ಭೃಗುಋಷಿಯ ವೃತ್ತಾಂತ>
15-573

<ಭೃಗುಗಳು>
5-669

<ಭೃಗು ಮೊದಲಾದ ಋಷಿಗಳು ಅಗ್ನಿಯನ್ನು ಕಂಡ ವಿಚಾರ>
28-104

<ಭೃಗು, ಅಂಗಿರಸರು, ಅತ್ರಿ ಎಂಬ ಋಷಿಗಳ ಜನ್ಮವೃತ್ತಾಂತ (ಬೃ.ದೇ.)>
28-874

<ಭೃಮಯಃ ಎಂಬ ಶಬ್ದದ ವಿವರಣೆ>
17-528

[ಮ]
<ಮಂಗಲಕರವಾದ ಹೆಸರುಗಳು (ಬೃ.ದೇ.) ನಾನಾವಿಧ ಮಂತ್ರಗಳು>
28-685

<ಮಘವತ್‍ ಶಬ್ದದ ಅರ್ಥವಿವರಣೆ, ಪ್ರಯೋಗ>
13-205

<ಮಘವಾನಾ ಎಂಬ ಶಬ್ದದ ವಿವರಣೆ>
14-14

<ಮಜ್ಮನಾ ಎಂಬ ಶಬ್ದದ ವಿವರಣೆ>
14-804

<ಮಂಡಲ ವಿಭಾಗ ಕ್ರಮ>
2-4

<ಮಂಡಲಾಂತ್ಯದಲ್ಲಿ ಪಠಿಸಬೇಕಾದ ಕೆಲವು ಶ್ಲೋಕಗಳು>
20-337

<ಮಂಡಲದ (ಹತ್ತನೆಯ) ಪೀಠಿಕೆ>
27-177

<ಮಂಡೂಕ>
5-684

<ಮಂಡೂಕ ಶಬ್ದದ ರೂಪನಿಷ್ಪತ್ತಿ, ಯಾಸ್ಕರ ನಿರ್ವಚನ ಇತ್ಯಾದಿ>
23-226

<ಮಂತ್ರಗಳಿಗೆ ಇರುವ ಲಕ್ಷಣ ಮೊದಲಾದ ಸ್ವರೂಪ ನಿರ್ಣಯ>
1-476

<ಮಂತ್ರಗಳ ಋಷಿ ದೇವತಾ ಛಂದಸ್ಸುಗಳನ್ನು ತಿಳಿದುಕೊಳ್ಳಬೇಕಾದ ವಿಚಾರ>
1-595

<ಮಂತ್ರಗಳಲ್ಲಿ ನಿರ್ದಿಷ್ಟರಾದ ದೇವತೆಗಳನ್ನು ತಿಳಿದುಕೊಳ್ಳಬೇಕಾದ ಅವಶ್ಯಕತೆ (ಬೃ.ದೇ.)>
28-677
28-1030

<ಮಂತ್ರಗಳಲ್ಲಿ ಸ್ತುತರಾದ ದೇವತೆಗಳನ್ನು ತಿಳಿಯುವ ಬಗೆ (ಬೃ.ದೇ.)>
28-777

<ಮಂತ್ರಗಳ ವಿಭಾಗ ಕ್ರಮ–ದೈವಿಕ ಮಂತ್ರ, ಸ್ತುತಿಮಂತ್ರ, ಆಶೀರ್ಮಂತ್ರ, ಶಪಥ, ಅಭಿಶಾಪ, ಕಶ್ಚಿದ್ಭಾವ, ಪರಿದೇವನಾ, ನಿಂದಾ, ಪ್ರಶಂಸಾ ಇತ್ಯಾದಿ ಅರ್ಥಾನುಸಾರವಾಗಿ>
2-205

<ಮಂತ್ರಗಳ ಅರ್ಥಾನುಸಾರವಾಗಿ ಪರೋಕ್ಷಕೃತ, ಪ್ರತ್ಯಕ್ಷಕೃತ, ಆಧ್ಯಾತ್ಮಿಕ ಎಂಬ ಪ್ರಭೇದಗಳ ವಿವರಣೆ–ಉದಾಹರಣೆ ಸಹಿತವಾಗಿ>
23-262

<ಮಂದೇಹಾರಣ ಎಂಬ ದ್ವೀಪವಸಿಗಳಾದ ಅಸುರರ ವಿಷಯ>
19-838
30-610

<ಮಧುವಿದ್ಯೆಯ ವಿಚಾರದಲ್ಲಿ ಸ್ಕಂದಸ್ವಾಮಿಗಳೂ, ಸಾಯಣರೂ ಹೇಳಿರುವ ಪೂರ್ತೀತಿಹಾಸವು>
20-109

<ಮಧು ಶಬ್ದದ ನಾನಾರ್ಥಗಳು>
17-17

<ಮಧುಮತೀ ಎಂಬ ಶಬ್ದದ ವಿವರಣೆ>
17-405

<ಮಂಧಾತಾ>
8-784

<ಮಧ್ಯಮಸ್ಥಾನ ದೇವತೆಗಳು>
28-712

<ಮಧ್ಯಮಸ್ಥನ (ಅಂತರಿಕ್ಷ) ದೇವತೆಗಳು ಮತ್ತು ಇಂದ್ರನಿಗೆ ಸಂಬಂಧಪಟ್ಟ ಇತರ ದೇವತೆಗಳು>
28-709

<ಮನೀಷಾ ಎಂಬ ಶಬ್ದದ ಅರ್ಥವಿವರಣೆ>
6-262
17-382

<ಮನುವಿನ ಉತ್ಪತ್ತಿ>
27-534

<ಮನುವು ತನ್ನ ಪುತ್ರರಿಗೆ ದಾಯಭಾಗವನ್ನು ಹಂಚಿಕೊಟ್ಟ ವಿಚಾರ ಮತ್ತು ನಾಭಾನೇದಿಷ್ಠನ ವಿಷಯ>
28-399

<ಮನುವಿನ ವಿಷಯ>
3-536
4-545 
5-668 
6-617 
9-107
14-574
18-178
26-675

<ಮನುಷ್ಯರಿಗಿರುವಂತೆಯೇ ದೇವತೆಗಳಿಗೆ ದೇಹಾದ್ಯಾಕಾರಗಳು ಹಸ್ತಾದ್ಯವಯವಗಳು ಆಹಾರ ಭಕ್ಷ್ಯ ಇತ್ಯಾದಿಗಳಿರುವದೇ ಎಂಬ ವಿಷಯ ವಿಮರ್ಶೆ>
30-575


<ಮನೋತಾ ದೇವತೆಗಳ ವಿಷಯ ಇತ್ಯಾದಿ>
20-362

<ಮನ್ಮನ ಶಬ್ದಾರ್ಥ ವಿವರಣೆ>
11-144
16-194

<ಮನ್ಮಾನಿ ಎಂಬ ಶಬ್ದದ ವಿವರಣೆ>
12-719

<ಮನ್ಯುದೇವತೆ>
5-644
28-724

<ಮನ್ಯು ದೇವತೆಯ ಸ್ವರೂಪ>
29-314

<ಮನ್ಯು ಶಬ್ದ ನಿರ್ವಚನ>
8-198 
15-16

<ಮನ್ಯುಸೂಕ್ತದ ವಿಷಯದಲ್ಲಿ ವಿನಿಯೋಗ, ಫಲಶೃತಿ ಇತ್ಯಾದಿ>
29-315

<ಮಮತಾ ಎಂಬ ಸ್ತ್ರೀಯ ಪುತ್ರನಾದ ದೀರ್ಘತಮಾಃ ಎಂಬ ಋಷಿಯ ಅಂಧತ್ವವನ್ನು ಅಗ್ನಿಯು ಪರಿಹಾರ ಮಾಡಿದ ವಿಚಾರ>
17-782

<ಮರುದ್ದೇವತೆಗಳ ವಿಷಯ>
4-253
5-593
7-147 
19-594
22-333

<ಮರುದ್ದೇವತೆಗಳು ರುದ್ರಪುತ್ರರೆಂಬ ವಿಚಾರ>
6-102 
8-193
13-13
15-439

<ಮರುದ್ದೇವತೆಗಳ ವಿಶೇಷ ವಿವರಣೆ>
19-597
22-236

<ಮರುದ್ದೇವತೆಗಳ ವೈಶಿಷ್ಟ್ಯ ಮತ್ತು ಗುಣವರ್ಣನೆ>
19-599

<ಮರುದ್ದೇವತೆಗಳ ವಿವರಣೆ–ಸೂಕ್ತಗಳಲ್ಲಿಯೂ ಋಕ್ಕಗಳಲ್ಲಿಯೂ ಕಂಡುಬರುವಂತೆ>
22-237

<ಮರುತಃ ಎಂಬ ಶಬ್ದದ ನಿಷ್ಪತ್ತಿ, ಅರ್ಥವಿವರಣೆ ಇತ್ಯಾದಿ>
7-146 
11-660
16-278
19-598
22-337

<ಮರುದ್ದೇವತೆಗಳ ಉತ್ಪತ್ತಿ ವಿಚಾರ>
13-148
17-188

<ಮರುದ್ಗಣಗಳ ವಿಷಯ>
26-731

<ಮರುದ್ದೇವತೆಗಳಲ್ಲಿ ಸಪ್ತಗಣಗಳಿರುವ ವಿಚಾರ>
24-189

<ಮರುತ್ತುಗಳು ಇಂದ್ರನಿಗೆ ಸಹಾಯ ಮಾಡಿದ ವಿಚಾರ>
7-172

<ಮರುದ್ದೇವತೆಗಳ ವಿಷಯದಲ್ಲಿ ಒಂದು ಇತಿಹಾಸವು (ಬೃ.ದೇ.)>
28-816

<ಮರುತ್ತುಗಳು, ಇಂದ್ರ ಮತ್ತು ಅಗಸ್ತ್ಯ ಋಷಿ–ಸೂಕ್ತ ೧೬೯–೧೭೦ (ಬೃ.ದೇ.)>
28-817

<ಮರುದ್ದೇವತೆಗಳ ಆಭರಣ ವಿವರಣೆ>
19-635 
19-786

<ಮರುದ್ದೇವತೆಗಳಿಗೆ ರುದ್ರಾಃ ಎಂಬ ಹೆಸರು ಬರಲು ಕಾರಣ>
19-66O

<ಮರುದ್ದೇವತೆಗಳು ಗೋತಮ ಋಷಿಯ ದಾಹಶಮನಾರ್ಥವಾಗಿ ಜಲಪೂರ್ಣವಾದ ಒಂದು ಬಾವಿಯನ್ನು ತಂದು ಒದಗಿಸಿದ ವಿಚಾರ>
19-727

<ಮರುದ್ದೇವತೆಗಳಲ್ಲಿ ಎಲ್ಲರೂ ಒಂದೇ ವಿಧವಾಗಿಯೂ ಸಮಾನರಾಗಿಯೂ ಇರುವರೆಂಬ ವಿಚಾರ>
19-754

<ಮರುದ್ದೇವತೆಗಳಲ್ಲಿ ಜ್ಯೇಷ್ಠ, ಕನಿಷ್ಠ ಎಂಬ ಭೇದವಿಲ್ಲದೆ ಎಲ್ಲರೂ ಸಮಾನರೆಂಬ ವಿಷಯ>
19-775

<ಮರುತ್ತುಗಳಿಗೂ ಇಂದ್ರನಿಗೂ ಇರುವ ಸಾಹಚರ್ಯ>
19-596

<ಮರುತ್‍ ಶಬ್ದದ ಅರ್ಥ ವಿಷಯದಲ್ಲಿ ಪಾಶ್ಚಾತ್ಯ ಪಂಡಿತರ ಅಭಿಪ್ರಾಯ>
16-278

<ಮರುತ್ವತೀಯ ಶಸ್ತ್ರದ ಸ್ವರೂಪ ಮತ್ತು ವಿವರಣೆ>
13-238

<ಮರ್ತಭೋಜನಂ ಎಂಬ ಶಬ್ದದ ಅರ್ಥವಿವರಣೆ–ಉದಾಹರಣೆ ಸಹಿತವಾಗಿ>
7-22

<ಮರ್ತ್ಯ ಎಂಬ ಶಬ್ದದ ವಿವರಣೆ>
6-489

<ಮರ್ಯಃ ಎಂಬ ಶಬ್ದದ ವಿವರಣೆ>
27-974

<ಮಹಾಕೌಷೀತಕೀಶಾಖಾ>
1-78

<ಮಹಾನಾಈ ಋಕ್ಕುಗಳು (ಬೃ.ದೇ.)>
28-1019

<ಮಹಾನಿಂದ್ರ ಎಂದು ಇಂದ್ರನಿಗೆ ಹೆಸರುಬರಲು ಕಾರಣ ಈ ವಿಷಯದಲ್ಲಿ ಪೂರ್ವೇತಿಹಾಸ ಇತ್ಯಾದಿ>
20-702
28-142

<ಮಹಾ ಪಿತೃಯಜ್ಞ>
27-446

<ಮಹಾವ್ರತದ ವಿವರಣೆ>
2-150

<ಮಹೀಧರ>
1-264

<ಮಾಂಡೂಕೇಯ ಶಾಖೆಗಳು>
1-80

<ಮಾತರಿಶ್ವಾ ಎಂಬ ದೇವತೆ>
5-586
29-441

<ಮಾತರಿಶ್ವಾ ಎಂಬ ಶಬ್ದದ ರೂಪನಿಷ್ಪತ್ತಿ, ಅರ್ಥವಿವರಣೆ, ಮಾತರಿಶ್ವನು ಅಗ್ನಿಯನ್ನು ಕಂಡ ವಿಚಾರ ಇತ್ಯಾದಿ>
8-16
10-331 
15-598
15-766 
16-611
6-300

<ಮಾತರಿಶ್ವನು ಅಗ್ನಿಯನ್ನು ಭೂಮಿಗೆ ತಂದ ವಿಚಾರ>
15-769
28-114

<ಮಾತ್ವಾ ಸೋಮಸ್ಯ ಎಂಬ ಋಕ್ಕಿನ ನಿರುಕ್ತ ಮತ್ತು ಗಲ್ದಯಾ ಎಂಬ ಶಬ್ದದ ನಿರ್ವಚನ>
23-315

<ಮಾದಯಧ್ಯೈ ಎಂಬ ಶಬ್ದದ ಅರ್ಥವಿವರಣೆ>
12-116

<ಮಾಂದಾರ್ಯಸ್ಯ ಎಂಬ ಶಬ್ದ>
17-725

<ಮಾಧ್ಯಂದಿನ ಸವನ>
26-47

<ಮಾಧ್ಯಂದಿನ ಸವನದ ವಿವರಣೆಯು>
26-59

<ಮಾಧ್ಯಂದಿನ ಸವನ ಕಾಲದಲ್ಲಿ ಪಠಿಸಬೇಕಾದ ಮಂತ್ರಗಳು>
15-694
27-169

<ಮಾಧ್ಯಂದಿನಸವನ ಕಾಲದಲ್ಲಿ ಪಠಿಸಬೇಕಾದ ಉನ್ನೀಯಮಾನ ಸೂಕ್ತವು>
15-700
26-62

<ಮಾಧ್ಯಂದಿನ ಸವನದ ಮಂತ್ರಗಳು ಉದಾಹರಣೆ ಸಹಿತವಾಗಿ>
15-706

<ಮಾನಃ ಸಮಸ್ಯ ಎಂಬ ಋಕ್ಕಿನ ನಿರುಕ್ತ>
25-139


<ಮಾನಸ್ಯ ಸೂನುಃ ಎಂಬ ಶಬ್ದಗಳ ವಿವರಣೆ>
14-295

<ಮಾನಾನಃ ಮಾನ್ಯ ಎಂಬ ಶಬ್ದಗಳ ನಾನಾರ್ಥಗಳು ಮತ್ತು ಪ್ರಯೋಗ>
13-215
14-36

<ಮಾನುಷ ಎಂಬ ಶಬ್ದದ ರೂಪನಿಷ್ಪತ್ತಿ>
27-364

<ಮಾನ್ಯಸ್ಯ ಎಂಬ ಶಬ್ದ>
12-725

<ಮಾನ್ಯನೆಂಬ ಹೆಸರು ಅಗಸ್ತ್ಯಋಷಿಗೆ ಬರಲು ಕಾರಣ>
13-621

<ಮಾಯಾ ಮಾಯಾಭಿಃ, ಮಾಯಿನಂ ಎಂಬ ಶಬ್ದಗಳ ವಿವರಣೆ>
6-586 
11-514
17-324

<ಮಾರುತಂ ಗಣಂ, ಮಾರುತಂ ಶರ್ಧಃ ಎಂಬ ಶಬ್ದಗಳ ವಿವರಣೆ>
14-615
19-744

<ಮಿತದ್ರವಃ ಎಂಬ ಶಬ್ದದ ಅರ್ಥವಿವರಣೆ>
22-214

<ಮಿತ್ರ ಎಂಬ ದೇವತೆ>
5-539

<ಮಿತ್ರದೇವನ ವಿಷಯ, ಮಹಿಮೆ ಇತ್ಯಾದಿ>
14-200

<ಮಿತ್ರನು ಅಹರಭಿಮಾನಿ ಎಂಬ ವಿಷಯ>
11-201

<ಮಿತ್ರ ಶಬ್ದಾರ್ಥ ವಿವರಣೆ>
14-614

<ಮಿತ್ರನ ವೈಯಕ್ತಿಕವಾದ ಸ್ವರೂಪ ವರ್ಣನೆ>
20-12

<ಮಿತ್ರದೇವನ ಶಕ್ತಿಸ್ವರೂಪಾದಿಗಳು>
17-463

<ಮಿತ್ರಸ್ಯ ಚರ್ಷಣೀಧೃತಃ ಎಂಬ ಋಕ್ಕಿನ ವಿಶೇಷ ವಿವರಣೆ>
17-469

<ಮಿತ್ರ, ವರುಣ, ಅರ್ಯಮಾ ಎಂಬ ದೇವತೆಗಳ ವಿಷಯ>
14-119
15-355 
28-506
29-45

<ಮಿತ್ರಾವರುಣರ ಸೂಕ್ಷ್ಮಪರಿಚಯ>
4-395
19-827
20-1

<ಮಿತ್ರಾವರುಣರ ಸಾಹಚರ್ಯದ ಸ್ವರೂಪ ಮತ್ತು ಅವರ ಕ್ರಿಯೆಗಳು>
19-831

<ಮಿತ್ರಾವರುಣರು ರಾಜಾನಾ ಎಂಬ ಶಬ್ದದಿಂದ ಕರೆಯಲ್ಪಡುವ ವಿಚಾರ>
6-326

<ಮಿತ್ರಾವರುಣರ ವಿಮರ್ಶೆ>
22-402

<ಮಿತ್ರಾವರುಣರಿಗೆ ವಿರಾಟ್‍ ಎಂಬ ಛಂದಸ್ಸು, ಇಂದ್ರನಿಗೆ ತ್ರಿಷ್ಟುಪ್‍ಛಂದಸ್ಸು, ವಿಶ್ವೇದೇವತೆಗಳಿಗೆ ಜಗತೀಛಂದಸ್ಸು ಪ್ರಿಯವೆಂಬ ವಿಚಾರ>
30-796

<ಮಿತ್ರಾವರುಣರ ವಿಷದಲ್ಲಿ Roth ಎಂಬ ಜರ್ಮನ್‍ ಪಂಡಿತನ ಅಭಿಪ್ರಾಯ>
19-833

<ಮಿತ್ರೋ ಜನಾನ್‍ ಎಂಬ ಋಕ್ಕಿನ ನಿರುಕ್ತ>
17-442

<ಮಿಥುನಾ ಎಂಬ ಶಬ್ದದ ನಿರ್ವಚನ, ರೂಪನಿಷ್ಪತ್ತಿ, ಅರ್ಥವಿವರಣೆ ಇತ್ಯಾದಿ>
27-989
29-539

<ಮಿಮ್ಯಕ್ಷ ಯೇಷು ರೋದಸೀ ಎಂಬ ಋಕ್ಕಿನ ನಿರುಕ್ತ>
21-283

<ಮುಂಜ ಎಂಬ ಶಬ್ದದ ವಿವರಣೆ>
27-857

<ಮುದ್ಗಲಋಷಿ ಮತ್ತು ಅವನ ದ್ರುಘಣದ ವಿಷಯ>
30-171

<ಮುದ್ಗಲಶಾಖಾ>
1-54

<ಮುಹೂರ್ತ ಶಬ್ದಾರ್ಥ ವಿಚಾರ>
27-218

<ಮೂರಾಃ ಎಂಬ ಶಬ್ದದ ವಿವರಣೆ>
17-59

<ಮೂರುವಿಧ ಅನ್ನಗಳು>
18-459

<ಮೂರುವಿಧ ಅಗ್ನಿಗಳು>
28-56 
28-700

<ಮೂರುವಿಧವಾದ ಅಗ್ನಿಗಳು, ಸಂವತ್ಸರ (ಬೃ.ದೇ.)>
28-813

<ಮೂರುವಿಧವಾದ ವಿಶ್ವೇದೇವಸೂಕ್ತಗಳು (ಬೃ.ದೇ.)>
28-767

<ಮೂರ್ತಾಮೂರ್ತ, ಮರ್ತ್ಯಾಮರ್ತ್ಯ, ಸ್ಥಾವರ ಜಂಗಮ ಇತ್ಯಾದಿ ವಿವರಣೆ>
12-559

<ಮೂವತ್ತಮೂರು ದೇವತೆಗಳ ವಿಷಯ>
4-123 
15-790
26-669

<ಮೃಗಃ ಎಂಬ ಶಬ್ದ>
13-255

<ಮೃಗೋನ ಭೀಮಃ ಎಂಬ ಋಕ್ಕಿನ ನಿರುಕ್ತ>
30-1226

<ಮೃತನ ಪತ್ನಿಯು ತನ್ನ ಪತಿಯೊಡನೆ ಸಹಗಮನ ಮಾಡುವುದನ್ನು ಅವಳ ತಡೆಯಬೇಕೆಂಬ ವಿಚಾರ>
27-611

<ಮೃತ್ಯು>
5-707
28-727

<ಮೃತ್ಯು, ಕಾಲವಿಭಗ ಇತ್ಯಾದಿ>
30-762

<ಮೇಧಾಃ ಎಂಬ ಶಬ್ದದ ವಿವರಣೆ>
17-417

<ಮೇಧಾಸೂಕ್ತ (ಬೃ.ದೇ.)>
28-1008

<ಮೇಧ್ಯಾತಿಥಿಃ>
4-213
5-677

<ಮೇಧ್ಯಾಶ್ವಕ್ಕೂ ಅಗ್ನೀಂದ್ರಾದಿದೇವತೆಗಳಿಗೂ ಇರುವ ಸಂಬಂಧ>
12-237


<ಮೇಹನಾ ಎಂಬ ಶಬ್ದಕ್ಕೆ ಯಾಸ್ಕರ ನಿರ್ವಚನ>
19-309

<ಮೈತ್ರಾಯಣೀಯ ಸಂಹಿತೆಯ ಪದಪಾಠಕಾರರು>
1-288

<ಮೈನಮಗ್ನೇ ಎಂಬ ಸೂಕ್ತದ ಪೀಠಿಕೆ>
27-510


[ಯ]
<ಯಃ (ಗಾಯತ್ರಿಯಲ್ಲಿರುವ) ಶಬ್ದದ ವಿವರಣೆ>
17-634

<ಯ ಇಮಾ ಎಂಬ ಋಕ್ಕಿನ ವಿಶೇಷಾರ್ಥವಿವರಣೆ>
29-193

<ಯ ಇಮೇ ದ್ಯಾವಾಪೃಥಿವೀ ಎಂಬ ಋಕ್ಕಿನ ನಿರುಕ್ತ>
30-363

<ಯಂ ಕುಮಾರ ಎಂಬ ಋಕ್ಕಿನ ಆದಿತ್ಯ ಪರವಾದ ಅರ್ಥ>
30-857

<ಯಂ ಕುಮಾರ ಎಂಬ ಋಕ್ಕಿಗೆ ಯಮನ ಪರವಾದ ಅರ್ಥ>
30-857

<ಯಂ ಕುಮಾರ ಪ್ರ ಎಂಬ ಋಕ್ಕಿಗೆ ಆದಿತ್ಯಪರವಾದ ಅರ್ಥ>
30-860

<ಯಂ ಕುಮಾರ ಪ್ರ ಎಂಬ ಋಕ್ಕಿಗೆ ಯಮನ ಪರವಾದ ಅರ್ಥ>
30-859

<ಯಂ ಕ್ರದಸೀ ಎಂಬ ಋಕ್ಕಿನ ಅರ್ಥವಿಮರ್ಶೆ>
14-667

<ಯಜುರ್ವೇದ ಸಂಹಿತಾ>
1-17

<ಯಜುರ್ವೇದದ ಶಾಖೆಗಳು>
1-95

<ಯಜುರ್ವೇದದ ಭಾಷ್ಯಕಾರರು>
1-261

<ಯಜುರ್ವೇದದ ಪದಪಾಠಕಾರರು>
1-287

<ಯಜುರ್ವೇದಕ್ಕೆ ಸಾಯಣಾಚಾರ್ಯರು ಮೊದಲು ಭಾಷ್ಯವನ್ನು ಬರೆದ ವಿಷಯದಲ್ಲಿ ಪೂರ್ವಪಕ್ಷವು>
1-345

<ಯಜುರ್ವೇದಕ್ಕೆ ಸಾಯಣಾಚಾರ್ಯರು ಮೊದಲು ಭಾಷ್ಯವನ್ನು ಬರೆದ ವಿಷಯದಲ್ಲಿ ಆಕ್ಷೇಪಕ್ಕೆ ಸಮಾಧಾನವು>
1-355

<ಯಜ್ಞಗಳ ಪ್ರಭೇದಗಳು-ಇಪ್ಪತ್ತೊಂದು ವಿಧ>
3-32
6-359

<ಯಜ್ಞದಲ್ಲಿ ಹವಿಸ್ಸು ಅತಿಭಾರವೆಂಬ ವಿಚಾರ>
3-531

<ಯಜ್ಞಶಬ್ದ ಸ್ವರೂಪವು>
8-564

<ಯಜ್ಞಕಾಲದಲ್ಲಿ ಬ್ರಹ್ಮೋದ್ಯವೆಂಬ ತತ್ತ್ವವಿಮರ್ಶೆ>
12-515
28-225

<ಯಜ್ಞವೇ ಲೋಕಕ್ಕೆಲ್ಲಾ ಶ್ರೇಯಸ್ಸಾಧನವು, ಯಜ್ಞವೇ ಪ್ರಜಾಪ್ರತಿಯು ಯಜ್ಞವೇ ಪರಬ್ರಹ್ಮವು ಇತ್ಯಾದಿ>
12-523

<ಯಜ್ಞ ನಿರ್ವಹಣದಲ್ಲಿ ಅಶ್ವಿನೀದೇವತೆಗಳ ಪಾತ್ರ>
13-545

<ಯಜ್ಞದಲ್ಲಿ ಭಾಗವಹಿಸುವ ನಾನಾ ಋತ್ವಿಕ್ಕುಗಳು>
14-394

<ಯಜ್ಞಶಬ್ದ ನಿರ್ವಚನ>
14-429
16-155

<ಯಜ್ಞದಲ್ಲಿ ಭಾಗವಹಿಸುವ ಹೋತು ಮೊದಲಾದ ಋತ್ವಿಕ್ಕುಗಳ ಕರ್ತವ್ಯ>
16-155

<ಯಜ್ಞವು ದೇವತೆಗಳನ್ನೂ ಬಿಟ್ಟು ಹೊರಟು ಹೋದ ವಿಚಾರ ಉಪಾಖ್ಯಾನ ಇತ್ಯಾದಿ>
30-793

<ಯಜ್ಞವು ವಿಷ್ಣುಸ್ವರೂಪದಿಂದ ಹೊರಟುಹೋದಾಗ ಇಂದ್ರನಿಗೂ ವಿಷ್ಣುವಿಗೂ ನಡೆದ ಸಂವಾದ>
26-765

<ಯಜ್ಞವೇದಿಯು ವಿಶ್ವದ ನಾಭಿಸ್ಥಾನ ಎಂಬ ವಿಷಯ>
27-208

<ಯಜ್ಞಕ್ಕೂ ಬ್ರಾಹ್ಮಣರಿಗೂ ಇರುವ ನಿಕಟವಾದ ಸಂಬಂಧ>
29-921

<ಯಜ್ಞಾಯುಧಗಳ (ದಶ) ವಿವರಣೆ>
3-414
26-728

<ಯಜ್ಞಾನುಷ್ಠಾನ ಫಲದಾತೃಗಳಾದ ದೇವತೆಗಳು>
8-486

<ಯಜ್ಞಾಶ್ವದ ವರ್ಣನೆ ಇತ್ಯಾದಿ>
12-147

<ಯಜ್ಞಾಶ್ವದ ಮಹತ್ತ್ವ>
12-215

<ಯಜ್ಞಾ ಯಜ್ಞಾ ವಃ ಎಂಬ ಋಕ್ಕಿನ ವಿಷಯದಲ್ಲಿ ಐತರೇಯ ಬ್ರಾಹ್ಮಣದಲ್ಲಿರುವ ವಿವರಣೆಯು>
21-231

<ಯಜ್ಞೇನ ಯಜ್ಞಮಯಜಂತ ಎಂಬ ಋಕ್ಕಿನ ವಿವರಣೆ>
29-935

<ಯಜ್ಞೇನ ಯಜ್ಞಮಯಜಂತ ಎಂಬ ವಾಕ್ಯದ ವಿವರಣೆ>
29-937

<ಯತ್ತ್ವಾದೇವ ಎಂಬ ಋಕ್ಕಿನ ನಿರ್ವಚನ - ಔಷಧಿ ಮತ್ತು ಚಂದ್ರನ ಪರವಾದ ಅರ್ಥವಿವರಣೆ>
29-364

<ಯತ್ಪುರುಷೇಣ ಹವಿಷಾ ಎಂಬ ಋಕ್ಕಿನ ವಿವರಣೆ>
29-831

<ಯತ್ಪುರುಷಂ ವ್ಯವಧುಃ ಎಂಬ ಋಕ್ಕಿನ ವಿವರಣೆ>
29-907

<ಯಥಾಭವತ್‍ ಎಂಬ ಋಕ್ಕಿಗೆ ಆದಿತ್ಯ ಪರವಾದ ಅರ್ಥ>
30-865

<ಯಥಾಭವತ್‍ ಎಂಬ ಋಕ್ಕಿಗೆ ಯಮನಪರವಾದ ಅರ್ಥ>
30-864

<ಯದುವಿನ ವಿಷಯ>
5-322
8-604

<ಯದು ತುರ್ವಶ ಎಂಬುವರ ವಿಷಯ>
20-735

<ಯದುದಂಚುಃ ನಿರುಕ್ತ ಎಂಬ ಋಕ್ಕಿನ ನಿರುಕ್ತ>
29-487

<ಯದ್ವಾಗ್ವದಂತೀ ಎಂಬ ಋಕ್ಕಿನ ನಿರುಕ್ತ>
25-511

<ಯದ್ಭೂತಂ ಯಚ್ಚ ಭವ್ಯಂ ಎಂಬ ವಾಕ್ಯದ ವಿವರಣೆ>
29-743

<ಯದ್ದೇವಾಪಿಃ ಶಂತನವೇ ಎಂಬ ಋಕ್ಕಿನ ನಿರುಕ್ತ>
30-112

<ಯಮ>
5-706

<ಯಮ (ಯುಗಲ) ಶಬ್ದಾರ್ಥ ವಿವರಣೆ>
19-733

<ಯಮನ ಮಹಾತ್ಮೆ>
27-335

<ಯಮನ ಸ್ಥಾನಮಾನಗಳು>
27-336

<ಯಮನ ಮಾತಾಪಿತೃಗಳು>
27-338

<ಯಮನ ಸ್ವರೂಪ ವರ್ಣನೆ>
27-338

<ಯಮನ ದೂತರು, ಅವರ ಸ್ವರೂಪ>
27-339

<ಯಮನ ವಿಷಯವಾಗಿ ಕೆಲವು ಋಕ್ಕುಗಳಲ್ಲಿ ಕಂಡು ಬರುವ ವಿವರಣೆಗಳು>
27-349

<ಯಮನು ಯಾರು ? ಅವನ ಕರ್ತವ್ಯವೇನು?>
27-455

<ಯಮ ಶಬ್ದ ನಿರ್ವಚನ>
30-853

<ಯಮ ಮತ್ತು ಯಈ>
27-340

<ಯಮನಿಗೂ ಅವನ ಸಹೋದರಿಯಾದ ಯಮಿಗೂ ಸಂಭೋಗಾರ್ಥವಾಗಿ ನಡೆದ ಸಂಭಾಷಣೆ>
27-353

<ಯಮನಿಗೂ ನಚಿಕೇತಋಷಿಗೂ ನಡೆದ ಸಂಭಾಷಣೆ–ಕಠೋಪನಿಷತ್ತಿನವಿವರಣೆ>
27-582

<ಯಯಾತಿರಾಜನ ವಿಷಯ>
3-583

<ಯಸ್ತೇ ಗರ್ಭಮಈವ ಎಂಬ ಋಕ್ಕಿನ ನಿರುಕ್ತ>
30-1096

<ಯಸ್ತ್ವದ್ಧೋತಾ ಎಂಬ ಋಕ್ಕಿಗೆ ಯಾಸ್ಕರ ವಿವರಣೆ>
16-361

<ಯಸ್ಮಿನ್ವೃಕ್ಷೇ ಎಂಬ ಋಕ್ಕಿಗೆ ಆದಿತ್ಯಪರರಾದ ಅರ್ಥ>
30-852

<ಯಸ್ಮಿನ್ಮೃಕ್ಷೇ ಎಂಬ ಋಕ್ಕಿಗೆ ಯಮನಪರವಾದ ಅರ್ಥ>
30-851

<ಯಸ್ಮಿನ್ವೃಕ್ಷೇ ಎಂಬ ಋಕ್ಕಿನ ನಿರುಕ್ತ>
30-853

<ಯಸ್ಯ ವ್ರತಂ ಎಂಬ ಪರಿಶಿಷ್ಟಮಂತ್ರ>
23-278

<ಯಹ್ವೀಃ>
6-316
6-368

<ಯಾ ಔಷಧೀಃ ಪೂರ್ವ ಜಾತಾ ಎಂಬ ಋಕ್ಕಿನ ನಿರುಕ್ತ>
30-73

<ಯಾಜ್ಞವಲ್ಕ್ಯರ ವಿಷಯ>
1-102

<ಯಾಜ್ಞವಲ್ಕ್ಯ ಋಷಿಗೂ ಗಾರ್ಗ್ಯನಿಗೂ ಲೋಕಗಳ ವಿಷಯದಲ್ಲಿ ನಡೆದ ಸಂಭಾಷಣೆ>
29-278

<ಯಾತುಧಾನರ ವಿಷಯ>
10-553

<ಯಾತುಧಾನ್ಯ ಶಬ್ದಾರ್ಥ>
4-170 
14-352

<ಯಾ ತೇ ದಿದ್ಯುತ್‍ ಎಂಬ ಋಕ್ಕಿನ ನಿರುಕ್ತ>
22-275

<ಯಾದೃಗೇವ ದದೃಶೇ ತಾದೃಗುಚ್ಯತೇ ಎಂಬ ವಾಕ್ಯದ ವಿಮರ್ಶೆ>
19-467

<ಯಾವನ್ಮಾತ್ರಮುಷಸಃ ಎಂಬ ಋಕ್ಕಿನ ನಿರ್ವಚನ>
29-573

<ಯಾವ ದೇವತೆಯ ಹೆಸರೂ ಇಲ್ಲದ ಕೆಲವು ಮಂತ್ರಗಳು (ಬೃ.ದೇ.)>
29-951

<ಯುಗ ಶಬ್ದಾರ್ಥ ವಿವರಣೆ>
9-157

<ಯುಗ ಶಬ್ದದ ನಾನಾರ್ಥಗಳು ಮತ್ತು ಪ್ರಯೋಗಗಳು>
29-6

<ಯುಗ, ಮನ್ವಂತರ, ಕಲ್ಪ ಇತ್ಯಾದಿ>
29-15

<ಯುಜೇ ವಾಂ ಬ್ರಹ್ಮ ಎಂಬ ಸೂಕ್ತ ಪೀಠಿಕೆ>
27-407

<ಯುವತಿ ಶಬ್ದಾರ್ಥ>
14-79

<ಯುವಾನಂ, ಯವಿಷ್ಠಂ ಎಂಬ ಶಬ್ದಗಳು ಅಗ್ನಿಗೆ ಅನ್ವಯಿಸಿರುವ ವಿಚಾರ>
20-421

<ಯೂಪ ಅಥವಾ ವನಸ್ಪತಿಯ ವಿವರಣೆ>
16-51

<ಯೂಪಚ್ಛೇದನ, ಸಂಸ್ಕರಣ ಇತ್ಯಾದಿ>
12-172

<ಯೂವಚ್ಛೇದ ಮತ್ತು ಯೂಪಸ್ಥಾಪನ ಮಂತ್ರಗಳು (ತೈತ್ತಿರೀಯ ಸಂಹಿತೆಯಲ್ಲಿರುವಂತೆ)>
16-60

<ಯೂವಪದ ವಿಷಯದಲ್ಲಿ ಐತರೇಯ ಬ್ರಾಹ್ಮಣದಲ್ಲಿರುವ ಪೂರ್ವೇತಿಹಾಸ>
16-52

<ಯೂಪವನ್ನು ನೆಡುವ ವಿಧಿ>
16-53

<ಯೂಪದ ನಾನಾ ಪ್ರಭೇದಗಳು–ಖಾದಿರ, ಬೈಲ್ವ, ಪಾಲಾಶ ಇತ್ಯಾದಿ>
16-53

<ಯೂಪವನ್ನು ಘೃತದಿಂದ ಅಭ್ಯಕ್ತವನ್ನಾಗಿ ಮಾಡಿ ಸಂಸ್ಕರಿಸುವ ವಿಧಾನ>
16-54

<ಯೂಪಾಂಜನ ಕರ್ಮದ ಇತರ ವಿಧಿನಿಯಮಗಳು>
16-57

<ಯೂಪ ಪ್ರಾರ್ಥನಾ>
16-55

<ಯೂಪ ಸ್ಥಾಪನ ವಿಧಿ>
16-62

<ಯೂಪ ಪ್ರೋಕ್ಷಣ ವಿಧಿ>
16-66

<ಯೇ ದೇವಾಸಃ ಎಂಬ ಋಕ್ಕಿನ ವಿಶೇಷವಿನಿಯೋಗ>
11-95

<ಯೋ ಅನಿಧ್ಮೋ ಎಂಬ ಋಕ್ಕಿನ ನಿರುಕ್ತ>
27-799

<ಯೋನಿ ಶಬ್ದದ ನಾನಾರ್ಥಗಳು ಮತ್ತು ಪ್ರಯೋಗಗಳು>
11-345 
11-451 
3-33 
13-455
16-607

<ಯೋ ವಾಂ ಪರಿಜ್ಮಾ ಎಂಬ ಸೂಕ್ತದ ಪೀಠಿಕೆ>
27-930

<ಯೋಷಾ ಎಂಬ ಶಬ್ದದ ನಿರ್ವಜನ>
27-974

[ರ]
<ರಜಃ ಶಬ್ದದ ನಾನಾರ್ಥಗಳು>
10-58
13-131
13-499 
14-195 
19-660


<ರಜಸಃ ಎಂಬ ಶಬ್ದದ ನಿರ್ವಚನ ಮತ್ತು ಅರ್ಥ>
6-525

<ರಜಸೀ ಎಂಬ ಶಬ್ದದ ಅರ್ಥವಿವರಣೆ ಇತ್ಯಾದಿ>
12-77

<ರಣ್ವಾ>
6-168

<ರಥ ಶಬ್ದ ವಿವರಣೆ>
8-711
13-42
14-664
15-568


<ರಥ, ಚಕ್ರ ಇತ್ಯಾದಿ ಶಬ್ದಗಳ ರೂಪನಿಷ್ಪತ್ತಿ>
12-301


<ರಥಂತರಸಾಮಕ್ಕೂ ಆದಿತ್ಯನಿಗೂ ಇರುವ ಸಂಬಂಧ>
12-461

<ರಥವೀತಿರಾಜನು ತಪೋವನಕ್ಕೆ ಹೋದ ವಿಷಯವರ್ಣನೆ>
19-826

<ರಥಗೋಪರ ವಿವರಣೆ>
21-676

<ರಥೇ ತಿಷ್ಠನ್‍ ಎಂಬ ಋಕ್ಕಿನ ನಿರುಕ್ತ>
21-571

<ರಭಸಶಬ್ದಾರ್ಥ>
13-49

<ರಂಭಶಬ್ದದ ನಿರ್ವಚನೆ ಮತ್ತು ಅರ್ಥವಿವರಣೆ>
24-468

<ರಯಿಶಬ್ದಾರ್ಥ>
14-277

<ರಸಾ ಎಂಬ ಜಲರಹಿತವಾದ ನದಿಗೆ ನೀರು ಬರುವಂತೆ ಮಾಡಿದ ವಿಚಾರ>
S-780 
19-643

<ರಕ್ಷಃ>
10-549

<ರಕ್ಷಃ, ಪಿಶಾಚ ಮುಂತಾದವರ ವಿಷಯ>
26-771

<ರಕ್ಷಕ ದೇವತೆಗಳು>
5-667

<ರಕ್ಷಸಃ>
6-696

<ರಕ್ಷೋಹಾಗ್ನಿ ಎಂಬ ದೇವತೆ–ಇದು ಯಜ್ಞಾದಿ ಕರ್ಮಗಳಲ್ಲಿ ರಾಕ್ಷಸರ ಬಾಧೆಯಿಂದ ಕಾಪಾಡುವ ಅಗ್ನಿಯ ಹೆಸರು>
17-759 
18-849
29-489
30-569

<ರಾಕಾ ಎಂಬ ಹೆಸರಿನ ದೇವತೆ ಮತ್ತು ರಾಕಾ ಶಬ್ದಾರ್ಥ>
15-186
19-431

<ರಾಘವೇಂದ್ರಯತಿಗಳ ಮಂತ್ರಾರ್ಥಮಂಜರೀ>
1-243

<ರಾಜನ ರಾಜ್ಯಾಭಿಷೇಕ ವಿಷಯ>
30-1168

<ರಾಜನು ಶತ್ರುಸೇನೆಗಳನ್ನು ಸೋಲಿಸುವ ಬಗೆ>
18-53

<ರಾಜರ ದಾನಪ್ರಶಂಸೆ–ನಾರಾಶಂಸೀ ಮಂತ್ರಗಳು ಇತ್ಯಾದಿ>
28-800

<ರಾತ್ರಿ ಮತ್ತು ಉಷಸ್ಸು ಸಹೋದರಿಯರೆಂಬ ವಿಚಾರ>
7-446

<ರಾತ್ರೀ ಸೂಕ್ತದ ಪರಿಶಿಷ್ಟ ಮಂತ್ರಗಳು>
30-726

<ರಾತ್ರೀ ಸೂಕ್ತದ ಮಹಿಮೆ, ಜಪವಿಧಾನ ಇತ್ಯಾದಿ>
30-713

<ರಾತ್ರಿ ಮತ್ತು ಉಷಸ್ಸುಗಳ ಸಮಾನತ್ವಾಸಮಾನತ್ವವಿಚಾರ>
9-13

<ರಾತ್ರಿ ಮತ್ತು ಉಷಸ್ಸುಗಳ ಭಿನ್ನಾಭಿನ್ನತೆಗಳು>
9-20

<ರಿಕ್ತ, ಧಾಮಚ್ಛತ್‍, ವಜ್ರ ಎಂಬ ವಷಟ್ಕಾರದ ಮೂರು ಪ್ರಭೇದಗಳು>
23-206

<ರಿಶಾದಸಃ ಎಂಬ ಶಬ್ದದ ನಿರ್ವಚನ ಮತ್ತು ಅರ್ಥವಿವರಣೆ>
24-169

<ರುಕ್ಮ ಎಂಬ ಶಬ್ದದ ವಿವರಣೆ>
13-49

<ರುದ್ರ ಎಂಬ ದೇವತೆಯ ವಿಷಯ>
5-589
22-267

<ರುದ್ರಶಬ್ದಾರ್ಥವಿಚಾರ, ರುದ್ರಶಬ್ದದ ರೂಪನಿಷ್ಪತ್ತಿ ಇತ್ಯಾದಿ>
4-467
6-350 
9-101
10-379
13-541 
14-402 
15-196
22-272

<ರುದ್ರನು ತ್ರಿಪುರ ಸಂಹಾರ ಮಾಡಿದ ವಿಷಯ-ಐತರೇಯ ಬ್ರಾಹ್ಮಣ ಮತ್ತು ತೈತ್ತಿರೀಯ ಸಂಹಿತೆಯಲ್ಲಿರುವಂತೆ>
20-625


<ರುದ್ರನು ಮರುದ್ದೇವತೆಗಳಿಗೆ ತಂದೆಯೆಂಬ ವಿಚಾರ>
9-126
19-595

<ರುದ್ರಾಃ ಎಂಬ ಹೆಸರು ಮರುದ್ದೇವತೆಗಳಿಗೆ ಬರಲು ಕಾರಣ>
19-660

<ರುಧಿಕ್ರಾ ಎಂಬ ಅಸುರನನ್ನು ಇಂದ್ರನು ಸಂಹಾರ ಮಾಡಿದ ವಿಷಯ>
14-733

<ರೂಪಸಮೃದ್ಧವೆಂಬ ಶಬ್ದದ ಅರ್ಥವಿವರಣೆ ಇತ್ಯಾದಿ>
29-432

<ರೇತಃ ಶಬ್ದಾರ್ಥ ವಿವರಣೆ>
11-617 
12-537
17-310

<ರೇಭ>
8-746
9-360
9-443
22-483

<ರೇಭನೆಂಬ ಋಷಿಯನ್ನು ಅಸುರರು ಭಾವಿಯಲ್ಲಿ ಹಾಕಿದ್ದಾಗ ಅವನು ಹತ್ತುದಿನಗಳವರೆಗೂ ದುಃಖಪಡುತ್ತಿದ್ದು ಅಶ್ವಿನೀ ದೇವತೆಗಳನ್ನು ಪ್ರಾರ್ಥಿಸಲು ಅವರು ಬಂದು ಅವನನ್ನು ಬಾವಿಯಿಂದ ಮೇಲಕ್ಕೆತ್ತಿ ರಕ್ಷಿಸಿದ ವಿಚಾರ>
9-302 
9-324
9-475
27-960

<ರೋದಸೀ ಎಂಬ ಶಬ್ದದ ವಿವರಣೆ>
17-369

<ರೋದಸೀ ಎಂಬುದು ರುದ್ರಪತ್ನಿಯ ಹೆಸರು ಇತ್ಯಾದಿ>
13-275

<ರೋದಸೀ ಎಂಬ ಶಬ್ದದ ನಾನಾರ್ಥಗಳು>
13-86 
13-275 
16-232
19-718

<ರೋಮಶಾ ಮತ್ತು ಇಂದ್ರ ಇವರ ವೃತ್ತಾಂತ (ಬೃ.ದೇ)>
28-803

<ರೋಮಶೆಗೂ ಅವಳ ಪತಿ ಭಾವಯವ್ಯನಿಗೂ ನಡೆದ ಸಂಭಾಷಣೆ>
10-242

[ಲ]
<ಲಿಂಗಗಳು ಮತ್ತು ಉಪಸರ್ಗಗಳು (ಬೃ.ದೇ.)>
28-736

<ಲೋಕಗಳ ಸ್ವರೂಪ ಇತ್ಯಾದಿ>
29-789

<ಲೋಪಾಮುದ್ರಾ ಮತ್ತು ಅಗಸ್ತ್ಯ>
28-819

<ಲೋಪಮುದ್ರೆಗೂ ಅಗಸ್ತ್ಯನಿಗೂ ನಡೆದ ಸಂಭಾಷಣೆ>
28-819

<ಲೋಪಾಮುದ್ರೆಯು ರತಿವಿಷಯವಾಗಿ ಅಗಸ್ತ್ಯನನ್ನು ಪ್ರಾರ್ಥಿಸುವುದು>
13-473

[ವ]
<ವಕ್ವರೀ ಶಬ್ದಾರ್ಥ>
11-353

<ವಜ್ರ, ಧಾಮಚ್ಛತ್‍, ರಿಕ್ತ ಎಂಬ ವಷಟ್ಕಾರದ ಮೂರು ಪ್ರಭೇದಗಳು>
23-206

<ವಜ್ರ ಎಂಬ ಶಬ್ದದ ವಿವರಣೆ>
17-71

<ವತ್ಸಂ>
6-340

<ವತ್ಸ ಪ್ರಿಯಋಷಿ ದೃಷ್ಟವಾದ ಸೂಕ್ತಗಳ ಮಹಿಮೆ>
28-93

<ವಂದನ>
8-746

<ವಂದನನೆಂಬ ಋಷಿಯು ಕಾಡಿನ ಮಧ್ಯದಲ್ಲಿದ್ದ ಒಂದು ಭಾವಿಯಲ್ಲಿ ಬಿದ್ದು ಪರಿತಪಿಸುತ್ತಿರುವಾಗ ಅಶ್ವಿನೀ ದೇವತೆಗಳು ಬಂದು ಅವನನ್ನು ಭಾವಿಯಿಂದ ಮೇಲಕ್ಕೆತ್ತಿ ರಕ್ಷಿಸಿದ ವಿಚಾರ>
9-239 
9-329
22-488 
27-956

<ವಧ್ರಿಮತೀ ಎಂಬ ಸ್ತ್ರೀಗೆ ಅಶ್ವಿನೀ ದೇವತೆಗಳು ಹಿರಣ್ಯಹಸ್ತನೆಂಬ ಪುತ್ರನನ್ನು ಅನುಗ್ರಹಿಸಿದ ವಿಚಾರ>
9-252 
9-408 
22-488 
27-954

<ವನಶಬ್ದಾರ್ಥ ವಿಚಾರ>
8-380 
13-207

<ವನಸ್ಪತಿ ಎಂಬ ದೇವತೆಯ ವಿಷಯ>
11-274 
14-266 
15-740 
28-641
30-360

<ವನಸ್ಪತಿ ಅಥವಾ ಯೂಪದ ವರ್ಣನೆ>
16-51

<ವನಸ್ಪತೇ ವೀಡ್ವಂಗಃ ಎಂಬ ಋಕ್ಕಿಗೆ ಯಾಸ್ಕರ ನಿರ್ವಚನ>
21-219

<ವನಸ್ಪತಿ ಮತ್ತು ಸ್ವಾಹಾಕೃತಿ ಎಂಬ ದೇವತೆಗಳು (ಬೃ.ದೇ.)>
28-763

<ವನುಷ್ಯತೇ ಎಂಬ ಶಬ್ದದ ನಿರ್ವಚನ>
23-11

<ವನೇ ನ ವಾ ಎಂಬ ಋಕ್ಕಿನ ನಿರುಕ್ತ>
27-773

<ವನೋತಿ ಹಿ ಎಂಬ ಋಕ್ಕಿನ ವಿಶೇಷ ವಿನಿಯೋಗ>
10-564

<ವದ್ರು ಎಂಬುವನನ್ನು ಅಶ್ವಿನೀದೇವತೆಗಳು ರಕ್ಷಿಸಿದ ವಿಚಾರ>
8-792

<ವಈ ಶಬ್ದಾರ್ಥ ವಿವರಣೆ>
25-556

<ವಯಃ ಸುಪರ್ಣಾ ಎಂಬ ಋಕ್ಕಿನ ನಿರುಕ್ತ>
29-74

<ವಯುನ ಶಬ್ದ ವಿವರಣೆ>
14-277 
15-235
20-748

<ವಯ್ಯ ಎಂಬುವನ ವಿಷಯ>
5-322
8-753
14-714
18-19
20-156

<ವಯ್ಯ ಎಂಬುವನಿಗೆ ಸತ್ಯಶ್ರವಾಃ ಎಂಬ ಋಷಿಯು ಪುತ್ರನೆಂಬ ವಿಚಾರ>
20-156

<ವರಾಹ ಶಬ್ದದ ರೂಪನಿಷ್ಪತ್ತಿ ಅರ್ಥವಿವರಣೆ ಇತ್ಯಾದಿ>
5-501 
5-534
9-120
25-175
26-762

<ವರಾಹಾಸುರನನ್ನು ಇಂದ್ರನು ಸಂಹಾರ ಮಾಡಿದ ವಿಷಯ>
26-763
28-181

<ವರುಣ ಎಂಬ ದೇವನ ವಿಷಯ>
5-532
20-222

<ವರುಣಾನೀ ಎಂಬ ವರುಣನ ಪತ್ನಿ>
3-100

<ವರುಣ, ಮಿತ್ರ, ಅರ್ಯಮಾ ಎಂಬ ದೇವತೆಗಳ ವಿಷಯ>
14-119

<ವರುಣನ ವೈಯಕ್ತಿಕವಾದ ವಿಶೇಷಗಳು ಮತ್ತು ಕ್ರಿಯೆಗಳು>
20-5

<ವರುಣನು ಅಗ್ನಿಗೆ ಹೇಗೆ ಸಹೋದರನು ಎಂಬ ವಿಷಯ>
17-671

<ವರೇಣ್ಯಂ (ಗಾಯತ್ರಿಯಲ್ಲಿರುವ) ಎಂಬ ಶಬ್ದದ ವಿವರಣೆ>
17-630

<ವರ್ಚಿಃ>
5-696

<ವರ್ಚೀ ಎಂಬ ಅಸುರನ ವಿಷಯ>
19-202

<ವರ್ಣ ವಿಭಾಗವು ಹೇಗೆ ಉಂಟಾಯಿತು ಎಂಬ ವಿಷಯದಲ್ಲಿ ಪಾಶ್ಚಾತ್ಯ ಪಂಡಿತರ ಅಭಿಪ್ರಾಯಗಳು>
29-913

<ವರ್ತಿಕಾ ಎಂಬ ಗುಬ್ಬಚ್ಚಿಯನ್ನು ತೋಳನ ಬಾಯಿಂದ ಅಶ್ವಿನೀದೇವತೆಗಳು ರಕ್ಷಿಸಿದ ವಿಚಾರ>
7-764
9-377 
9-440


<ವರ್ತಿ ಶಬ್ದದ ವಿವರಣೆ>
13-635

<ವರ್ಷಂತು ತೇ ವಿಭಾವರಿ ಎಂಬ ಪರಿಶಿಷ್ಟಮಂತ್ರದ ಅರ್ಥವಿವರಣೆ>
20-222

<ವಲನೆಂಬ ಅಸುರನ ವೃತ್ತಾಂತ>
4-465
5-232
5-692

<ವಲ ಮತ್ತು ಪಣಿಗಳ ವೃತ್ತಾಂತ>
14-634

<ವಲನೆಂಬ ಅಸುರನು ದೇವತೆಗಳ ಗೋವುಗಳನ್ನು ಅಪಹರಿಸಿ ಬಚ್ಚಿಟ್ಟಿದ್ದ ವಿಷಯ–ಪಣಿಗಳ ವಿಷಯ>
20-680

<ವಶನೆಂಬ ರಾಜನನ್ನು ಅಶ್ವಿನೀದೇವತೆಗಳು ರಕ್ಷಿಸಿದ ವಿಚಾರ>
80-772

<ವಶನೆಂಬುವನನ್ನು ಪೃಥುಶ್ರವಸ್ಸೆಂಬ ರಾಜನನ್ನೂ ಅಶ್ವಿನೀದೇವತೆಗಳು ರಕ್ಷಿಸಿದ ವಿಚಾರ>
9-289
27-981

<ವಷಟ್ಕಾರಸ್ವರೂಪ ಮತ್ತು ಅರ್ಥವಿವರಣೆ>
23-203
27-267

<ವಷಟ್ಕಾರಸ್ವರೂಪ ಮತ್ತು ಪ್ರಯೋಗದ ವಿಷಯದಲ್ಲಿ ಐತರೇಯ ಬ್ರಾಹ್ಮಣದ ವಿಸ್ತಾರವಾದ ವಿವರಣೆ>
23-204

<ವಷಟ್ಕಾರವನ್ನು ಉಪಯೋಗಿಸುವ ವಿಧಾನ>
23-205

<ವಷಟ್ಕಾರಗಳ (ಮೂರುವಿಧ) ಪ್ರಯೋಗದಿಂದ ಲಭಿಸಿದ ಫಲಭೇದಗಳು>
23-207

<ವಷಟ್ಕಾರವನ್ನು ಪಠಿಸುವಾಗ ಅನುಸರಿಸಬೇಕಾದ ವಿಧಿನಿಯಮಗಳು>
23-208

<ವಷಟ್ಕಾರಗಳ ದೇವತೆಗಳು (ಬೃ.ದೇ.)>
28-1023

<ವಸತಿ ಶಬ್ದದ ರೂಪನಿಷ್ಪತ್ತಿ>
6-201 
7-321

<ವಸತೀವರೀ ಎಂಬ ಉದಕಗಳ ವಿಷಯದಲ್ಲಿ ಅಪೋನಪ್ತ್ರೀಯವೆಂಬ ಯಜ್ಞಾಂಗ ಕರ್ಮದ ವಿವರಣೆ>
27-841

<ವಸಾತಿ ಶಬ್ದದ ಅರ್ಥವಿವರಣೆ>
9-316

<ವಸಿಷ್ಠ ಋಷಿಯ ಸೂಕ್ಷ್ಮ ಪರಿಚಯ>
21-596

<ವಸಿಷ್ಠ ಋಷಿಯ ಉತ್ಪತ್ತಿ ಮತ್ತು ಮಾಹಾತ್ಮ್ಯ>
21-596 
22-138

<ವಸಿಷ್ಠ ಮತ್ತು ವಿಶ್ವಾಮಿತ್ರ>
21-609

<ವಸಿಷ್ಠ ಮತ್ತು ಸುದಾಸ>
21-606

<ವಸಿಷ್ಠೋತ್ಪತ್ತಿ ಕ್ರಮ–ಋಗ್ವೇದದಲ್ಲಿ ಹೇಳಿರುವಂತೆ>
22-142

<ವಸಿಷ್ಠೋತ್ಪತ್ತಿ ಕ್ರಮ–ಬೃಹದ್ದೇವತಾ ಗ್ರಂಥದಲ್ಲಿ ಹೇಳಿರುವಂತೆ>
22-143

<ವಸಿಷ್ಠಋಷಿಗೂ ಅಶ್ವಿನೀದೇವತೆಗಳಿಗೂ ಇರುವ ಬಂಧುತ್ವದ ವಿಮರ್ಶೆ>
22-539

<ವಸಿಷ್ಠ ಋಷಿಯ ಮಹಿಮೆ–ಕಶ್ಯಪನ ಪತ್ನಿಯರು (ಬೃ.ದೇ.)>
28-888

<ವಸಿಷ್ಠ ಮತ್ತು ವಸಿಷ್ಠಕುಲದವರು (ಬೃ.ದೇ.)>
28-892

<ವಸಿಷ್ಠ ಮತ್ತು ವರುಣನ ನಾಯಿ (ಬೃ.ದೇ.)>
28-901

<ವಸಿಷ್ಠ ಎಂಬ ಋಷಿಯನ್ನು ಅಶ್ವಿನೀದೇವತೆಗಳು ಕಾಪಾಡಿದ ವಿಚಾರ>
8-768

<ವಸಿಷ್ಠ ಪುತ್ರನಾದ ಶಕ್ತಿ ಎಂಬುವನು ವಿಶ್ವಾಮಿತ್ರನ ವಾಕ್ಸ್ತಂಭನವನ್ನು ಮಾಡಿದ ವಿಚಾರ ಮತ್ತು ಸಸರ್ಪರೀ ಎಂಬುವಳ ವಿಷಯ>
17-212

<ವಸುಶಬ್ದದ ಅನೇಕಾರ್ಥಗಳು>
13-190

<ವಸು, ರುದ್ರ, ಆದಿತ್ಯರ ವಿಷಯ>
4-543
16-417

<ವಸು ಶಬ್ದದ ರೂಪನಿಷ್ಪತ್ತಿ ಇತ್ಯಾದಿ>
22-221

<ವಸುಗಳು–(ಅಷ್ಟ)>
17-386
19-577

<ವಸ್ಯ ಶಬ್ದಾರ್ಥ>
8-623

<ವಕ್ಷ್ಯಂತೀ ವೇದಾಗನೀಕಂತಿ ಎಂಬ ಋಕ್ಕಿನ ನಿರುಕ್ತ>
21-563

<ವ್ಯಕ್ತಿಗೆ ಹೆಸರುಗಳು ವ್ಯಕ್ತಿಯ ಕರ್ಮಗಳಿಂದಲೇ ಉಂಟಾಗುವವು–ಇದು ಶೌನಕರ ಮತವು (ಬೃ.ದೇ.)>
28-684

<ವ್ಯಚಯಾ ಎಂಬ ಸ್ತ್ರೀಯ ವೃತ್ತಾಂತ>
5-201

<ವ್ಯಚಸ್ವತೀರುರ್ವಿಯಾ ಎಂಬ ಋಕ್ಕಿನ ನಿರುಕ್ತ>
30-349

<ವ್ಯಂತಃ ಎಂಬ ಶಬ್ದದ ನಿರ್ವಚನ>
20-368

<ವ್ಯಶ್ವ>
8-792

<ವ್ರತಶಬ್ದವಿವರಣೆ>
13-57 
14-498 
14-727
14-925 
15-3 
15-107
15-145

<ವಾ ಎಂಬ ಶಬ್ದದ ಅರ್ಥ ಮತ್ತು ಪ್ರಯೋಗ>
30-593

<ವಾಕ್ಕಿನ ಮೂರುವಿಧ ಸ್ವರೂಪ (ಬೃ.ದೇ.)>
28-730

<ವಾಕ್ಕೇ ಸಮಸ್ತ ದೇವತೆಗಳಿಗೂ ಯಜ್ಞಯಾಗಾದಿಗಳಿಗೂ ಆಶ್ರಯವು>
12-594

<ವಾಗ್ದೇವತೆಯ ಇತರ ರೂಪಗಳು, ನಾಲ್ಕೂ ಸ್ವರ್ಗೀಯ ರೂಪಗಳು (ಬೃ.ದೇ.)>
28-731

<ವಾಗ್ದೇವತೆಯ ಸ್ವರೂಪ>
12-593

<ವಾಕ್ಯರಚನೆ (ಬೃ.ದೇ.)>
28-737

<ವಾಜಪ್ರಮಹಃ ಎಂಬ ಶಬ್ದ>
9-601

<ವಾಜಃ ಶಬ್ದ ವಿವರಣೆ>
10-439

<ವಾಜಸನೇಯ ಶಾಖೆಗಳು–೧೫>
1-109

<ವಾಜಿ ಶಬ್ದದ ವಿವರಣೆ>
6-117

<ವಾಜಿನೀವತಿ ಶಬ್ದಾರ್ಥವಿವರಣೆ>
9-530

<ವಾಜಿನಃ ಎಂಬ ಶಬ್ದದ ವಿವರಣೆ>
22-214

<ವಾಜನೀವಸೂ ಎಂಬ ಶಬ್ದದ ಅರ್ಥವಿವರಣೆ>
27-988

<ವಾತ>
5-598

<ವಾತ ಶಬ್ದದ ನಿರ್ವಚನ>
30-1137

<ವಾತ ಆವಾತು ಎಂಬ ಋಕ್ಕಿನ ನಿರುಕ್ತ>
30-1251

<ವಾತ್ಸ್ಯಶಾಖಾ>
1-59

<ವಾನರ>
5-684

<ವಾಮದೇವಋಷಿಯ ಸೂಕ್ಷ್ಮಪರಿಚಯ>
17-659
18-1

<ವಾಮದೇವಋಷಿಯ ಜನನವೃತ್ತಾಂತ>
17-989
28-843

<ವಾಮದೇವಋಷಿಯನ್ನು ತಾಯಿಯ ಗರ್ಭದಿಂದ ಜನಿಸುವಂತೆ ಮಾಡಿದ ವಿಚಾರ>
9-479

<ವಾಮದೇವಋಷಿಯು ಗರ್ಭಸ್ಥನಾಗಿದ್ದಾಗ ಇಂದ್ರ ಅದಿತಿಯರ ಸಂವಾದವು>
17-990

<ವಾಮದೇವಋಷಿಯು ಅತ್ಯಂತದರಿದ್ರಾವಸ್ಥೆಯಲ್ಲಿದ್ದಾಗ ಆಹಾರಾಭಾವದಿಂದ ನಾಯಿಯ ಮಾಂಸವನ್ನು ಆಹಾರಕ್ಕಾಗಿ ಬೇಯಿಸಿದ ವಿಚಾರ ಮತ್ತು ಇಂದ್ರನ ಸಹಾಯ ಇತ್ಯಾದಿ>
17-1006

<ವಾಯುದೇವತೆಯ ವಿಷಯ>
5-598
30-1137


<ವಾಯುವು ಮಧ್ಯಮಸ್ಥಾನ ದೇವತೆಯೆಂಬ ವಿಚಾರ>
10-568

<ವಾಯುವಿಗೆ ಯಜ್ಞದಲ್ಲಿ ಸೋಮಪಾನ ಮಾಡಿದಾಗ ಪ್ರಥಮಸ್ಥಾನವು ಹೇಗೆ ಬಂತೆಂಬ ವಿಚಾರ>
10-603
]8-469
19-568 
26-89
26-795

<ವಾಯುವು ವೃಷ್ಟಿಪತನಕ್ಕೆ ಸಹಾಯಕನಾಗಿರುವ ವಿಚಾರ>
12-138


<ವಾಯಿವಿನ ಪ್ರಥಮ ಸೋಮಪಾನಾರ್ಹತೆಯ ವಿಷಯದಲ್ಲಿ ಐತರೇಯ ಬ್ರಾಹ್ಮಣದಲ್ಲಿರುವ ಪೂರ್ವೇತಿಹಾಸವು>
18-469

<ವಾರುಣಾಶ್ವವರ್ಣನೆ ಇತ್ಯಾದಿ>
12-233
13-557

<ವಾಲಖಿಲ್ಯಸೂಕ್ತಗಳ ಪೀಠಿಕೆ>
24-567

<ವಾಶೀಮಂತಃ ಎಂಬ ಶಬ್ದಕ್ಕೆ ಯಾಸ್ಕರ ನಿರ್ವಚನ>
19-729

<ವಾಶ್ರಾಃ ಎಂಬ ಶಬ್ದದ ವಿವರಣೆ>
8-35

<ವಾಸರಾಣಿ ಶಬ್ದದ ನಿರ್ವಚನ>
24-556

<ವಾಸಿಷ್ಠಶಾಖಾ>
1-91

<ವಾಸ್ತೋಷ್ಪತಿ ಎಂಬ ದೇವತೆ (ಬೃ.ದೇ.)>
28-723

<ವಾಸ್ತೋಷ್ಪತಿ ಶಬ್ದದ ನಿರ್ವಚನ>
22-326

<ವಾಹನಾದಿಗಳು (ನಾನಾದೇವತೆಗಳ)>
6-68

<ವಾಹಿಷ್ಠಃ ಎಂಬ ಶಬ್ದದ ನಿರ್ವಚನ ಮತ್ತು ಅರ್ಥವಿವರಣೆ>
24-147
27-251

<ವ್ಯಾಕರಣ>
1-28

<ವ್ಯಾಕರಣವಿಶೇಷಗಳು (ಋಗ್ವೇದದಲ್ಲಿ)>
1-3318

<ವ್ಯಾಕರಣಶಾಸ್ತ್ರದ ಲಕ್ಷಣ ಇತ್ಯಾದಿ>
1-55

<ವ್ಯಾಹೃತಿ (ಓಂ ಎಂಬ) ಮತ್ತು ಅದರ ದೇವತೆಗಳು (ಬೃ.ದೇ.)>
28-745

<ವ್ರಾಥತ್‍ ಎಂಬ ಶಬ್ದ>
8-212

<ವ್ರಾಥತಃ ಎಂಬ ಶಬ್ದದ ಅರ್ಥವಿವರಣೆ>
10-46 
10-663

<ವಿಕಟೇ ಎಂಬ ಶಬ್ದದ ವಿವರಣೆ>
30-1044

<ವಿಚರ್ಷಣಿಃ ಎಂಬ ಶಬ್ದದ ವಿವರಣೆ>
6-561

<ವಿಜಾಮಾತುಃ ಎಂಬ ಶಬ್ದಾರ್ಥ>
8-629

<ವಿದಥ ಶಬ್ದದ ನಾನಾರ್ಥಗಳು ಮತ್ತು ಪ್ರಯೋಗಗಳು>
11-321 
13-9 
14-104 
16-261
18-77

<ವಿದ್ಯಾ ಅವಿದ್ಯಾ ಇವುಗಳ ಸ್ವರೂಪ>
12-572

<ವಿದ್ಯುನ್ನಯಾ ಪತಂತೀ ಎಂಬ ಋಕ್ಕಿನ ನಿರುಕ್ತ>
30-38

<ವಿಧವಾ ಎಂಬ ಶಬ್ದದ ನಿರ್ವಚನ>
27-984

<ವಿಪ್ರುವಿನ ವಿಷಯ>
14-733

<ವಿಭೇಧಕಾಃ ಎಂಬ ಶಬ್ದದ ವಿವರಣೆ>
27-857

<ವಿಮರನೆಂಬ ರಾಜರ್ಷಿಯ ವಿಚಾರ>
9-181

<ವಿಮದ ಎಂಬುವನ ಪತ್ನಿಯು ಅಸುರರಿಂದ ಅಪಹರಿಸಲ್ಪಟ್ಟಾಗ ಅಶ್ವಿನೀದೇವತೆಗಳು ಅವಳನ್ನು ರಕ್ಷಿಸಿ ಕರೆತಂದ ವಿಚಾರ>
9-396
27-953

<ವಿರಪ್ಶಿನಃ ಎಂಬ ಶಬ್ದ>
6-132

<ವಿರಾಟ್‍ಶಬ್ದಾರ್ಥ ಇತ್ಯಾದಿ>
29-821

<ವಿರಾಟ್‍ಛಂದಸ್ಸಿನ ಪ್ರಾಮುಖ್ಯತೆ>
18-54

<ವಿರಾಟ್‍ಛಂದಸ್ಸಿನಲ್ಲಿರುವ ಪ್ರಭೇದಗಳು>
18-58

<ವಿರಾಟ್‍ ಛಂದಸ್ಸಿನ ಯಾಜ್ಯಾಮಂತ್ರದಲ್ಲಿ ಎಲ್ಲಾ ದೇವತೆಗಳಿಗೂ ಹವಿರ್ಭಾಗವಿರುವುದು>
18-53

<ವಿರೂಪಾಃ>
4-552
5-674

<ವಿರೂಪೇ>
6-412

<ವಿವಸ್ವಂತನು ಅದಿತಿಯಲ್ಲಿ ಹುಟ್ಟಿದ ಬಗೆ>
27-525

<ವಿವಸ್ವಾನ್‍>
27-555
5-707

<ವಿವಿಧ ಜಾತಿಯ ಸರ್ಪಗಳ ವಿಷಯ>
14-340

<ವಿಶ್ಪಲಾ ಎಂಬ ಸ್ತ್ರೀಯ ಕತ್ತರಿಸಿಹೋದ ಕಾಲನ್ನು ಅಶ್ವಿನೀ ದೇವತೆಗಳು ಸರಿಪಡಿಸಿದ ವಿಚಾರ>
8-771
9-262 
9-440
22-487 
27-958

<ವಿಶ್ಪತಿ ಶಬ್ದದ ಅರ್ಥವಿವರಣೆ>
10-357

<ವಿಶ್ವಕರ್ಮಾ>
5-642

<ವಿಶ್ವಕರ್ಮನೆಂಬ ದೇವತೆಯ ವಿಷಯವಿಮರ್ಶೆ (ಬೃ.ದೇ.)>
28-724 
29-169

<ವಿಶ್ವಕರ್ಮನ ಸ್ವರೂಪ, ಮಹತ್ತ್ವ ಇತ್ಯಾದಿ>
29-263

<ವಿಶ್ವಕರ್ಮನ ವಿಭೂತಿ ವಿಶೇಷಗಳು, ಶಕ್ತಿಸಾಮರ್ಥ್ಯಗಳು ಇತ್ಯಾದಿ>
29-290


<ವಿಶ್ವಕರ್ಮನು ಸಕಲ ಜಗದುತ್ಪತ್ತಿಗೆ ಕಾರಣನಾದರೂ ಅವನು ಜಗದತಿರಿಕ್ತನು>
29-274

<ವಿಶ್ವಕರ್ಮನು ಸ್ವರೂಪವನ್ನೂ, ಮಹಿಮೆಯನ್ನೂ ಇತರ ದೇವಾದಿಗಳೂ ಪ್ರಾಣಿಗಳೂ ತಿಳಿಯಲಾರರು>
29-302

<ವಿಶ್ವಕರ್ಮನು ನೆರವೇರಿಸಿದ ಸಂಹಾರ ಮತ್ತು ಸೃಷ್ಟಿಕ್ರಮ ಇತ್ಯಾದಿ>
29-181

<ವಿಶ್ವಕರ್ಮನು ಜಗತ್ಸೃಷ್ಟಿ ಮಾಡುವಾಗ ಅವನಿಗಿದ್ದ ಸಾಧನಗಳಾವುವು?>
29-198

<ವಿಶ್ವಕರ್ಮನು ಸಾಧನ ಸಾಮಗ್ರಿಗಳೊಂದೂ ಇಲ್ಲದೆ ಸೃಷ್ಟಿಮಾಡಿದನು>
29-206

<ವಿಶ್ವಕರ್ಮನ್‍ ಹವಿಷಾ ಎಂಬ ಋಕ್ಕಿನ ನಿರುಕ್ತ>
29-225

<ವಿಶ್ವಕರ್ಮನು ನೆರವೇರಿಸಿದ ಸರ್ನಹಾತ್‍ ಎಂಬ ಯಜ್ಞದ ಸ್ವರೂಪ>
29-226

<ವಿಶ್ವಕರ್ಮನು ನೆರವೇರಿಸಿದ ವಿಶ್ವಸೃಷ್ಟಿಯ ವಿಷಯವಿಮರ್ಶೆ ಇತ್ಯಾದಿ>
29-211

<ವಿಶ್ವಕರ್ಮನ ಆತ್ಮಯಜ್ಞವೇ ಜಗತ್ಸೃಷ್ಟಿಗೆ ಮೂಲಕಾರಣ ಇತ್ಯಾದಿ>
29-237

<ವಿಶ್ವಕರ್ಮಾ ವಿಮನಾ ಎಂಬ ಋಕ್ಕಿನ ನಿರುಕ್ತ>
29-262

<ವಿಶ್ವಕರ್ಮ ಸಂಬಂಧವಾದ ಸೂಕ್ತ–ತೈತ್ತಿರೀಯ ಸಂಹಿತೆಯಲ್ಲಿರುವಂತೆ>
29-308

<ವಿಶ್ವಕನೆಂಬ ರಾಜನ ಮಗನಾದ ವಿಷ್ಣ್ವಾಪಿ ಎಂಬುವನು ತಪ್ಪಿಸಿಕೊಂಡುಹೋಗಿದ್ದಾಗ ಅಶ್ವಿನೀದೇವತೆಗಳು ಅವನನ್ನು ಹುಡುಕಿತಂದು ಅವನ ತಂದೆಗೆ ಕೊಟ್ಟರೆಂಬ ವಿಚಾರ>
9-298

<ವಿಶ್ವಜನ್ಯಾ ಎಂಬ ಶಬ್ದದ ಅರ್ಥವಿವರಣೆ>
13-174

<ವಿಶ್ವದರ್ಶತಃ ಎಂಬ ಶಬ್ದದ ವಿವರಣೆ>
5-118

<ವಿಶ್ವಧಾಯಾಃ ಎಂಬ ಶಬ್ದ>
6-394

<ವಿಶ್ವರೂಪ>
5-693

<ವಿಶ್ವರೂಪನನ್ನೂ (ತ್ರಿಶಿರಾಃ ಎಂಬ ತ್ವಷ್ಟೃ ಪುತ್ರನಾದ) ಇಂದ್ರನು ಸಂಹಾರ ಮಾಡಿದ ವಿಷಯ>
14-626
27-300 
27-303
28-184

<ವಿಶ್ವವಾರಂ ಎಂಬ ಶಬ್ದದ ಅರ್ಥವಿವರಣೆ>
5-77

<ವಿಶ್ವಾಂಚನೆಂಬ ಅಸುರನನ್ನು ಸಂಹಾರಮಾಡಿದ ವಿಚಾರ>
9-378

<ವಿಶ್ವಾಮಿತ್ರ ಋಷಿಯ ಪರಿಚಯ>
16-1 
17-1
28-832

<ವಿಶ್ವಾಮಿತ್ರ ಶಬ್ದದ ಅವಯವಾರ್ಥ>
16-376

<ವಿಶ್ವಾಮಿತ್ರನಿಗೂ ಸುದಾಸನಿಗೂ ಇರುವ ರಾಜಪುರೋಹಿತ-ಸಂಬಂಧ (ಬೃ.ದೇ.)>
16-774

<ವಿಶ್ವಾಮಿತ್ರನಿಗೂ ವಿಪಾಟ್ಫುತುದ್ರಿನದಿಗಳಿಗೂ ನಡೆದ ಸಂಭಾಷಣೆ>
16-775 
17-195
26-94

<ವಿಶ್ವಾಮಿತ್ರನು ವಿಪಾಟ್‍ ಮತ್ತು ಶುತುದ್ರಿ ಎಂಬ ನದಿಗಳನ್ನು ದಾಟಲು ಅವುಗಳ ಪ್ರವಾಹವನ್ನು ನಿಲ್ಲಿಸುವಂತೆ ಅವುಗಳನ್ನು ಸ್ತುತಿಸಿದ ವಿಚಾರ>
17-194

<ವಿಶ್ವಾಮಿತ್ರನಿಗೂ ಭರತರಾಜನಿಗೂ ಇರುವ ಸಂಬಂಧ>
17-204


<ವಿಶ್ವಾಮಿತ್ರನು ವಸಿಷ್ಠಋಷಿಗೆ ಶಾಪವನ್ನು ಕೊಟ್ಟು ಬೈದ ವಿಚಾರ>
17-227

<ವಿಶ್ವಾಮಿತ್ರನನ್ನು ವಸಿಷ್ಠನ ಕಡೆ ಜನರು ಹಗ್ಗಗಳಿಂದ ಬಿಗಿದುಕಟ್ಟಿ ಹೊತ್ತುಕೊಂಡುಹೋದ ವಿಚಾರ>
17-230

<ವಿಶ್ವಾಮಿತ್ರ, ಸುದಾಸ ಮತ್ತು ನದಿಗಳು (ಬೃ.ದೇ.)>
28-825

<ವಿಶ್ವಾಮಿತ್ರ ಮತ್ತು ಶಕ್ತಿ (ಬೃ.ದೇ.)>
28-838

<ವಿಶ್ವಾಮಿತ್ರ ಮತ್ತು ವಾಕ್‍ ಸಸರ್ಪರೀ>
28-838

<ವಿಶ್ವಾನರ (ಬೃ.ದೇ.)>
28-720

<ವಿಶ್ವಾಯುಃ>
6-230

<ವಿಶ್ವಾವಸು ಮೊದಲಾದ ಗಂಧರ್ವರ ವಿಷಯ>
29-401

<ವಿಶ್ವೇತ್ತಾ ವಿಷ್ಣುಃ ಎಂಬ ಋಕ್ಕಿಗೆ ನೈರುಕ್ತಪಕ್ಷದ ಅರ್ಥವಿವರಣೆ>
25-171

<ವಿಶ್ವೇತ್ತಾ ವಿಷ್ಣುಃ ಎಂಬ ಋಕ್ಕಿಗೆ ಐತಿಹಾಸಿಕ ಪಕ್ಷದ ವಿವರಣೆ ಮತ್ತು ಈ ಸಂದರ್ಭದಲ್ಲಿ ಚರಕ ಬ್ರಾಹ್ಮಣದಲ್ಲಿರುವ ಪೂರ್ವೇತಿಹಾಸವು>
25-172

<ವಿಶ್ವೇತ್ವಾ ವಿಷ್ಣುಃ ಎಂಬ ಋಕ್ಕಿನ ವಿವರಣೆಯ ಸಂದರ್ಭದಲ್ಲಿ ಕೃಷ್ಣಯಜುರ್ವೇದದ ತೈತ್ತಿರೀಯ ಸಂಹಿತೆಯಲ್ಲಿ ಹೇಳಿರುವ ಉಪಾಖ್ಯಾನವು>
25-174

<ವಿಶ್ವೇದೇವತೆಗಳ ವಿಷಯ>
2-49
14-102
15-434 
19-245
19-336
28-272

<ವಿಶ್ವೇದೇವತೆಗಳ ವಿಷಯದಲ್ಲಿ ವಿದಗ್ಧಶಾಕಲ್ಯಋಷಿಗೂ ಯಾಜ್ಞವಲ್ಕ್ಯಋಷಿಗೂ ನಡೆದ ಸಂವಾದ>
19-346

<ವಿಶ್ವೇದೇವತಾಕಸೂಕ್ತದ್ರಷ್ಟೃಗಳಾದ ಋಷಿಗಳ ಹೆಸರುಗಳು ಮತ್ತು ಸೂಕ್ತಗಳು (ಬೃ.ದೇ.)>
28-746 
28-771
28-280

<ವಿಶ್ವೇದೇವತಾಕವಾದ ಸೂಕ್ತಗಳನ್ನು ನೈಮಿತ್ತಿಕವಾಗಿ (ಪ್ರಸಂಗವಶದಿಂದ) ಸ್ತುತರಾಗುವ ದೇವತೆಗಳು>
28-289

<ವಿಷನಾಶಕವಾದ ಓಷಧಿಗಳ ಸ್ವರೂಪ>
14-338

<ವಿಷಾಪಹಾರಕ ಸಣ್ಣ ಪಕ್ಷಿಗಳು>
14-361
14-363

<ವಿಷೂಚೀ ಎಂಬ ಶಬ್ದದ ವಿವರಣೆ>
12-498

<ವಿಷ್ಣ್ವಾಪಿಯ ವಿಷಯ>
9-298


<ವಿಷ್ಣ್ವಾಪ್ವೇ ಎಂಬ ಶಬ್ದದ ಅರ್ಥವಿವರಣೆ>
25-270

<ವಿಷ್ಣ್ವಾಪು ಎಂಬುವನನ್ನು ಅಶ್ವಿನೀದೇವತೆಗಳು ಕರೆತಂದು ಅವನ ತಂದೆಯಾದ ವಿಶ್ವಕನಿಗೆ ಒಪ್ಪಿಸಿದ ವಿಚಾರ ಮತ್ತು ಘೋಷಾ ಎಂಬ ಸ್ತ್ರೀಗೆ ಪತಿಯನ್ನು ಒದಗಿಸಿ ಕೊಟ್ಟ ವಿಚಾರ>
9-220

<ವಿಷ್ಣು ಎಂಬ ದೇವತೆಯ ವಿಷಯ>
5-550
14-164
28-728


<ವಿಷ್ಣುಶಬ್ದದ ರೂಪನಿಷ್ಪತ್ತಿ>
3-119
11-570

<ವಿಷ್ಣುವಿನ ಪಾದಪ್ರಕ್ಷೇಪ>
11-575

<ವಿಷ್ಣುವಿನ ಪರಮಪದಸ್ವರೂಪ>
11-589

<ವಿಷ್ಣುವಿನ ಪರಮಪದ ಮತ್ತು ಪಾದಪ್ರಕ್ಷೇಪಗಳ ವಿವರಣೆ>
11-596

<ವಿಷ್ಣುವಿನ ತ್ರಿವಿಕ್ರಮ ವರ್ಣನೆ>
11-623
14-166

<ವಿಷ್ಣುವಿಗೂ ವರುಣನಿಗೂ, ವಿಷ್ಣುವಿಗೂ ಅಶ್ವಿನೀದೇವತೆಗಳಿಗೂ ಇರುವ ಸಂಬಂಧ>
11-656

<ವಿಷ್ಣುವಿನ ಸರ್ವವ್ಯಾಪಕತ್ವ>
12-538

<ವಿಷ್ಣುವಿನ ಸಹಾಯವನ್ನು ವೃತ್ರವಧಾರ್ಥವಾಗಿ ಇಂದ್ರನು ಪ್ರಾರ್ಥಿಸಿದ ವಿಷಯ>
25-515

<ವಿಷ್ಣುವಿಗೆ ಶಿಪಿವಿಷ್ಟ ಎಂಬ ಹೆಸರು ಬರಲು ಕಾರಣ ಮತ್ತು ಶಿಪಿವಿಷ್ಟ ಶಬ್ದದ ನಿರ್ವಚನ>
23-199

<ವಿಷ್ಣುವು ವಸಿಷ್ಠ ಋಷಿಗೆ ಸಹಾಯ ಮಾಡಿದ ವಿಚಾರ–ಶಿಪಿವಿಷ್ಟ ಶಬ್ದದ ಅರ್ಥವಿವರಣೆ ಇತ್ಯಾದಿ>
23-201

<ವಿಷ್ಣುವು ಯಜ್ಞರೂಪದಿಂದ ಹೊರಟುಹೋದಾಗ ಇಂದ್ರನಿಗೂ ವಿಷ್ಣುವಿಗೂ ನಡೆದ ಸಂವಾದ>
20-765

<ವಿಷ್ಣುವು ಇಂದ್ರನಿಗೆ ಸಹಾಯ ಮಾಡಿದ ವಿಷಯ (ಬೃ.ದೇ.)>
28-932

<ವಿಷ್ಣುಸೂಕ್ತಗಳ ವೈಶಿಷ್ಟ್ಯ>
11-566

<ವಿಷ್ಣೋರ್ನುಕಂ ಎಂಬ ಋಕ್ಕಿನ ವಿಶೇಷ ವಿನಿಯೋಗ>
11-566

<ವಿಸದೃಶಾ ಎಂಬ ಶಬ್ದ>
9-34

<ವಿ ಹಿ ಸೋತೋರಸೃಕ್ಷತ ಎಂಬ ಸೂಕ್ತದ ವಿಷಯದಲ್ಲಿ ಋಗ್ವಿಧಾನದಲ್ಲಿ ಹೇಳಿರುವ ವಿನಿಯೋಗ ಇತ್ಯಾದಿ>
29-454

<ವೀತಹವ್ಯ ಶಬ್ದವಿವರಣೆ>
20-535

<ವೀರಶಬ್ದ ನಿರ್ವಚನ ಮತ್ತು ಅರ್ಥವಿವರಣೆ>
15-481

<ವೀಳು ಎಂಬ ಶಬ್ದ>
6-290

<ವೃಕ ಶಬ್ದಾರ್ಥವಿವರಣೆ>
9-257 
9-520
18-638 
14-527
15-122
27-967

<ವೃಕ್ತಬರ್ಹಿಷಃ ಎಂಬ ಶಬ್ದದ ಅರ್ಥವಿವರಣೆ>
15-577
17-476

<ವೃಜನ ಶಬ್ದದ ನಾನಾರ್ಥಗಳು ಮತ್ತು ಪ್ರಯೋಗ>
6-389
13-223

<ವೃತಂಚಯ ಎಂಬ ಶಬ್ದದ ವಿವರಣೆ>
14-881

<ವೃತ್ರ, ವೃತ್ರಹನ್‍ ಎಂಬ ಶಬ್ದಗಳು>
8-179

<ವೃತ್ರ ಶಬ್ದದ ನಿಷ್ಪತ್ತಿ>
6-580

<ವೃತ್ರ ಶಬ್ದದ ನಾನಾರ್ಥಗಳು>
16-212

<ವೃತ್ರಹಂತಮಂ ಎಂಬ ಶಬ್ದವು ಅಗ್ನಿಗೆ ವಿಶೇಷಣವಾಗಿರುವ ಸಂದರ್ಭ>
6-518

<ವೃತ್ರಹನನ ಕಾಲದಲ್ಲಿ ಮರುತ್ತುಗಳು ಇಂದ್ರನಿಗೆ ಸಹಾಯಮಾಡಿದ ವಿಚಾರ>
12-681

<ವೃತ್ರಹನನಾರ್ಥವಾಗಿ ದೇವತೆಗಳು ತಮ್ಮ ಶಕ್ತಿಯನ್ನು ಇಂದ್ರನಿಗೆ ಕೊಟ್ಟ ವಿಚಾರ ವೃತ್ರವಧಾರ್ಥವಾಗಿ ಇಂದ್ರನು ವಿಷ್ಣುವಿನ ಸಹಾಯವನ್ನು ಪ್ರಾರ್ಥಿಸಿದ ವಿಚಾರ>
25-515

<ವೃತ್ರಾಸುರನ ವೃತ್ತಾಂತ>
2-123
5-691

<ವೃತ್ರಾಸುರನ ಉತ್ಪತ್ತಿ ವಿಷಯ>
3-132

<ವೃತ್ರಾಸುರವಧ ವಿಚಾರ (ಐತರೇಯ ಬ್ರಾಹ್ಮಣ ಮತ್ತು ತೈತ್ತಿರೀಯ ಸಂಹಿತೆಯಲ್ಲಿರುವಂತೆ)>
18-9

<ವೃತಾಸುರವಧವರ್ಣನೆ>
28-161

<ವೃಶಃ ಎಂಬ ಋಷಿ>
18-730

<ವೃಶಋಷಿಯು ಅಗ್ನಿಯನ್ನು ಪ್ರಾರ್ಥಿಸಿದ ಸನ್ನಿವೇಶ>
18-755

<ವೃಶಜಾನ ಮತ್ತು ತ್ರ್ಯರುಣ ಎಂಬುವರ ವೃತ್ತಾಂತ (ಬೃ.ದೇ.)>
28-852

<ವೃಷಖಾದಯಃ ಎಂಬ ಶಬ್ದ>
6-132

<ವೃಷಖಾದಯಃ, ಖಾದಿ ಮೊದಲಾದ ಶಬ್ದಗಳ ಅರ್ಥವಿಚಾರದಲ್ಲಿ ಆಂಗ್ಲಪಂಡಿತರ ಅಭಿಪ್ರಾಯಗಳು>
13-43

<ವೃಷಣ ಶಬ್ದದ ನಾನಾರ್ಥಗಳು ಮತ್ತು ಪ್ರಯೋಗ>
13-252

<ವೃಷಣಶ್ವ ಎಂಬ ರಾಜನ ವಿಷಯ>
5-201

<ವೃಷನ್‍, ವೃಷಾ ಎಂಬ ಶಬ್ದಗಳು>
8-l68
13-252

<ವೃಷನ್‍ ಶಬ್ದದ ವೈಶಿಷ್ಟ್ಯ>
l3-436

<ವೃಷನ್ನಿಂದ್ರವೃಷಪಾಣಾಸಃ ಎಂಬ ಋಕ್ಕಿನ ವಿಶೇಷ ವಿನಿಯೋಗ>
11-65

<ವೃಷಭಶಬ್ದಾರ್ಥ ಇತ್ಯಾದಿ>
5-682
6-530 
10-336 
13-219 
13-252
14-783
14-883

<ವೃಷಾಕಪಿಯ ವಿಷಯ (ಬೃ.ದೇ.)>
28-728
29-448

<ವೃಷಾಕಪಿ, ಇಂದ್ರ, ಇಂದ್ರಾಣಿ ಇವರುಗಳಿಗೆ ನಡೆದ ಸಂವಾದ>
29-451

<ವೃಷಾಕಪಾಯಿ ರೇವತಿ ಎಂಬ ಋಕ್ಕಿನ ನಿರುಕ್ತ>
29-474

<ವೃಕ್ಷಶಬ್ದದ ಅವಯವಾರ್ಥ>
19-664

<ವೃಕ್ಷಸ್ಯ ನು ತೇ ಪುರುಹೂತವಯಾಃ ಎಂಬ ವಾಕ್ಯಕ್ಕೆ ಯಾಸ್ಕರ ನಿರ್ವಚನ ಮತ್ತು ಅರ್ಥವಿವರಣೆ>
20-778

<ವೆಂಕಟಮಾಧವ>
1-235

<ವೇತಸು, ತುಗ, ದಶೋಣಿ, ತೂತುಜಿ, ಇಭ ಎಂಬ ಅಸುರನನ್ನು ಇಂದ್ರನು ಸೋಲಿಸಿದ ವಿಚಾರ>
20-730

<ವೇತಸೂನ್‍ ಎಂಬ ಶಬ್ದದ ವಿವರಣೆ>
28-201

<ವೇದವೆಂದರೇನು?>
1-1
28-724

<ವೇದ ಶಬ್ದದ ವ್ಯುತ್ಪತ್ತಿ>
1-1
6-277

<ವೇದದ ಶಾಖೆಗಳು>
1-44

<ವೇದಭಾಷ್ಯ ಭೂಮಿಕೆಗಳು>
1-264

<ವೇದವನ್ನು ವ್ಯಾಸಂಗಮಾಡುವ ಕ್ರಮ>
1-314

<ವೇದದಲ್ಲಿ ಪ್ರತಿಪಾದಿತವಾದ ವಿಷಯಗಳು>
1-317

<ವೇದಕಾಲ ವಿಮರ್ಶೆ>
1-336

<ವೇದದ ಕಾಲದಲ್ಲಿ ಸಾಮಾಜಿಕ ಜೀವನ>
1-326

<ವೇದಕ್ಕೆ ಲಕ್ಷಣಪ್ರಮಾಣಾದಿಗಳು ಇಲ್ಲದ್ದರಿಂದ ವೇದವೆಂಬ ಗ್ರಂಥವೇ ಇಲ್ಲವೆಂಬ ಪೂರ್ವಪಕ್ಷವು>
1-362

<ವೇದಕ್ಕೆ ಲಕ್ಷಣಪ್ರಮಾಣಾದಿಗಳು ಇವೆಯೆಂದು ಸಿದ್ಧಾಂತಸಮರ್ಥನೆ>
1-367

<ವೇದಕ್ಕೆ ವ್ಯಾಖ್ಯಾನವೇಕೆ ? ಅದರ ಮಹತ್ವ ಮತ್ತು ಉಪಯೋಗವೇನು? ಇತ್ಯಾದಿ ವಿಷಯಗಳ ವಿಮರ್ಶೆ>
1-368

<ವೇದಮಂತ್ರಗಳಲ್ಲಿ ಕೆಲವಕ್ಕೆ ಸರಿಯಾದ ಅರ್ಥವಿಲ್ಲ, ಕೆಲವು ಮಂತ್ರಗಳು ಸಂದಿಗ್ಧಾರ್ಥವುಳ್ಳವು, ಕೆಲವು ನಿರ್ಜೀವವಸ್ತುಗಳನ್ನು ಸಂಬೋಧಿಸುವವು ಇತ್ಯಾದಿ ವಿಷಯಗಳಲ್ಲಿ ಪೂರ್ವಪಕ್ಷ ಮತ್ತು ಸಿದ್ಧಾಂತ>
1-369

<ವೇದಮಂತ್ರಗಳ ಲಕ್ಷಣಾದಿವಿಚಾರ–ಈ (ಸಂಬಂಧವಾದ) ಈಮಾಂಸಾ ಸೂತ್ರಗಳ ವಿಮರ್ಶೆ–ಪೂರ್ವಪಕ್ಷ ಮತ್ತು ಸಿದ್ಧಾಂತ>
1-374

<ವೇದವು ಪುರುಷರಚಿತವೆಂಬ ವಿಷಯನಿರಾಕರಣೆ>
1-465

<ವೇದಮಂತ್ರವಿಭಾಗಗಳು–ಋಗ್ವೇದ ಇತ್ಯಾದಿ>
1-487

<ವೇದವಿಷಯದಲ್ಲಿ ವಿಷಯ, ಪ್ರಯೋಜನ, ಸಂಬಂಧ, ಅಧಿಕಾರಿ ಎಂಬ ಅನುಬಂಧ ಚತುಷ್ಟಯದ ನಿರೂಪಣೆ>
1-539

<ವೇದವನ್ನು ಓದುವುದರಿಂದ ಆಗುವ ಪ್ರಯೋಜನವೇನು? ಇದನ್ನು ಎಂತಹ ಪುರುಷನು ಕಲಿಯಬೇಕು? ಇತ್ಯಾದಿ ವಿಷಯವಿಮರ್ಶೆ>
1-539

<ವೇದಗಳ, ಷಡಂಗಗಳ ಉಪಯೋಗವೇನು ಇತ್ಯಾದಿ ವಿಷಯ>
1-546

<ವೇದಮಂತ್ರಗಳ ವಿಭಾಗಕ್ರಮ, ಋಷಿಕ್ರಮ, ದೇವತಾಕ್ರಮ, ಛಂದಃಕ್ರಮ>
2-3

<ವೇದಗಳ ದೇವತೆಗಳು (ಬೃ.ದೇ.)>
28-1023

<ವೇದಾಂಗಗಳು>
1-27

<ವೇದಾಭ್ಯಾಸಕ್ರಮ>
1-304

<ವೇದಾಭ್ಯಾಸ ವಿಚಾರದಲ್ಲಿ ಆಧುನಿಕರ ಪ್ರಯತ್ನಗಳು>
1-309

<ವೇದಾರ್ಥ ವಿಚಾರದಲ್ಲಿ ಪ್ರಾಚೀನರ ಪ್ರಯತ್ನಗಳು>
1-296

<ವೇದಾಪೌರುಷೇಯತ್ವಸಮರ್ಥನೆ>
1-465

<ವೇದಾಧ್ಯಯನವನ್ನು ಅರ್ಥಸಹಿತವಾಗಿ ಮಾಡಬೇಕೆಂಬ ವಿಚಾರದಲ್ಲಿ ವಿಷಯವಿಮರ್ಶೆ>
1-490

<ವೇದಾರ್ಥಜ್ಞಾನವಿಲ್ಲದೆ ವೇದವನ್ನು ಕೇವಲ ಕಂಠಪಾಠ ಮಾಡುವ ಪುರುಷನ ನಿಂದೆ>
1-520

<ವೇದಾರ್ಥಜ್ಞಾನದ ಪ್ರಶಂಸೆ>
1-520

<ವೇದಾಂಗಗಳ ಸ್ವರೂಪನಿರೂಪಣೆ>
1-547

<ವೇದಾರ್ಥಜ್ಞಾನಕ್ಕೆ ಬೇಕಾಗುವ ಸಹಾಯ ಗ್ರಂಥಗಳು, ಅಧಿಕಾರಿಲಕ್ಷಣ, ವೇದಾರ್ಥಜ್ಞಾನದಿಂದ ಆಗುವ ಪ್ರಯೋಜನ ಇತ್ಯಾದಿ>
1-585

<ವೇದಾರ್ಥಜ್ಞಾನಕ್ಕೆ ಅಧಿಕಾರಿಯಾದವನ ಲಕ್ಷಣ>
1-588

<ವೇದಾರ್ಥವನ್ನು ತಿಳಿದವರಿಗೆ ಇರುವ ಗೌರವ>
12-579

<ವೇದ ಶಬ್ದದ ನಿರ್ವಚನ>
30-659

<ವೈತಸ ಶಬ್ದನಿರ್ವಚನ>
29-425
30-28

<ವೈದಿಕಮತ>
1-317

<ವೈಲಸ್ಥಾನ ಶಬ್ದವಿವರಣೆ>
10-538

<ವೈಶ್ವದೇವಸೂಕ್ತಗಳಲ್ಲಿ ನೈಮಿತ್ತಿಕವಾಗಿ ಪ್ರಶಂಸಿತರಾಗತಕ್ಕ ದೇವತೆಗಳು>
19-247
28-1027

<ವೈಶ್ವದೇವಸೂಕ್ತಗಳ ದ್ರಷ್ಟೃಗಳಾದ ಋಷಿಗಳು (ಬೃ.ದೇ.)>
28-746
28-771

<ವೈಶ್ವದೇವಸೂಕ್ತಗಳ ಸ್ವರೂಪ ಮತ್ತು ವಿವರಣೆ (ಬೃ.ದೇ.)>
29-805
28-747


<ವೈಶ್ವದೇವಸೂಕ್ತಗಳ ಲಕ್ಷಣ (ಬೃ.ದೇ.)>
28-768

<ವೈಶ್ವಾನರಾಗ್ನಿಯ ವಿಷಯ>
14-107
20-440
29-525


<ವೈಶ್ವಾನರಾಗ್ನಿಯ ಮಹಿಮೆ>
8-137
15-551

<ವೈಶ್ವಾನರ ಶಬ್ದದ ರೂಪನಿಷ್ಪತ್ತಿ>
5-455

<ವೈಶ್ವಾನರಾಗ್ನಿಯ ಸ್ವರೂಪಕಥನ (ಶತಪಥಬ್ರಾಹ್ಮಣದಲ್ಲಿರುವಂತೆ)>
8-140

<ವೈಶ್ವಾನರ ಶಬ್ದಾರ್ಥ ವಿಮರ್ಶೆ>
16-496

<ವೈಶ್ವಾನರಾಗ್ನಿಯು ಯಾರು? ಇವನು ಮಧ್ಯಮಸ್ಥಾನ ದೇವತೆಯೇ>
17-788 
20-440

<ವೈಶ್ವಾನರನೇ ಆದಿತ್ಯನು ಎಂದು ಪೂರ್ವದ ಯಾಜ್ಞಿಕರ ಮತ, ಈ ವಿಷಯದಲ್ಲಿ ಆಕ್ಷೇಪ ಸಮಾಧಾನಗಳು>
17-789 
20-442

<ವೈಶ್ವಾನರನೇ ಪಾರ್ಥಿವಾಗ್ನಿಯು, ಆದಿತ್ಯನಲ್ಲ ಎಂಬ ಶಾಕಪೂಣಿ ಎಂಬ ಆಚಾರ್ಯರ ಮತವು ಈ ವಿಷಯದಲ್ಲಿ ಆಕ್ಷೇಪ ಸಮಾಧಾನಗಳು>
17-791 
20-444

<ವೈಶ್ವಾನರಾಗ್ನಿಯ ವಿಷಯವಿಮರ್ಶೆ (ಯಾಸ್ಕರ ನಿರುಕ್ತದಂತೆ)>
17-787

[ಶ]
<ಶಕಮಯಂ ಧೂಮಂ ಎಂಬ ಶಬ್ದಗಳ ಅರ್ಥವಿವರಣೆ>
12-602

<ಶಕುನ ಅಥವಾ ಕಪಿಂಜಲಪಕ್ಷಿಯು ಶುಭಶಕುನವನ್ನು ನುಡಿಯುವ ವಿಷಯ>
15-488

<ಶಚೀವಸು ಶಬ್ದದ ವಿವರಣೆ>
11-61

<ಶತಬಲಾಕ್ಷ ಶಾಖಾ>
1-88

<ಶತಹಿಮಾಃ>
6-420

<ಶಂತನು ಮತ್ತು ದೇವಾಪಿ ಎಂಬ ರಾಜರ ವಿಚಾರ>
14-301
30-97

<ಶತಕ್ರತು ಶಬ್ದದ ವಿವರಣೆ>
14-792
17-138

<ಶತರ್ಚಿನರು ಯಾರು? (ಬೃ.ದೇ.)>
28-789


<ಶತಂ ಜೀವ ಶರದಃ ಎಂಬ ಋಕ್ಕಿನ ನಿರುಕ್ತ>
30-1090

<ಶಂಬರನೆಂಬ ಅಸುರನ ವಿಷಯ>
5-174 
5-694
8-274 
10-550
14-672
17-100
18-176
18-231
19-201
20-687

<ಶಂಬರಾಸುರನೊಡನೆ ಇಂದ್ರನು ಯುದ್ಧಮಾಡಿದ ವಿಷಯ>
5-315

<ಶಂಬರನೆಂಬ ಅಸುರನ ನೂರು ಪಟ್ಟಣಗಳನ್ನು ಇಂದ್ರನು ನಾಶಮಾಡಿದ ವಿಚಾರ>
14-739

<ಶಂಬರಾಣಿ ಎಂಬ ಶಬ್ದದ ವಿವರಣೆ>
15-16

<ಶಬ್ದಗಳ ಅರ್ಥವಿವರಣೆ (ಬೃ.ದೇ.)>
28-740

<ಶಬ್ದ ಬ್ರಹ್ಮರೂಪವಾದ ವಾಕ್ಕಿಗೆ ಹೃದಯಾಕಾಶವೇ ಸ್ಥಾನವು>
12-595

<ಶ್ರಮಯುವಃ ಎಂಬ ಶಬ್ದ>
6-341

<ಶಯುವೆಂಬ ಋಷಿಯ ಬಂಜೆಯಾದ ಹಸುವನ್ನು ಕರುಹಾಕಿ ಹಾಲುಕೊಡುವಂತೆ ಅಶ್ವಿನೀದೇವತೆಗಳು ಮಾಡಿದ ವಿಚಾರ>
8-736
9-294
9-393
9-440
13-509
22-489
27-957

<ಶಂಯುವಿನ ವಿಷಯ>
4-477
8-553

<ಶಂಯು ಶಬ್ದಾರ್ಥ ವಿಚಾರ>
16-354
27-495

<ಶರದಃ>
6-346

<ಶರತ್‍ ಶಬ್ದಾರ್ಥ>
7-205
13-260
14-673


<ಶರನೆಂಬ ಋಷಿಯು ಬಾಯಾರಿಕೆಯಿಂದ ಪರಿತಪಿಸುತ್ತಾ ದಾರಿಯಲ್ಲಿ ಸಿಕ್ಕಿದ ಆಳವಾದ ಬಾವಿಯಲ್ಲಿರುವ ನೀರನ್ನು ಕುಡಿಯಲು ಅಸಮರ್ಥನಾಗಿ ಅಶ್ವಿನೀದೇವತೆಗಳನ್ನು ಸ್ತುತಿಸಲು ಅವರು ಬಂದು ಬಾವಿಯಲ್ಲಿದ್ದ ನೀರು ಮೇಲಕ್ಕೆ ಉಕ್ಕಿಬರುವಂತೆ ಮಾಡಿ ಶರನ ದಾಹವನ್ನು ಶಮನ ಮಾಡಿದರೆಂಬ ವಿಚಾರ>
9-293

<ಶರ್ಧ ಶಬ್ದದ ನಾನಾರ್ಥ ಮತ್ತು ಪ್ರಯೋಗ>
10-54
11-39

<ಶರ್ಯಾತ(ತಿ) ಎಂಬ ರಾಜನ ವೃತ್ತಾಂತ>
8-800
17-147

<ಶರ್ಯಾತ ಶಬ್ದದ ಅರ್ಥವಿವರಣೆ>
11-439

<ಶರ್ಯಾಃ ಎಂಬ ಶಬ್ದದ ನಿರ್ವಚನ ಮತ್ತು ಅರ್ಥವಿವರಣೆ>
27-127

<ಶರ್ಯಾತಿ ಎಂಬ ಋಷಿಯ ಯಜ್ಞದಲ್ಲಿ ಇಂದ್ರನು ಸೋಮಪಾನ ಮಾಡಿದ ವಿಚಾರ>
17-147

<ಶವಃ>
6-403

<ಶವಂತೀಃ ಪರಾಯಂತೀ ಎಂಬ ಪರಿಶಿಷ್ಟ ಮಂತ್ರಗಳು>
23-276

<ಶಶೀಯಸಿಯ (ತರಂತರಾಜನ ಮಹಿಷಿಯಾದ) ದಾನಸ್ತುತಿ>
19-803

<ಶಂಸ್ತಾ ಎಂಬ ಋತ್ವಿಜನ ಕರ್ತವ್ಯ>
12-166

<ಶ್ರದ್ಧಧಾನಃ>
8-370

<ಶ್ರದ್ಧಯಾಗ್ನಿಃ ಎಂಬ ಋಕ್ಕಿನ ನಿರುಕ್ತ>
30-1015

<ಶ್ರದ್ಧಾ ಎಂಬ ದೇವತೆ>
5-644
30-1012

<ಶ್ರದ್ಧಾ ಶಬ್ದದ ರೂಪನಿಷ್ಪತ್ತಿ>
30-1014

<ಶ್ರದ್ಧಾರಹಿತರಿಗೆ ಆಗತಕ್ಕ ಬಾಧಕಗಳಿಂದ ಇಂದ್ರನ ಅಸ್ತಿತ್ವವು ಸ್ಥಿರಪಡುವುದು ಎಂಬ ವಿಷಯ>
16-666

<ಶ್ರವಃ ಶಬ್ದಾರ್ಥ ವಿವರಣೆ>
10-353
12-81

<ಶ್ವಫ್ನೀ ಎಂಬ ಶಬ್ದದ ಅರ್ಥವಿವರಣೆ>
18-34
28-31

<ಶಾಕಲಶಾಖೆಗಳು>
1-52

<ಶಾಕಲ್ಯಸಂಹಿತೆ>
1-60

<ಶಾಂಖ್ಯಾಯನ ಶಾಖೆಗಳು>
1-72

<ಶಾರ್ಯಾತನ ವಿಷಯ>
5-197

<ಶಾಲೀಯಶಾಖಾ>
1-58

<ಶಾಸದ್ವಹ್ನಿರ್ದುಹಿತುಃ ಎಂಬ ಋಕ್ಕಿನ ನಿರುಕ್ತ>
16-719

<ಶ್ಯಾವನೆಂಬ ಋಷಿಯು ಕುಷ್ಠರೋಗದಿಂದ ನರಳುತ್ತಿದ್ದಾಗ ಅಶ್ವಿನೀದೇವತೆಗಳು ಅವನ ರೋಗವನ್ನು ಪರಿಹಾರಮಾಡಿ ರಕ್ಷಿಸಿದ ವಿಚಾರ>
9-341

<ಶ್ಯಾವಾಶ್ವ ಋಷಿಯ ವಿಷಯ>
13-436
29-591
28-862

<ಶ್ಯಾವಾಶ್ವ ಋಷಿಯ ವಿವಾಹ ವಿಷಯ>
19-794

<ಶ್ಯಾವಾಶ್ವ ಋಷಿಯ ವಿವಾಹ–ಶೌನಕಮಹರ್ಷಿಯು ಹೇಳಿರುವಂತೆ>
19-822

<ಶ್ಯಾವಾಶ್ವ ಋಷಿಯ ತನ್ನ ವಿವಾಹ ವಿಚಾರವಾಗಿ ರಾತ್ರಿದೇವತೆಯನ್ನು ರಥ ವೀತಿ ರಾಜನಲ್ಲಿಗೆ ಕಳುಹಿಸುವುದು>
19-824

<ಶ್ರಾಯಂತ ಇವ ಸೂರ್ಯಂ ಎಂಬ ಋಕ್ಕಿನ ನಿರುಕ್ತ>
25-488

<ಶಿಕ್ವಭಿಃ ಎಂಬ ಶಬ್ದದ ವಿವರಣೆ>
11-193

<ಶಿರಂಬಿಠಸ್ಯ ಎಂಬ ಶಬ್ದದ ವಿವರಣೆ>
30-1144

<ಶಿಷ್ಯನ ಕರ್ತವ್ಯ ಕರ್ಮಗಳು, ಶಿಷ್ಯನಲ್ಲಿರಬೇಕಾದ ಗುಣಗಳು ಇತ್ಯಾದಿ>
1-588

<ಶಿಕ್ಷತಿ ಶಬ್ದದ ನಾನಾರ್ಥಗಳು>
8-651

<ಶಿಕ್ಷಾ ಎಂಬ ವೇದಾಂಗ>
1-27
1-547

<ಶೀರಂ ಮತ್ತು ಪಾವಕಶೋಚಿಷಂ ಎಂಬ ಶಬ್ದಗಳ ನಿರ್ವಚನ>
24-422
25-546

<ಶ್ರೀ ಸೂಕ್ತವು–(೧೫ ಋಕ್ಕು) ಮತ್ತು ಕೆಲವು ಶ್ಲೋಕಗಳು ಸಹಿತ. ಈ ಸೂಕ್ತಕ್ಕೆ ಪ್ರತಿ ಋಕ್ಕಿಗೂ ಋಷಿದೇವತಾಛಂದಸ್ಸುಗಳು, ಅಂಗನ್ಯಾಸ, ಕರನ್ಯಾಸ, ಧ್ಯಾನ ಮೊದಲಾದ ಜಪಕ್ರಮಗಳು, ಸಸ್ವರಮಂತ್ರ, ವಿದ್ಯಾರಣ್ಯಭಾಷ್ಯ, ಪೃಥಿವೀಧರಾಚಾರ್ಯಭಾಷ್ಯ, ಪ್ರತಿಪದಾರ್ಥ ಭಾವಾರ್ಥಗಳು, ಆಯಾ ಮಂತ್ರಗಳ ಜಪಕ್ರಮ, ಫಲಸ್ತುತಿ ಇವುಗಳ ವಿಷಯದಲ್ಲಿ ಇತರ ಗ್ರಂಥಗಳಲ್ಲಿ ಹೇಳಿರುವ ವಿಶೇಷ ವಿಷಯಗಳ ಸಹಿತ ವಿಸ್ತಾರವಾದ ವಿವರಣೆ>
20-267

<ಶುಕ್ಲ ಯಜುರ್ವೇದದ ಪ್ರಾಚೀನತೆ>
1-95

<ಶುಕ್ಲ ಯಜುರ್ವೇದದ ಮಾಧ್ಯಂದಿನಶಾಖೆಯ ಹದಿನೇಳು ಮುಖ್ಯ ಭೇದಗಳು>
1-97

<ಶುಕ್ಲ ಯಜುರ್ವೇದದ ಕಾಣ್ವಶಾಖೆಯ ಹದಿನೈದು ಮುಖ್ಯ ಭೇದಗಳು>
1-100

<ಶುಕ್ರಂ ತೇ ಎಂಬ ಋಕ್ಕಿನ ನಿರ್ವಚನ>
21-375

<ಶುಕ್ರಾಸಃ ಎಂಬ ಶಬ್ದ>
10-596

<ಶುಚಯಃ ಎಂಬ ಶಬ್ದ>
10-596

<ಶುಚಂತಿ ಎಂಬ ರಾಜನನ್ನು ರಕ್ಷಿಸಿದ ವಿಚಾರ>
8-757

<ಶುಚಿ ಶಬ್ದಾರ್ಥ ವಿವರಣೆ>
17-540

<ಶುನಃಶೇಪೋಪಾಖ್ಯಾನ>
3-243

<ಶುನಃಶೇಪೋಪಾಖ್ಯಾನದ ವಿಷಯದಲ್ಲಿ (ನರಮೇಧಯಜ್ಞದ ವಿಷಯದಲ್ಲಿ) ಕೆಲವು ಪಾಶ್ಚಾತ್ಯ ಪಂಡಿತರ ಅಭಿಪ್ರಾಯವು ಸರಿಯಾದುದಲ್ಲವೆಂಬ ವಿಷಯವಿಮರ್ಶೆ>
18-757

<ಶುನ ಮತ್ತು ಶುನಾಸೀರ ದೇವತೆಗಳು>
18-631

<ಶುನಹೋತ್ರ ಶಬ್ದಾರ್ಥ>
14-80

<ಶುನಾಸೀರೀಯ ಹವಿರ್ಧಾನವೆಂಬ ಕರ್ಮದ ವಿಷಯ>
18-634

<ಶುಲ್ಪಸೂತ್ರಗಳು>
1-41

<ಶಿಶುಋಷಿಯು ಪಾಪಪರಿಹಾರಾರ್ಥವಾದ ತನ್ನ ಮತ್ತು ತನ್ನ ಮನೆಯವರ ಕಾರ್ಯಗಳನ್ನು ವಿವರಿಸುವುದು>
27-144

<ಶುಷ್ಣನೆಂಬ ಅಸುರನ ವಿಷಯ>
6-72
8-275 
10-550 
13-380
18-230
19-201
20-686
20-724


<ಶುಷ್ಣ ಶಬ್ದದ ವಿವರಣೆ>
9-581


<ಶುಷ್ಣನೆಂಬ ಅಸುರನೊಡನೆ ಇಂದ್ರನು ಯುದ್ಧಮಾಡಿದ ವಿಷಯ>
5-319

<ಶುಷ್ಣನೆಂಬ ಅಸುರನನ್ನು ಇಂದ್ರನು ಸಂಹಾರ ಮಾಡಿದ ವಿಷಯ>
14-733
20-724

<ಶ್ರುಷ್ಟಿ, ಆಯುಃ ಎಂಬ ಶಬ್ದಗಳ ನಾನಾರ್ಥಗಳು>
13-448

<ಶೇಪಂ ಎಂಬ ಶಬ್ದದ ವಿವರಣೆ ಮತ್ತು ನಿರ್ವಚನ>
2-425
30-28

<ಶ್ಯೇನಪಕ್ಷಿಯು ಸ್ವರ್ಗದಿಂದ ಸೋಮಲತೆಗಳನ್ನು ತಂದ ವಿಷಯ>
6-569
25-229

<ಶ್ವೇತಕೇತು, ಬಾಲಾಕಿ ಇವರ ವಿಷಯ>
12-327

<ಶ್ವೈತ್ರೇಯ ಮತ್ತು ದಶದ್ಯು ಎಂಬ ಋಷಿಗಳ ವಿಷಯ>
4-64

<ಶೈಲಾಲಕಶಾಖಾ>
1-88

<ಶೈಶಿರಿಶಾಖಾ>
1-59

<ಶೈಶಿರಿಶಾಖೆಯ ಪರಿಮಾಣ>
1-60

<ಶ್ರೋಣ>
8-763

<ಶ್ಲೋಕ ಶಬ್ದದ ವಿವರಣೆಗಳು, ಉದಾಹರಣೆಗಳು ಇತ್ಯಾದಿ>
17-421

<ಶೌನಕಶಾಖಾ>
1-92

<ಶೌನಕ ಆಂಗಿರಸರ ಸಂವಾದ>
12-571

<ಶ್ರೌತ ಸೂತ್ರಗಳು>
1-33

[ಷ]
<ಷಡ್ಢೋತೃ>
29-836

<ಷೋಳ್ಹಾ ಎಂಬ ಶಬ್ದದ ಅರ್ಥ ವಿವರಣೆ>
17-312

[ಸ]
<ಸಕಲ ವರ್ಣಗಳೂ ಬ್ರಹ್ಮನಿಂದಲೇ ಉತ್ಪನ್ನವಾದವು, ಅವುಗಳ ಶ್ರೇಷ್ಠತೆಯಲ್ಲಿ ವ್ಯತ್ಯಾಸವಿಲ್ಲ ಎಂಬ ವಿಷಯದಲ್ಲಿ ಶ್ರುತಿ ವಾಕ್ಯಗಳು>
29-922

<ಸಕ್ತುಮಿವ ತಿತ ಉನಾ ಎಂಬ ಋಕ್ಕಿನ ನಿರುಕ್ತ>
28-654

<ಸಜಾತರು ಮತ್ತು ಜ್ಞಾಸರಿಗೆ ಇರುವ ವ್ಯತ್ಯಾಸ>
8-634

<ಸಜೋಷಃ ಶಬ್ದಾರ್ಥ ವಿವರಣೆ>
14-615

<ಸಂಜ್ಞಾನಂ ಎಂಬ ಖಿಲಸೂಕ್ತ (ಬೃ.ದೇ.)>
28-1017

<ಸತ್‍, ಅಸತ್‍ ಎಂಬ ತತ್ತ್ವಗಳ ವಿಷಯ>
27-279
30-755

<ಸತ್ಪತಿ ಶಬ್ದಾರ್ಥ>
13-317

<ಸತ್ಯ ಶಬ್ದಾರ್ಥ>
11-521
13-317

<ಸತ್ಯ ಧರ್ಮಾಣಂ ಎಂಬ ಶಬ್ದ>
19-564

<ಸತ್ಯಶ್ರವಾಃ ಎಂಬ ಋಷಿಯು ವಯ್ಯ ಎಂಬುವನ ಪುತ್ರನೆಂಬ ವಿಷಯ>
20-156

<ಸತ್ರಾಸಾಹಂ ಎಂಬ ಶಬ್ದ>
6-550

<ಸದಾನ್ವೇ ಎಂಬ ಶಬ್ದ ವಿವರಣೆ>
30-1044

<ಸದ್ಯಶ್ಚಿದ್ಯಃ ಶವಸಾ ಎಂಬ ಋಕ್ಕಿನ ನಿರುಕ್ತ>
30-1217

<ಸದ್ಯೋ ಜಾತೋ ವ್ಯಮಿಈತ ಎಂಬ ಋಕ್ಕಿನ ನಿರುಕ್ತ>
30-368

<ಸಧಮಾದಃ>
9-601

<ಸನಕರೆಂಬ ವೃತ್ರಾನುಚರರು>
4-19

<ಸಧ್ರೀಚೀಃ ಎಂಬ ಶಬ್ದದ ವಿವರಣೆ>
4-48 
12-498

<ಸನೀಳಾಃ>
6-286

<ಸನುತಃ ಎಂಬ ಶಬ್ದದ ಅರ್ಥವಿವರಣೆ>
18-750

<ಸನೇಮಿ ಶಬ್ದಾರ್ಥ>
13-156

<ಸಪ್ತ ಚಕ್ರಂ ಎಂಬ ಶಬ್ದದ ವಿವರಣೆ>
15-401

<ಸಪ್ತಹೋತೃಗಳ ವಿಷಯ ವಿಮರ್ಶೆ>
15-693 
16-30
17-689
29-830

<ಸಪ್ತಹೋತೃಗಳು ಪ್ರಾತಸ್ಸವನ ಕಾಲದಲ್ಲಿ ಪಠಿಸಬೇಕಾದ ಮಂತ್ರಗಳು ಇತ್ಯಾದಿ>
15-703
26-56

<ಸಪ್ತ ಜಿಹ್ವೆಗಳು (ಅಗ್ನಿಯ)>
15-777

<ಸಪ್ತದಶ ಸೋಮದ ವಿವರಣೆ>
2-144

<ಸಪ್ತರ್ಷಿಗಳು>
5-625
6-23

<ಸಪ್ತರಶ್ಮಿಗಳ ವಿಚಾರ>
8-482 
11-383
12-365
14-675
14-820

<ಸಪ್ತವಧ್ರಿಯ ವೃತ್ತಾಂತ–ಈ ಋಷಿಯನ್ನೂ ಇವನ ಜ್ಞಾತಿಗಳು ಒಂದು ಪೆಟ್ಟಿಗೆಯಲ್ಲಿಟ್ಟು ಬಂಧಿಸುತ್ತಿದ್ದ ವಿಷಯ ಇತ್ಯಾದಿ>
20-145
28-870

<ಸಪ್ತಗಣಗಳು (ಮರುದ್ದೇವತೆಗಳ)>
24-189

<ಸಪ್ತಪರ್ವತಗಳನ್ನೂ ಇಂದ್ರನು ಸೀಳಿದ ಪೂರ್ತೀತಿಹಾಸ>
25-425

<ಸಪ್ತಮರ್ಯಾದಾಃ ಎಂಬ ಏಳುವಿಧ ಪಾಪಗಳ ವಿವರಣೆ>
27-273

<ಸಪ್ತಮಾತೃ, ದಿವಃ ಮೊದಲಾದ ಶಬ್ದಗಳ ಅರ್ಥವಿವರಣೆ>
4-110

<ಸಪ್ತನಿಧಿಗಳು (ಬೃ.ದೇ.)>
28-381

<ಸಪ್ತತಂತುಂ ಎಂಬ ಶಬ್ದದ ವಿವರಣೆ>
30-678

<ಸಪ್ತಾಸ್ಯಾಸನ್ಪರಿಧಯಃ ಎಂಬ ಋಕ್ಕಿನ ವಿವರಣೆ>
29-933

<ಸಮರ್ಯ ಶಬ್ದಾರ್ಥ ವಿಚಾರ>
13-463

<ಸಮಂತೇ ಶಬ್ದದ ವಿವರಣೆ>
14-79

<ಸುಮತಿ ಶಬ್ದದ ಅರ್ಥವಿವರಣೆ>
6-412

<ಸಮಂ, ಸಮಸ್ಯ ಎಂಬ ಶಬ್ದಗಳ ಅರ್ಥವಿವರಣೆ>
25-139

<ಸಮಸ್ಮಿಞ್ಜಾಯಮಾನೇ ಎಂಬ ಋಕ್ಕಿನ ನಿರುಕ್ತ>
30-32

<ಸಮಸ್ತವಸ್ತುಗಳಲ್ಲಿಯೂ ಚೈತನ್ಯವು ಅಡಗಿರುವುದು ಎಂದರೆ ಪರಮಾತ್ಮನು ಸಮಸ್ತವಸ್ತುಗಳನ್ನೂ ಪ್ರವೇಶಿಸಿ ವ್ಯಾಪಿಸಿರುವನು>
30-756

<ಸಮಾಸ ಪದಗಳು (ಆರುವಿಧ) (ಬೃ.ದೇ.)>
28-737

<ಸಮಿದ್ಧಶಬ್ದದ ವಿವರಣೆ>
14-227

<ಸಮಿದ್ಧಃ ಅಗ್ನಿಃ ಎಂಬ ದೇವತೆಯ ವಿಷಯ>
15-677
28-612

<ಸಮಿದ್ಧೋ ಅದ್ಯ ಎಂಬ ಋಕ್ಕಿನ ನಿರುಕ್ತ>
30-335

<ಸಮುದ್ರಾದೂರ್ಮಿಃ ಎಂಬ ಋಕ್ಕಿಗೆ ಪಾರ್ಥಿವಾಗ್ನಿ, ವೈದ್ಯುತಾಗ್ನಿ ಘೃತ, ಆಪಃ ಮೊದಲಾದ ದೇವತೆಗಳ ಪರವಾದ ವಿವರಣೆ>
18-645

<ಸಮುದ್ರಾದೂರ್ಮಿಃ ಎಂಬ ಋಕ್ಕಿನ ನಿರುಕ್ತ>
18-646

<ಸಮುದ್ರಜ್ಯೇಷ್ಠಾಃ ಎಂಬ ಶಬ್ದದ ಅರ್ಥ ವಿವರಣೆ>
22-291

<ಸಮುದ್ರಶಬ್ದಾರ್ಥ ವಿವರಣೆ>
3-504 
8-660
9-201 
12-54
12-599
13-78
13-291
19-691 
26-736
27-262


<ಸಮುದ್ರಮಧ್ಯದಲ್ಲಿ ಮುಳುಗುತ್ತಿದ್ದ ಭುಜ್ಯುವೆಂಬ ರಾಜನನ್ನು ಅಶ್ವಿನೀದೇವತೆಗಳು ನಾಲ್ಕು ದೋಣಿಗಳ ಮೂಲಕ ರಕ್ಷಿಸಿದ ವಿಚಾರ>
13-616

<ಸರಯು ಎಂಬ ನದಿಯ ವಿಷಯ>
19-643

<ಸರಣ್ಯೂದೇವಿಯ ವೃತ್ತಾಂತ (ಬೃ.ದೇ.)>
28-945 
28-947

<ಸರಣ್ಯೂ ಮತ್ತು ವಿವಸ್ವಾನ್‍>
27-531

<ಸರಣ್ಯೂವನ್ನು ವಿವಸ್ವಂತನು ಮದುವೆಯಾದ ವಿಚಾರ>
27-531

<ಸರಣ್ಯೂವಿನಲ್ಲಿ ಯಮ ಮತ್ತು ಯಮಿ ಎಂಬ ಇಬ್ಬರು ಅವಳೀಮಕ್ಕಳ ಉತ್ಪತ್ತಿ>
27-533

<ಸರಸ್ವತೀದೇವತೆಯ ವಿಷಯ, ಸರಸ್ವತೀ ಶಬ್ದದ ರೂಪನಿಷ್ಪತ್ತಿ ಇತ್ಯಾದಿ>
2-68
2-771
15-444
15-447

<ಸರಮಾ ಎಂಬ ದೇವಶುನಿಯ ವಿಷಯ>
5-683 
19-502
30-288

<ಸರಮೆಗೂ ಪಣಿಗಳಿಗೂ ನಡೆದ ಸಂಭಾಷಣೆ ಇತ್ಯಾದಿ>
6-15
19-503
28-999

<ಸರಸ್ವತೀ, ಇಳಾ, ಭಾರತೀ (ತಿಸ್ರೋದೇವ್ಯಃ) ಎಂಬ ದೇವತೆಗಳ ವಿಷಯ>
11-263 
30-359

<ಸರಸ್ವತೀ ಎಂಬ ನದೀ ಮತ್ತು ದೇವತೆಯ ಸ್ವರೂಪ>
12-642
29-84

<ಸರಸ್ವತೀ ಮತ್ತು ಸರಸ್ವಾನ್‍ ಎಂಬ ದೇವತೆಗಳ ವಿಷಯ>
12-670

<ಸರಸ್ವತೀ ಮತ್ತು ನಾಹುಷ (ಬೃ.ದೇ.)>
28-903

<ಸರಸ್ವಾನ್‍ ಎಂಬ ದೇವತೆ (ಬೃ.ದೇ.)>
28-724

<ಸರಸ್ವತೀದೇವತಾಕವಾದ ಮಂತ್ರಗಳು ಮತ್ತು ಇಂದ್ರದೇವತಾಕವಾದ ಸೂಕ್ತಗಳು (ಬೃ.ದೇ.)>
28-748

<ಸರಾಂಸಿ, ತ್ರಿಂಶತಂ ಎಂಬ ಶಬ್ದಗಳ ವಿವರಣೆ>
25-163

<ಸರ್ಪಿಃ ಎಂಬ ಶಬ್ದದ ವಿವರಣೆ>
14-525

<ಸರ್ವವೂ ಬ್ರಹ್ಮಸ್ವರೂಪವೆಂಬ ವಿಷಯ>
12-548

<ಸರ್ವನಾಮಗಳು (ಬೃ.ದೇ.)>
28-737

<ಸರ್ವಹುತ್‍ ಅಥವಾ ಸರ್ವಮೇಧವೆಂಬ ಯಜ್ಞದ ವಿಚಾರ>
29-878

<ಸವನತ್ರಯಗಳ ವಿವರಣೆ>
2-6
13-285
26-40

<ಸವನತ್ರಯಗಳ ಸ್ವರೂಪ ಮತ್ತು ಅವುಗಳನ್ನು ಪಠಿಸುವ ಸ್ತೋತ್ರ ಮತ್ತು ಶಸ್ತ್ರಮಂತ್ರಗಳ ವಿವರಣೆ>
1-652

<ಸವತಿಯ ಬಾಧಾಪರಿಹಾರಾರ್ಥವಾಗಿ ಮೂಲಿಕೆಯನ್ನು ಉಪಯೋಗಿಸುವ ವಿಧಾನ ಇತ್ಯಾದಿ>
30-964

<ಸಂವತ್ಸರ ಕಾಲ ಇತ್ಯಾದಿ ವಿವರಣೆ>
11-633 
12-308
17-330
17-351

<ಸಂವತ್ಸರ, ಪರಿವತ್ಸರ, ಇದುವತ್ಸರ ಇತ್ಯಾದಿ ವಿವರಣೆ>
12-370
26-824

<ಸಂವತ್ಸರಂ ಶಶಯಾನಾ ಎಂಬ ಋಕ್ಕಿನ ನಿರುಕ್ತ>
23-228

<ಸವ್ಯನ ವೃತ್ತಾಂತ (ಬೃ.ದೇ.)>
28-789

<ಸಂವಾದಗಳು (ಬೃ.ದೇ.)>
28-734

<ಸವಿತಾ ಯಂತ್ರೈಃ ಎಂಬ ಋಕ್ಕಿನ ನಿರುಕ್ತ>
30-997

<ಸವಿತುಃ (ಗಾಯತ್ರಿಯಲ್ಲಿರುವ) ಎಂಬ ಶಬ್ದದ ವಿವರಣೆ>
17-628

<ಸವಿತೃ ದೇವತೆಯ ವಿಷಯ>
5-542
15-332
18-562
20-175

<ಸವಿತೃಶಬ್ದಕ್ಕೆ ಯಾಸ್ಕರ ನಿರ್ವಚನ>
16-416
30-997

<ಸವಿತೃ ಶಬ್ದದ ನಿರ್ವಚನವನ್ನು ತಿಳಿಸುವ ಶ್ರುತಿವಾಕ್ಯಗಳು>
18-565

<ಸವಿತೃವಿಗೂ ಸೂರ್ಯನಿಗೂ ತಾದಾತ್ಮ್ಯ ಮತ್ತು ಭಿನ್ನತೆಯನ್ನು ಪ್ರತಿಪಾದಿಸುವ ಶ್ರುತಿವಾಕ್ಯಗಳು>
18-568

<ಸವಿತೃದೇವತೆಗೆ ಹಿರಣ್ಯಪಾಣಿಃ ಎಂಬ ಹೆಸರು ಬರಲು ಕಾರಣ>
17-253

<ಸವಿತೃ ಅಥವಾ ಆದಿತ್ಯನ ವಿಶ್ವಸಂಚಾರ ವಿಷಯ>
17-552

<ಸವಿತೃದೇವತೆಯ ಪ್ರತಿಪಾದಕವಾದುದರಿಂದ ಗಾಯತ್ರಿಯು ಸಾವಿತ್ರೀ ಶಬ್ದವಾಚ್ಯವು>
17-557

<ಸವಿತೃವಿನ ಏಕತತ್ತ್ವಪ್ರತಿಪಾದಕ ಸ್ವರೂಪ>
17-558

<ಸವಿತೃಶಬ್ದದ ನಿರ್ವಚನ ಮತ್ತು ಸವಿತೃವಿನ ವಿಶ್ವನಿಯಮಕವಾದ ದಿವ್ಯಸ್ವರೂಪವರ್ಣನೆ>
17-560

<ಸವಿತೃ ಮತ್ತು ವರುಣನ ಸಾಧಾರಣ ಧರ್ಮವನ್ನು ತೋರಿಸುವ ವಿಶೇಷಣಗಳು>
17-564

<ಸವಿತೃವಿನ ಭರ್ಗವೆಂದು ಪ್ರಸಿದ್ಧವಾದ ಈ ಜ್ಯೋತಿಯ ಸಾಕ್ಷಾತ್ಕಾರವು ಧ್ಯಾನದಿಂದ ಮಾತ್ರ ಸಾಧ್ಯ>
17-614

<ಸವಿತೃ, ಸೂರ್ಯ, ಆದಿತ್ಯ ಇವರಿಗಿರುವ ವ್ಯತ್ಯಾಸ>
30-903

<ಸಸರ್ಪರೀ ಎಂಬವಳ ವಿಷಯ>
17-212

<ಸಸಸ್ಯ ಎಂಬ ಶಬ್ದದ ನಿರ್ವಚನ, ಅರ್ಥವಿವರಣೆ>
15-760

<ಸಂಸ್ತವಿಕ ಮಂತ್ರಗಳ ಸ್ವರೂಪ ಮತ್ತು ಅವುಗಳ ಉದಾಹರಣೆಗಳು>
2-710
15-389

<ಸಸ್ನಿಮವಿಂದಚ್ಚರಣೇ ನದೀನಾಂ ಎಂಬ ಋಕ್ಕಿನ ನಿರುಕ್ತ>
30-910

<ಸಹಗಮನದ ವಿಷಯ>
27-615

<ಸಹವಸು ಎಂಬ ಅಸುರನ ವಿಷಯ>
14-704

<ಸಹಸ್ರಾಕ್ಷ ಶಬ್ದದ ಅರ್ಥ ವಿವರಣೆ>
3-152
6-561

<ಸಹಸೋ ಯಹೋ, ಸಹಸಸ್ಪುತ್ರ ಎಂಬ ಶಬ್ದಗಳ ವಿವರಣೆ>
3-372
6-443
20-421

<ಸಹಸ್ಕೃತ ಶಬ್ದ ವಿವರಣೆ>
4-566

<ಸಹಸ್ವಾನ್‍ ಎಂಬ ಶಬ್ದ>
8-125

<ಸಹರಕ್ಷ (ರಕ್ಷೋಹಾಗ್ನಿ) ಯಜ್ಞಾದಿಕರ್ಮಗಳಲ್ಲಿ ರಾಕ್ಷಸರ ಬಾಧೆಯಿಂದ ಕಾಪಾಡುವ ಅಗ್ನಿ>
18-849

<ಸಹದೇವನೆಂಬ ರಾಜನ ಪುತ್ರನಾದ ಸೋಮಕನ ವೃತ್ತಾಂತ>
17-924

<ಸಹಸ್ರ ಶಬ್ದಾರ್ಥ ವಿವರಣೆ>
29-698

<ಸಹಸ್ರಶೀರ್ಷಾ ಪುರುಷಃ ಎಂಬ ಋಕ್ಕಿನ ವಿವರಣೆ>
29-716

<ಸಕ್ಷಣಿ ಎಂಬ ಶಬ್ದ>
8-712

<ಸ್ಕಂದಸ್ವಾಮಿ>
2-232

<ಸ್ಕಂಭದೇಷ್ಣ ಶಬ್ದದ ವಿವರಣೆ>
13-33

<ಸ್ಕಂಭನೇಭಿಃ ಎಂಬ ಶಬ್ದದ ಅರ್ಥ ವಿವರಣೆ>
12-77

<ಸ್ತನಾಭುಜಃ ಎಂಬ ಶಬ್ದದ ಅರ್ಥವಿವರಣೆ>
9-524

<ಸ್ಯಃತ್ಯಂ ಸ್ಮ ಎಂಬ ಶಬ್ದಗಳ ಅರ್ಥ>
19-715

<ಸ್ವಃ ಎಂಬ ಶಬ್ದದ ನಾನಾರ್ಥಗಳು ಮತ್ತು ಪ್ರಯೋಗಗಳು ಉದಾಹರಣೆಸಹಿತವಾಗಿ>
8-455
9-225
11-627
13-246 
14-532
17-517

<ಸ್ವಂಚಾಃ ಎಂಬ ಶಬ್ದದ ವಿವರಣೆ>
19-293

<ಸ್ವತವಸಃ ಎಂಬ ಶಬ್ದದ ವಿವರಣೆ>
7-172

<ಸ್ವಧಾ ಶಬ್ದ ವಿವರಣೆ>
11-582 
12-695
13-273
13-402

<ಸ್ವಧಿತಿಶಬ್ದಾರ್ಥ ವಿವರಣೆ>
16-97

<ಸ್ವನಯ ಮತ್ತು ಕಕ್ಷೀವನ್‍ ಎಂಬುವರ ವೃತ್ತಾಂತ (ಬೃ.ದೇ.)>
28-798

<ಸ್ವನಯನ ದಾನಪ್ರಶಂಸೆ>
10-192

<ಸ್ವಪ್ನಸ್ವಪ್ನಾಧಿಕರಣೆ ಇತ್ಯಾದಿ ಪರಿಶಿಷ್ಟಮಂತ್ರಗಳು>
23-277

<ಸ್ವಪ್ರಶಂಸೆ ಮಾಡಿಕೊಂಡಿರುವ ದೇವತೆಗಳು (ಬೃ.ದೇ.)>
28-734

<ಸ್ವರಗಳ ದೇವತೆಗಳು (ಬೃ.ದೇ.)>
28-1023

<ಸ್ವರಸಾಮದ ವಿಚಾರ>
13-363

<ಸ್ವರವ್ಯತ್ಯಾಸದಿಂದ ಮಂತ್ರಗಳ ಅರ್ಥವ್ಯತ್ಯಾಸವಾಗುವುದೆಂಬ ವಿಚಾರ>
18-8

<ಸ್ವರಾಜ್ಯಶಬ್ದದ ವಿವರಣೆ>
6-566

<ಸ್ವರಾಟ್‍ ಶಬ್ದಾರ್ಥ ವಿಚಾರ>
17-51

<ಸ್ವರುಶಬ್ದದ ವಿವರಣೆ>
7-419

<ಸ್ವರ್ಕಾಃ ಎಂಬ ಶಬ್ದದ ಅರ್ಥವಿವರಣೆ>
22-214

<ಸ್ವರ್ಗ>
5-701

<ಸ್ವರ್ಗ ಮತ್ತು ಸ್ವರ್ಗವಾಸಿಗಳ ವಿಷಯ>
27-511

<ಸ್ವರ್ಭಾನು (ರಾಹು) ಎಂಬ ಅಸುರನ ವಿಷಯ>
5-693
19-502


<ಸ್ವಈಳ್ಹೇ ಎಂಬ ಶಬ್ದದ ಅರ್ಥವಿವರಣೆ>
6-82

<ಸ್ವರ್ವತ್‍ ಎಂಬ ಶಬ್ದದ ವಿವರಣೆ>
14-69

<ಸ್ವಶ್ವ ಎಂಬ ರಾಜನ ವಿಷಯ>
5-525

<ಸ್ವಶ್ವ ಎಂಬ ರಾಜಪುತ್ರನೊಡನೆ ಏತಶನೆಂಬ ಋಷಿಯು ಯುದ್ಧ ಮಾಡಿದಾಗ ಇಂದ್ರನು ಅವನಿಗೆ ಸಹಾಯ ಮಾಡಿದ ವಿಷಯ>
17-975

<ಸ್ವಸರೇಷು ಎಂಬ ಶಬ್ದದ ಅರ್ಥವಿವರಣೆ>
25-285

<ಸ್ವಸಾರಃ ಎಂಬ ಶಬ್ದ>
6-286 
14-499
26-680

<ಸ್ವಸ್ತಿ ಎಂಬ ಶಬ್ದದ ಅರ್ಥವಿವರಣೆ>
8-133
13-356 
18-805 
23-210
26-647


<ಸ್ವಸ್ತಿರಿದ್ಧಿ ಎಂಬ ಋಕ್ಕಿನ ನಿರುಕ್ತ>
28-481

<ಸಾಂಖ್ಯ ತತ್ತ್ವಗಳು>
12-540

<ಸಾಂಖ್ಯಾಯನ ಸಂಹಿತಾಕ್ರಮ>
1-45

<ಸಾಧ್ಯರ ವಿಶೇಷ ವಿವರಣೆ>
29-871

<ಸಾಧ್ಯರು ಯಜ್ಞವನ್ನು ಮಾಡಿದ ವಿಚಾರ>
12-657

<ಸಾಧ್ಯಾ ಎಂಬ ದೇವತೆಗಳ ವಿಷಯ>
29-861
29-943

<ಸಾನು ಶಬ್ದ ವಿವರಣೆ ಇತ್ಯಾದಿ>
11-602


<ಸಾಮಗಾನ ಮಾಡುವ ಋತ್ವಿಜರು ಏತಕ್ಕಾಗಿ ಮೂರು ಋಕ್ಕುಗಳನ್ನು ಪಠಿಸಿ ಗಾನ ಮಾಡುವರು?>
18-56

<ಸಾಮ ಶಬ್ದ ನಿರ್ವಚನ>
13-242
14-947

<ಸಾಮಕ್ಕೂ ಋಕ್ಕಿಗೂ ಇರುವ ಸಂಬಂಧ>
12-450

<ಸಾಮವೇದ ಸಂಹಿತಾ>
1-21

<ಸಾಮವೇದದ ಶಾಖೆಗಳು>
1-144

<ಸಾಮವೇದದ ಭಾಷ್ಯಕಾರರು>
1-279

<ಸಾಮವೇದದ ಪದಪಾಠಕಾರರು>
1-290

<ಸಾಯಣಾಚಾರ್ಯರು>
1-246

<ಸಾರಥಿ ಮತ್ತು ಲಗಾಮುಗಳ ವಿವರಣೆಯು>
21-571

<ಸಾರಮೇಯಗಳೆಂಬ ಯಮನ ಬಳಿ ಇರುವ ಎರಡು ನಾಯಿಗಳ ವಿಷಯ>
22-327 
27-436
27-474

<ಸ್ಥಾವರ ಜಂಗಮ>
12-559

<ಸ್ವಾದುಷಂ ಸದಃ ಎಂಬ ಋಕ್ಕಿನ ನಿರುಕ್ತ>
21-576

<ಸ್ವಾಹಾಕಾರಗಳ ದೇವತೆಗಳು (ಬೃ.ದೇ.)>
28-1023

<ಸ್ವಾಹಾಕೃತಿ ಶಬ್ದವಿವರಣೆ>
2-580
11-28l
14-271
15-744 
28-646
28-763
30-367

<ಸಿಂಧು ಶಬ್ದಾರ್ಥವಿವರಣೆ>
7-487 
8-647
13-139 
13-612
15-111


<ಸಿನೀವಾಲೀ ಎಂಬ ಹೆಸರಿನ ದೇವತೆ>
15-189

<ಸಿಮ ಶಬ್ದಾರ್ಥ>
8-335

<ಸಿಲಿಕ ಮಧ್ಯಮಾಸಃ ಎಂಬ ಶಬ್ದದ ಅರ್ಥವಿವರಣೆ>
12-270

<ಸೀಂ ಎಂಬ ಶಬ್ದದ ಅರ್ಥವಿವರಣೆ ಮತ್ತು ಪ್ರಯೋಗ>
9-389
16-459
20-362

<ಸೀತಾ ಎಂಬ ದೇವತೆಯ ವಿಷಯ>
18-639

<ಸ್ತ್ರೀದೇವತೆಗಳು>
5-649

<ಸ್ತ್ರೀಯರಿಗೆ ವೇದಾಧಿಕಾರದಲ್ಲಿ ವಿಷಯದ ವಿಮರ್ಶೆ>
10-481

<ಸುಕನ್ಯೆ ಎಂಬುವಳ ವೃದ್ಧನಾದ ಪತಿಯನ್ನು ಅಶ್ವಿನೀದೇವತೆಗಳು ಯುವಕನನ್ನಾಗಿ ಮಾಡಿದ ವಿಚಾರ>
12-692

<ಸುಕಿಂಶುಕಂ ಎಂಬ ಋಕ್ಕಿನ ನಿರುಕ್ತ>
29-399

<ಸುಕೇತುಭಿಃ, ಸುತುಕ ಎಂಬ ಶಬ್ದಗಳ ನಿರ್ವಚನ ಇತ್ಯಾದಿ>
27-247

<ಸುಖರಥಂ ಎಂಬ ಶಬ್ದದ ಅರ್ಥವಿವರಣೆ>
19-180

<ಸುತುಕ ಶಬ್ದದ ವಿವರಣೆ>
11-462

<ಸುದಾನು ಶಬ್ದಾರ್ಥ>
8-533 
13-228
14-36

<ಸುದಾಸನೆಂಬ ರಾಜನ ವಿಷಯ>
8-808
17-186

<ಸುದೇವೋ ಅಸಿ ವರುಣ ಎಂಬ ಋಕ್ಕಿನ ನಿರುಕ್ತ>
25-37

<ಸುಪರ್ಣಂ ವಸ್ತೇ ಎಂಬ ಋಕ್ಕಿನ ನಿರುಕ್ತ>
21-580

<ಸುಪರ್ಣ ಶಬ್ದಾರ್ಥ ವಿವರಣೆ>
8-481
12-672
27-247
30-954

<ಸುಬಂಧು ಋಷಿಯ ವೃತ್ತಾಂತ>
28-350
28-971

<ಸುಬಂಧುವಿನ ವಿಷಯದಲ್ಲಿ ಶಾಟ್ಯಾಯನ ಬ್ರಾಹ್ಮಣದಲ್ಲಿ ಹೇಳಿರುವ ಪೂರ್ವೇತಿಹಾಸವು>
28-356

<ಸುಭಗಾಸಃ ಎಂಬ ಶಬ್ದದ ವಿವರಣೆ>
15-107

<ಸುಮಂಗಲ ಶಬ್ದಾರ್ಥ ವಿವರಣೆ>
15-479

<ಸುಮತ್‍ ಶಬ್ದವಿವರಣೆ>
11-646

<ಸುಮತಿಶಬ್ದಾರ್ಥ ವಿವರಣೆ>
6-412 
13-197
15-534 
26-798
27-988

<ಸುಮ್ನಾವರೀ ಎಂಬ ಶಬ್ದ>
9-59

<ಸುಲಭಶಾಖಾ>
1-92

<ಸುವೃಕ್ತಿ ಶಬ್ದಾರ್ಥವಿವರಣೆ>
13-115
17-640

<ಸುಶೇವ ಶಬ್ದದ ಅರ್ಥವಿವರಣೆ>
17-467


<ಸುಶಿಪ್ರಃ ಎಂಬ ಶಬ್ದ>
8-308
15-205

<ಸುಶ್ರವಸ್‍ ಎಂಬ ರಾಜನ ವೃತ್ತಾಂತ>
5-295

<ಸ್ತುತಿಯ ಸ್ವರೂಪ (ಬೃ.ದೇ.)>
28-679

<ಸ್ತುಷೇಯ್ಯಂ ಪುರುವರ್ಪಸಂ ಎಂಬ ಋಕ್ಕಿನ ನಿರುಕ್ತ>
30-619

<ಸ್ರುಕ್‍ ಶಬ್ದಾರ್ಥ>
15-775

<ಸೂಕ್ತಭಾಂಜಿ (ಬೃ.ದೇ.)>
28-681

<ಸೂಕ್ತಭಾಗಿನೀ ಮಂತ್ರಗಳು (ಬೃ.ದೇ.)>
28-770

<ಸೂಕ್ತದ ಮುಖ್ಯದೇವತೆ (ಬೃ.ದೇ.)>
28-696

<ಸೂನರೀ ಶಬ್ದದ ಅರ್ಥವಿವರಣೆ>
25-48

<ಸೂನಾಃ ಎಂಬ ಶಬ್ದದ ವಿವರಣೆ>
12-195

<ಸೂನೃತಾ ಶಬ್ದಾರ್ಥ>
9-59

<ಸೂನೃತಾ ಶಬ್ದಕ್ಕೆ Max Muller ಪಂಡಿತನ ವಿವರಣೆ ಇತ್ಯಾದಿ>
10-570

<ಸೂರಿ ಶಬ್ದಾರ್ಥ ವಿಚಾರ, ನಿರ್ವಚನ, ಪ್ರಯೋಗ ಇತ್ಯಾದಿ>
8-120
10-216
13-381
13-412

<ಸೂರ್ಯದೇವತೆಯ ವಿಷಯ>
5-540
22-397
29-522

<ಸೂರ್ಯ ಶಬ್ದನಿಷ್ಪತ್ತಿ>
5-108
16-277

<ಸೂರ್ಯನ ಕುದುರೆಗಳ ಮಹಿಮೆ>
9-161

<ಸೂರ್ಯರಥ ವರ್ಣನೆ ಇತ್ಯಾದಿ>
4-144

<ಸೂರ್ಯನ ವಿಷಯವಾಗಿ ಋಗ್ವೇದದ ಋಕ್ಕುಗಳಲ್ಲಿರುವ ವಿವರಣೆ>
22-400

<ಸೂರ್ಯನ ವಿಷಪರಿಹಾರ ಶಕ್ತಿಯ ವರ್ಣನೆ>
14-357

<ಸೂರ್ಯನೊಡನೆ ಹೋರಾಡುತ್ತಿದ್ದ ಏತಶಋಷಿಗೆ ಇಂದ್ರನು ಸಹಾಯ ಮಾಡಿದ ವಿಚಾರ>
14-846

<ಸೂರ್ಯನು ಜಗಚ್ಛಕ್ಷುವು ಎಂಬ ವಿಷಯ>
12-381

<ಸೂರ್ಯನ ಉತ್ತರಾಯಣ ದಕ್ಷಿಣಾಯನ ಗತಿಗಳು>
12-385

<ಸೂರ್ಯನೇ ಚರಾಚರಾತ್ಮಕರಾದ ಜಗತ್ತಿಗೆ ಆತ್ಮಸ್ವರೂಪನಾಗಿರುವನೆಂಬ ವಿಚಾರ>
9-153

<ಸೂರ್ಯನ ಉದಯಾಸ್ತಗಳ ಮಹಿಮೆ>
9-165

<ಸೂರ್ಯನು ಯಾವಾಗಲೂ ಮುಳುಗುವುದೇ ಇಲ್ಲ, ಉದಯಿಸುವುದೂ ಇಲ್ಲ ಎಂಬ ವಿಷಯದಲ್ಲಿ ಐತರೇಯ ಬ್ರಾಹ್ಮಣದ ವಿವರಣೆಯು>
27-581

<ಸೂರ್ಯನೇ (ಪ್ರಜಾಪತಿ) ಸರ್ವಕ್ಕೂ ಮೂಲವು (ಬೃ.ದೇ.)>
28-689

<ಸೂರ್ಯನ ಗತಿಗೆ ವೇದಗಳು ಕಾರಣವು ಎಂಬ ವಿಚಾರ>
30-894

<ಸೂರ್ಯ ಪುತ್ರಿಯ ವಿಚಾರ>
4-94

<ಸೂರ್ಯ ಪುತ್ರಿಯು ಅಶ್ವಿನೀದೇವತೆಗಳ ರಥವನ್ನು ಏರಿದ ವಿಷಯ>
9-471
9-365

<ಸೂರ್ಯಾದೇವಿಯ ವಿವಾಹ ವರ್ಣನೆ>
14-26
29-366

<ಸೂರ್ಯಾ ವಿವಾಹದ ವಿಷಯದಲ್ಲಿ ಆಧ್ಯಾತ್ಮಿಕತತ್ತ್ವದ ವಿವರಣೆ>
29-345

<ಸೂರ್ಯಾದೇವಿಯ ಸ್ವಯಂವರ>
9-189

<ಸೂರ್ಯಾಸೂಕ್ತದ (ಋ.ಸಂ. ೧೦-೪೫) ವಿವರಣೆ (ಬೃ.ದೇ.)>
28-982

<ಸೂರ್ಯೆಯನ್ನು (ಪ್ರಜಾಪತಿಯ ಪುತ್ರಿಯಾದ) ಪಡೆಯಲು ದೇವತೆಗಳು ನಡೆಸಿದ ಸ್ಪರ್ಧೆಯ ವಿಚಾರ>
11-333
29-341

<ಸ್ಥೂರಂ ರಾಧಃ ಎಂಬ ಋಕ್ಕಿನ ನಿರುಕ್ತ>
23-435

<ಸೃಂಜಯನೆಂಬ ರಾಜನ ವಿಷಯ>
17-921
20-831

<ಸೃಣಿ ಶಬ್ದದ ನಿರ್ವಚನ, ಅರ್ಥವಿವರಣೆ>
18-38
30-154

<ಸೃಣ್ಯೇವ ಜರ್ಭರೀ ಎಂಬ ಋಕ್ಕಿನ ನಿರುಕ್ತ>
30-260

<ಸೃಪ್ರ, ಬೃಬದುಕ್ಥಂ ಎಂಬ ಶಬ್ದಗಳ ನಿರ್ವಚನ, ಅರ್ಥವಿವರಣೆ>
24-236

<ಸೃಷ್ಟಿಕ್ರಮ>
30-75

<ಸೃಷ್ಟಿಗೆ ಪೂರ್ವದಲ್ಲಿ ಏನಿದ್ದಿತು?>
30-758

<ಸೃಷ್ಟಿಗೆ ಪೂರ್ವದಲ್ಲಿ ಇದ್ದ ಸ್ಥಿತಿ ಇತ್ಯಾದಿ>
12-315 
30-749
30-629

<ಸೃಷ್ಟಿವಿಷಯ>
30-1273

<ಸೃಷ್ಟಿಯು ರೂಪುಗೊಂಡ ವಿಧಾನ–ಸೃಷ್ಟಿಸಲ್ಪಟ್ಟ ವಸ್ತುಗಳಲ್ಲಿ ಭೋಕ್ತೃ, ಭೋಗ್ಯ ಎಂದು ಎರಡು ವಿಧ>
30-774

<ಸೃಷ್ಟಿಕ್ರಮವನ್ನು ಬ್ರಹ್ಮನು ಮಾತ್ರ ತಿಳಿಯಬಲ್ಲನು, ಬೇರೆ ಯಾರೂ ತಿಳಿಯಲಾರರು>
30-780

<ಸೃಷ್ಟಿಕ್ರಮವನ್ನು ಯಜ್ಞಕ್ಕೂ ಯಜ್ಞವನ್ನು ವಸ್ತ್ರನಿರ್ಮಾಣಕ್ಕೂ ಹೋಲಿಸಿ ಮಾಡಿರುವ ವಿವರಣೆ>
30-785

<ಸ್ತೇನಶಬ್ದನಿಷ್ಪತ್ತಿ, ನಿರ್ವಚನ, ಅರ್ಥವಿವರಣೆ ಇತ್ಯಾದಿ>
14-947
15-453

<ಸ್ನೇಹಿತನ ಕರ್ತವ್ಯ, ಸ್ನೇಹಿತನ ಲಕ್ಷಣ>
30-559

<ಸ್ವೇದವಹ್ಯಶಬ್ದಾರ್ಥ>
9-562

<ಸೋಬರಿ ಮತ್ತು ಚಿತ್ರ ಎಂಬುವರ ವೃತ್ತಾಂತ (ಬೃ.ದೇ.)>
28-914

<ಸೋಮಕನ (ಸಹದೇವಪುತ್ರನಾದ) ವೃತ್ತಾಂತ>
17-924

<ಸೋಮದೇವತೆಯ ವಿಷಯ>
5-626

<ಸೋಮದೇವತೆಯ ವಿಷಯದಲ್ಲಿ ಎರಡು ವಿಧವಾದ ವಿವರಣೆ>
29-359

<ಸೋಮಲತೆಯ ವಿಷಯ>
1-674
3-411
15-390 
26-68
29-351

<ಸೋಮಲತೆ ಮತ್ತು ಗ್ರಾವಗಳ ವಿಷಯ>
29-115

<ಸೋಮಲತೆಗಳನ್ನು ಜಜ್ಜುವ ಗ್ರಾವಗಳ ವಿಷಯ>
29-1003

<ಸೋಮಶಬ್ದನಿರ್ವಚನ ಮತ್ತು ನಾನಾರ್ಥಗಳು>
4-88 
16-208
26-5 
29-351

<ಸೋಮಂ ಗಾವಃ ಎಂಬ ಋಕ್ಕಿಗೆ ಅಧಿದೈವತ ಮತ್ತು ಅಧ್ಯಾತ್ಮ ಪಕ್ಷಗಳಲ್ಲಿ ಅರ್ಥವಿವರಣೆ>
26-812

<ಸೋಮಪಾನದಿಂದ ಉಂಟಾದ ಹರ್ಷದಿಂದ ಇಂದ್ರನು ಮಾಡಿದ ಅನೇಕ ಸಾಹಸ ಕೃತ್ಯಗಳ ವರ್ಣನೆ>
14-755

<ಸೋಮಪಾನ ಮಾಡುವುದಕ್ಕೆ ಉಪಯೋಗಿಸುವ ಗ್ರಹಪಾತ್ರೆಗಳು>
30-499

<ಸೋಮಪಾನದಲ್ಲಿ ಇಂದ್ರನಿಗೆ ಇರುವ ಆದರ>
16-644 
26-80

<ಸೋಮಪಾನದಲ್ಲಿ ಉಪಯೋಗಿಸುವ ನಾನಾವಿಧ ಪಾತ್ರೆಗಳು>
3-412

<ಸೋಮಪಾನದಲ್ಲಿ (ಯಜ್ಞದಲ್ಲಿ ನಡೆಯುವ) ಪ್ರಥಮಸ್ಥಾನವು ಯಾರಿಗೆ ಸೇರಬೇಕೆಂಬ ವಿಚಾರದಲ್ಲಿ ಐತರೇಯ ಬ್ರಾಹ್ಮಣದಲ್ಲಿರುವ ಪೂರ್ವೇತಿಹಾಸ ಇತ್ಯಾದಿ>
10-624

<ಸೋಮಪ್ರವಾಹಕರ್ಮದ ಪೂರ್ವಪೀಠಿಕೆ>
15-457 
26-24

<ಸೋಮಪ್ರವಾಹವೆಂಬ ಯಜ್ಞಾಂಗಕರ್ಮ>
26-26

<ಸೋಮಪ್ರವಾಹಣಕರ್ಮದ ಉತ್ತರಪೀಠಿಕೆ>
15-470
26-36

<ಸೋಮಪ್ರವಾಹವೆಂಬ ಸೋಮವನ್ನು ಉತ್ತರವೇದಿಯ ಬಳಿಗೆ ತರುವ ಕರ್ಮದ ವಿವರಣೆ (ಐತರೇಯ ಬ್ರಾಹ್ಮಣದಲ್ಲಿರುವಂತೆ)>
15-459

<ಸೋಮರಸದ ವಿಷಯ>
14-206 
29-1003

<ಸೋಮರಸಕ್ಕೆ ಮಿಶ್ರಮಾಡುವ ಆಶಿರದ್ರವ್ಯಗಳು>
14-209

<ಸೋಮರಸವನ್ನು ಸಿದ್ಧಪಡಿಸುವ ಕ್ರಮ>
10-631
14-744 
26-80
29-351

<ಸೋಮರಸದ ಸ್ವರೂಪ ಮತ್ತು ಅದನ್ನು ಯಜ್ಞದಲ್ಲಿ ಸಂಸ್ಕರಿಸುವ ವಿಧಾನ ಇತ್ಯಾದಿ>
10-637 
26-74

<ಸೋಮರಸವನ್ನಿಡುವ ಪಾತ್ರೆಗಳು>
26-71

<ಸೋಮರಸಗಳು ನಾನಾವಿಧ>
26-682

<ಸೋಮಯಾಗದ ಸಪ್ತಸಂಸ್ಥೆಗಳ ವಿವರಣೆ>
20-564
26-93

<ಸೋಮಯಾಗಫಲ ಪ್ರಾಪ್ತಿವರ್ಣನೆ>
11-147
26-93

<ಸೋಮವಿಕ್ರಯಣ (ಎಂಬ ಯಜ್ಞಾಂಗ) ಕರ್ಮದ ವಿಚಾರ ಇತ್ಯಾದಿ>
24-246
27-153

<ಸೋಮಮಂತ್ರಗಳನ್ನು ಅಧ್ಯಯನಮಾಡಿದವರ ನಾನಾವಿಧ ಪಾಪಪರಿಹಾರ>
26-95

<ಸೋಮಯಾಗದ ಪ್ರಭೇದಗಳು>
2-136

<ಸೋಮನಿಗೂ ಸೂರ್ಯಾದೇವಿಗೂ ನಡೆದ ವಿವಾಹವಿಷಯದಲ್ಲಿ ಪೂರ್ವೇತಿಹಾಸವು>
29-339

<ಸೋಮನು ದೇವತೆಗಳನ್ನು ಬಿಟ್ಟು ಓಡಿಹೋದ ವಿಚಾರ (ಬೃ.ದೇ.)>
28-989

<ಸೋಮನು ಓಷಧಿಗಳಿಗೆ ರಾಜನೆಂಬ ವಿಚಾರ>
6-177

<ಸೋಮಲತೆಗಳನ್ನು ಶ್ಯೇನಪಕ್ಷಿಯು ಸ್ವರ್ಗದಿಂದ ತರುತ್ತಿದ್ದಾಗ ಅವುಗಳನ್ನು ಗಂಧರ್ವನು ಅಪಹರಿಸಿದ ವಿಚಾರ>
27-153

<ಸೋಮವನ್ನು (ಆದಿತ್ಯನ ತೇಜೋರೂಪವಾದ) ರಕ್ಷಿಸಿದ ಗಂಧರ್ವನೇ ಹಿರಣ್ಯಗರ್ಭನು>
17-583

<ಸೋಮವೇ ಆದಿತ್ಯನ ತೇಜಸ್ಸು, ಗಾಯತ್ರಿಯೇ ಇದನ್ನರಿಯಲು ಮುಖ್ಯ ಸಾಧನ>
17-581

<ಸೋಮಾಭಿಷವಣ ವಿಷಯವಾಗಿ ತೃತೀಯ ಸವನದಲ್ಲಿ ಆಚರಿಸಬೇಕಾದ ಕೆಲವು ವಿಶೇಷ ಸಂಗತಿಗಳು>
15-711

<ಸೋಮಾಹರಣೋಪಾಖ್ಯಾನ–ಶ್ಯೇನಪಕ್ಷಿಯು ಸ್ವರ್ಗದಿಂದ ಸೋಮವನ್ನು ತರುವಾಗ ಸೋಮಪಾಲಕರಲ್ಲಿ ಒಬ್ಬನಾದ ಕೃಶಾನುವು ಆ ಪಕ್ಷಿಯ ರೆಕ್ಕೆಯನ್ನು ಕತ್ತರಿಸಿದ ವಿಚಾರ>
18-190
26-6

<ಸೋಮಾಹರಣ ಪ್ರಸಂಗದಲ್ಲಿ ಋಗ್ವೇದಾದಿಗಳಲ್ಲಿರುವ ಕೆಲವು ಋಕ್ಕುಗಳ ವಿವರಣೆ>
26-15

<ಸೋಮಾಹರಣಪ್ರಸಂಗದಲ್ಲಿ ಶ್ಯೇನಪಕ್ಷಿಯ ಪ್ರಯತ್ನ>
17-569
26-15

<ಸೋಮಾಹರಣಕ್ಕಾಗಿ ಛಂದಸ್ಸುಗಳ ಪ್ರಯತ್ನ ಮತ್ತು ಪರಾಭವ>
17-574 
26-20

<ಸ್ತೋತ್ರ ಮತ್ತು ಶಸ್ತ್ರಮಂತ್ರಗಳ ವಿವರಣೆ>
1-651

<ಸ್ತೋಮಗಳ (ಒಂಭತ್ತು ವಿಧ) ವಿವರಣೆ>
2-138
14-586
29-370

<ಸ್ತೋಮಶಬ್ದವಿವರಣೆ–ಸ್ತೋಮಗಳಲ್ಲಿ ಎಷ್ಟುವಿಧ ಇತ್ಯಾದಿ>
16-303

<ಸ್ತೋಮ, ವಿಷ್ಟುತಿ, ತೃಚ, ಪರ್ಯಾಯ ಇವುಗಳ ವಿವರಣೆ>
2-137

<ಸ್ತೋಮಗಳ ಸ್ವರೂಪ, ಮಹಿಮೆ ಇತ್ಯಾದಿ>
16-697

<ಸ್ತೋಮಗಳ ವಿಷಯವಾಗಿ ಕೆಲವು ವಿಶೇಷ ಸಂಗತಿಗಳು>
10-222
11-50

<ಸ್ತೋಮವೃದ್ಧಿ>
2-149

<ಸ್ತೋಮಾವಾಪ>
2-149

<ಸೌಚಿಕಾಗ್ನಿಯ ವಿಷಯ, ಅಗ್ನಿಯು ದೇವತೆಗಳನ್ನು ಬಿಟ್ಟು ಓಡಿಹೋಗಿ ಅವಿತುಕೊಂಡಿದ್ದ ವಿಚಾರ (ಬೃ.ದೇ.)>
28-964

<ಸೌರ್ಯವೈಶ್ವಾನರನೆಂದರೆ ಅಗ್ನಿಯ ರೂಪಾಂತರ (ಬೃ.ದೇ.)>
28-716

[ಹ]
<ಹಗಲಿಗೆ ಆದಿತ್ಯನೂ ರಾತ್ರಿಗೆ ಅಗ್ನಿಯೂ ಅಧಿಪತಿಗಳೆಂಬ ವಿಷಯ>
27-236

<ಹತ್ತನೆಯ ಮಂಡಲದ ಪೀಠಿಕೆ>
27-177

<ಹತ್ತನೆಯ ಮಂಡಲದ ಋಷಿಗಳು ಮತ್ತು ಅವರಿಂದ ದೃಷ್ಟವಾದ ಸೂಕ್ತಗಳ ವಿವರಣೆ>
27-178

<ಹಯೇ ಎಂಬ ಶಬ್ದದ ಅರ್ಥವಿವರಣೆ ಮತ್ತು ಪ್ರಯೋಗ ಉದಾಹರಣೆ ಸಹಿತ>
19-739

<ಹರಯಾಣಃ ಎಂಬ ಶಬ್ದದ ನಿರ್ವಚನ>
24-128

<ಹರ್ಮ್ಯ ಶಬ್ದಾರ್ಥವಿವರಣೆ>
13-21

<ಹವ್ಯವಾಟ್‍>
6-206

<ಹವ್ಯವಾಹನ ದೇವತೆಗಳಿಗೆ ಹವಿಸ್ಸನ್ನು ವಹಿಸುವ ಅಗ್ನಿ>
18-849

<ಹವಿರ್ಧಾನದ ವಿಷಯ>
17-643

<ಹವಿರ್ಧಾನದಸ್ವರೂಪ, ಐತರೇಯ ಬ್ರಾಹ್ಮಣದಲ್ಲಿ ವಿವರಿಸಿರುವಂತೆ>
27-414

<ಹವಿರ್ಧಾನಶಕಟಗಳು>
27-407

<ಹವಿರ್ಧಾನಸ್ತುತಿಯು–ತೈತ್ತಿರೀಯ ಸಂಹಿತೆಯಲ್ಲಿರುವಂತೆ>
27-414

<ಹಂಸಃ ಶುಚಿಷತ್‍ ಎಂಬ ಋಕ್ಕಿನ ಪ್ರತಿಪದಕ್ಕೂ ವಿಶೇಷಾರ್ಥ ವಿವರಣೆ>
18-397

<ಹಂಸಃ ಶುಚಿಷತ್‍ ಎಂಬ ಋಕ್ಕಿಗೆ ಅಗ್ನಿ ಮತ್ತು ಆದಿತ್ಯಪರವಾದ ಅರ್ಥ>
18-397

<ಹಂಸಃ ಶುಚಿಷತ್‍ ಎಂಬ ಋಕ್ಕನ್ನು ಹಂಸವತೀ ಋಕ್ಕೆಂದು ಐತರೇಯಬ್ರಾಹ್ಮಣದಲ್ಲಿ ವರ್ಣಿಸಿ ಹೇಳಿರುವ ವಿವರಣೆ ಇತ್ಯಾದಿ>
18-400

<ಹಂಸಃ ಶುಚಿಷತ್‍ ಎಂಬ ಋಕ್ಕಿಗೆ ನಿರುಕ್ತ>
18-402

<ಹಂಸಃ ಶುಚಿಷತ್‍ ಎಂಬ ಋಕ್ಕಿನ ವಿಷಯದಲ್ಲಿ Wilson ಎಂಬ ಆಂಗ್ಲಪಂಡಿತನ ಅಭಿಪ್ರಾಯ>
18-403

<ಹಸ್ತಘ್ನದ ವಿವರಣೆ>
21-586

<ಹಾರಿಯೋಜನಂ ಎಂಬ ಶಬ್ದದ ಅರ್ಥವಿವರಣೆ>
7-47

<ಹಾವುಗಳಲ್ಲಿರುವ ನಾನಾ ಭೇದಗಳು>
14-335

<ಹ್ವಾರ ಶಬ್ದಾರ್ಥ ವಿವರಣೆ>
11-187
13-507

<ಹಿನೋತಾ ನೋ ಅಧ್ವರಂ ಎಂಬ ಋಕ್ಕಿನ ನಿರುಕ್ತ>
27-808

<ಹಿಮ ಶಬ್ದಕ್ಕೆ ಸಂವತ್ಸರಾರ್ಥವಿಚಾರ>
6-148

<ಹಿರಣ್ಯಗರ್ಭ ಶಬ್ದಾರ್ಥ ವಿಚಾರ>
30-629

<ಹಿರಣ್ಯಗರ್ಭಃ ಸಮವರ್ತತ ಎಂಬ ಋಕ್ಕಿನ ನಿರುಕ್ತ>
30-630

<ಹಿರಣ್ಯಪಾಣಿಶಬ್ದದ ಅರ್ಥವಿವರಣೆ, ರೂಪನಿಷ್ಪತ್ತಿ ಇತ್ಯಾದಿ>
3-75

<ಹಿರಣ್ಯಯೇಭಿಃ ಎಂಬ ಶಬ್ದ>
6-136

<ಹಿರಣ್ಯವರ್ಣಾಂ ಎಂಬ ಶ್ರೀ ಸೂಕ್ತವು (೧೧ನೇ ಪರಿಶಿಷ್ಟ ಸೂಕ್ತ) ೧೫ ಋಕ್ಕುಗಳು ಮತ್ತು ಕೆಲವು ಶ್ಲೋಕಗಳಸಹಿತ, ಪ್ರತಿಪದಾರ್ಥ ವಿಶೇಷ ವಿಷಯಗಳು ಇತ್ಯಾದಿ>
20-267

<ಹಿರಣ್ಯಸ್ತೂಪಃ ಸವಿತಃ ಎಂಬ ಋಕ್ಕಿನ ನಿರುಕ್ತ>
30-1005

<ಹೀಳಿತಃ ಎಂಬ ಶಬ್ದ>
6-579

<ಹೃದಯಾಕಾಶ, ದಹರಾಕಾಶ>
29-797

<ಹೆಸರಿನ ಸ್ವರೂಪ ಮತ್ತು ಅವು ಹೇಗೆ ಉತ್ಪತ್ತಿಯಾದವು (ಬೃ.ದೇ.)>
28-682

<ಹೋತಾ ಎಂಬ ಶಬ್ದದ ಅರ್ಥಗಳು ಮತ್ತು ಪ್ರಯೋಗಗಳು>
13-258

<ಹೋತೃ ಮತ್ತು ಹೋತೃಕರ್ಮಗಳು>
6-482

<ಹೋತೃಶಬ್ದದ ವಿವಿಧಾರ್ಥಗಳು ಮತ್ತು ಪ್ರಯೋಗಗಳು>
11-182
15-626 
17-532
26-661

<ಹೋತೃವಿನ ಕರ್ತವ್ಯ>
12-166

[ಕ್ಷ]
<ಕ್ಷತ್ರಶಬ್ದದ ವಿಶೇಷಾರ್ಥ ವಿವರಣೆ>
11-677 
12-81

<ಕ್ಷಪಾ ಶಬ್ದಾರ್ಥ ವಿಚಾರ>
6-269 
17-310

<ಕ್ಷಯದ್ವೀರ ಶಬ್ದಾರ್ಥ ವಿವರಣೆ>
9-101

<ಕ್ಷಾಮದೇವತೆಯ ವಿಷಯ>
30-1043

<ಕ್ಷುರ ಶಬ್ದದ ಅರ್ಥ ವಿವರಣೆ>
13-49

<ಕ್ಷೇತ್ರಪತಿ ಎಂಬ ದೇವತೆಯ ವಿಷಯ>
18-623

<ಕ್ಷೇತ್ರಪತಿಯು ಇಂದ್ರ ಅಥವಾ ಅಗ್ನಿಯ ವಿಭೂತಿ ವಿಶೇಷವೆಂಬ ವಿಷಯ>
18-627

<ಕ್ಷೇತ್ರಸ್ಯ ಪತಿ (ಬೃ.ದೇ.)>
28-721

<ಕ್ಷೇತ್ರಸ್ಯ ಪತಿನಾ ಎಂಬ ಋಕ್ಕಿಗೆ ಯಾಸ್ಕರ ನಿರ್ವಚನವು>
18-624

.......
